\chapter{CONNECTEDNESS AND THE INTERMEDIATE VALUE THEOREM}\label{ivt}

\section{CONNECTED SUBSETS OF $\R$}
There appears to be no way of finding an exact solution to such an equation as
\begin{equation}\label{ivt_eqn1}  \sin x = x - 1.
\end{equation}
No algebraic trick or trigonometric identity seems to help.  To be entirely honest we must ask
what reason we have for thinking that \eqref{ivt_eqn1} \emph{has} a solution.  True, even a
rough sketch of the graphs of the functions $x \mapsto \sin x$ and $x \mapsto x - 1$ seems to
indicate that the two curves cross not far from $x = 2$. But is it not possible in the vast
perversity of the-way-things-are that the two curves might somehow skip over one another
without actually having a point in common?  To say that it seems unlikely is one thing, to say
that we \emph{know} it can not happen is another.  The solution to our dilemma lies in a
result familiar from nearly every beginning calculus course: the \emph{intermediate value
theorem.} Despite the complicated symbol-ridden formulations in most calculus texts what this
theorem really says is that continuous functions from $\R$ to $\R$ take intervals to
intervals.  In order to prove this fact we will first need to introduce an important
topological property of all intervals, \emph{connectedness}.  Once we have proved the
\emph{intermediate value theorem} not only will we have the intellectual satisfaction of
saying that we \emph{know} \eqref{ivt_eqn1} has a solution, but we can utilize this same
theorem to find approximations to the solution of \eqref{ivt_eqn1} whose accuracy is limited
only by the computational means at our disposal.

So, first we define ``connected''.  Or rather, we define ``disconnected'', and then agree that
sets are connected if they are not disconnected.

\begin{defn} A subset $A$ of $\R$ is
 \index{disconnected!subset of $\R$}%
\df{disconnected} if there exist disjoint nonempty sets $U$ and $V$ both open in $A$ whose
union is~$A$.  In this case we say that the sets $U$ and $V$ \emph{disconnect}~$A$.  A subset
$A$ of $\R$ is
 \index{connected!subset of $\R$}%
\df{connected} if it is not disconnected.
\end{defn}

\begin{prop}\label{prop_nasc_conn}  A set $A \subseteq \R$ is connected if and only if the
only subsets of $A$ which are both open in $A$ and closed in $A$ are the null set and $A$
itself.
\end{prop}

\begin{proof} Exercise. (Solution~\ref{sol_nasc_conn}.)  \ns  \end{proof}

\begin{exam} The set $\{1,4,8\}$ is disconnected.
\end{exam}

\begin{proof} Let $A = \{1,4,8\} \subseteq \R$.  Notice that the sets $\{1\}$ and $\{4,8\}$
are open subsets of~$A$.  Reason: $(-\infty, 2) \cap A = \{1\}$ and $(2,\infty) \cap A =
\{4,8\}$; so $\{1\}$ is the intersection of an open subset of $\R$ with $A$, and so is
$\{4,8\}$ (see definition~\ref{rel_open}). Thus $\{1\}$ and $\{4,8\}$ are disjoint nonempty
open subsets of $A$ whose union is~$A$.  That is, the sets $\{1\}$ and $\{4,8\}$
disconnect~$A$.
\end{proof}

\begin{exam} The set $\Q$ of rational numbers is disconnected.
\end{exam}

\begin{proof} The sets $\{x \in \Q \colon x < \pi\}$ and $\{x \in \Q \colon x > \pi\}$
disconnect~$\Q$.
\end{proof}

\begin{exam} The set $\{\frac1n \colon n \in \N\}$ is disconnected.
\end{exam}

\begin{proof} Problem.  \ns  \end{proof}

If $A$ is a subset of $\R$, it is somewhat awkward that ``connected'' is defined in terms of
(relatively) open subsets of~$A$.  In general, it is a nuisance to deal with relatively open
subsets.  It would be much more convenient to deal only with the familiar topology of~$\R$.
Happily, this can be arranged.

It is an elementary observation that $A$ is disconnected if we can find disjoint sets $U$ and
$V$, both open in $\R$, whose intersections with $A$ are nonempty and whose union
contains~$A$. (For then the sets $U \cap A$ and $V \cap A$ disconnect $A$.) Somewhat less
obvious is the fact that $A$ is disconnected if and only if it can be written as the union of
two nonempty subsets $C$ and $D$ of $\R$ such that
 \begin{equation}\label{eqn_mut_sep} C \cap \clo D  = \clo C \cap D
              = \emptyset.  \end{equation}
(The indicated closures are in $\R$.)

\begin{defn} Two nonempty subsets $C$ and $D$ of $\R$ which satisfy
equation~\eqref{eqn_mut_sep} are said to be
 \index{mutually separated}%
 \index{separated!mutually}%
\df{mutually separated} in~$\R$.
\end{defn}

\begin{prop}\label{prop_mut_sep} A subset $A$ of $\R$ is disconnected if and only if it is the
union of two nonempty sets mutually separated in~$\R$.
\end{prop}

\begin{proof} If $A$ is disconnected, it can be written as the union of two disjoint nonempty
sets $U$ and $V$ which are open in~$A$.  (These sets need not, of course, be open in $\R$.) We
show that $U$ and $V$ are mutually separated.  It suffices to prove that $U \cap \clo V$ is
empty, that is, $U \subseteq \R \setminus \clo V$.  To this end suppose that $u \in U$. Since
$U$ is open in $A$, there exists $\delta > 0$ such that
   \[A \cap J_\delta(u) = \{x \in A \colon \abs{x-u} < \delta\} \subseteq U
                             \subseteq \R \setminus V.\]
The interval $J_\delta(u) = (u-\delta,u+\delta)$ is the union of two sets: $A \cap
J_\delta(u)$ and $A^c \cap J_\delta(u)$.  We have just shown that the first of these belongs
to $\R \setminus V$. Certainly the second piece contains no points of $A$ and therefore no
points of~$V$.  Thus $J_\delta(u) \subseteq \R \setminus V$. This shows that $u$ does not
belong to the closure (in $\R$) of the set~$V$; so $u \in \R \setminus \clo V$.  Since $u$ was
an arbitrary point of $U$, we conclude that $U \subseteq \R \setminus \clo V$.

Conversely, suppose that $A = U \cup V$ where $U$ and $V$ are nonempty sets mutually separated
in~$\R$.  To show that the sets $U$ and $V$ disconnect $A$, we need only show that they are
open in $A$, since they are obviously disjoint.

Let us prove that $U$ is open in $A$.  Let $u \in U$ and notice that since $U \cap \clo V$ is
empty, $u$ cannot belong to $\clo V$.

Thus there exists $\delta > 0$ such that $J_\delta(u)$ is disjoint from $V$. Then certainly $A
\cap J_\delta(u)$ is disjoint from $V$. Thus $A \cap J_\delta(u)$ is contained in $U$.
Conclusion: $U$ is open in~$A$.
\end{proof}

The importance of proposition~\ref{prop_mut_sep} is that it allows us to avoid the definition
of ``connected'', which involves \emph{relatively} open subsets, and replace it with an
equivalent condition which refers only to closures in~$\R$.  There is no important idea here,
only a matter of convenience.  We will use this characterization in the proof of our next
proposition, which identifies the connected subsets of the real line.  (To appreciate the
convenience that proposition ~\ref{prop_mut_sep} brings our way, try to prove the next result
using only the definition of ``connected'' and not \ref{prop_mut_sep}.)  As mentioned earlier
we need to know that in the real line the intervals are the connected sets.  This probably
agrees with your intuition in the matter.  It is plausible, and it is true; but it is not
obvious, and its proof requires a little thought.

\begin{defn} A subset $J$ of $\R$ is an
 \index{interval}%
\df{interval} provided that it satisfies the following condition: if $c$, $d \in J$ and $c < z
< d$, then $z \in J$. (Thus, in particular, the empty set and sets containing a single point
are intervals.)
\end{defn}

\begin{prop}\label{prop_conn_int}  A subset $J$ of $\R$ is connected if and only if it is
an interval.
\end{prop}

\begin{proof} To show that intervals are connected argue by contradiction. Assume that there
exists an interval $J$ in $\R$ which is not connected.  By proposition~\ref{prop_mut_sep}
there exist nonempty sets $C$ and $D$ in $\R$ such that $J = C \cup D$ and
    \[ C \cap \clo D = \clo C \cap D = \emptyset \]
(closures are taken in~$\R$).  Choose $c \in C$ and $d \in D$. Without loss of generality
suppose $c < d$. Let $A = (-\infty, d) \cap C$ and $z = \sup A$.  Certainly $c \le z \le d$.
Now $z \in \clo C$. (We know from example~\ref{sup_in_clo} that $z$ belongs to $\clo A$ and
therefore to $\clo C$.)   Furthermore $z \in \clo D$. (If $z = d$, then $z \in D \subseteq
\clo D$.  If $z < d$, then the interval $(z,d)$ is contained in $J$ and, since $z$ is an upper
bound for $A$, this interval contains no point of $C$. Thus $(z,d) \subseteq D$ and $z \in
\clo D$.)  Finally, since $z$ belongs to $J$ it is a member of either $C$ or $D$. But $z \in
C$ implies $z \in C \cap \clo D = \emptyset$, which is impossible; and $z \in D$ implies $z
\in \clo C \cap D = \emptyset$, which is also impossible.

The converse is easy. If $J$ is not an interval, then there exist numbers $c < z < d$ such
that $c$, $d \in J$ and $z \notin J$. It is easy to see that the sets $(-\infty, z) \cap J$
and $(z, \infty) \cap J$ disconnect~$J$.
\end{proof}

\section{CONTINUOUS IMAGES OF CONNECTED SETS} Some of the facts we prove in this chapter are
quite specific to the real line~$\R$.  When we move to more complicated metric spaces no
reasonable analog of these facts is true. For example, even in the plane $\R^2$ nothing
remotely like proposition~\ref{prop_conn_int} holds. While it is not unreasonable to guess
that the connected subsets of the plane are those in which we can move continuously between
any two points of the set without leaving the set, this conjecture turns out to be wrong.  The
latter property, \emph{arcwise connectedness}, is sufficient for connectedness to hold---but
is not necessary.  In chapter~\ref{conn} we will give an example of a connected set which is
not arcwise connected.

Despite the fact that some of our results are specific to~$\R$, others will turn out to be
true in very general settings.  The next theorem, for example, which says that continuity
preserves connectedness, is true in~$\R$, in~$\R^n$, in metric spaces, and even in general
topological spaces.  More important, the same proof works in all these cases!  Thus when you
get to chapter~\ref{conn}, where connectedness in metric spaces is discussed,  you will
already know the proof that the continuous image of a connected set is itself connected.

\begin{thm}\label{thm_contimg_conn} The continuous image of a connected set is connected.
\end{thm}

\begin{proof} Exercise. \emph{Hint.}  Prove the contrapositive.  Let $f\colon A \sto \R$ be
a continuous function where $A \subseteq \R$.  Show that if $\ran f$ is disconnected, then so
is~$A$.  (Solution~\ref{sol_contimg_conn}.)   \ns
\end{proof}

The important \emph{intermediate value theorem}, is an obvious corollary of the preceding
theorem.

\begin{thm}[Intermediate Value Theorem: Conceptual Version]\label{thm_ivt1} The continuous
 \index{intermediate value theorem}%
image in $\R$ of an interval is an interval.
\end{thm}

\begin{proof} Obvious from \ref{prop_conn_int} and \ref{thm_contimg_conn}.
\end{proof}

A slight variant of this theorem, familiar from beginning calculus, helps explain the
\emph{name} of the result.  It says that if a continuous real valued function defined on an
interval takes on two values, then it takes on every intermediate value, that is, every value
between them. It is useful in establishing the existence of solutions to certain equations and
also in approximating these solutions.


\begin{thm}[Intermediate Value Theorem: Calculus Text Version]\label{thm_ivt2}  Let $f \colon J
\sto \R$ be a continuous function defined on an interval $J \subseteq \R$. If $a$, $b \in \ran
f$ and $a < z < b$, then $z \in \ran f$.
\end{thm}

\begin{proof} Exercise.  (Solution~\ref{sol_ivt2}.)   \ns  \end{proof}

Here is a typical application of the \emph{intermediate value theorem}.

\begin{exam}\label{exam_ivt_eqn} The equation
 \begin{equation}\label{ivt_soln_eqn2}
   x^{27} + 5x^{13} + x = x^3+ x^5 + \frac2{\sqrt{1+3x^2}}
 \end{equation}
has at least one real solution.
\end{exam}

\begin{proof}  Exercise.  \emph{Hint.}  Consider the function $f$
whose value at $x$ is the left side minus the right side of~\eqref{ivt_soln_eqn2}. What can
you say without much thought about $f(2)$ and $f(-2)$?  (Solution~\ref{sol_ivt_eqn}.)   \ns
\end{proof}

As another application of the \empty{intermediate value theorem} we prove a fixed point
theorem. (\textbf{Definition:} Let $f \colon S \sto S$ be a mapping from a set $S$ into
itself. A point $c$ in $S$ is a
 \index{fixed point}%
\df{fixed point} of the function $f$ if $f(c) = c$.)  The next result is a (very) special case
of the celebrated Brouwer fixed point theorem, which says that every continuous map from the
closed unit ball of $\R^n$ into itself has a fixed point. The proof of this more general
result is rather complicated and will not be given here.

\begin{prop}\label{prop_ivt_fp} Let $a < b$ in $\R$.  Every continuous map of the interval
$[a,b]$ into itself has a fixed point.
\end{prop}

\begin{proof} Exercise.  (Solution~\ref{sol_ivt_fp}.)    \ns  \end{proof}

\begin{exam} The equation
    \[ x^{180} + \frac{84}{1+x^2+\cos^2x} = 119 \]
has at least two solutions in $\R$.
\end{exam}

\begin{proof} Problem.  \ns   \end{proof}

\begin{prob} Show that the equation
    \[ \frac1{\sqrt{4x^2+x+4}} - 1 = x - x^5 \]
has at least one real solution.  Locate such a solution between consecutive integers.
\end{prob}

\begin{prob}  We return to the problem we discussed at the beginning of this chapter.
Use the \emph{intermediate value theorem} to find a solution to the equation
   \[ \sin x = 1 - x \]
accurate to within $10^{-5}$. \emph{Hint.}  You may assume, for the purposes of this problem,
that the function $x \mapsto \sin x$ is continuous.  You will not want to do the computations
by hand; write a program for a computer or programmable calculator.  Notice to begin with that
there is a solution in the interval $[0, 1]$. Divide the interval in half and decide which
half, $[0,\frac12]$ or $[\frac12,1]$, contains the solution.  Then take the appropriate half
and divide \emph{it} in half. Proceed in this way until you have achieved the desired
accuracy.  Alternatively, you may find it convenient to divide each interval into tenths
rather than halves.
\end{prob}







\section{HOMEOMORPHISMS}
\begin{defn} Two subsets $A$ and $B$ of $\R$ are
 \index{homeomorphic}%
\df{homeomorphic} if there exists a continuous bijection $f$ from $A$ onto $B$ such that
$f^{-1}$ is also continuous.  In this case the function $f$ is a
 \index{homeomorphism}%
\df{homeomorphism}.
\end{defn}

Notice that if two subsets $A$ and $B$ are homeomorphic, then there is a one-to-one
correspondence between the open subsets of $A$ and those of~$B$.  In terms of topology, the
two sets are identical. Thus if we know that the open intervals $(0,1)$ and $(3,7)$ are
homeomorphic (see the next problem), then we treat these two intervals for all topological
purposes as indistinguishable.  (Of course, a concept such as distance is another matter; it
is \emph{not} a topological property.  When we consider the distance between points of our two
intervals, we can certainly distinguish between them: one is four times the length of the
other.)  A homeomorphism is sometimes called a
 \index{topological!isomorphism}%
 \index{isomorphism!topological}%
\df{topological isomorphism}.

\begin{prob} Discuss the homeomorphism classes of intervals in $\R$.  That is, tell, as generally
as you can, which intervals in $\R$ are homeomorphic to which others---and, of course, explain
why.  It might be easiest to start with some examples.  Show that
 \begin{enumerate}
  \item[(a)] the open intervals $(0,1)$ and $(3,7)$ are homeomorphic; and
  \item[(b)] the three intervals $(0,1)$, $(0,\infty)$, and $\R$ are homeomorphic.
 \end{enumerate}
Then do some counterexamples.  Show that
 \begin{enumerate}
  \item[(c)] no two of the intervals $(0,1)$, $(0,1]$, and $[0,1]$ are homeomorphic.
 \end{enumerate}
\emph{Hint.}  For (a) consider the function $x \mapsto 4x+3$.  For part of (c) suppose that
$f:(0,1] \sto (0,1)$ is a homeomorphism. What can you say about the restriction of $f$
to~$(0,1)$?

When you feel comfortable with the examples, then try to prove more general statements.  For
example, show that any two bounded open intervals are homeomorphic.

Finally, try to find the \emph{most general} possible homeomorphism classes.  (A
 \index{homeomorphism!class}
\df{homeomorphism class} is a family of intervals any two of which
are homeomorphic.)
\end{prob}

\begin{prob}  Describe the class of all continuous mappings from $\R$ into~$\Q$.
\end{prob}


\endinput
