\chapter{SPECIAL SUBSETS OF $\R$}


We denote by
 \index{realnumbers@$\R$ (the real numbers)}%
 \index{numbers!real}%
$\R$ the set of real numbers.  Certain subsets of $\R$ have standard names.  We list some of
them here for reference. The set
 \index{positive@$\Po$ (the strictly positive real numbers)}%
 \index{positive!strictly}%
$\Po = \{x \in \R\colon  x > 0\}$ of strictly positive numbers is discussed in
appendix~\ref{order_R}. The set $\{1,2,3,\dots\}$ of all
 \index{naturalnumbers@$\N$ (the natural numbers)}%
 \index{natural numbers}%
 \index{numbers!natural}
natural numbers is denoted by~$\N$, initial segments $\{1,2,3,\dots,m\}$ of this set by
$\N_m$, and the set $\{\dots,-3,-2,-1,0,1,2,3,\dots\}$ of all
 \index{z@$\Z$ (the integers)}%
 \index{integers}%
 \index{numbers!integers}%
integers by~$\Z$.  The set of all rational numbers (numbers of the form $p/q$ where $p,q \in
\Z$ and $q \ne 0$) is denoted
 \index{q@$\Q$ (the rational numbers}%
 \index{rational numbers}%
 \index{numbers!rational}%
by~$\Q$. There are the \emph{open intervals}
 \index{<@$(a,b)$ (open interval)}%
 \index{interval!open}%
 \index{open!interval}%
\begin{align*}
            (a,b) &:= \{x \in \R\colon a < x < b\}, \\
      (-\infty,b) &:= \{x \in \R\colon x < b\},\quad\text{and} \\
      (a, \infty) &:= \{x \in \R\colon x > a\}.
\end{align*}
There are the \emph{closed intervals}
 \index{<@$[a,b]$ (closed interval)}%
 \index{interval!closed}%
 \index{closed!interval}%
\begin{align*}
            [a,b] &:= \{x \in \R\colon a \le x \le b\}, \\
      (-\infty,b] &:= \{x \in \R\colon x \le b\}, \quad\text{and} \\
      [a, \infty) &:= \{x \in \R\colon x \ge a\}.
\end{align*}
And there are the intervals which (if $a < b$) are neither open nor closed:
\begin{align*}
            [a,b) &:= \{x \in \R\colon a \le x < b\} \quad\text{and} \\
            (a,b] &:= \{x \in \R\colon a < x \le b\}.
\end{align*}

The set $\R$ of all real numbers may be written in interval notation as $(-\infty, \infty)$.
(As an interval it is considered both open and closed.  The reason for applying the words
``open'' and  ``closed'' to intervals is discussed in chapter~\ref{nbhds_in_r}.)


A subset $A$ of $\R$ is
 \index{bounded!subset of $\R$}%
\df{bounded} if there is a positive number $M$ such that $\abs a \le M$ for all $a \in A$.
Thus intervals of the form $[a,b]$, $(a,b]$, $[a,b)$, and~$(a,b)$ are
 \index{bounded!interval}%
 \index{interval!bounded}%
bounded.  The other intervals are \emph{unbounded}.

If $A$ is a subset of $\R$, then
 \index{<@$A^+$ (positive elements of a set~$A$)}%
$A^+ := A \cap [0,\infty)$.  These are the
 \index{positive!elements of a set}%
\df{positive} elements of~$A$.  Notice, in particular, that $\Z^+$, the set of positive integers,
contains $0$, but $\N$, the set of natural numbers, does not.

 \endinput
