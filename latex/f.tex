\chapter{SET OPERATIONS}


\section{UNIONS}
Recall that if $S$ and $T$ are sets, then the
 \index{<@$\cup$ (union)}%
\df{union} of $S$ and $T$, denoted by $S \cup T$, is defined to be the set of all those
elements $x$ such that $x \in S$ or $x \in T$.

That is,
   \[ S \cup T  :=  \{x \colon x \in S \text{ or } x \in T\}\,. \]

\begin{exam} If $S = [0,3]$ and $T = [2,5]$, then $S \cup T = [0,5]$.
\end{exam}

The operation of taking unions of sets has several essentially obvious properties.  In the
next proposition we list some of these.

\begin{prop}\label{prop_union_sets} Let $S$, $T$, $U$, and $V$ be sets.  Then
 \begin{enumerate}
   \item[(a)] $S \cup (T \cup U) = (S \cup T) \cup U$ \qquad (associativity);
   \item[(b)] $S \cup T = T \cup S$ \qquad (commutativity);
   \item[(c)] $S \cup \emptyset = S$;
   \item[(d)] $S \subseteq S \cup T$;
   \item[(e)] $S = S \cup T$ if and only if $T \subseteq S$; and
   \item[(f)] If $S \subseteq T$ and $U \subseteq V$, then $S \cup U \subseteq T \cup V$.
 \end{enumerate}
\end{prop}

\begin{proof} We prove parts (a), (c), (d), and (e).  Ordinarily one would probably regard
these results as too obvious to require proof.  The arguments here are presented only to
display some techniques used in writing formal proofs.  Elsewhere in the text references will
not be given to this proposition when the facts (a)--(f) are used.  When results are
considered obvious, they may be mentioned but are seldom cited.  The proofs of the remaining
parts (b) and (f) are left as problems.

\begin{proof} (a)  A standard way to show that two sets are equal is to show that an element
$x$ belongs to one if and only if it belongs to the other.  In the present case
 \begin{align*}
x \in S \cup(T \cup U) &\text{ iff } x \in S \text{ or } x \in T
\cup U \\
            &\text{ iff } x \in S \text{ or } (x \in T \text{ or }
x \in U)\\
            &\text{ iff } (x \in S \text{ or } x \in T) \text{ or
} x \in U\\
            &\text{ iff } x \in S \cup T \text{ or } x \in U\\
            &\text{ iff } x \in (S \cup T) \cup U.
 \end{align*}
Notice that the proof of the associativity of union $\cup$ depends on the associativity of
``or'' as a logical connective.

Since we are asked to show that two sets are equal, some persons feel it necessary to write a
chain of equalities between sets:
 \begin{align*}
     S \cup (T \cup U)  &= \{x\colon x \in S \cup (T \cup U)\}\\
                 &= \{x\colon x \in S \text{ or } x \in T \cup U\}\\
                 &= \{x\colon x \in S \text{ or }(x \in T \text{ or } x \in U)\}\\
                 &= \{x\colon (x \in S \text{ or } x \in T)\text{ or }x \in U\}\\
                 &= \{x\colon x \in S \cup T \text{ or } x \in U\}\\
                 &= \{x\colon x \in (S \cup T) \cup U\}\\
                 &= (S \cup T) \cup U.
 \end{align*}
This second proof is virtually identical to the first; it is just a bit more cluttered.  Try
to avoid clutter; mathematics is hard enough without it. \ns\end{proof}

\begin{proof} (c) An element $x$ belongs to $S \cup \emptyset$ if and only if $x \in S$ or
$x \in \emptyset$.  Since $x \in \emptyset$ is never true, $x \in S \cup \emptyset$ if and
only if $x \in S$. That is, $S \cup \emptyset = S$. \ns\end{proof}

\begin{proof} (d)  To prove that $S \subseteq S \cup T$, show that $x \in S$ implies
$x \in S \cup T$.  Suppose $x \in S$.  Then it is certainly true that $x \in S$ or $x \in T$;
that is, $x \in S \cup T$. \ns\end{proof}

\begin{proof} (e) First show that $S = S \cup T$ implies $T \subseteq S$; then prove the converse,
if $T \subseteq S$, then $S = S \cup T$.  To prove that $S = S \cup T$ implies $T \subseteq
S$, it suffices to prove the contrapositive.  We suppose that $T \not\subseteq S$ and show
that $S \ne S \cup T$.  If $T \not\subseteq S$, then there is at least one element $t$ in $T$
which does not belong to $S$.  Thus (by parts (d) and (b))
   \[t \in T  \subseteq T \cup S  =  S \cup T;\]
but $t \notin S$. Since $t$ belongs to $S \cup T$ but not to $S$ these sets are not equal.

Now for the converse.  Suppose $T \subseteq S$.  Since we already know that $S \subseteq S
\cup T$ (by part (d)), we need only show that $S \cup T \subseteq S$ in order to prove that
the sets $S$ and $S \cup T$ are identical.  To this end suppose that $x \in S \cup T$.  Then
$x \in S$ or $x \in T \subseteq S$. In either case $x \in S$. Thus $S \cup T \subseteq S$.
\ns\end{proof}\end{proof}

\begin{prob} Prove parts (b) and (f) of proposition~\ref{prop_union_sets}.
\end{prob}

On numerous occasions it is necessary for us to take the union of a large (perhaps
infinite) family of sets.  When we consider a family of sets (that is, a set whose
members are themselves sets), it is important to keep one thing in mind.  If $x$ is a
member of a set $S$ and $S$ is in turn a member of a family $\sfml S$ of sets, it does
not follow that $x \in \sfml S$.  For example, let $S = \{0, 1, 2\}$, $T = \{2, 3, 4\}$,
$U = \{5, 6\}$, and $\sfml S = \{S, T, U\}$.  Then $1$ is a member of $S$ and $S$ belongs
to $\sfml S$; but $1$ is not a member of $\sfml S$ (because $\sfml S$ has only 3 members:
$S$, $T$, and $U$).

\begin{defn} Let $\sfml S$ be a family of sets.  We define the
 \index{union}%
\df{union} of the family $\sfml S$ to be the set of all $x$ such that $x \in S$ for at
least one set $S$ in $\sfml S$.  We denote the union of the family $\sfml S$ by  $\bigcup
\sfml S$ (or by $\bigcup_{S \in \sfml S}S$, or by $\bigcup\{S: S \in \sfml S\}$). Thus $x
\in \bigcup\sfml S$ if and only if there exists $S \in \sfml S$ such that $x \in S$.
\end{defn}

\begin{notn} If $\sfml S$ is a finite family of sets $S_1, \dots,S_n$, then we may write
$\bigcup_{k=1}^n S_k$ or $S_1 \cup S_2 \cup \dots \cup S_n$ for $\bigcup \sfml S$.
\end{notn}

\begin{exam} Let $S = \{0,1,3\}$, $T = \{1,2,3\}$, $U = \{1,3,4,5\}$, and $\sfml S = \{S,T,U\}$.  Then
   \[ \bigcup \sfml S = S \cup T \cup U = \{0,1,2,3,4,5\}\,. \]
\end{exam}

The following very simple observations are worthy of note.

\begin{prop} If $\sfml S$ is a family of sets and $T \in \sfml S$, then
$T \subseteq \bigcup \sfml S$.
\end{prop}

\begin{proof} If $x \in T$, then $x$ belongs to at least one of the sets in $\sfml S$,
namely~$T$.
\end{proof}

\begin{prop}\label{union_of_fam1} If $\sfml S$ is a family of sets and each member of
$\sfml S$ is contained in a set~$U$, then $\bigcup\sfml S \subseteq U$.
\end{prop}

\begin{proof} Problem. \ns \end{proof}






\section{INTERSECTIONS}
\begin{defn} Let $S$ and $T$ be sets.  The
 \index{<@$\cap$ (intersection)}%
\df{intersection} of $S$ and $T$ is the set of all $x$ such that $x \in S$ and $x \in T$.
\end{defn}

\begin{exam} If $S = [0,3]$ and $T = [2,5]$, then $S \cap T = [2,3]$.
\end{exam}

\begin{prop}\label{prop_intersection_sets} Let $S$, $T$, $U$, and $V$ be sets.  Then
 \begin{enumerate}
   \item[(a)] $S \cap (T \cap U) = (S \cap T) \cap U$; \qquad (associativity)
   \item[(b)] $S \cap T = T \cap S$; \qquad (commutativity)
   \item[(c)] $S \cap \emptyset = \emptyset$;
   \item[(d)] $S \cap T \subseteq S$;
   \item[(e)] $S = S \cap T$ if and only if $S \subseteq T$;
   \item[(f)] If $S \subseteq T$ and $U \subseteq V$, then $S \cap U \subseteq T \cap V$.
 \end{enumerate}
\end{prop}

\begin{proof} Problem. \ns \end{proof}


There are two distributive laws for sets: union distributes over intersection (proposition
\ref{uoveri_fin} below) and intersection distributes over union (proposition
\ref{ioveru_fin}).

\begin{prop}\label{uoveri_fin} Let $S$, $T$, and $U$ be sets.  Then
  \[ S \cup (T \cap U)  =  (S \cup T) \cap (S \cup U)\,. \]
\end{prop}

\begin{proof} Exercise. \emph{Hint.} Use problem~\ref{prob_disj_conj}.
(Solution~\ref{sol_uoveri_fin}.) \ns
\end{proof}

\begin{prop}\label{ioveru_fin}  Let $S$, $T$, and $U$ be sets. Then
   \[ S \cap (T \cup U)  =  (S \cap T) \cup (S \cap U)\,. \]
\end{prop}

\begin{proof} Problem. \ns \end{proof}

Just as we may take the union of an arbitrary family of sets, we may also take its
intersection.

\begin{defn}  Let $\sfml S$ be a family of sets.  We define the
 \index{intersection}%
\df{intersection} of the family $\sfml S$ to be the set of all $x$ such that $x \in \sfml
S$ for every $S$ in $\sfml S$.  We denote the intersection of $\sfml S$ by $\bigcap\sfml
S$ (or by $\bigcap_{S \in \sfml{S}}S$, or by $\bigcap\{S: S \in \sfml S\}$).
\end{defn}

\begin{notn} If $S$ is a finite family of sets $S_1,\dots,S_n$, then we may write
$\bigcap_{k=1}^n S_k$ or $S_1 \cap S_2 \cap \dots \cap S_n$ for $\bigcap\sfml S$.
Similarly, if $\sfml S = \{S_1,S_2,\dots\}$, then we may write
$\smash[t]{\bigcap_{k=1}^\infty S_k}$ or $S_1 \cap S_2 \cap \dots$ for $\bigcap\sfml S$.
\end{notn}


\begin{exam} Let $S = \{0,1,3\}$, $T = \{1,2,3\}$, $U = \{1,3,4,5\}$, and $S = \{S,T,U\}$.
Then
   \[ \bigcap S = S \cap T \cap U = \{1,3\}\,. \]
\end{exam}

Proposition \ref{uoveri_fin} may be generalized to say that union distributes over the
intersection of an arbitrary family of sets. Similarly there is a more general form of
proposition \ref{ioveru_fin} which says that intersection distributes over the union of an
arbitrary family of sets.  These two facts, which are stated precisely in the next two
propositions, are known as
 \index{generalized distributive law}%
 \index{distributive law!generalized}%
\df{generalized distributive laws}.

\begin{prop}\label{uoveri_inf} Let $T$ be a set and $\sfml S$ be a family of sets.  Then
   \[ T \cup \bigl(\,\bigcap\sfml S\,\bigr) = \bigcap\{T \cup S \colon  S \in \sfml S\}\,. \]
\end{prop}

\begin{proof} Exercise.  (Solution~\ref{sol_uoveri_inf}.) \ns \end{proof}

\begin{prop}\label{ioveru_inf}  Let $T$ be a set and $\sfml S$ be a family of sets.  Then
   \[ T \cap \bigl(\,\bigcup \sfml S\,\bigr) = \bigcup\{T \cap S: S \in \sfml S\}\,. \]
\end{prop}

\begin{proof} Problem. \ns \end{proof}

\begin{defn} Sets $S$ and $T$ are said to be
 \index{disjoint}%
\df{disjoint} if $S \cap T = \emptyset$.  More generally, a family $\sfml S$ of sets is a
 \index{disjoint!family}%
\df{disjoint family} (or a
 \index{pairwise disjoint}%
 \index{disjoint!pairwise}%
\df{pairwise disjoint family}) if $S \cap T = \emptyset$ whenever $S$ and $T$ are
distinct (that is, not equal) sets which belong to~$\sfml S$.
\end{defn}

\begin{cau} Let $\sfml S$ be a family of sets.  Do not confuse the following two statements.
 \begin{enumerate}
   \item[(a)] $\sfml S$ is a (pairwise) disjoint family.
   \item[(b)] $\bigcap \sfml S = \emptyset$.
 \end{enumerate}
Certainly, if $\sfml S$ contains more than a single set, then (a) implies (b).  But if
$\sfml S$ contains three or more sets the converse need not hold.  For example, let $S =
\{0,1\}$, $T = \{3,4\}$, $U = \{0,2\}$, and $\sfml S = \{S,T,U\}$.  Then $\sfml S$ is not
a disjoint family (because $S \cap U$ is nonempty), but $\bigcap\sfml S = \emptyset$.
\end{cau}

\begin{exam} Let $S$, $T$, $U$, and $V$ be sets.
 \begin{enumerate}
   \item[(a)] Then $(S \cap T) \cup (U \cap V) \subseteq (S \cup U)
\cap (T \cup V)$.
   \item[(b)] Give an example to show that equality need not hold
in~(a).
 \end{enumerate}
\end{exam}

\begin{proof} Problem.  \emph{Hint.}  Use propositions \ref{prop_union_sets}(d)  and
\ref{prop_intersection_sets}(f) to show that $S \cap T$ and $U \cap V$ are contained in $(S
\cup U) \cap (T \cup V)$.  Then use \ref{prop_union_sets}(f).   \ns
\end{proof}







\section{COMPLEMENTS} Recall that we regard all the sets with which we work in a particular
situation as being subsets of some appropriate ``universal'' set.  For each set  $S$  we
define the
 \index{<@$S^c$ (complement of a set $S$)}%
\df{complement} of  $S$ , denoted by  $S^c$, to be the set of all members of our universal set
which do not belong to $S$.  That is, we write $x \in S^c$ if and only if $x \notin S$.

\begin{exam} Let $S$ be the closed interval  $(-\infty, 3]$.  If nothing else is specified,
we think of this interval as being a subset of the real  line $\R$ (our universal set).  Thus
$S^c$ is the set of all  $x $ in $\R$ such that $x$ is not less than or equal to~$3$.  Thus
$S^c$ is the interval $(3,\infty)$.
\end{exam}

\begin{exam} Let $S$ be the set of all points $(x,y)$ in the plane such that $x \ge 0$ and
$y \ge 0$.  Then $S^c$ is the set of all points $(x,y)$  in the plane such that either $x < 0$
or $y < 0$. That is ,
   \[ S^c  =  \{(x,y)\colon x < 0\} \cup \{(x,y)\colon y < 0\}\,. \]
\end{exam}

The two following propositions are
 \index{De Morgan's laws!for sets}%
\df{De Morgan's laws} for sets.  As you may expect, they are obtained by translating into the
language of sets the facts of logic which go under the same name. (See \ref{exam_demorg1} and
\ref{prob_demorg2}.)

\begin{prop}\label{prop_compl_union} Let $S$ and $T$ be sets.  Then
    \[ (S \cup T)^c = S^c \cap T^c\,. \]
\end{prop}

\begin{proof} Exercise.  \emph{Hint.}  Use example~\ref{exam_demorg1}.
(Solution~\ref{sol_prop_compl_union}.)   \ns
\end{proof}

\begin{prop} Let $S$ and $T$ be sets.  Then
   \[ (S \cap T)^c = S^c \cup T^c\,. \]
\end{prop}

\begin{proof} Problem. \ns \end{proof}

Just as the distributive laws can be generalized to arbitrary families of sets, so too can De
Morgan's laws.  The complement of the union of a family is the intersection of the complements
(proposition \ref{comp_of_union}), and the complement of the intersection of a family is the
union of the complements (proposition \ref{comp_of_intersection}).

\begin{prop}\label{comp_of_union} Let $\sfml S$ be a family of sets. Then
   \[ (\bigcup \sfml S)^c = \bigcap\{S^c\colon S \in \sfml S\}\,. \]
\end{prop}

\begin{proof}  Exercise. (Solution~\ref{sol_comp_of_union}.) \ns \end{proof}

\begin{prop}\label{comp_of_intersection} Let $\sfml S$ be a family of sets.  Then
   \[ (\bigcap \sfml S)^c  = \bigcup\{S^c\colon S \in \sfml S\}\,. \]
\end{prop}

\begin{proof} Problem. \ns \end{proof}

\begin{defn}If $S$ and $T$ are sets we define the
 \index{complement!relative}%
 \index{relative!complement}%
\df{complement of} $T$ \df{relative to} $S$, denoted by $S \setminus T$, to be the set of all
$x$ which belong to $S$ but not to~$T$.  That is,
   \[ S \setminus T :=  S \cap T^c\,. \]
The operation ${}\setminus{}$ is usually called
 \index{set!subtraction}%
\df{set subtraction} and $S \setminus T$ is read as ``$S$ minus $T$''.
\end{defn}

\begin{exam} Let $S = [0,5]$ and $T = [3,10]$.  Then $S \setminus T = [0,3)$.
\end{exam}

It is a frequently useful fact that the union of two sets can be rewritten as a disjoint union
(that is, the union of two disjoint sets).

\begin{prop}\label{prop_decomp_union}  Let $S$ and $T$ be sets. Then $S \setminus T$ and $T$ are
disjoint sets whose union is $S \cup T$.
\end{prop}

\begin{proof} Exercise. (Solution~\ref{sol_prop_decomp_union}.) \ns \end{proof}

\begin{exer}\label{exer_diff_union} Show that $(S \setminus T) \cup T = S$  if and only if
$T \subseteq S$. (Solution~\ref{sol_exer_diff_union}.)
\end{exer}

\begin{prob} Let $S = (3,\infty)$, $T = (0,10]$, $U = (-4,5)$, $V = [-2,8]$, and $\sfml S =
\{S^c, T, U, V\}$.
 \begin{enumerate}
   \item[(a)]  Find $\bigcup \sfml S$.
   \item[(b)]  Find $\bigcap \sfml S$.
 \end{enumerate}
\end{prob}

\begin{prob} If $S$, $T$, and $U$ are sets, then
   \[ (S \cap T) \setminus U  = (S \setminus U) \cap (T \setminus U)\,. \]
\end{prob}

\begin{prob} If $S$, $T$, and $U$ are sets, then
  \[ S \cap (T \setminus U)  = (S  \cap T)  \setminus (S \cap U)\,. \]
\end{prob}

\begin{prob}\label{prob_sdiff1} If $S$ and $T$ are sets, then $T \setminus S$ and $T \cap S$
are disjoint and
   \[ T = (T \setminus S) \cup (T \cap S)\,. \]
\end{prob}

\begin{prob} If $S$ and $T$ are sets, then $S \cap T  =  S \setminus (S \setminus T)$.
\end{prob}

\begin{defn} A family $\sfml S$ of sets
 \index{covers}%
\df{covers} (or is a
 \index{cover}%
\df{cover for}, or is a
 \index{covering}%
\df{covering for}) a set $T$ if $T \subseteq \bigcup S$.
\end{defn}

\begin{prob}  Find a family of open intervals which covers the set $\N$ of natural numbers and has
the property that the sum of the lengths of the intervals is~$1$.  \emph{Hint.}
$\sum_{k=1}^\infty 2^{-k} = 1$.
\end{prob}


\endinput
