\chapter*{PREFACE}

In American universities two distinct types of courses are often called ``Advanced
Calculus'': one, largely for engineers, emphasizes advanced computational techniques in
calculus; the other, a more ``theoretical'' course, usually taken by majors in
mathematics and physical sciences (and often called \emph{elementary analysis} or
\emph{intermediate analysis}), concentrates on conceptual development and proofs. This
\emph{ProblemText} is a book of the latter type. It is not a place to look for
post-calculus material on Fourier series, Laplace transforms, and the like.  It is
intended for students of mathematics and others who have completed (or nearly completed)
a standard introductory calculus sequence and who wish to understand where all those
rules and formulas come from.

Many advanced calculus texts contain more topics than this \emph{ProblemText}.  When
students are encouraged to develop much of the subject matter for themselves, it is not
possible to ``cover'' material at the same breathtaking pace that can be achieved by a
truly determined lecturer. But, while no attempt has been made to make the book
encyclopedic, I do think it nevertheless provides an integrated overview of Calculus and,
for those who continue, a solid foundation for a first year graduate course in Real
Analysis.


As the title of the present document, \emph{ProblemText in Advanced Calculus}, is
intended to suggest, it is as much an extended problem set as a textbook.  The proofs of
most of the major results are either \emph{exercises} or \emph{problems}.  The
distinction here is that solutions to \emph{exercises} are written out in a separate
chapter in the ProblemText while solutions to \emph{problems} are not given.  I hope that
this arrangement will provide flexibility for instructors who wish to use it as a text.
For those who prefer a (modified) Moore-style development, where students work out and
present most of the material, there is a quite large collection of problems for them to
hone their skills on.  For instructors who prefer a lecture format, it should be easy to
base a coherent series of lectures on the presentation of solutions to thoughtfully
chosen problems.

I have tried to make the \emph{ProblemText} (in a rather highly qualified sense discussed
below) ``self-contained''. In it we investigate how the edifice of calculus can be
grounded in a carefully developed substrata of sets, logic, and numbers.  Will it be a
``complete'' or ``totally rigorous'' development of the subject? Absolutely not.  I am
not aware of any serious enthusiasm among mathematicians I know for requiring rigorous
courses in \emph{Mathematical Logic} and \emph{Axiomatic Set Theory} as prerequisites for
a first introduction to analysis.  In the use of the tools from set theory and formal
logic there are many topics that because of their complexity and depth are cheated, or
not even mentioned. (For example, though used often, the \emph{axiom of choice} is
mentioned only once.) Even everyday topics such as ``arithmetic,'' see
appendix~\ref{arithmetic}, are not developed in any great detail.

Before embarking on the main ideas of Calculus proper one \emph{ideally} should have a
good background in all sorts of things: quantifiers, logical connectives, set operations,
writing proofs, the arithmetic and order properties of the real numbers, mathematical
induction, least upper bounds, functions, composition of functions, images and inverse
images of sets under functions, finite and infinite sets, countable and uncountable sets.
On the one hand all these are technically prerequisite to a careful discussion of the
foundations of calculus. On the other hand any attempt to do all this business
systematically at the beginning of a course will defer the discussion of anything
concerning calculus proper to the middle of the academic year and may very well both bore
and discourage students.  Furthermore, in many schools there may be students who have
already studied much of this material (in a ``proofs'' course, for example). In a spirit
of compromise and flexibility I have relegated this material to appendices. Treat it any
way you like.  I teach in a large university where students show up for Advanced Calculus
with a wide variety of backgrounds, so it is my practice to go over the appendices first,
covering many of them in a quite rapid and casual way, my goal being to provide just
enough detail so that everyone will know where to find the relevant material when they
need it later in the course.  After a rapid traversal of the appendices I start
Chapter~$1$.

For this text to be useful a student should have previously studied introductory
calculus, more for mathematical maturity than anything else.  Familiarity with properties
of elementary functions and techniques of differentiation and integration may be assumed
and made use of in \emph{examples}---but is never relied upon in the logical development
of the material.

One motive for my writing this text is to make available in fairly simple form material
that I think of as ``calculus done right.''  For example, differential calculus as it
appears in many texts is a morass of tedious epsilon-delta arguments and partial
derivatives, the net effect of which is to almost totally obscure the beauty and elegance
which results from a careful and patient elaboration of the concept of tangency. On the
other hand texts in which things are done right (for example Loomis and
Sternberg~\cite{LoomisS:1990}) tend to be rather forbidding.  I have tried to write a
text which will be helpful to a determined student with an average background.  (I
seriously doubt that it will be of much use to the chronically lazy or totally
unengaged.)

In my mind one aspect of doing calculus ``correctly'' is arrange things so that there is
nothing to unlearn later.  For example, in this text topological properties of the real
line are discussed early on.  Later, topological things (continuity, compactness,
connectedness, and so on) are discussed in the context of metric spaces (because they
unify the material conceptually and greatly simplify subsequent arguments).  But the
important thing is the \emph{definitions} in the single variable case and the metric
space case are the same. Students do not have to ``unlearn'' material as they go to more
general settings.  Similarly, the differential calculus is eventually developed in its
natural habitat of normed linear spaces.  But here again, the student who has mastered
the one-dimensional case, which occurs earlier in the text, will encounter definitions
and theorems and proofs that are virtually identical to the ones with which (s)he is
already familiar.  There is nothing to unlearn.

In the process of writing this \emph{ProblemText} I have rethought the proofs of many
standard theorems. Although some, perhaps most, results in advanced calculus have reached
a final, optimal form, there are many others that, despite dozens of different approaches
over the years, have proofs that are genuinely confusing to most students.  To mention
just one example there is the theorem concerning change of order of differentiation whose
conclusion is
  \[ \frac{\partial^2f}{\partial x \, \partial y} =
               \frac{\partial^2f}{\partial y \, \partial x}\,. \]
This well-known result to this day still receives clumsy, impenetrable, and even
incorrect proofs. I make an attempt, here and elsewhere in the text, to lead students to
proofs that are conceptually clear, even when they may not be the shortest or most
elegant.  And throughout I try very hard to show that mathematics is about ideas and not
about manipulation of symbols.

There are of course a number of advantages and disadvantages in consigning a document to
electronic life. One slight advantage is the rapidity with which links implement
cross-references.  Hunting about in a book for \emph{lemma 3.14.23} can be time-consuming
(especially when an author engages in the entirely logical but utterly infuriating
practice of numbering lemmas, propositions, theorems, corollaries, \emph{etc.}
separately).  A perhaps more substantial advantage is the ability to correct errors, add
missing bits, clarify opaque arguments, and remedy infelicities of style in a timely
fashion.  The correlative disadvantage is that a reader returning to the web page after a
short time may find everything (pages, definitions, theorems, sections) numbered
differently.  (\LaTeX is an amazing tool.) I will change the date on the title page to
inform the reader of the date of the last nontrivial update (that is, one that affects
numbers or cross-references).

The most serious disadvantage of electronic life is impermanence.  In most cases when a
web page vanishes so, for all practical purposes, does the information it contains.  For
this reason (and the fact that I want this material to be freely available to anyone who
wants it) I am making use of a ``Share Alike'' license from Creative Commons. It is my
hope that anyone who finds this text useful will correct what is wrong, add what is
missing, and improve what is clumsy.  To make this possible I am also including the
\LaTeX source code on my web page. For more information on creative commons licenses see
   \[ \text{http://creativecommons.org/} \]

I extend my gratitude to the many students over the years who have endured various
versions of this ProblemText and to my colleagues who have generously nudged me when they
found me napping.  I want especially to thank Dan Streeter who provided me with so much
help in the technical aspects of getting this document produced.  The text was prepared
using \AmS-\LaTeX. For the diagrams I used the macro package \Xy-pic by Kristoffer H.
Rose and Ross Moore supplemented by additional macros in the diagxy package by Michael
Barr.

Finally it remains only to say that I will be delighted to receive, and will consider,
any comments, suggestions, or error reports. My e-mail address is
  \[  \text{erdman@pdx.edu} \]
