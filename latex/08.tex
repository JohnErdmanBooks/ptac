\chapter{DIFFERENTIATION OF REAL VALUED FUNCTIONS}\label{dorf}

Differential calculus is a highly geometric subject---a fact which is not always made entirely
clear in elementary texts, where the study of derivatives as numbers often usurps the place of
the fundamental notion of linear approximation. The contemporary French mathematician Jean
Dieudonn\'e comments on the problem in chapter~8 of his magisterial multivolume treatise on
the \emph{Foundations of Modern Analysis}\cite{Dieudonne:1962}
 \begin{quote}
\ldots\ the fundamental idea of calculus [is] the ``local'' approximation of functions by
\emph{linear} functions. In the classical teaching of Calculus, this idea is immediately
obscured by the accidental fact that, on a one-dimensional vector space, there is a one-to-one
correspondence between linear forms and numbers, and therefore the derivative at a point is
defined as a \emph{number} instead of a \emph{linear form}.  This slavish subservience to the
shibboleth of numerical interpretation at any cost becomes much worse when dealing with
functions of several variables \ldots
 \end{quote}
The goal of this chapter is to display as vividly as possible the geometric underpinnings of
the differential calculus. The emphasis is on ``tangency'' and ``linear approximation'', not
on number.






\section{THE FAMILIES $\lobo O$ AND $\lobo o$}
\begin{notn}  Let $a \in \R$. We denote
 \index{f@$\fml F_a$ (functions defined in a neighborhood of~$a$)}%
by~$\fml F_a$ the family of all real valued functions defined on a neighborhood of~$a$.  That
is, $f$ belongs to $\fml F_a$ if there exists an open set $U$ such that $a \in U \subseteq
\dom f$.
\end{notn}

Notice that for each $a \in \R$, the set $\fml F_a$ is closed under addition and
multiplication. (We define the sum of two functions $f$ and $g$ in $\fml F_a$ to be the
function $f + g$ whose value at $x$ is $f(x) + g(x)$ whenever $x$ belongs to $\dom f \,\cap\,
\dom g$.  A similar convention holds for multiplication.)

Among the functions defined on a neighborhood of zero are two subfamilies of crucial
importance; they are $\lobo O$ (the family of ``\emph{big-oh}'' functions) and $\lobo o$ (the
family of ``\emph{little-oh}'' functions).

\begin{defn} A function $f$ in $\fml F_0$ belongs to
 \index{ohbig@$\lobo O$ functions}%
$\lobo O$ if there exist numbers $c > 0$ and $\delta > 0$ such that
   \[ \abs{f(x)} \le c\,\abs{x} \]
whenever $\abs{x} < \delta$.

A function $f$ in $\fml F_0$ belongs to
 \index{ohlittle@$\lobo o$ functions}%
$\lobo o$ if for every $c > 0$ there exists $\delta > 0$ such that
   \[ \abs{f(x)} \le c\,\abs{x} \]
whenever $\abs{x} < \delta$. Notice that $f$ belongs to $\lobo o$ if and only if $f(0) = 0$
and
   \[ \lim_{h \sto 0}\frac{\abs{f(h)}}{\abs{h}} = 0\,. \]
\end{defn}

\begin{exam}  Let $f(x) = \sqrt{\abs{x}}$. Then $f$ belongs to neither $\lobo O$ nor $\lobo o$.
(A function belongs to $\lobo O$ only if in some neighborhood of the origin its graph lies
between two lines of the form $y = cx$ and $y = -cx$.)
\end{exam}

\begin{exam}  Let $g(x) = \abs{x}$. Then $g$ belongs to $\lobo O$ but not to $\lobo o$.
\end{exam}

\begin{exam}  Let $h(x) = x^2$. Then $h$ is a member of both $\lobo O$ and $\lobo o$.
\end{exam}

Much of the elementary theory of differential calculus rests on a few simple properties of the
families $\lobo O$ and $\lobo o$. These are given in propositions \ref{o_in_O}--\ref{mult_o}.

\begin{defn} A function $L \colon \R \sto \R$ is
 \index{linear}%
\df{linear} if
   \[ L(x + y) = L(x) + L(y) \]
and
   \[ L(cx) = cL(x) \]
for all $x$, $y$, $c \in \R$.  The family of all linear functions from $\R$ into $\R$ will be
denoted
 \index{linear@$\ofml L$ (linear functions from $\R$ into $\R$)}%
by~$\ofml L$.
\end{defn}

The collection of linear functions from $\R$ into $\R$ is not very impressive, as the next
problem shows.  When we get to spaces of higher dimension the situation will become more
interesting.

\begin{exam}\label{lin_R} A function $f \colon \R \sto \R$ is linear if and only if its graph is a
(nonvertical) line through the origin.
\end{exam}

\begin{proof} Problem. \ns \end{proof}

\begin{cau} Since linear functions must pass through the origin, straight lines are not in
general graphs of linear functions.
\end{cau}

\begin{prop}\label{o_in_O} Every member of $\lobo o$ belongs to $\lobo O$; so does every member
of $\ofml L$. Every member of $\lobo O$ is continuous at 0.
\end{prop}

\begin{proof} Obvious from the definitions.
\end{proof}

\begin{prop}\label{lin_vs_o} Other than the constant function zero, no linear function belongs to~$\lobo o$.
\end{prop}

\begin{proof} Exercise.  (Solution~\ref{sol_lin_vs_o}.)   \ns \end{proof}

\begin{prop}\label{O_closed} The family $\lobo O$ is closed under addition and multiplication
by constants.
\end{prop}

\begin{proof} Exercise.   (Solution~\ref{sol_O_closed}.)  \ns \end{proof}

\begin{prop}\label{o_closed} The family $\lobo o$ is closed under addition and multiplication
by constants.
\end{prop}

\begin{proof} Problem. \ns \end{proof}

The next two propositions say that the composite of a function in $\lobo O$ with one in $\lobo
o$ (in either order) ends up in $\lobo o$.

\begin{prop} If $g \in \lobo O$ and $f \in \lobo o$, then $f \circ g \in \lobo o$.
\end{prop}

\begin{proof} Problem. \ns \end{proof}

\begin{prop}\label{O_comp_o} If $g \in \lobo o$ and $f \in \lobo O$, then $f \circ g \in \lobo o$.
\end{prop}

\begin{proof} Exercise.   (Solution~\ref{sol_O_comp_o}.)   \ns \end{proof}

\begin{prop}\label{mult_o} If $\phi,f \in \lobo O$, then $\phi f \in \lobo o$.
\end{prop}

\begin{proof} Exercise.  (Solution~\ref{sol_mult_o}.)   \ns \end{proof}

\begin{rem}\label{rem_oO_R} The preceding facts can be summarized rather concisely. (Notation:
 \index{continuous@$\fml C_0$ (functions continuous at~$0$)}%
$\fml C_0$ is the set of all functions in $\fml F_0$ which are continuous at 0.)
 \begin{align*}
   &(1)\qquad \ofml L \cup \lobo o \subseteq \lobo O \subseteq
                   \fml C_0 \, .\\
   &(2)\qquad \ofml L \cap \lobo o = {0}\,. \\
   &(3)\qquad \lobo O + \lobo O \subseteq \lobo O\,; \qquad
                    \alpha\,\lobo O \subseteq \lobo O\,. \\
   &(4)\qquad \lobo o + \lobo o \subseteq \lobo o\,; \qquad
                    \alpha\,\lobo o \subseteq \lobo o\,. \\
   &(5)\qquad \lobo o \circ \lobo O \subseteq \lobo o\,. \\
   &(6)\qquad \lobo O \circ \lobo o \subseteq \lobo o\,. \\
   &(7)\qquad \lobo O \cdot \lobo O \subseteq \lobo o\,.
 \end{align*}
\end{rem}

\begin{prob}\label{O_comp_O} Show that $\lobo O\circ\lobo O\subseteq\lobo O$.  That is, if
$g \in \lobo O$ and $f \in \lobo O$, then $f\circ g \in \lobo O$. (As usual, the domain of $f
\circ g$ is taken to be $\{x \colon g(x) \in \dom f\}$.)
\end{prob}






\section{TANGENCY}
The fundamental idea of differential calculus is the local approximation of a ``smooth''
function by a translate of a linear one. Certainly the expression ``local approximation''
could be taken to mean many different things. One sense of this expression which has stood the
test of usefulness over time is ``tangency''. Two functions are said to be tangent at zero if
their difference lies in the family $\lobo o$. We can of course define tangency of functions
at an arbitrary  point (see project \ref{tan_at_a} below); but for our purposes, ``tangency at
0'' will suffice. All the facts we need to know concerning this relation turn out to be
trivial consequences of the results we have just proved.

\begin{defn} Two functions $f$ and $g$ in $\fml F_0$ are
 \index{tangent!at $0$}%
\df{tangent at zero}, in which case we write
 \index{<@$\simeq$ (tangency)}%
$f \simeq g$, if $f - g \in \lobo o$.
\end{defn}

\begin{exam} Let $f(x) = x$ and $g(x) = \sin x$. Then $f \simeq g$ since $f(0) = g(0) = 0$ and
$\lim\limits_{x \sto 0} \dfrac{x - \sin x}x = \lim\limits_{x \sto 0}\biggl(1 - \dfrac{\sin
x}x\biggr) = 0$.
\end{exam}

\begin{exam}\label{8exer1} If $f(x) = x^2 - 4x - 1$ and $g(x) = \bigl(3x^2 + 4x - 1\bigr)^{-1}$,
then~$f \simeq g$.
\end{exam}

\begin{proof} Exercise.   (Solution~\ref{sol_8exer1}.)
  \ns \end{proof}

\begin{prop}\label{tan_er} The relation ``tangency at zero'' is an equivalence relation
on~$\fml F_0$.
\end{prop}

\begin{proof} Exercise.  (Solution~\ref{sol_tan_er}.)   \ns \end{proof}

The next result shows that at most one linear function can be tangent at zero to a given
function.

\begin{prop}\label{tan_uniq} Let $S$, $T \in \ofml L$ and $f \in \fml F_0$. If $S \simeq f$
and $T \simeq f$, then $S = T$.
\end{prop}

\begin{proof} Exercise.  (Solution~\ref{sol_tan_uniq}.)   \ns \end{proof}

\begin{prop}\label{tan_alg} If $f \simeq g$ and $j \simeq k$, then $f + j \simeq g + k$,
and furthermore, $\alpha f \simeq \alpha g$ for all $\alpha \in \R$.
\end{prop}

\begin{proof} Problem. \ns \end{proof}

Suppose that $f$ and $g$ are tangent at zero. Under what circumstances are $h \circ f$ and $h
\circ g$ tangent at zero? And when are $f \circ j$ and $g \circ j$ tangent at zero? We prove
next that sufficient conditions are: $h$ is linear and $j$ belongs to $\lobo O$.

\begin{prop}\label{L_tan} Let $f,g \in \fml F_0$ and $T \in \ofml L$. If $f \simeq g$, then
$T \circ f \simeq T \circ g$.
\end{prop}

\begin{proof} Problem. \ns \end{proof}

\begin{prop}\label{tan_O} Let $h \in \lobo O$ and $f,g \in \fml F_0$. If $f \simeq g$, then
$f \circ h \simeq g \circ h$.
\end{prop}

\begin{proof} Problem. \ns \end{proof}

\begin{exam} Let $f(x) = 3x^2 - 2x + 3$ and $g(x) = \sqrt{-20x + 25} - 2$ for $x \le 1$.
Then $f \simeq g$.
\end{exam}

\begin{proof} Problem.  \ns \end{proof}

\begin{prob} Let $f(x) = x^3 - 6x^2 + 7x$. Find a linear function $T \colon \R \sto \R$
which is tangent to $f$ at~0.
\end{prob}

\begin{prob} Let $f(x) = \abs x$. Show that there is no linear function $T \colon \R \sto \R$
which is tangent to $f$ at~$0$.
\end{prob}

\begin{prob}\label{tan_at_a} Let $T_a \colon x \mapsto x + a$. The mapping $T_a$ is called
 \index{translation}%
\df{translation by}~$a$. Note that it is \emph{not} linear (unless, of course, $a = 0$). We
say that functions $f$ and $g$ in $\fml F_a$ are
 \index{tangent!at points other than $0$}%
\df{tangent at} $a$ if the functions $f \circ T_a$ and $g \circ T_a$ are tangent at zero.
 \begin{enumerate}
  \item[(a)] Let $f(x) = 3x^2 + 10x + 13$ and $g(x) = \sqrt{-20x -15}$. Show that $f$ and
$g$ are tangent at~$-2$.
  \item[(b)] Develop a theory  for the relationship ``tangency at $a$'' which generalizes
our work on ``tangency at~$0$''.
 \end{enumerate}
\end{prob}


\begin{prob} Each of the following is an abbreviated version of a proposition.  Formulate
precisely and prove.
 \begin{enumerate}
  \item[(a)] $\fml C_0 + \lobo O \subseteq \fml C_0$.
  \item[(b)] $\fml C_0 + \lobo o \subseteq \fml C_0$.
  \item[(c)] $\lobo O + \lobo o \subseteq \lobo O$.
 \end{enumerate}
\end{prob}

\begin{prob}  Suppose that $f \simeq g$. Then the following hold.
 \begin{enumerate}
  \item[(a)] If $g$ is continuous at 0, so is $f$.
  \item[(b)] If $g$ belongs to $\lobo O$, so does $f$.
  \item[(c)] If $g$ belongs to $\lobo o$, so does $f$.
 \end{enumerate}
\end{prob}






\section{LINEAR APPROXIMATION}
One often hears that differentiation of a smooth function $f$ at a point $a$ in its domain is
the process of finding the best ``linear approximation'' to $f$ at~$a$. This informal
assertion is not quite correct. For example, as we know from beginning calculus, the tangent
line at $x=1$ to the curve $y = 4 + x^2$ is the line $y = 2x + 3$, which is not a linear
function since it does not pass through the origin. To rectify this rather minor shortcoming
we first translate the graph of the function $f$ so that the point $(a,f(a))$ goes to the
origin, and \emph{then} find the best linear approximation at the origin. The operation of
translation is carried out by a somewhat notorious acquaintance from beginning calculus
$\Delta y$. The source of its notoriety is two-fold: first, in many texts it is inadequately
defined; and second, the notation $\Delta y$ fails to alert the reader to the fact that under
consideration is a function of \emph{two} variables. We will be careful on both counts.

\begin{defn}\label{Delta} Let $f \in \fml F_a$. Define the function
 \index{<@$\Delta F_a$}%
 \index{delta@$\Delta F_a$}%
$\Delta f_a$ by
  \[ \Delta f_a(h) := f(a + h) - f(a) \]
for all $h$ such that $a + h$ is in the domain of $f$. Notice that since $f$ is defined in a
neighborhood of $a$, the function $\Delta f_a$ is defined in a neighborhood of 0; that is,
$\Delta f_a$ belongs to $\fml F_0$. Notice also that $\Delta f_a(0) = 0$.
\end{defn}

\begin{prob} Let $f(x) = \cos x$ for $0 \le x \le 2\pi$.
 \begin{enumerate}
  \item[(a)] Sketch the graph of the function $f$.
  \item[(b)] Sketch the graph of the function $\Delta f_{\pi}$.
 \end{enumerate}
\end{prob}

\begin{prop}\label{del_sm} If $f \in \fml F_a$ and $\alpha \in \R$, then
   \[ \Delta(\alpha f)_a = \alpha\,\Delta f_a\,. \]
\end{prop}

\begin{proof} To show that two functions are equal show that they agree at each point in their
domain.  Here
 \begin{align*}
     \Delta(\alpha f)_a(h) &= (\alpha f)(a + h) - (\alpha f)(a) \\
                           &= \alpha f(a + h) - \alpha f(a) \\
                           &= \alpha(f(a + h) - f(a)) \\
                           &= \alpha\,\Delta f_a(h)
 \end{align*}
for every $h$ in the domain of~$\Delta f_a$.
\end{proof}

\begin{prop}\label{del_sum} If $f,g \in \fml F_a$, then
   \[ \Delta(f+g)_a = \Delta f_a + \Delta g_a\,. \]
\end{prop}

\begin{proof} Exercise.   (Solution~\ref{sol_del_sum}.)  \ns \end{proof}

The last two propositions prefigure the fact that differentiation is a linear operator; the
next result will lead to \emph{Leibniz's rule} for differentiating products.

\begin{prop}\label{del_mult} If $\phi, f \in \fml F_a$, then
  \[\Delta(\phi f)_a = \phi(a)\cdot\Delta f_a \,+\, \Delta\phi_a\cdot f(a)
                        \,+\, \Delta\phi_a\cdot\Delta f_a\,. \]
\end{prop}

\begin{proof} Problem. \ns \end{proof}

Finally, we present a result which prepares the way for the \emph{chain rule}.

\begin{prop}\label{del_comp} If $f \in \fml F_a$, $g \in \fml F_{f(a)}$, and
$g \circ f \in \fml F_a$, then
   \[ \Delta(g \circ f)_a = \Delta g_{{}_{\scriptstyle{f(a)}}} \circ \Delta f_a\,. \]
\end{prop}

\begin{proof} Exercise.  (Solution~\ref{sol_del_comp}.)   \ns \end{proof}

\begin{prop} Let $A \subseteq \R$.  A function $f \colon A \sto \R$ is continuous at the
point $a$ in $A$ if and only if $\Delta f_a$ is continuous at~$0$.
\end{prop}

\begin{proof} Problem. \ns  \end{proof}

\begin{prop} If $f \colon U \sto U_1$ is a bijection between subsets of $\R$, then for each
$a$ in $U$ the function $\Delta f_a \colon  U - a \sto U_1 - f(a)$ is invertible and
   \[ \bigl(\Delta f_a\bigr)^{-1}=\Delta\bigl(f^{-1}\bigr)_{f(a)}\,. \]
\end{prop}

\begin{proof} Problem.  \ns  \end{proof}






\section{DIFFERENTIABILITY}
We now have developed enough machinery to talk sensibly about \emph{differentiating} real
valued functions.

\begin{defn}\label{df_diff} Let $f \in \fml F_a$.  We say that $f$ is
 \index{differentiable!at a point}%
\df{differentiable at}~$a$ if there exists a linear function which is tangent at $0$
to~$\Delta f_a$.  If such a function exists, it is called the
 \index{differential}%
\df{differential} of $f$ at $a$ and is denoted
 \index{dfa@$df_a$ (differential of $f$ at~$a$)}%
by~$df_a$. (Don't be put off by the slightly complicated notation; $df_a$ is just a
member of $\ofml L$ satisfying $df_a \simeq \Delta f_a$.)  We denote by $\fml D_a$ the
family of all functions in $\fml F_a$ which are differentiable at~$a$.
\end{defn}

The next proposition justifies the use of the definite article which modifies ``differential''
in the preceding paragraph.

\begin{prop}\label{diff_uniq} Let $f \in \fml F_a$. If $f$ is differentiable at $a$, then its
differential is unique. (That is, there is at most one linear map tangent at $0$ to $\Delta
f_a$.)
\end{prop}

\begin{proof} Proposition \ref{tan_uniq}.  \end{proof}

\begin{exam}\label{deriv} It is instructive to examine the relationship between the differential
of $f$ at $a$, which we defined in \ref{df_diff}, and the derivative of $f$ at $a$ as defined
in beginning calculus.  For $f \in \fml F_a$ to be differentiable at $a$ it is necessary that
there be a linear function $T \colon \R \sto \R$ which is tangent at $0$ to $\Delta f_a$.
According to \ref{lin_R} there must exist a constant $c$ such that $Tx = cx$ for all $x$
in~$\R$.  For $T$ to be tangent to $\Delta f_a$, it must be the case that
  \[ \Delta f_a - T \in \lobo o\,; \]
that is,
  \[ \lim_{h \sto 0}\frac{\Delta f_a(h) - ch}h = 0\,. \]
Equivalently,
   \[ \lim_{h \sto 0}\frac{f(a+h) - f(a)}h
              = \lim_{h \sto 0}\frac{\Delta f_a(h)}h = c\,. \]
In other words, the function $T$, which is tangent to $\Delta f_a$ at $0$, must be a line
through the origin whose slope is
   \[ \lim_{h \sto 0}\frac{f(a + h) - f(a)}h\,. \]
This is, of course, the familiar ``derivative of $f$ at $a$'' from beginning calculus.  Thus
for any real valued function $f$ which is differentiable at $a$ in~$\R$
 \index{<@$f'$ (derivative of $f$)}%
   \[ df_a(h) = f'(a)\cdot h \]
for all $h \in \R$.
\end{exam}

\begin{prob} Explain carefully the quotation from Dieudonn\'e given at the beginning of the chapter.
\end{prob}

\begin{exam} Let $f(x) =3x^2 - 7x + 5$ and $a=2$. Then $f$ is differentiable at~$a$.  (Sketch the
graph of the differential~$df_a$.)
\end{exam}

\begin{proof} Problem.  \ns  \end{proof}

\begin{exam} Let $f(x) = \sin x$ and $a = \pi/3$.  Then $f$ is differentiable at~$a$.  (Sketch the
graph of the differential~$df_a$.)
\end{exam}

\begin{proof} Problem.  \ns  \end{proof}

\begin{prop} Let $T \in \ofml L$ and $a \in \R$. Then $dT_a = T$.
\end{prop}

\begin{proof} Problem.  \ns  \end{proof}

\begin{prop}\label{del_in_O} If $f \in \fml D_a$, then $\Delta f_a \in \lobo O$.
\end{prop}

\begin{proof} Exercise.   (Solution~\ref{sol_del_in_O}.) \ns \end{proof}

\begin{cor}\label{diff_cont} Every function which is differentiable at a point is continuous there.
\end{cor}

\begin{proof} Exercise.  (Solution~\ref{sol_diff_cont}.)    \ns \end{proof}

\begin{prop}\label{diff_sm} If $f$ is differentiable at $a$ and $\alpha \in \R$, then $\alpha f$
is differentiable at $a$ and
   \[ d(\alpha f)_a = \alpha\,df_a\,. \]
\end{prop}

\begin{proof} Exercise.  (Solution~\ref{sol_diff_sm}.)   \ns \end{proof}

\begin{prop}\label{diff_sum} If $f$ and $g$ are differentiable at $a$, then $f + g$ is
differentiable at $a$ and
   \[ d(f+g)_a = df_a + dg_a\,. \]
\end{prop}

\begin{proof} Problem.  \ns \end{proof}

\begin{prop}[Leibniz's Rule.]\label{diff_mult} If $\phi, f \in \fml D_a$,
 \index{Leibniz's rule}%
then $\phi f \in \fml D_a$ and
   \[ d(\phi f)_a = d\phi_a \cdot f(a) \,+\, \phi(a)\,df_a\,. \]
\end{prop}

\begin{proof} Exercise.  (Solution~\ref{sol_diff_mult}.) \ns \end{proof}

 \index{chain rule}%
\begin{thm}[The Chain Rule]\label{ch_rul} If $f \in \fml D_a$ and $g \in \fml D_{f(a)}$, then
$g \circ f \in \fml D_a$ and
   \[ d(g \circ f)_a = dg_{{}_{\scriptstyle{f(a)}}} \circ df_a\,. \]
\end{thm}

\begin{proof} Exercise.  (Solution~\ref{sol_ch_rul}.) \ns \end{proof}

\begin{prob}[A Problem Set on Functions from $\R$ into $\R$] We are
now in a position to derive the standard results, usually contained
in the first term of a beginning calculus course, concerning the
differentiation of real valued functions of a single real variable.
Having at our disposal the machinery developed earlier in this
chapter, we may derive these results quite easily; and so the proof
of each is a problem.

\begin{defn} If $f \in \fml D_a$, the
 \index{derivative!of a function from $\R$ into $\R$}%
\df{derivative of $f$ at} $a$, denoted by $f'(a)$ or $Df(a)$, is defined to be ${\D\lim_{h
\sto 0}\frac{f(a + h) - f(a)}h}$. By \ref{trnsl_lim} this is the same as ${\D\lim_{x \sto
a}\frac{f(x) - f(a)}{x - a}}$.
\end{defn}

\begin{prop}\label{diff_deriv} If $f \in \fml D_a$, then $Df(a) = df_a(1)$.
\end{prop}

\begin{proof} Problem.  \ns  \end{proof}

\begin{prop} If $f$, $g \in \fml D_a$, then
   \[ D(fg)(a) = Df(a) \cdot g(a) + f(a) \cdot Dg(a)\,. \]
\end{prop}

\begin{proof} Problem. \emph{Hint.} Use \emph{Leibniz's rule} (\ref{diff_mult}) and
proposition~\ref{diff_deriv}.  \ns
\end{proof}

\begin{exam}\label{8exam1} Let $r(t) = \dfrac1t$ for all $t \ne 0$. Then $r$ is differentiable
and $Dr(t) = -\dfrac1{t^2}$ for all $t \ne 0$.
\end{exam}

\begin{proof} Problem.  \ns  \end{proof}

 \index{chain rule}%
\begin{prop}\label{c_ru}  If $f \in \fml D_a$ and $g \in \fml D_{f(a)}$, then $g \circ f \in
\fml D_a$
and
  \[ D(g \circ f)(a) = (Dg)(f(a)) \cdot Df(a)\,. \]
\end{prop}

\begin{proof} Problem. \ns  \end{proof}

\begin{prop} If $f$, $g \in \fml D_a$ and $g(a) \ne 0$, then
  \[ D\biggl(\frac fg\biggr)(a) = \frac{g(a)\,Df(a) - f(a)\,Dg(a)}{(g(a))^2}\;. \]
\end{prop}

\begin{proof} Problem. \emph{Hint.} Write $\dfrac fg$ as $f \cdot (r \circ g)$ and use
\ref{diff_deriv}, \ref{c_ru}, and~\ref{8exam1}. \ns
\end{proof}

\begin{prop}\label{pos_deriv} If $f \in \fml D_a$ and $Df(a) > 0$, then there exists $r > 0$
such that
 \begin{enumerate}
  \item[(i)] $f(x) > f(a)$ whenever $a < x < a+r$, and
  \item[(ii)] $f(x) < f(a)$ whenever $a-r <x < a$.
 \end{enumerate}
\end{prop}

\begin{proof} Problem.  \emph{Hint.} Define $g(h) = h^{-1}\,\Delta f_a(h)$ if $h \ne 0$ and
$g(0) = Df(a)$. Use proposition~\ref{cnt_vs_lim} to show that $g$ is continuous at $0$. Then
apply proposition~\ref{pos_at_pt}.  \ns
\end{proof}

\begin{prop}\label{neg_deriv}  If $f \in \fml D_a$ and $Df(a) < 0$, then there exists $r > 0$
such that
 \begin{enumerate}
  \item[(i)] $f(x) < f(a)$ whenever $a < x < a+r$, and
  \item[(ii)] $f(x) > f(a)$ whenever $a-r < x < a$.
 \end{enumerate}
\end{prop}

\begin{proof} Problem. \emph{Hint.} Of course it is possible to obtain this result by doing
\ref{pos_deriv} again with some inequalities reversed. That is the hard way.   \ns
\end{proof}

\begin{defn}  Let $f \colon A \sto \R$ where $A \subseteq \R$. The function $f$ has a
 \index{local!maximum}%
 \index{maximum!local}%
\df{local} (or
 \index{relative!maximum}%
 \index{maximum!relative}%
\df{relative}) \df{maximum} at a point $a \in A$ if there exists $r > 0$ such that $f(a) \ge
f(x)$ whenever $\abs{x - a} < r$ and $x \in \dom f$. It has a
 \index{local!minimum}%
 \index{minimum!local}%
\df{local} (or
 \index{relative!minimum}%
 \index{minimum!relative}%
\df{relative}) \df{minimum} at a point $a \in A$ if there exists $r > 0$ such that $f(a) \le
f(x)$ whenever $\abs{x - a} < r$ and $x \in \dom f$.

Recall from chapter \ref{evt} that $f \colon A \sto \R$ is said to attain a
 \index{maximum}%
\df{maximum} at $a$ if $f(a) \ge f(x)$ for all $x \in \dom f$.
This is often called a
 \index{global!maximum}%
 \index{maximum!global}%
\df{global} (or
 \index{absolute!maximum}%
 \index{maximum!absolute}%
\df{absolute}) \df{maximum} to help distinguish it from the local version just defined. It is
clear that every global maximum is also a local maximum but not \emph{vice versa}. (Of course
a similar remark holds for minima.)
\end{defn}

\begin{prop}\label{extr_crit}  If $f \in \fml D_a$ and $f$ has either a local maximum or a
local minimum at $a$, then $Df(a) = 0$.
\end{prop}

\begin{proof} Problem. \emph{Hint.} Use propositions \ref{pos_deriv} and \ref{neg_deriv}.)
\ns  \end{proof}

\index{Rolle's theorem}%
\begin{prop}[Rolle's Theorem]\label{Rthm} Let $a<b$. If $f \colon [a,b] \sto \R$ is continuous,
if it is differentiable on $(a,b)$, and if $f(a) = f(b)$, then there exists a point $c$ in
$(a,b)$ such that $Df(c) = 0$.
\end{prop}

\begin{proof} Problem. \emph{Hint.} Argue by contradiction. Use the \emph{extreme value
theorem}~\ref{evthm} and proposition~\ref{extr_crit}.  \ns
\end{proof}

 \index{mean value theorem}%
\begin{thm}[Mean Value Theorem]\label{mvthm} Let $a < b$. If $f \colon [a,b] \sto \R$ is continuous
and if it is differentiable on $(a,b)$, then there exists a point $c$ in $(a,b)$ such that
   \[ Df(c) = \frac{f(b) - f(a)}{b - a}\,. \]
\end{thm}

\begin{proof} Problem. \emph{Hint.} Let $y = g(x)$ be the equation of the line which passes through
the points $(a,f(a))$ and $(b,f(b))$. Show that the function $f - g$ satisfies the hypotheses
of \emph{Rolle's theorem}~(\ref{Rthm})  \ns
\end{proof}

\begin{prop} Let $J$ be an open interval in $\R$. If $f \colon J \sto \R$ is differentiable and
$Df(x) = 0$ for every $x \in J$, then $f$ is constant on $J$.
\end{prop}

\begin{proof} Problem. \emph{Hint.} Use the \emph{mean value theorem}~(\ref{mvthm}). \ns
\end{proof}
\end{prob}



\endinput
