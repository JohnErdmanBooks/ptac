\chapter{SETS}\label{sets}

 \setcounter{section}{1}
 \setcounter{thm}{0}

In this text everything is defined ultimately in terms of two primitive (that is, undefined)
concepts:
 \index{set}%
set and
 \index{set!membership in a}%
set membership. We assume that these are already familiar to the reader.  In particular, it is
assumed to be understood that distinct
 \index{element}%
elements (or
 \index{member}%
members, or points) can be regarded collectively as a single set (or family, or class, or
collection). To indicate that $x$ belongs to a set $A$ (or that $x$ is a member of $A$) we
write $x \in A$; to indicate that it does not belong to $A$ we write $x \notin A$.

We specify a set by listing its members between braces  (for instance, $\{1, 2, 3, 4, 5\}$ is
the set of the first five natural numbers), by listing some of its members between braces with
an ellipsis (three dots) indicating the missing members (\emph{e.g.\ }$\{1, 2, 3, \dots\}$ is
the set of all natural numbers), or by writing $\{x\colon P(x)\}$ where $P(x)$ is an open
sentence which specifies what property the variable $x$ must satisfy in order to be included
in the set (\emph{e.g.\ }$\{x\colon 0 \le x \le 1\}$ is the closed unit interval $[0,1]$).

\begin{prob} Let $\N$ be the set of natural numbers $1$, $2$, $3$, \dots and let
   \[ S = \{x \colon x < 30 \text{ and $x = n^2$ for some $n \in \N$}\}\,. \]
List all the elements of~$S$.
\end{prob}

\begin{prob} Let $\N$ be the set of natural numbers $1$, $2$, $3$, \dots and let
  \[ S = \{x \colon x = n + 2 \text{ for some $n \in \N$ such that $n < 6$}\}\,. \]
List all the elements of~$S$.
\end{prob}

\begin{prob} Suppose that
  \[ S = \{x \colon x = n^2 + 2 \text{ for some $n \in \N$}\} \]
and that
  \[ T = \{3,6,11,18,27,33,38,51\}\,. \]
 \begin{enumerate}
  \item[(a)] Find an element of $S$ that is not in~$T$.
  \item[(b)] Find an element of $T$ that is not in~$S$.
 \end{enumerate}
\end{prob}

\begin{prob} Suppose that
  \[ S = \{x \colon x = n^2 + 2n \text{ for some $n \in \N$}\} \]
and that
  \[ T = \{x\colon x = 5n - 1 \text{ for some $n \in \N$}\}\,. \]
Find an element which belongs to both $S$ and ~$T$.
\end{prob}

Since all of our subsequent work depends on the notions of set and of set membership, it is
not altogether satisfactory to rely on intuition and shared understanding to provide a
foundation for these crucial concepts. It is possible in fact to arrive at paradoxes using a
naive approach to sets. (For example, ask yourself the question, ``If $S$ is the set of all
sets which do not contain themselves as members, then does $S$ belong to $S$?'' If the answer
is \emph{yes}, then it must be \emph{no}, and \emph{vice versa}.)  One satisfactory
alternative to our intuitive approach to sets is axiomatic set theory.  There are many ways of
axiomatizing set theory to provide a secure foundation for subsequent mathematical
development.  Unfortunately, each of these ways turns out to be extremely intricate, and it is
generally felt that the abstract complexities of axiomatic set theory do not serve well as a
beginning to an introductory course in advanced calculus.

Most of the paradoxes inherent in an intuitive approach to sets have to do with sets that are
too ``large''. For example, the set $S$ mentioned above is enormous. Thus in the sequel we
will assume that in each situation all the mathematical objects we are then considering (sets,
functions, \emph{etc.}) belong to some appropriate ``universal'' set, which is ``small''
enough to avoid set theoretic paradoxes. (Think of ``universal'' in terms of ``universe of
discourse'', not ``all-encompassing''.)  In many cases an appropriate universal set is clear
from the context. Previously we considered a statement
    \[ (\forall y)(\exists x)\, x < y\,. \]
The appearance of the symbol ``$<$'' suggests to most readers that $x$ and $y$ are real
numbers. Thus the universal set from which the variables are chosen is the set $\R$ of all
real numbers. When there is doubt that the universal set will be properly interpreted, it may
be specified. In the example just mentioned we might write
   \[(\forall y \in \R)(\exists x \in \R)\, x < y.\]
This makes explicit the intended restriction that $x$ and $y$ be real numbers.

As another example recall that in appendix~\ref{quan} we defined a real valued function $f$ to
be continuous at a point $a \in \R$ if
  \[(\forall\epsilon > 0)(\exists \delta > 0) \text{ such that
        $(\forall x)\, \abs{f(x)-f(a)} < \epsilon$  whenever
            $\abs{x - a} < \delta$}.\]
Here the first two variables, $\epsilon$ and $\delta$, are restricted to lie in the open
interval $(0, \infty)$. Thus we might rewrite the definition as follows:
  \[\begin{split}
    \forall\epsilon\in (0,\infty)\,\, &\exists\delta\in (0,\infty) \\
        &\text{such that ($\forall x \in \R)\, \abs{f(x)-f(a)}
        <\epsilon$ whenever $\abs{x-a} < \delta$}.
  \end{split}\]
The expressions ``$\forall x \in \R$'', ``\,$\exists\delta\in (0, \infty)$'', \emph{etc.}\ are
called
 \index{restricted quantifiers}%
 \index{quantifier!restricted}%
\df{restricted quantifiers}.

\begin{defn} Let $S$ and $T$ be sets. We say that $S$ is a
 \index{subset}%
\df{subset} of $T$ and write
 \index{subset@$\subseteq$ (subset of)}%
 \index{<@$\subseteq$ (subset of)}%
$S \subseteq T$ (or $T \supseteq S$) if every member of $S$ belongs to $T$. If $S \subseteq T$
we also say that $S$ is \df{contained in} $T$ or that $T$
 \index{contains}%
\df{contains}~$S$.  Notice that the relation $\subseteq$ is
 \index{reflexive}%
\df{reflexive} (that is, $S \subseteq S$ for all $S$) and
 \index{transitive}%
\df{transitive} (that is, if $S \subseteq T$ and $T \subseteq U$,
then $S \subseteq U$). It is also
 \index{antisymmetric}%
\df{antisymmetric} (that is, if $S \subseteq T$ and $T \subseteq S$, then $S = T$). If we wish
to claim that $S$ is \emph{not} a subset of $T$ we may write $S \not\subseteq T$. In this case
there is at least one member of $S$ which does not belong to $T$.
\end{defn}

\begin{exam} Since every number in the closed interval $[0,1]$ also belongs to the interval
$[0,5]$, it is correct to write $[0, 1] \subseteq [0,5]$.  Since the number $\pi$ belongs to
$[0,5]$ but not to $[0,1]$, we may also write $[0,5] \not\subseteq [0,1]$.
\end{exam}

\begin{defn} If $S \subseteq T$ but $S \ne T$, then we say that $S$ is a
 \index{subset!proper}%
 \index{proper subset}%
\df{proper subset} of $T$ (or that $S$ is \df{properly contained in} $T$, or that~$T$
 \index{properly contains}%
\df{properly contains} $S$) and write
 \index{subsetn@$\subsetneqq$ (proper subset of)}%
 \index{<@$\subsetneqq$ (proper subset of)}%
$S \subsetneqq T$.
\end{defn}

\begin{prob} Suppose that $S = \{x\colon x = 2n + 3 \text{ for some $n \in \N$}\}$ and that
$T$ is the set of all odd natural numbers $1$, $3$, $5$, $\dots$\,.
 \begin{enumerate}
  \item[(a)] Is $S \subseteq T$?  If not, find an element of $S$ which does not
belong to~$T$.
  \item[(b)] Is $T \subseteq S$?  If not, find an element of $T$ which does not
belong to~$S$.
 \end{enumerate}
\end{prob}


\begin{defn}  The
 \index{empty set}%
\df{empty set} (or
 \index{null set}%
\df{null set}), which is denoted
 \index{emptyset@$\emptyset$ (empty set, null set)}%
 \index{<@$\emptyset$ (empty set, null set)}%
 \index{nullset@$\emptyset$ (empty set, null set)}%
by~$\emptyset$, is defined to be the set which has no elements. (Or, if you like, define it to
be $\{x\colon x \ne x\}$.) It is regarded as a subset of every set, so that
$\emptyset\subseteq S$ is always true.  (Note: ``$\emptyset$'' is a letter of the Danish
alphabet, not the Greek letter ``phi''.)
\end{defn}

\begin{defn} If $S$ is a set, then the \df{power set} of $S$, which we denote
 \index{powerset@$\sfml P$ (power set)}%
by~$\sfml P(S)$, is the set of all subsets of $S$.
\end{defn}

\begin{exam} Let $S = \{a,b,c\}$.  Then the members of the power set of $S$ are the empty set,
the three one-element subsets, the three two-element subsets, and the set $S$ itself.  That
is,
  \[ \sfml P(S) = \bigl\{\emptyset,\{a\},\{b\}, \{c\}, \{a,b\},
            \{a,c\}, \{b,c\},S\bigr\}\,. \]
\end{exam}

\begin{prob} In each of the words (a)--(d) below let $S$ be the set of letters in the word.
In each case find the number of members of $S$ and the number of members of $\sfml P(S)$
the power set of~$S$.
 \begin{enumerate}
  \item[(a)] lull
  \item[(b)] appall
  \item[(c)] attract
  \item[(d)] calculus
 \end{enumerate}
\end{prob}


\begin{cau} In attempting to prove a theorem which has as a hypothesis ``Let
$S$ be a set'' do not include in your proof something like ``Suppose $S = \{s_1, s_2, \dots,
s_n\}$'' or ``Suppose $S = \{s_1, s_2, \dots\}$''.  In the first case you are tacitly assuming
that $S$ is finite and in the second that it is countable. Neither is justified by the
hypothesis.
\end{cau}



\begin{cau} A single letter $\sfml S$ (an S in fraktur font) is an acceptable symbol
in printed documents. Don't try to imitate it in hand-written work or on the blackboard.  Use
script letters instead.
\end{cau}






Finally a word on the use of the symbols $=$ and~$:=$.  In this text
 \index{equality}%
 \index{<@$=$ (equals)}%
equality is used in the sense of identity.  We write $x = y$ to indicate that $x$ and $y$ are
two names for the same object.  For example, $0.5 = 1/2 = 3/6 = 1/\sqrt 4$ because $0.5$,
$1/2$, $3/6$, and $1/\sqrt 4$ are different names for the same real number.  You have probably
encountered other uses of the term \emph{equality}. In many high school geometry texts, for
example, one finds statements to the effect that a triangle is isosceles if it has two equal
sides (or two equal angles).  What is meant of course is that a triangle is isosceles if it
has two sides of \emph{equal length} (or two angles of \emph{equal angular measure}).  We also
make occasional use of the
 \index{equality!by definition}%
 \index{<@$:=$ (equality by definition)}%
symbol~$:=$ to indicate \emph{equality by definition}.  Thus when we write $a := b$ we are
giving a new name $a$ to an object $b$ with which we are presumably already familiar.

 \endinput
