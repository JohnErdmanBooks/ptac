\chapter{SEQUENCES OF REAL NUMBERS}\label{seqs_rn}

Sequences are an extremely useful tool in dealing with topological properties of sets in $\R$
and, as we will see later, in general metric spaces.  A major goal of this chapter is to
illustrate this usefulness by showing how sequences may be used to characterize open sets,
closed sets, and continuous functions.



\section{CONVERGENCE OF SEQUENCES}

A
 \index{sequence}%
\df{sequence} is a function whose domain is the set $\N$ of natural numbers.  In this chapter
the sequences we consider will all be sequences of real numbers, that is, functions from $\N$
into $\R$. If $a$ is a sequence, it is conventional to write its value at a natural number $n$
as $a_n$ rather than as $a(n)$. The sequence itself may be written in a number of ways:
   \[a = (a_n)_{n=1}^\infty = (a_n) = (a_1,a_2,a_3, \dots).\]
Care should be exercised in using the last of these notations.  It would not be clear, for
example, whether $(\frac13,\frac15,\frac17,\dots)$ is intended to be the sequence of
reciprocals of odd primes (in which case the next term would be $\frac1{11}$) or the sequence
of reciprocals of odd integers greater than $1$ (in which case the next term would be
$\frac19$). The element $a_n$ in the range of a sequence is the $n^{\text{th}}$
 \index{term!of a sequence}%
\df{term} of the sequence.

It is important to distinguish in one's thinking between a sequence and its range.  Think of a
sequence $(x_1,x_2,x_3,\dots)$ as an \emph{ordered} set: there is a first element $x_1$, and a
second element $x_2$, and so on.  The range $\{x_1,x_2,x_3,\dots\}$ is just a set.  There is
no ``first'' element.  For example, the sequences $(1,2,3,4,5,6,\dots)$ and
$(2,1,4,3,6,5,\dots)$ are different whereas the sets $\{1,2,3,4,5,6,\dots\}$ and
$\{2,1,4,3,6,5,\dots\}$ are exactly the same (both are $\N$).

\begin{rem} Occasionally in the sequel it will be convenient to alter the preceding definition
a bit to allow the domain of a sequence to be the set $\N \cup \{0\}$ of all positive
integers. It is worth noticing as we go along that this in no way affects the correctness of
the results we prove in this chapter.
\end{rem}

\begin{defn} A sequence $(x_n)$ of real numbers is
 \index{eventually in}%
\df{eventually in} a set $A$ if there exists a natural number $n_0$ such that $x_n \in A$
whenever $n \ge n_0$.
\end{defn}

\begin{exam} For each $n \in \N$ let $x_n = 3 + \dfrac{7-n}2$.  Then the sequence
$(x_n)_{n=1}^\infty$ is eventually strictly negative; that is, the sequence is eventually in
the interval $(-\infty,0)$.
\end{exam}

\begin{proof} If $n \ge 14$, then $\dfrac{7-n}2 \le -\dfrac72$ and $x_n = 3 + \dfrac{7-n}2
\le -\dfrac12 < 0$.  So $x_n \in (-\infty,0)$ whenever $n \ge 14$.
\end{proof}

\begin{exam} For each $n \in \N$ let $x_n = \dfrac{2n-3}n$.  Then the sequence $(x_n)_{n=1}^\infty$
is eventually in the interval $(\frac32,2)$.
\end{exam}

\begin{proof} Problem. \ns \end{proof}

\begin{defn}\label{df_conv_seq} A sequence $(x_n)_{n=1}^\infty$ of real numbers
 \index{convergence!of a sequence}%
\df{converges} to $a \in \R$ if it is eventually in every $\epsilon$-neighborhood of~$a$. When
the sequence converges to $a$ we write
 \index{<@$x_n \sto a \text{ as $n \sto \infty$}$ (limit of a sequence)}%
 \begin{equation}\label{xn_to_a} x_n \sto a \text{ as $n \sto \infty$.}
 \end{equation}
These symbols may be read, ``$x_n$ approaches $a$ as $n$ gets large.''  If $(x_n)$ converges
to $a$, the number $a$ is the
 \index{limit!of a sequence}%
\df{limit} of the sequence~$(x_n)$.  It would not be proper to refer to \emph{the} limit of a
sequence if it were possible for a sequence to converge to two different points.  We show now
that this cannot happen; limits of sequences are unique.
\end{defn}

\begin{prop} Let $(x_n)$ be a sequence in $\R$.  If $x_n \sto a$ and $x_n \sto b$ as $n \sto \infty$,
then $a=b$.
\end{prop}

\begin{proof} Argue by contradiction.  Suppose $a \ne b$, and let $\epsilon = \abs{a-b}$. Then
$\epsilon > 0$. Since $(x_n)$ is eventually in $J_{\epsilon/2}(a)$, there exists $n_0 \in \N$
such that $x_n \in J_{\epsilon/2}(a)$ for $n \ge n_0$.  That is, $\abs{x_n - a} <
\frac\epsilon2$ for $n \ge n_0$.  Similarly, since $(x_n)$ is eventually in
$J_{\epsilon/2}(b)$, there is an $m_0 \in \N$ such that $\abs{x_n-b} < \frac\epsilon2$ for $n
\ge m_0$.  Now if $n$ is any integer larger than both $n_0$ and $m_0$, then
   \[\epsilon = \abs{a-b} = \abs{a-x_n+x_n-b} \le \abs{a-x_n} +
              \abs{x_n-b} < \tfrac\epsilon2 + \tfrac\epsilon2 = \epsilon.\]
But $\epsilon < \epsilon$ is impossible. Therefore, our initial supposition was wrong, and $a
= b$.
\end{proof}

Since limits are unique, we may use an alternative notation to \eqref{xn_to_a}: if $(x_n)$
converges to $a$ we may write
 \index{limit@$\lim_{n \sto \infty}x_n$ (limit of a sequence)}%
              \[\lim_{n \sto \infty}x_n = a.\]
(Notice how inappropriate this notation would be if limits were not unique.)

\begin{defn} If a sequence $(x_n)$ does not converge (that is, if there exists no $a \in \R$
such that $(x_n)$ converges to $a$), then the sequence
 \index{divergence!of a sequence}%
\df{diverges}.  Sometimes a divergent sequence $(x_n)$ has the property that it is eventually
in every interval of the form $(n,\infty)$ where $n \in \N$. In this case we write
   \[x_n \sto \infty \text{ as $n \sto \infty$} \qquad \text{or} \qquad
                          \lim_{n \sto \infty}x_n = \infty.\]
If a divergent sequence $(x_n)$ is eventually in every interval of
the form $(-\infty,-n)$ for $n \in \N$, we write
   \[x_n \sto -\infty \text{ as $n \sto \infty$} \qquad \text{or} \qquad
                               \lim_{n \sto \infty}x_n = -\infty.\]
\end{defn}

\begin{cau} It is \emph{not} true that every divergent sequence satisfies either $x_n \sto \infty$
or $x_n \sto -\infty$.  See \ref{exam_div_seq} below.
\end{cau}

It is easy to rephrase the definition of convergence of a sequence in slightly different
language.  The next problem gives two such variants.  Sometimes one of these alternative
``definitions'' is more convenient than \ref{df_conv_seq}.

\begin{prob}\label{cond_conv_seq} Let $(x_n)$ be a sequence of real numbers and $a \in \R$.
 \begin{enumerate}
  \item[(a)]  Show that $x_n \sto a$ if and only if for every $\epsilon > 0$ there exists
$n_0 \in \N$ such that $\abs{x_n - a} < \epsilon$ whenever $n \ge n_0$.
  \item[(b)] Show that $x_n \sto a$ if and only if $(x_n)$ is eventually in every
neighborhood of~$a$.
 \end{enumerate}
\end{prob}


\begin{exam}\label{seq_exam1} The sequence $(\frac1n)_{n=1}^\infty$ converges to~$0$.
\end{exam}

\begin{proof} Let $\epsilon > 0$. Use the \emph{Archimedean property}~\ref{arch_prop} of the
real numbers to choose $N \in \N$ large enough that $N > \frac1\epsilon$. Then
   \[\left\lvert\frac1n - 0\right\rvert = \frac1n \le \frac1N < \epsilon.\]
whenever $n \ge N$.
\end{proof}

\begin{exam}\label{exam_div_seq} The sequence $\bigl((-1)^n\bigr)_{n=0}^\infty$ diverges.
\end{exam}

\begin{proof} Argue by contradiction. If we assume that $(-1)^n \sto a$ for some $a \in \R$,
then there exists $N \in \N$ such that $n \ge N$ implies $\abs{(-1)^n - a} < 1$. Thus for
every $n \ge N$
 \begin{align*}
        2 &=   \abs{(-1)^n - (-1)^{n+1}} \\
          &=   \abs{(-1)^n - a + a - (-1)^{n+1}} \\
          &\le \abs{(-1)^n - a} + \abs{a - (-1)^{n+1}} \\
          &<   1 + 1  = 2
 \end{align*}
which is not possible.
\end{proof}

\begin{exam} The sequence  $\left(\dfrac n{n+1}\right)_{n=1}^\infty$ converges to $1$.
\end{exam}

\begin{proof} Problem. \ns  \end{proof}

\begin{prop}\label{abs_seq} Let $(x_n)$ be a sequence in $\R$. Then $x_n \sto 0$ if and
only if $\abs{x_n} \sto 0$.
\end{prop}

\begin{proof} Exercise.  (Solution~\ref{sol_abs_seq}.)  \ns  \end{proof}








\section{ALGEBRAIC COMBINATIONS OF SEQUENCES}

As you will recall from beginning calculus, one standard way of computing the limit of a
complicated expression is to find the limits of the constituent parts of the expression and
then combine them algebraically.  Suppose, for example, we are given sequences $(x_n)$ and
$(y_n)$ and are asked to find the limit of the sequence $(x_n+y_n)$.  What do we do?  First we
try to find the limits of the individual sequences $(x_n)$ and  $(y_n)$.  Then we add.  This
process is justified by a proposition which says, roughly, that \emph{the limit of a sum is
the sum of the limits.} Limits with respect to other algebraic operations behave similarly.

The aim of the following problem is to develop the theory detailing how limits of sequences
interact with algebraic operations.

\begin{prob}\label{seq_proj} Suppose that $(x_n)$ and $(y_n)$ are sequences of real numbers,
that $x_n \sto a$ and $y_n \sto b$, and that $c \in \R$.  For the case where $a$ and $b$ are
real numbers derive the following:
 \begin{enumerate}
  \item[(a)] $x_n + y_n \sto a + b$,
  \item[(b)] $x_n - y_n \sto a - b$,
  \item[(c)] $x_n y_n \sto ab$,
  \item[(d)] $c x_n \sto ca$, and
  \item[(e)] $\dfrac{x_n}{y_n} \sto \dfrac ab$ if $b \ne 0$.
 \end{enumerate}
Then consider what happens in case $a = \pm\infty$ or $b = \pm\infty$ (or both).  What can you
say (if anything) about the limits of the left hand expressions of (a)--(e)?  In those cases
in which nothing can be said, give examples to demonstrate as many outcomes as possible.  For
example, if $a = \infty$ and $b = -\infty$, then nothing can be concluded about the limit as
$n$ gets large of $x_n+y_n$.  All of the following are possible:
 \begin{enumerate}
  \item[(i)] $x_n + y_n \sto -\infty$. [Let $x_n = n$ and $y_n = -2n$.]
  \item[(ii)] $x_n + y_n \sto \alpha$, where $\alpha$ is any real number. [Let $x_n = \alpha + n$
and $y_n = -n$.]
  \item[(iii)] $x_n + y_n \sto \infty$. [Let $x_n = 2n$ and $y_n = -n$.]
  \item[(iv)] None of the above. [Let $(x_n) = (1,2,5,6,9,10,13,14,\dots)$ and
$y_n = (0,-3,-4,-7,-8,-11,-12,-15,\dots)$.]
\end{enumerate}
\end{prob}

\begin{exam} If $x_n \sto a$ in $\R$ and $p \in  \N$, then ${x_n}^p \sto a^p$.
\end{exam}

\begin{proof} Use induction on $p$.  The conclusion obviously holds when $p=1$.  Suppose
${x_n}^{p-1} \sto a^{p-1}$.  Apply part (c) of the preceding problem:
   \[{x_n}^p = {x_n}^{p-1} \cdot x_n \sto a^{p-1} \cdot a = a^p.\]
\end{proof}

\begin{exam} If $x_n = \dfrac{2-5n+7n^2-6n^3}{4-3n+5n^2+4n^3}$ for each $n \in \N$, then
$x_n \sto -\frac32$ as $n \sto \infty$.
\end{exam}

\begin{proof} Problem.  \ns  \end{proof}

\begin{prob} Find $\lim_{n \sto \infty}(\sqrt{n^2 + 5n} - n)$.
\end{prob}

Another very useful tool in computing limits is the ``sandwich theorem'',  which says that a
sequence sandwiched between two other sequences with a common limit has that same limit.

\begin{prop}[Sandwich theorem]\label{sand_seq}
 \index{sandwich theorem}%
Let $a$ be a real number or one of the symbols $+\infty$ or~$-\infty$.  If $x_n \sto a$ and
$z_n \sto a$ and if $x_n \le y_n \le z_n$ for all $n$, then $y_n \sto a$.
\end{prop}

\begin{proof} Problem. \ns  \end{proof}

\begin{exam} If $x_n = \dfrac{\sin\bigl(3 + \pi^{n^2}\bigr)}{n^{3/2}}$ for each $n \in \N$,
then $x_n \sto 0$ as $n \sto \infty$.
\end{exam}

\begin{proof} Problem. \emph{Hint.} Use \ref{abs_seq}, \ref{seq_exam1}, and \ref{sand_seq}.
  \ns \end{proof}


\begin{exam} If $(x_n)$ is a sequence in $(0,\infty)$  and $x_n \sto a$, then
$\sqrt{x_n} \sto \sqrt a$.
\end{exam}

\begin{proof} Problem. \emph{Hint.}  There are two possibilities: treat the cases
$a = 0$ and $a > 0$ separately. For the first use problem~\ref{cond_conv_seq}(a).  For the
second use \ref{seq_proj}(b) and \ref{abs_seq}; write $\sqrt{x_n} - \sqrt a$ as
$\abs{x_n~-~a}/(\sqrt{x_n} + \sqrt a)$. Then find an inequality which allows you to use the
\emph{sandwich theorem}(proposition \ref{sand_seq}). \ns
\end{proof}

\begin{exam}\label{4exam2} The sequence $\bigl(n^{1/n}\bigr)$ converges to~$1$.
\end{exam}

\begin{proof} Problem. \emph{Hint.}  For each $n$ let $a_n = n^{1/n}-1$.  Apply the
\emph{binomial theorem} \ref{binom_thm} to $(1+a_n)^n$ to obtain the inequality
$n>\frac12n(n-1)\,{a_n}^2$ and hence to conclude that $0<a_n<\sqrt{\dfrac2{n-1}}$ for every
$n$. Use the \emph{sandwich theorem} \ref{sand_seq}.  \ns
\end{proof}










\section{SUFFICIENT CONDITION FOR CONVERGENCE}

An amusing situation sometimes arises in which we know what the limit of a sequence must be
\emph{if} it exists, but we have no idea whether the sequence actually converges. Here is an
example of this odd behavior.  The sequence will be recursively defined. Sequences are said to
be
 \index{recursive definition}%
\df{recursively} defined if only the first term or first few terms
are given explicitly and the remaining terms are defined by means
of the preceding term(s).

Consider the sequence $(x_n)$ defined so that $x_1=1$ and for $n \in \N$
  \begin{equation}\label{recur_seq} x_{n+1} =
\frac{3(1+x_n)}{3+x_n}\,.
  \end{equation}
It is obvious that \emph{IF the sequence converges} to a limit, say $\ell$, then $\ell$ must
satisfy
   \[\ell = \frac{3(1+\ell)}{3+\ell}\,. \]
(This is obtained by taking limits as $n \sto \infty$ on both sides of \eqref{recur_seq}.)
Cross-multiplying and solving for $\ell$ leads us to the conclusion that $\ell = \pm\sqrt3$.
Since the first term $x_1$ is positive, equation \eqref{recur_seq} makes it clear that all the
terms will be positive; so $\ell$ cannot be $-\sqrt3$.  Thus it is entirely clear that if the
limit of the sequence exists, it must be $\sqrt3$.  What is \emph{not} clear is whether the
limit exists at all.  This indicates how useful it is to know some conditions sufficient to
guarantee convergence of a sequence.  The next proposition gives one important example of such
a condition: it says that bounded monotone sequences converge.

First, the relevant definitions.

\begin{defn} A sequence $(x_n)$ of real numbers is
 \index{bounded!sequence}%
\df{bounded} if its range is a bounded subset of $\R$.  Another way of saying the same thing:
a sequence $(x_n)$ in $\R$ is bounded if there exists a number $M$ such that $\abs{x_n} \le M$
for all $n \in \N$.
\end{defn}

\begin{defn} A sequence $(a_n)$ of real numbers is
 \index{increasing!sequence}%
\df{increasing} if $a_{n+1} \ge a_n$ for every $n \in \N$; it is
 \index{strictly increasing!sequence}%
\df{strictly increasing} if $a_{n+1} > a_n$ for every $n$. A sequence is
 \index{decreasing!sequence}%
\df{decreasing} if $a_{n+1} \le a_n$  for every $n$, and is
 \index{strictly decreasing!sequence}%
\df{strictly decreasing} if $a_{n+1} < a_n$  for every $n$. A sequence is
 \index{monotone!sequence}%
\df{monotone} if it is either increasing or decreasing.
\end{defn}

\begin{prop}\label{bdd_monotone} Every bounded monotone sequence of real numbers converges.
In fact, if a sequence is bounded and increasing, then it converges to the least upper bound
of its range.  Similarly, if it is bounded and decreasing, then it converges to the greatest
lower bound of its range.
\end{prop}

\begin{proof} Let $(x_n)$ be a bounded increasing sequence.  Let $R$ be the range of the
sequence and $\ell$ be the least upper bound of $R$; that is, $\ell = \sup\{x_n \colon n \in
\N\}$. We have seen in example \ref{sup_in_clo} that the least upper bound of a nonempty
subset of $\R$ belongs to the closure of that set.  In particular, $\ell \in \clo R$. Thus
given any $\epsilon > 0$ we can find a number $x_{n_0}$ of $R$ which lies in the
interval~$J_\epsilon(\ell)$.  In fact, $x_{n_0} \in (\ell-\epsilon, \ell]$.  Since the
sequence is increasing and bounded above by $\ell$, we have $x_n \in (\ell-\epsilon, \ell]
\subseteq J_\epsilon(\ell)$ for every $n \ge n_0$.  What we have just proved is that the
sequence $(x_n)$ is eventually in every $\epsilon$-neighborhood of $\ell$.  That is, $x_n \sto
\ell$.

If $(x_n)$ is bounded and decreasing, then the sequence $(-x_n)$ is bounded and increasing. If
$R$ is the range of $(x_n)$ and $g$ is the greatest lower bound of $R$, then $-R$ is the range
of $(-x_n)$ and $-g$ is the least upper bound of~$-R$.  By what we have just proved, $-x_n
\sto -g$.  So $x_n \sto g$ as desired.
\end{proof}

\begin{exam}\label{exam_recur} Let $(x_n)$ be the sequence recursively defined above: $x_1 = 1$
and \eqref{recur_seq} holds for all $n \ge 1$.  Then $x_n \sto \sqrt3$ as~$n~\sto~\infty$.
\end{exam}

\begin{proof} We have argued previously that the limit of the sequence is $\sqrt3$ \emph{if} it
exists.  So what we must show is that the sequence does converge.  We have thus far one
tool---proposition \ref{bdd_monotone}.  Thus we \emph{hope} that we will be able to prove that
the sequence is bounded and monotone. If the sequence starts at $1$, is monotone, and
approaches $\sqrt3$, then it must be increasing.  How do we prove that the sequence is
bounded?   Bounded by what?  We observed earlier that $x_1 \ge 0$ and equation
\eqref{recur_seq} then guarantees that all the remaining $x_n$'s will be positive.  So $0$ is
a lower bound for the sequence.  If the sequence is increasing and approaches $\sqrt 3$, then
$\sqrt3$ will be an upper bound for the terms of the sequence.  Thus it appears that the way
to make use of proposition \ref{bdd_monotone} is to prove two things: (1) $0 \le x_n \le
\sqrt3$ for all $n$; and (2) $(x_n)$ is increasing.  If we succeed in establishing both of
these claims, we are done.  \ns
\end{proof}

\textbf{Claim 1.}  $0 \le x_n \le \sqrt3$ for all $n \in \N$.

\begin{proof} We have already observed that $x_n \ge 0$ for all $n$.  We use mathematical
induction to show that $x_n \le \sqrt3$ for all $n$.  Since $x_1=1$, it is clear that $x_1 \le
\sqrt3$. For the inductive hypothesis suppose that for some particular $k$ we have
  \begin{equation}\label{4loc1} x_k \le \sqrt3\,.
  \end{equation}
We wish to show that $x_{k+1} \le \sqrt3$.  To start, multiply both sides of \eqref{4loc1} by
$3-\sqrt3$.  (If you are wondering, ``How did you know to do that?'' consult the next
problem.)  This gives us
   \[3x_k - \sqrt3 x_k \le 3\sqrt3 - 3.\]
Rearrange so that all terms are positive
   \[3 + 3x_k \le 3\sqrt3 + \sqrt3 x_k = \sqrt3(3 + x_k). \]
But then clearly
   \[x_{k+1} = \frac{3+3x_k}{3+x_k} \le \sqrt3, \]
which was to be proved.  \ns
\end{proof}

\textbf{Claim 2.} The sequence $(x_n)$ is increasing.

\begin{proof} We will show that $x_{n+1} - x_n \ge 0$ for all $n$.  For each $n$
   \[x_{n+1}-x_n = \frac{3+3x_n}{3+x_n} - x_n = \frac{3-{x_n}^2}{3+x_n} \ge 0.\]
(We know that $3-{x_n}^2 \ge 0$ from claim 1.)
\end{proof}

\begin{prob} A student, Fred R.\ Dimm, tried on an exam to prove the claims made in
example \ref{exam_recur}.  For the inductive proof of claim 1 that $x_n \le \sqrt3$ he
offered the following ``proof'':
 \begin{align*}
                  x_{k+1} &\le \sqrt3 \\
     \frac{3+3x_k}{3+x_k} &\le \sqrt3 \\
                   3+3x_k &\le \sqrt3(3+x_k) \\
         3x_k - \sqrt3x_k &\le 3\sqrt3 - 3 \\
            (3-\sqrt3)x_k &\le (3-\sqrt3)\sqrt3 \\
                      x_k &\le \sqrt3, \qquad\text{which is true by hypothesis.}
 \end{align*}
 \begin{enumerate}
  \item[(a)] Aside from his regrettable lack of explanation, Fred seems to be making a
serious logical error.  Explain to poor Fred why his offering is not a proof.
\emph{Hint.}  What would you say about a proof that $1=2$, which goes as follows?
   \begin{align*}
                    1 &= 2 \\
            0 \cdot 1 &= 0 \cdot 2 \\
                    0 &= 0, \qquad\text{which is true.}
   \end{align*}
  \item[(b)] Now explain why Fred's computation in (a) is really quite useful scratch
work, even if it is not a proof.  \emph{Hint.} In the correct proof of claim 1, how
\emph{might} its author have been inspired to ``multiply both sides of \ref{4loc1} by $3
- \sqrt3$''?
 \end{enumerate}
\end{prob}

\begin{exam}\label{4exam1} The condition (bounded and monotone) given in proposition
\ref{bdd_monotone}, while sufficient to guarantee the convergence of a sequence, is not
necessary.
\end{exam}

\begin{proof} Problem.  (Give an explicit example.) \ns
\end{proof}

Expressed as a conditional, proposition \ref{bdd_monotone} says that if a sequence is bounded
and monotone, then it converges. Example \ref{4exam1} shows that the converse of this
conditional is not correct.  A partial converse does hold however: if a sequence converges, it
must be bounded.

\begin{prop}\label{conv_bdd} Every convergent sequence in $\R$ is bounded.
\end{prop}

\begin{proof} Exercise.  (Solution~\ref{sol_conv_bdd}.)  \ns \end{proof}

We will encounter many situations when it is important to know the limit as $n$ gets large of
$r^n$ where $r$ is a number in the interval $(-1,1)$ and the limit of $r^{1/n}$ where $r$ is a
number greater than $0$.  The next two propositions show that the respective limits are always
$0$ and $1$.

\begin{prop}\label{prop_lim_pwrs} If $\abs{r} < 1$, then $r^n \sto 0$ as $n \sto \infty$.
If $\abs{r} > 1$, then $(r^n)$ diverges.
\end{prop}

\begin{proof} Suppose that $\abs{r} < 1$. If $r=0$, the proof is trivial, so we consider
only $r \ne 0$.  Since $0 < \abs r < 1$ whenever $-1 < r < 0$, proposition \ref{abs_seq}
allows us to restrict our attention to numbers lying in the interval~$(0,1)$. Thus we suppose
that $0 < r < 1$ and prove that $r^n \sto 0$.  Let $R$ be the range of the sequence $(r^n)$.
That is, let $R = \{r^n \colon n \in \N\}$.  Let $g = \inf R$. Notice that $g \ge 0$ since $0$
is a lower bound for $R$.  We use an argument by contradiction to prove that $g=0$. Assume to
the contrary that $g > 0$.  Since $0 < r < 1$, it must follow that $gr^{-1} > g$.  Since $g$
is the greatest lower bound of $R$, the number $gr^{-1}$ is not a lower bound for $R$. Thus
there exists a member, say $r^k$, of $R$ such that $r^k < gr^{-1}$. But then $r^{k+1} < g$,
which contradicts the choice of $g$ as a lower bound of $R$.  We conclude that $g = 0$.

The sequence $(r^n)$ is bounded and decreasing.  Thus by proposition \ref{bdd_monotone} it
converges to the greatest lower bound of its range; that is, $r^n \sto 0$ as $n \sto \infty$.

Now suppose $r>1$.  Again we argue by contradiction.  Suppose that $(r^n)$ converges.  Then
its range $R$ is a bounded subset of $\R$.

Let $\ell = \sup R$.  Since $r>1$, it is clear that $\ell r^{-1} < \ell$.  Since $\ell$ is the
least upper bound of $R$, there exists a number $r^k$ of $R$ such that $r^k > \ell r^{-1}$.
Then $r^{k+1} > \ell$, contrary to the choice of $\ell$ as an upper bound for~$R$.

Finally, suppose that $r<-1$.  If $(r^n)$ converges then its range is bounded.  In particular,
$\{r^{2n}\colon n \in \N\}$ is bounded.  As in the preceding paragraph, this is impossible.
\end{proof}

\begin{prop} If $r > 0$, then $r^{1/n} \sto 1$ as $n \sto \infty$.
\end{prop}

\begin{proof} Problem.  \emph{Hint.}  Show that $\frac1n < r < n$ for some natural number $n$.
Then use example \ref{4exam2} and proposition~\ref{sand_seq}.  (You will also make use of a
standard arithmetic fact---one that arises in problem~\ref{proj_nth_roots}---that if $0 < a <
b$, then $a^{1/n} < b^{1/n}$ for every natural number $n$.)   \ns
\end{proof}

\begin{prob} Find $\D\lim_{n \sto \infty}\frac{5^n + 3n + 1}{7^n - n - 2}$.
\end{prob}











\section{SUBSEQUENCES}

As example \ref{exam_div_seq} shows, boundedness of a sequence of real numbers does not
suffice to guarantee convergence.  It is interesting to note, however, that although the
sequence $\bigl((-1)^n\bigr)_{n=1}^\infty$ does not converge, it does have subsequences that
converge.  The odd numbered terms form a constant sequence $(-1,-1,-1,\dots)$, which of course
converges.  The even terms converge to~$+1$.  It is an interesting, if not entirely obvious,
fact that every bounded sequence has a convergent subsequence.  This is a consequence of our
next proposition.

Before proving proposition \ref{m_subseq} we discuss the notion of \emph{subsequence}. The
basic idea here is simple enough. Let $a = (a_1, a_2, a_3, \dots )$ be a sequence of real
numbers, and suppose that we construct a new sequence $b$ by taking some but not necessarily
all of the terms of $a$ and listing them in the same order in which they appear in $a$. Then
we say that this new sequence $b$ is a \emph{subsequence} of $a$.  We might, for example,
choose every fifth member of $a$ thereby obtaining the subsequence $b = (b_1, b_2, b_3, \dots
) = (a_5, a_{10}, a_{15}, \dots )$.  The following definition formalizes this idea.

\begin{defn}\label{subseq_r} Let $(a_n)$ be a sequence of real numbers. If
$\left(n_k\right)_{k=1}^\infty$ is a strictly increasing sequence in $\N$, then the sequence
 \index{<@$\left(a_{n_k}\right)$ (subsequence of a sequence)}%
   \[\left(a_{n_k}\right) = \left(a_{n_k}\right)_{k=1}^\infty =
                       \left(a_{n_1}, a_{n_2}, a_{n_3}, \dots \right)\]
is a
 \index{subsequence}%
\df{subsequence} of the sequence~$(a_n)$.
\end{defn}

\begin{exam} If $a_k = 2^{-k}$ and $b_k = 4^{-k}$ for all $k \in \N$, then $(b_k)$ is a subsequence
of $(a_k)$. Intuitively, this is clear, since the second sequence $(\frac14, \frac1{16},
\frac1{64}, \dots )$ just picks out the even-numbered terms of the first sequence $(\frac12,
\frac14, \frac18, \frac1{16}, \dots )$.  This ``picking out'' is implemented by the strictly
increasing function $n(k) = 2\,k$ (for $k \in \N$). Thus $b = a \circ n$ since
   \[a_{n_k} = a(n(k)) = a(2k) = 2^{-2k} = 4^{-k} = b_k\]
for all $k$ in~$\N$.
\end{exam}

\begin{prop}\label{m_subseq} Every sequence of real numbers has a monotone subsequence.
\end{prop}

\begin{proof} Exercise. \emph{Hint.}  A definition may be helpful. Say that a term $a_m$ of a
sequence in $\R$ is a
 \index{peak term}%
 \index{term!peak}%
\df{peak term} if it is greater than or equal to every succeeding term (that is, if $a_m \ge
a_{m+k}$ for all $k \in \N$). There are only two possibilities: either there is a subsequence
of the sequence $(a_n)$ consisting of peak terms, or else there is a last peak term in the
sequence. (Include in this second case the situation in which there are no peak terms.) Show
in each case how to select a monotone subsequence of $(a_n)$.  (Solution~\ref{sol_m_subseq}.)
\ns
\end{proof}

\begin{cor}\label{bdd_convss} Every bounded sequence of real numbers has a convergent subsequence.
\end{cor}

\begin{proof} This is an immediate consequence of propositions \ref{bdd_monotone} and \ref{m_subseq}.
\end{proof}

Our immediate purpose in studying sequences is to facilitate our investigation of the topology
of the set of real numbers.  We will first prove a key result \ref{Bolzano_thm}, usually known
as \emph{Bolzano's theorem}, which tells us that bounded infinite subsets of $\R$ have
accumulation points.  We then proceed to make available for future work sequential
characterizations of open sets \ref{sc_open}, closed sets \ref{sc_closed}, and continuity of
functions~\ref{sc_cont}.

\begin{defn}\label{nested} A sequence $(A_1, A_2, A_3, \dots)$ of sets is
 \index{nested sequence of sets}%
\df{nested} if $A_{k+1} \subseteq A_k$ for every~$k$.
\end{defn}

\begin{prop}\label{inter_nested} The intersection of a nested sequence of nonempty closed
bounded intervals whose lengths approach $0$ contains exactly one point.
\end{prop}

\begin{proof} Problem.  \emph{Hint.}  Suppose that $J_n = [a_n, b_n] \ne \emptyset$ for each
$n \in \N$, that $J_{n+1} \subseteq J_n$ for each $n$, and that $b_n - a_n \sto 0$ as $n \sto
\infty$. Show that $\cap_{n=1}^\infty J_n = \{c\}$ for some $c \in \R$. \ns
\end{proof}

\begin{prob} Show by example that the intersection of a nested sequence of nonempty closed intervals
may be empty.
\end{prob}

\begin{prob} Show by example that proposition \ref{inter_nested} no longer holds if ``closed'' is
replaced by ``open''.
\end{prob}

\begin{prop}[Bolzano's theorem]\label{Bolzano_thm} Every bounded infinite subset of $\R$ has at
least one accumulation point.
\end{prop}

\begin{proof} Let $A$ be a bounded infinite subset of $\R$. Since it is bounded it is contained in
some interval $J_0 = [a_0,b_0]$.  Let $c_0$ be the midpoint of $J_0$. Then at least one of the
intervals $[a_0,c_0]$ or $[c_0,b_0]$ contains infinitely many members of $A$ (see
\ref{prob_union_fin}).  Choose one that does and call it $J_1$. Now divide $J_1$ in half and
choose $J_2$ to be one of the resulting closed subintervals whose intersection with $A$ is
infinite.  Proceed inductively to obtain a nested sequence of closed intervals
   \[ J_0 \supseteq J_1 \supseteq J_2 \supseteq J_3 \supseteq \dots \]
each one of which contains infinitely many points of $A$. By proposition \ref{inter_nested}
the intersection of all the intervals $J_k$ consists of exactly one point $c$. Every open
interval about $c$ contains some interval $J_k$ and hence infinitely many points of $A$. Thus
$c$ is an accumulation point of~$A$.
\end{proof}

\begin{prop}\label{sc_open} A subset $U$ of $\R$ is open if and only if every sequence which
converges to an element of $U$ is eventually in $U$.
\end{prop}

\begin{proof} Suppose $U$ is open in $\R$.  Let $(x_n)$ be a sequence which converges to a point
$a$ in~$U$.  Since $U$ is a neighborhood of $a$, $(x_n)$ is eventually in $U$
by~\ref{cond_conv_seq}(b).

Conversely, suppose that $U$ is not open.  Then there is at least one point $a$ of $U$ which
is not an interior point of $U$.  Then for each $n \in \N$ there is a point $x_n$ which
belongs to $J_{1/n}(a)$ but \emph{not} to $U$.  Then the sequence $(x_n)$ converges to $a$ but
no term of the sequence belongs to~$U$.
\end{proof}

\begin{prop}\label{sc_closure} A point $b$ belongs to the closure of a set $A$ in $\R$ if and
only if there exists a sequence $(a_n)$ in $A$ which converges to~$b$.
\end{prop}

\begin{proof} Exercise. (Solution~\ref{sol_sc_closure}.)   \ns   \end{proof}

\begin{cor}\label{sc_closed} A subset $A$ of $\R$ is closed if and only if $b$ belongs to $A$
whenever there is a sequence $(a_n)$ in $A$ which converges to~$b$.
\end{cor}

\begin{prop}\label{sc_cont} Let $A \subseteq \R$. A function $f \colon A \sto \R$ is continuous
at $a$ if and only if $f(x_n) \sto f(a)$ whenever $(x_n)$ is a sequence in $A$ such that $x_n
\sto a$.
\end{prop}

\begin{proof} Problem.  \ns  \end{proof}

\begin{prob} Discuss in detail the continuity of algebraic combinations of continuous real valued
functions defined on subsets of $\R$.  Show, for example, that if functions $f$ and $g$ are
continuous at a point $a$ in $\R$, then such combinations as $f+g$ and $fg$ are also
continuous at $a$.  What can you say about the continuity of polynomials?  \emph{Hint.}  Use
problem \ref{seq_proj} and proposition~\ref{sc_cont}.
\end{prob}

We conclude this chapter with some problems.  The last six of these concern the convergence of
recursively defined sequences.  Most of these are pretty much like example \ref{exam_recur}
and require more in the way of patience than ingenuity.

\begin{prob} If $A$ is a nonempty subset of $\R$ which is bounded above, then there exists an
increasing sequence of elements of $A$ which converges to $\sup A$.  Similarly, if $A$ is
nonempty and bounded below, then there is a decreasing sequence in $A$ converging to~$\inf A$.
\end{prob}

\begin{prob}[Geometric Series] Let $\abs r < 1$ and $a \in \R$. For each $n \ge 0$ let
$s_n = \sum_{k=0}^n ar^k$.
 \begin{enumerate}
  \item[(a)] Show that the sequence $(s_n)_{n=0}^\infty$ converges. \emph{Hint.}  Consider $s_n - rs_n$.
  \item[(b)] The limit found in part (a) is usually denoted by $\sum_{k=0}^\infty ar^k$. (This is the
 \index{sum!of a geometric series}%
\df{sum} of a geometric series.)  Use (a) to find $\sum_{k=0}^\infty 2^{-k}$.
  \item[(c)] Show how (a) may be used to write the decimal $0.152424\overline{24}\dots$ as the
quotient of two natural numbers.
 \end{enumerate}
\end{prob}

\begin{exer}\label{4exer1} Suppose a sequence $(x_n)$ of real numbers satisfies
   \[ 4 x_{n+1} = {x_n}^3 \]
for all $n \in \N$.  For what values of $x_1$ does the sequence $(x_n)$ converge? For each
such $x_1$ what is $\lim_{n \sto \infty} x_n$? \emph{Hint.}  First establish that \emph{if}
the sequence $(x_n)$ converges, its limit must be $-2 ,0 ,\text{ or } 2$. This suggests
looking at several special cases: $x_1 < -2$, $x_1 = -2$, $-2 < x_1 < 0$, $x_1 = 0$, $0 < x_1
< 2$, $x_1 = 2$, and $x_1 > 2$. In case  $x_1 < -2$, for example, show that $x_n < -2$ for all
$n$. Use this to show that the sequence $(x_n)$ is decreasing and that it has no limit.  The
other cases can be treated in a similar fashion. (Solution~\ref{sol_4exer1}.)
\end{exer}

\begin{prob} Suppose a sequence $(x_n)$ in $\R$ satisfies
   \[ 5x_{n+1} = 3x_n + 4 \]
for all $n \in \N$. Show that $(x_n)$ converges. \emph{Hint.} First answer an easy question:
If $x_n \sto \ell$, what is $\ell$? Then look at three cases: $x_1 < \ell$, $x_1 = \ell$, and
$x_1 > \ell\,{}$. Show, for example, that if $x_1 < \ell$, then $(x_n)$ is bounded and
increasing.
\end{prob}

\begin{prob} Suppose a sequence $(x_n)$ in $\R$ satisfies
   \[ x_{n+1} = \sqrt{2 + x_n} \]
for all $n \in \N$. Show that $(x_n)$ converges. To what does it
converge?
\end{prob}

\begin{prob} Suppose that a sequence $(x_n)$ in $\R$ satisfies
    \[ x_{n+1} = 1 - \sqrt{1 - x_n} \]
for all $n \in \N$.  Show that $(x_n)$ converges. To what does it converge?  Does
$\left(\dfrac{x_{n+1}}{x_n}\right)$ converge?
\end{prob}

\begin{prob} Suppose that a sequence $(x_n)$ of real numbers satisfies
    \[ 3x_{n+1} = {x_n}^3 - 2 \]
for all $n \in \N$. For what choices of $x_1$  does the sequence converge?  To what?
\emph{Hint.}  Compute $x_{n+1} - x_n$.
\end{prob}

\begin{prob} Let $a > 1$. Suppose that a sequence $(x_n)$ in $\R$ satisfies
   \[ (a + x_n)x_{n+1} = a(1 + x_n) \]
for all $n$.  Show that if $x_1 > 0$, then $(x_n)$ converges. In this case find $\lim x_n$.
\end{prob}




\endinput
