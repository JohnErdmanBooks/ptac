\chapter{APPLICATIONS OF A FIXED POINT THEOREM}\label{fpt}

We now have enough information about metric spaces to consider some interesting applications.  We
will first prove a result known as the \emph{contractive mapping theorem} and then use it to find
solutions to systems of simultaneous linear equations and to certain integral equations.  Since
this chapter contains mostly examples, we will make liberal use of computations from beginning
calculus.  Although it would perhaps be more logical to defer these matters until we have developed
the necessary facts concerning integrals, derivatives, and power series, there is nevertheless much
to be said for presenting some nontrivial applications relatively early.



\section{THE CONTRACTIVE MAPPING THEOREM}
\begin{defn} A mapping $f \colon M \sto N$ between metric spaces is
 \index{contraction mapping}%
\df{contractive} if there exists a constant $c$ such that $0 < c < 1$ and
  \[ d(f(x),f(y))  \le  c\, d(x,y) \]
for all $x$, $y \in M$.  Such a number $c$ is a
 \index{contraction!constant}%
\df{contraction constant} for~$f$.  A contractive map is also
called a
 \index{contraction}%
\df{contraction}.
\end{defn}

\begin{exer}\label{cntr_cnt} Show that every contractive map is continuous.
(Solution~\ref{sol_cntr_cnt}.) \ns
\end{exer}

\begin{exam}\label{cntr_exm1} The map $f \colon \R^2 \sto \R^3$ defined by
  \[ f(x,y) = \bigl(1 - \tfrac13x, 1 +\tfrac13y, 2 + \tfrac13x - \tfrac13y\bigr) \]
is a contraction.
\end{exam}

\begin{proof} Exercise. (Solution~\ref{sol_cntr_exml}.)   \ns  \end{proof}

\begin{exam} The map
  \[ f \colon \R^2 \sto \R^2:(x,y) \mapsto \bigl(\tfrac12(1 + y), \tfrac12(3 - x)\bigr) \]
is a contraction on $\R^2$, where $\R^2$ has its usual (Euclidean) metric.
\end{exam}

\begin{proof} Problem.  \ns   \end{proof}

The next theorem is the basis for a number of interesting applications.  Also it will turn out to
be a crucial ingredient of the extremely important \emph{inverse function theorem} (in chapter
\ref{IFT}).  Although the statement of theorem \ref{cmthm} is important in applications, its proof
is even more so.  The theorem guarantees the existence (and uniqueness) of solutions to certain
kinds of equations; its proof allows us to approximate these solutions as closely as our
computational machinery permits. Recall from chapter \ref{ivt} that a
 \index{fixed point}%
\df{fixed point} of a mapping $f$ from a set $S$ into itself is a point $p \in S$ such that $f(p) =
p$.

 \index{contractive mapping theorem}%
\begin{thm}[Contractive Mapping Theorem]\label{cmthm} Every contraction from a complete metric
space into itself has a unique fixed point.
\end{thm}

\begin{proof}  Exercise. \emph{Hint.} Let $M$ be a complete metric space and $f \colon M \sto M$
be contractive.  Start with an arbitrary point $x_0$ in $M$.  Obtain a sequence
$(x_n)_{n=0}^\infty$ of points in $M$ by letting $x_1 = f(x_0)$, $x_2 = f(x_1)$, and so on. Show
that this sequence is Cauchy.  (Solution~\ref{sol_cmthm}.)
\end{proof}

\begin{exam}\label{fpt_exm1} We use the \emph{contractive mapping theorem} to solve the following
system of equations
  \begin{equation}\label{fpt_eq1}
                 \left\{   \begin{aligned}  9x - 2y  &=  7 \\
                                            3x + 8y  &=  11 \,.
                           \end{aligned}
                 \right.
  \end{equation}
Define
  \[ S \colon \R^2 \sto \R^2 \colon (x,y) \mapsto (9x - 2y, 3x + 8y)\,. \]
The system \eqref{fpt_eq1} may be written as a single equation
  \[ S(x,y) = (7, 11)\,, \]
or equivalently as
  \begin{equation}\label{fpt_eq2}
                (x,y) - S(x,y) + (7, 11) = (x, y)\,.
  \end{equation}
[\emph{Definition.}  Addition and subtraction on $\R^2$ are defined coordinatewise.  That is, if
$(x,y)$ and $(u,v)$ are points in $\R^2$, then $(x,y) + (u,v) := (x + u,y + v)$ and $(x,y) - (u,v)
:= (x - u,y - v)$.  Similar definitions hold for $\R^n$ with $n > 2$.] Let $T(x,y)$ be the left
hand side of \eqref{fpt_eq2}; that is, define
  \[ T \colon \R^2 \sto \R^2 \colon (x,y) \mapsto (x,y) - S(x,y) + (7,11)\,. \]
With this notation \eqref{fpt_eq2} becomes
  \[ T(x,y) = (x,y)\,. \]
Thus, and this is the crucial point, to solve \eqref{fpt_eq1} we need only find a fixed point of
the mapping~$T$.  If $T$ is contractive, then the preceding theorem guarantees that $T$ has a
unique fixed point and therefore that the system of equations \eqref{fpt_eq1} has a unique
solution.

Unfortunately, as things stand, $T$ is \emph{not} contractive with respect to the product metric on
$\R^2$.  (It is for convenience that we use the product metric $d_1$ on $\R^2$ rather than the
usual Euclidean metric.  Square roots are a nuisance.)  To see that $T$ is not contractive notice
that $d_1\bigl((1,0),(0,0)\bigr) = 1$ whereas $d_1\bigl(T(1,0),T(0,0)\bigr) =
d_1\bigl((-1,8),(7,11)\bigr) = 11$.  All is not lost however.  One simple-minded remedy is to
divide everything in \eqref{fpt_eq1} by a large constant~$c$.  A little experimentation shows that
$c = 10$ works.  Instead of working with the system \eqref{fpt_eq1} of equations, consider the
system
  \begin{equation}\label{fpt_eq4}
           \left\{     \begin{aligned}    0.9 x - 0.2 y = 0.7 \\
                                          0.3 x + 0.8 y = 1.1
                       \end{aligned}
           \right.
  \end{equation}
which obviously has the same solutions as~\eqref{fpt_eq1}. Redefine $S$ and $T$ in the obvious
fashion. Let
  \[ S \colon \R^2 \sto \R^2 \colon (x,y) \mapsto (0.9x - 0.2y, 0.3x + 0.8y) \]
and
  \[ T \colon \R^2 \sto \R^2 \colon (x,y) \mapsto (x,y) - S(x,y) + (0.7,1.1)\,. \]
Thus redefined, $T$ is contractive with respect to the product metric.  \emph{Proof:} Since
  \begin{equation}\label{fpt_eq5}
         T(x,y) = (0.1 x + 0.2 y + 0.7, -0.3 x + 0.2 y + 1.1)
  \end{equation}
for all $(x,y) \in \R^2$, we see that
  \begin{align}
    d_1\bigl(T(x,y),T(u,v)\bigr)
        &= 10^{-1}\bigl(\abs{(x+2y)-(u+2v)} + \abs{(-3x+2y)-(-3u+2v)}\bigr) \notag \\
        &\le 10^{-1}\bigl(\abs{x-u} + 2\abs{y-v} + 3\abs{x-u} + 2\abs{y-v}\bigr) \notag \\
        &= 0.4\,(\abs{x-u} + \abs{y-v}) \notag \\
        &= 0.4\,d_1\bigl((x,y),(u,v)\bigr) \label{fpt_eq6}
  \end{align}
for all points $(x,y)$ and $(u,v)$ in~$\R^2$.

Now since $T$ is contractive and $\R^2$ is complete (with respect to the product metric---see
\ref{prod_compl}), the \emph{contractive mapping theorem} (\ref{cmthm}) tells us that $T$ has a
unique fixed point.  But a fixed point of $T$ is a solution for the system \eqref{fpt_eq4} and
consequently for~\eqref{fpt_eq1}.

The construction used in the proof of \ref{cmthm} allows us to approximate the fixed point of $T$
to any desired degree of accuracy.  As in that proof choose $x_0$ to be any point whatever in
$\R^2$.  Then the points $x_0,x_1,x_2,\dots$ (where $x_n = T\bigl(x_{n-1}\bigr)$ for each $n$)
converge to the fixed point of~$T$.  This is a technique of
 \index{successive approximation}%
\df{successive approximation}.

For the present example let $x_0 = (0,0)$.  (The origin is chosen just for convenience.)  Now use
\eqref{fpt_eq5} and compute.
  \begin{align*}
           x_0 &= (0,0) \\
           x_1 &= T(0,0) = (0.7,1.1) \\
           x_2 &= T(x_1) = (1.021, 1.025) \\
           x_3 &= T(x_2) = (1.0071, 0.9987) \\
        \vdots &{}
  \end{align*}

It is reasonable to conjecture that the system \eqref{fpt_eq1} has a solution consisting of
rational numbers and then to guess that the points $x_0,x_1,x_2,\dots$ as computed above are
converging to the point $(1,1)$ in~$\R^2$.  Putting $x=1$ and $y=1$ in \eqref{fpt_eq5}, we see that
the point $(1,1)$ is indeed the fixed point of $T$ and therefore the solution to~\eqref{fpt_eq1}.
\end{exam}

In the preceding example we discovered an exact solution to a system of equations.  In general, of
course, we cannot hope that a successive approximation technique will yield exact answers.  In
those cases in which it does not, it is most important to have some idea how accurate our
approximations are.  After $n$ iterations, how close to the true solution are we?  How many
iterations must be computed in order to achieve a desired degree of accuracy?  The answer to these
questions in an easy consequence of the proof of theorem~\ref{cmthm}.

\begin{cor}\label{error1} Let the space $M$, the mapping $f$, the sequence $(x_n)$, the constant $c$,
and the point $p$ be as in theorem~\ref{cmthm} and its proof.  Then for every $m \ge 0$
  \[ d(x_m,p) \le d(x_0,x_1)\frac{c^m}{(1 - c)}\,. \]
\end{cor}

\begin{proof} Inequality \eqref{fpt_eq9} in the proof of \ref{cmthm} says that for $m < n$
  \[ d(x_m,x_n) \le d(x_0,x_1)c^m (1 - c)^{-1}\,. \]
Take limits as $n \sto \infty$.
\end{proof}

\begin{defn} Notation as in the preceding corollary.  If we think of the point $x_n$ as being the
$n^{\text{th}}$ approximation to~$p$, then the distance $d(x_n,p)$ between $x_n$ and $p$ is the
 \index{error}%
\df{error} associated with the $n^{\text{th}}$ approximation.
\end{defn}

Notice that because the product metric $d_1$ was chosen for $\R^2$ in example~\ref{fpt_exm1}, the
word ``error'' there means the \emph{sum} of the errors in $x$ and~$y$.  Had we wished for
``error'' to mean the \emph{maximum} of the errors in $x$ and $y$, we would have used the uniform
metric $d_u$ on~$\R^2$.  Similarly, if \emph{root-mean-square} ``error'' were desired (that is, the
square root of the sum of the squares of the errors in $x$ and $y$), then we would have used the
usual Euclidean metric on~$\R^2$.

\begin{exer}\label{fpt_exr1} Let $(x_n)$ be the sequence of points in $\R^2$ considered in
example~\ref{fpt_exm1}.  We showed that $(x_n)$ converges to the point $p = (1,1)$.
 \begin{enumerate}
  \item[(a)] Use corollary~\ref{error1} to find an upper bound for the error associated with the
approximation~$x_4$.
  \item[(b)] What is the actual error associated with $x_4$?
  \item[(c)] According to \ref{error1} how many terms of the sequence $(x_n)$ should we compute to
be sure of obtaining an approximation which is correct to within~$10^{-4}$?
 \end{enumerate}
(Solution~\ref{sol_fpt_exr1}.)
\end{exer}

\begin{prob} Show by example that the conclusion of the \emph{contractive mapping theorem} fails if:
 \begin{enumerate}
  \item[(a)] the contraction constant is allowed to have the value 1; or
  \item[(b)] the space is not complete.
 \end{enumerate}
\end{prob}

\begin{prob} Show that the map
  \[ g \colon [0, \infty) \sto [0, \infty) \colon x \mapsto \frac 1{x+1} \]
is not a contraction even though
  \[ d\bigl(g(x),g(y)\bigr) < d(x,y) \]
for all $x$, $y \ge 0$ with $x \ne y$.
\end{prob}

\begin{prob} Let $f(x) = (x/2) + (1/x)$ for $x \ge 1$.
 \begin{enumerate}
  \item[(a)] Show that $f$ maps the interval $[1,\infty)$ into itself.
  \item[(b)] Show that f is contractive.
  \item[(c)] Let $x_0 = 1$ and for $n \ge 0$ let
   \[ x_{n+1} = \frac{x_n}2 + \frac1{x_n}\,. \]
Show that the sequence $(x_n)$ converges.
  \item[(d)] Find $\lim_{n \sto \infty}x_n$.
  \item[(e)] Show that the distance between $x_n$ and the limit found in (d) is no greater
than~$2^{-n}$.
 \end{enumerate}
\end{prob}

\begin{prob} Solve the system of equations
  \begin{alignat*}{4}
         9&x  &   -&y  &  + 2&z &&= 37 \\
          &x  & +10&y  &  - 3&z &&= -69 \\
        -2&x  &  +3&y  &  +11&z &&= 58
  \end{alignat*}
following the procedure of example~\ref{fpt_exm1}.  \emph{Hint.} As in \ref{fpt_exm1} divide by
$10$ to obtain a contractive mapping. Before guessing at a rational solution, compute $10$ or $11$
successive approximations.  Since this involves a lot of arithmetic, it will be helpful to have
some computational assistance---a programmable calculator, for example.
\end{prob}

\begin{prob} Consider the following system of equations.
  \begin{alignat*}{4}
         75&x  & +16&y  &   -20&z &&= 40 \\
         33&x  & +80&y  &   +30&z &&= -48 \\
        -27&x  & +32&y  &   +80&z &&= 36
  \end{alignat*}
 \begin{enumerate}
  \item[(a)] Solve the system following the method of example \ref{fpt_exm1}.  \emph{Hint.} Because
the contraction constant is close to~1, the approximations converge slowly. It may take 20 or 30
iterations before it is clear what the exact (rational) solutions should be.  So as in the
preceding example, it will be desirable to use computational assistance.
  \item[(b)] Let $(x_n)$ be the sequence of approximations in $\R^3$ converging to the solution of
the system in~(a).  Use corollary \ref{error1} to find an upper bound for the error associated with
the approximation~$x_{25}$.
  \item[(c)] To 4 decimal places, what is the actual error associated with~$x_{25}$?
  \item[(d)] According to \ref{error1}, how many terms must be computed to be sure that the error
in our approximation is no greater than~$10^{-3}$?
 \end{enumerate}
\end{prob}

\begin{prob} Let $f:\R^2 \sto \R^2$ be the rotation of the plane about the point $(0,1)$ through an
angle of $\pi$ radians.  Let $g \colon \R^2 \sto \R^2$ be the map which takes the point $(x,y)$ to
the midpoint of the line segment connecting $(x,y)$ and~$(1,0)$.
 \begin{enumerate}
  \item[(a)] Prove that $g \circ f$ is a contraction on $\R^2$ (with its usual metric).
  \item[(b)] Find the unique fixed point of $g \circ f$.
  \item[(c)] Let $(x_0,y_0) = (0,1)$ and define (as in the proof of \ref{cmthm})
   \[ \bigl(x_{n+1},y_{n+1}\bigr) = (g \circ f)\bigl(x_n,y_n\bigr) \]
for all $n \ge 0$.  For each $n$ compute the exact Euclidean distance between $(x_n,y_n)$ and the
fixed point of~$g \circ f$.
 \end{enumerate}
\end{prob}






\section{APPLICATION TO INTEGRAL EQUATIONS}
\begin{exer}\label{fpt_inteq} Use theorem \ref{cmthm} to solve the integral equation
  \begin{equation}\label{fpt_eq7}
        f(x) = \tfrac13 x^3 + \int_0^x t^2 f(t)\,dt\,.
  \end{equation}
\emph{Hint.} We wish to find a continuous function $f$ which satisfies \eqref{fpt_eq7} for all $x
\in \R$.  Consider the mapping $T$ which takes each continuous function $f$ into the function $Tf$
whose value at $x$ is given by
  \[ Tf(x) = \tfrac13 x^3 + \int_0^x t^2 f(t)\,dt\,. \]
[It is important to keep in mind that $T$ acts on functions, not on numbers.  Thus $Tf(x)$ is to be
interpreted as $(T(f))(x)$ and not $T(f(x))$.]  In order to make use of theorem \ref{cmthm}, the
mapping $T$ must be contractive.  One way to achieve this is to restrict our attention to
continuous functions on the interval $[0,1]$ and use the uniform metric on $\fml C([0,1],\R)$.
Once a continuous function $f$ is found such that \eqref{fpt_eq7} is satisfied for all $x$ in
$[0,1]$, it is a simple matter to check whether \eqref{fpt_eq7} holds for all $x$ in $\R$.
Consider then the map
  \[ T \colon \fml C([0,1],\R) \sto \fml C([0,1],\R) \colon f \mapsto Tf \]
where
  \[ Tf(x) := \tfrac13 x^3 + \int_0^x t^2 f(t)\,dt \]
for all $x \in [0,1]$.  The space $\fml C([0,1],\R)$ is complete. (Why?)  Show that $T$ is
contractive by estimating $\abs{Tf(x) - Tg(x)}$, where $f$ and $g$ are continuous functions on
$[0,1]$, and taking the supremum over all $x$ in $[0,1]$.  What can be concluded from theorem
\ref{cmthm} about \eqref{fpt_eq7}?

To actually find a solution to \eqref{fpt_eq7}, use the \emph{proof} of \ref{cmthm} (that
is, successive approximations). For simplicity start with the zero function in $\fml
C([0,1],\R)$: let $g_0(x) = 0$ for $0 \le x \le 1$.  For $n \ge 0$ let $g_{n+1}(x) =
Tg_n(x)$ for $0 \le x \le 1$. Compute $g_1$, $g_2$, $g_3$, and~$g_4$.  You should be able
to guess what $g_n$ will be. (It is easy to verify the correctness of your guess by
induction, but it is not necessary to do this.) Next, let $f$ be the function which is
the uniform limit of the sequence~$(g_n)$.  That is, $f$ is the function whose power
series expansion has $g_n$ as its $n^{\text{th}}$ partial sum.  This power series
expansion should be one with which you are familiar from beginning calculus; what
elementary function does it represent?

Finally, show by direct computation that this elementary function does in fact satisfy
\eqref{fpt_eq7} for all $x$ in $\R$.   (Solution~\ref{sol_fpt_inteq}.)
\end{exer}

\begin{prob} Give a careful proof that there exists a unique continuous real valued
function $f$ on $[0,1]$ which satisfies the integral equation
  \[ f(x) =  x^2 + \int_0^x t^2 f(t)\,dt\,. \]
(You are not asked to \emph{find} the solution.)
\end{prob}

\begin{prob} Use theorem \ref{cmthm} to solve the integral equation
  \[ f(x) =  x + \int_0^x f(t)\,dt\,. \]
\emph{Hint.} Follow the procedure of exercise~\ref{fpt_inteq}. Keep in mind that the only
reason for choosing the particular interval $[0,1]$ in \ref{fpt_inteq} was to make the
map $T$ contractive.
\end{prob}

\begin{prob} For every $f$ in $\fml C([0,\pi/4],\R)$  define
  \[ Tf(x) =  x^2 - 2 - \int_0^x f(t)\,dt \]
where $0 \le x \le \pi/4$.  Show that $T$ is a contraction. Find the fixed point of~$T$.
What integral equation have you solved?
\end{prob}




\endinput
