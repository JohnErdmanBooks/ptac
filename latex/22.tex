\chapter{NORMS}\label{nls}
\section{NORMS ON LINEAR SPACES}  The last two chapters have been pure algebra.  In order to deal
with topics in analysis (\emph{e.g.\ }differentiation, integration, infinite series) we need
also a notion of convergence; that is, we need topology as well as algebra. As in earlier
chapters we consider only those topologies generated by metrics, in fact only those which
arise from norms on vector spaces. Norms, which we introduce in this chapter, are very natural
objects; in many concrete situations they abound.  Furthermore, they possess one extremely
pleasant property: a norm on a vector space generates a metric on the space, and this metric
is compatible with the algebraic structure in the sense that it makes the vector space
operations of addition and scalar multiplication continuous.  Just as metric is a
generalization of ordinary Euclidean distance, the concept of norm generalizes on vector
spaces the idea of length.

\begin{defn}  Let $V$ be a vector space. A function
 \index{<@$\norm{x}$ (norm of a vector)}%
$\norm{\quad}\colon V \sto \R\colon x \mapsto \norm{x}$ is a
 \index{norm}%
\df{norm} on $V$ if
 \begin{enumerate}
  \item $\norm{x + y} \le \norm{x} + \norm{y}$ \qquad  for all $x$, $y \in V$,
  \item $\norm{\alpha x} = \abs{\alpha}\,\norm{x}$\qquad for all $x \in V$ and $\alpha \in \R$, and
  \item If $\norm{x} = 0$, then $x = \vc 0$.
 \end{enumerate}
The expression $\norm{x}$ may be read as ``the norm of $x$'' or ``the
 \index{length}%
\df{length} of $x$''.  A vector space on which a norm has been defined is a
 \index{normed!linear space}%
 \index{space!normed linear}%
\df{normed linear space} (or \df{normed vector space}). A vector in a normed linear space
which has norm $1$ is a
 \index{unit vector}%
 \index{vector!unit}%
\df{unit vector}.
\end{defn}

\begin{exam}  The absolute value function is a norm on~$\R$.
\end{exam}

\begin{exam}  For $x = (x_1,\dots, x_n) \in \R^n$ let $\norm{x} =
\left(\sum_{k=1}^n {x_k}^2\right)^{1/2}$.  The only nonobvious part of the proof that this
defines a norm on $\R^n$ is the verification of the
 \index{triangle inequality}%
 \index{inequality!triangle}%
\df{triangle inequality} (that is, condition (1) in the preceding definition).  But we have
already done this: it is just Minkowski's inequality~\ref{Minkow}.  This is the
 \index{usual!norm!on $\R^n$}%
 \index{norm!usual!on $\R^n$}%
\df{usual norm} (or
 \index{Euclidean!norm}%
 \index{norm!Euclidean}%
\df{Euclidean norm}) on $\R^n$; unless the contrary is explicitly stated, $\R^n$ when regarded
as a normed linear space will always be assumed to possess this norm.
\end{exam}

\begin{exam}  For $x = (x_1,\dots,x_n) \in \R^n$ let
 \index{<@$\norm{x}_1$ (the $1$-norm on~$\R^n$)}%
$\norm{x}_1 = \sum_{k=1}^n \abs{x_k}$.  The function $x \mapsto \norm{x}_1$ is easily seen to
be a norm on~$\R^n$. It is sometimes called
 \index{norm!$1$- (on $\R^n$)}%
 \index{$1$-norm}%
the \df{$1$-norm} on~$\R^n$.
\end{exam}

\begin{exam}\label{exam_un_Rn}  For $x = (x_1,\dots,x_n) \in \R^n$ let
 \index{<@$\norm{x}_u$ (uniform norm on~$\R^n$)}%
$\norm{x}_u = \max\{\abs{x_k} \colon 1 \le k \le n\}$.  Again it is easy to see that this
defines a norm on $\R^n$; it is the
 \index{uniform!norm!on~$\R^n$}%
 \index{norm!uniform!on $\R^n$}%
\df{uniform norm} on~$\R^n$.
\end{exam}

\begin{exer}\label{exer_nls_r34}  Let $f \colon \R^3 \sto \R^4 \colon (x,y,z) \mapsto
(xz, x^2 + 3y,-x + y^2 - 3z,xyz-\sqrt 2\,x).$ Find $\norm{f(a + \lambda h)}$ when $a =
(4,2,-4)$, $h = (2,4,-4)$, and $\lambda = -1/2$.  (Solution~\ref{sol_exer_nls_r34}.)

\end{exer}

\begin{exer}\label{exer_nls_r32}  Let $f \colon  \R^3 \sto \R^2: x \mapsto
(3{x_1}^2, x_1x_2 -x_3)$ and let $m = \begin{bmatrix}
                                               6 & 0 &  0 \\
                                               0 & 1 & -1
                                      \end{bmatrix}$.  Find $\norm{f(a + h) - f(a) - mh}$ when
$a = (1, 0, -2)$ and $h$ is an arbitrary vector in~$\R^3$. (Solution~\ref{sol_exer_nls_r32}.)

\end{exer}

\begin{prob} Let $f \colon \R^3 \sto \R^3$ be defined by
  \[ f(x,y,z) = (xy^3 + yz^2, x\sin(3\pi y), 2z)\,. \]
Find $\norm{f(a)}$ when $a = (16,1/2,2)$.
\end{prob}

\begin{exam}\label{exam_un_set}  Let $S$ be a nonempty set.  For $f$ in $\fml B(S,\R)$ let
  \[ \norm f_u := \sup\{\abs{f(x)} \colon x \in S\}\,. \]
This is the
 \index{<@$\norm{f}_u$ (uniform norm on~$\fml B(S,\R)$)}%
 \index{uniform!norm!on~$\fml B(S,\R)$}%
 \index{norm!uniform!on~$\fml B(S,\R)$}%
\df{uniform norm} on $\fml B(S,\R)$. Notice that example~\ref{exam_un_Rn} is a special case of
this one.  [An $n$-tuple may be regarded as a function on the set $\{1,\dots,n\}$. Thus $\R^n
= \fml B(S,\R)$ where $S = \{1,\dots, n\}$.]
\end{exam}

\begin{exer}\label{exer_un_sin}  Define $f$, $g \colon [0, 2\pi] \sto \R$ by $f(x) = \sin x$ and
$g(x) = \cos x$. Find $\norm{f + g}_u$.   (Solution~\ref{sol_exer_un_sin}.)
\end{exer}

\begin{prob} Let $f(x) = x + x^2 - x^3$ for $0 \le x \le 3$.  Find $\norm f_u$.
\end{prob}

The following proposition lists some almost obvious properties of norms.

\begin{prop}\label{prop_norm_obv} If $V$ is a normed linear space, then
 \begin{enumerate}
  \item[(a)] $\norm{\vc 0}  =  0$;
  \item[(b)] $\norm{-x} = \norm{x}$ for all $x \in V$; and
  \item[(c)] $\norm{x} \ge 0$ for all $x \in V$.
 \end{enumerate}
\end{prop}

\begin{proof}  Exercise. \emph{Hint.} For part (a) use proposition~\ref{vs_exer2}.  For part (b) use
proposition~\ref{vs_prob1}; and for (c) use (a) and (b) together with the fact that $x + (-x)
= \vc 0$.  (Solution~\ref{sol_prop_norm_obv}.) \ns
\end{proof}






\section{NORMS INDUCE METRICS}
We now introduce a crucial fact: \emph{every normed linear space is a metric space.}  That is,
the norm on a normed linear space induces a
 \index{metric!induced by a norm}%
metric $d$ defined by $d(x,y) = \norm{x - y}$.  The distance between two vectors is the length
of their difference.
 \[ \xy
%      \Atriangle/<-`<-`>/[\phantom{|}`\phantom{o}`\phantom{o};x`x-y`y]
      \Atriangle/<-`<-`>/[``;x`x-y`y]
    \endxy \]
If no other metric is specified we always regard a normed linear space as a metric space under
this induced metric. Thus the concepts of compactness, open sets, continuity, completeness,
and so on, make sense on any normed linear space.

\begin{prop}  Let $V$ be a normed linear space.  Define $d\colon V \times V \sto \R$ by $d(x,y) =
\norm{x - y}$.  Then $d$ is a metric on~$V$.
\end{prop}

\begin{proof} Problem.  \ns \end{proof}


The existence of a metric on a normed linear space $V$ makes it possible to speak of
neighborhoods of points in $V$. These neighborhoods satisfy some simple algebraic properties.

\begin{prop}\label{prop_ind_metr} Let $V$ be a normed linear space, $x \in V$, and $r$, $s~>~0$. Then
 \begin{enumerate}
  \item[(a)] $B_r(0) = -B_r(0)$,
  \item[(b)] $B_{rs}(0) = r\,B_s(0)$,
  \item[(c)] $x + B_r(0) = B_r(x)$, and
  \item[(d)] $B_r(0) + B_r(0) = 2B_r(0)$.
 \end{enumerate}
\end{prop}

\begin{proof} Part (a) is an exercise. (Solution~\ref{sol_prop_ind_metr}.)
Parts (b), (c), and (d) are problems. \emph{Hint.}  For (d) divide the proof into two parts:
$B_r(0) + B_r(0) \subseteq 2B_r(0)$ and the reverse inclusion. For the first, suppose x
belongs to $B_r(0) + B_r(0)$. Then there exist $u$, $v \in B_r(0)$ such that $x = u + v$. Show
that $x = 2w$ for some $w$ in $B_r(0)$. You may wish to use problem~\ref{prob_ball_conv}.) \ns
\end{proof}

\begin{prop}\label{prop_norm_cont}  If $V$ is a normed linear space then the following hold.
 \begin{enumerate}
  \item[(a)] $\bigl|\,\norm x - \norm y \,\bigr| \le \norm{x - y}$ for all $x$, $y \in V$.
  \item[(b)] The norm $x \mapsto \norm x$ is a continuous function on~$V$.
  \item[(c)] If $x_n \sto a$ in $V$, then $\norm{x_n} \sto \norm a$.
  \item[(d)] $x_n \sto 0$ in $V$ if and only if $\norm{x_n} \sto 0$ in $\R$.
 \end{enumerate}
\end{prop}

\begin{proof} Problem.  \ns  \end{proof}

\begin{prob} Give an example to show that the converse of part (c) of
proposition~\ref{prop_norm_cont} does not hold.
\end{prob}

\begin{prob}\label{prob_ball_conv}  Prove that in a normed linear space every open ball is a convex
set.  And so is every closed ball.
\end{prob}










\section{PRODUCTS}\label{sec_prod_nls}
In this and the succeeding three sections we substantially increase our store of examples of
normed linear spaces by creating new spaces from old ones.  In particular, we will show that
each of the following can be made into a normed linear space:
 \begin{enumerate}
  \item[(i)] a vector subspace of a normed linear space,
  \item[(ii)] the product of two normed linear spaces,
  \item[(iii)] the set of bounded functions from a nonempty set into
a normed linear space, and
%\item[(iv)] the quotient of a normed linear space by a closed
%subspace, and
  \item[(iv)] the set of continuous linear maps between two normed
linear spaces.
 \end{enumerate}

It is obvious that (i) is a normed linear space: if $V$ is a normed linear space with norm
$\norm\quad$ and $W$ is a vector subspace of $V$, then the restriction of $\norm\quad$ to $W$
is a norm on~$W$. Now consider (ii).  Given normed linear spaces $V$ and $W$, we wish to make
the product vector space (see example~\ref{prod_vs}) into a normed linear space.  As a
preliminary we discuss \emph{equivalent norms}.

\begin{defn}  Two norms on a vector space are
 \index{equivalent!norms}%
 \index{norms!equivalent}%
\df{equivalent} if they induce equivalent metrics.  If two norms on a vector space $V$ are
equivalent, then, since they induce equivalent metrics, they induce identical topologies
on~$V$.  Thus properties such as continuity, compactness, and connectedness are unaltered when
norms are replaced by equivalent ones (see proposition~\ref{mtr_vs_top} and the discussion
preceding it).  In the next proposition we give a very simple necessary and sufficient
condition for two norms to be equivalent.
\end{defn}

\begin{prop}\label{prop_norm_equiv}  Two norms $\norm{\quad}_1$ and $\norm{\quad}_2$ on a vector
space $V$ are equivalent if and only if there exist numbers $\alpha$, $\beta > 0$ such that
  \[ \norm x_1 \le \alpha \norm x_2 \qquad\text{and}\qquad \norm x_2 \le \beta \norm x_1 \]
for all $x \in V$.
\end{prop}

\begin{proof}  Exercise. (Solution~\ref{sol_prop_norm_equiv}.)  \ns  \end{proof}

If $V$ and $W$ are normed linear spaces with norms ${\norm{\quad}}_{\ssst V}$ and
${\norm{\quad}}_{\ssst W}$ respectively, how do we provide the product vector space $V \times
W$ (see example~\ref{prod_vs}) with a norm?  There are at least three more or less obvious
candidates: for $v \in V$ and $w \in W$ let
 \begin{align*}
       \norm{(v,w)} &:= \bigl({{\norm v}_{\ssst V}}^2 + {{\norm w}_{\ssst W}}^2\bigr)^{1/2}, \\
   {\norm{(v,w)}}_1 &:= {\norm v}_{\ssst V} + {\norm w}_{\ssst W}, \qquad\text{and} \\
   {\norm{(v,w)}}_u &:= \max\{{\norm{v}}_{\ssst V}, {\norm w}_{\ssst W}\}.
 \end{align*}
First of all, are these really norms on $V \times W$?  Routine computations show that the
answer is \emph{yes}.  By way of illustration we write out the three verifications required
for the first of the candidate norms.  If $v$, $x \in V$, if $w$, $y \in W$, and if $\alpha
\in \R$, then
 \begin{align*}
     \text{(a)}\qquad \norm{(v,w) + (x,y)}
                  &= \norm{(v + x, w + y)} \\
                  &= \bigl({{\norm{v + x}}_{\ssst V}}^2 + {{\norm{w + y}}_{\ssst W}}^2\bigr)^{1/2} \\
                  &\le \bigl(({\norm v}_{\ssst V} + {\norm x}_{\ssst V})^2
                       + ({\norm w}_{\ssst W} + {\norm y}_{\ssst W})^2\bigr)^{1/2} \\
                  &\le \bigl({{\norm v}_{\ssst V}}^2 + {{\norm w}_{\ssst W}}^2\bigr)^{1/2}
                       + \bigl({{\norm x}_{\ssst V}}^2 + {{\norm y}_{\ssst W}}^2\bigr)^{1/2} \\
                  &= \norm{(v,w)} + \norm{(x,y)}.
 \end{align*}
The last inequality in this computation is, of course, Minkowski's inequality~\ref{Minkow}.
 \begin{align*}
   \text{(b)}\qquad \norm{\alpha(v,w)}
              &= \norm{(\alpha v, \alpha w)} \\
              &= \bigl({{\norm{\alpha v}}_{\ssst V}}^2
                           + {{\norm{\alpha w}}_{\ssst W}}^2\bigr)^{1/2} \\
              &= \left(\bigl(\abs{\alpha}\, {\norm v}_{\ssst V}\bigr)^2
                           + \bigl(\abs{\alpha}\, {\norm w}_{\ssst W}\bigr)^2\right)^{1/2} \\
              &= \abs{\alpha}\,\bigl({{\norm v}_{\ssst V}}^2
                           + {{\norm w}_{\ssst W}}^2\bigr)^{1/2} \\
              &= \abs{\alpha}\,\norm{(v,w)}.
 \end{align*}

(c) If $\norm{(v,w)} = 0$, then ${{\norm v}_{\ssst V}}^2 + {{\norm w}_{\ssst W}}^2 = 0$. This
implies that ${\norm v}_{\ssst V}$ and ${\norm w}_{\ssst W}$ are both zero. Thus $v$ is the
zero vector in $V$ and $w$ is the zero vector in $W$; so $(v,w) = (0,0)$, the zero vector in
$V \times W$.

Now which of these norms should we choose to be the product norm on $V \times W$?  The next
proposition shows that at least as far as topological considerations (continuity, compactness,
connectedness, \emph{etc.}) are concerned, it really doesn't matter.

\begin{prop}  The three norms on $V \times W$ defined above are equivalent.
\end{prop}

\begin{proof} Notice that the norms $\norm{\quad}$, ${\norm{\quad}}_1$, and ${\norm{\quad}}_u$
defined above induce, respectively, the metrics $d$, $d_1$, and $d_u$ defined in
chapter~\ref{metric}.  In proposition~\ref{3equiv} we proved that these three metrics are
equivalent. Thus the norms which induce them are equivalent.
\end{proof}


\begin{defn} Since $d_1$ was chosen (in~\ref{prod_met}) as our ``official'' product metric, we
choose ${\norm{\quad}}_1$, which induces $d_1$, as the
 \index{product!norm}%
 \index{norm!product}%
\df{product norm} on $V \times W$. In proposition~\ref{prop_add_cont} you are asked to show
that with this definition of the product norm, the operation of addition on a normed linear
space is continuous.  In the next proposition we verify that scalar multiplication (regarded
as a map from $\R \times V$ into $V$) is also continuous.
\end{defn}

\begin{prop}\label{prop_scmlt_cont}  If $V$ is a normed linear space, then the mapping
$(\beta,x) \mapsto \beta x$ from $\R \times V$ into $V$ is continuous.
\end{prop}

\begin{proof}  Exercise. \emph{Hint.}  To show that a map $f\colon U \sto W$ between two normed
linear space is continuous at a point $a$ in~$U$, it must be shown that for every $\epsilon >
0$ there exists $\delta > 0$ such that ${\norm{u - a}}_{\ssst U} < \delta$ implies
${\norm{f(u) - f(a)}}_{\ssst W} < \epsilon$.   (Solution~\ref{sol_prop_scmlt_cont}.)  \ns
\end{proof}

\begin{cor}\label{cor_scmlt}  Let $(\beta_n)$ be a sequence of real numbers and $(x_n)$ be a
sequence of vectors in a normed linear space~$V$.  If $\beta_n \sto \alpha$ in $\R$ and $x_n
\sto a$ in~$V$, then $\beta_n x_n \sto \alpha a$ in~$V$.
\end{cor}

\begin{proof} Exercise.  (Solution~\ref{sol_cor_scmlt}.) \ns  \end{proof}

\begin{cor} If $V$ is a normed linear space and $\alpha$ is a nonzero scalar, the map
  \[ M_\alpha \colon V \sto V\colon x \mapsto \alpha x \]
is a homeomorphism.
\end{cor}

\begin{proof} Problem.   \ns  \end{proof}

\begin{prop}\label{prop_add_cont}  Let $V$ be a normed linear space.  The operation of addition
  \[ A\colon V \times V \sto V\colon (x,y) \mapsto x + y \]
is continuous.
\end{prop}

\begin{proof} Problem.   \ns  \end{proof}

\begin{prob}\label{prob_clo_subsp}  Let $V$ be a normed linear space.  Prove the following:
 \begin{enumerate}
  \item[(a)] If $x_n \sto a$ and $y_n \sto b$ in $V$, then $x_n + y_n \sto a + b$.
  \item[(b)] If $S$ is a vector subspace of $V$, then so is $\clo S$.
 \end{enumerate}
\end{prob}

\begin{prob}\label{prob_sum_cpt} If $K$ and $L$ are compact subsets of a normed linear space,
then the set
  \[ K + L := \{k + l\colon k \in K \text{ and } l \in L\} \]
is compact. \emph{Hint.}  Let $A$ be as in proposition~\ref{prop_add_cont}.  What is $A^\sto(K
\times L)$?
\end{prob}

\begin{prob}\label{prob_sums_nls}  Let $B$ and $C$ be subsets of a normed linear space and
$\alpha \in \R$.  Prove the following:
 \begin{enumerate}
  \item[(a)] $\clo{\alpha B} = \alpha \clo B$.
  \item[(b)] $\clo B + \clo C \subseteq \clo{B + C}$.
  \item[(c)] $B + C$ need not be closed even if $B$ and $C$ are; thus equality need not hold
in~(b).  \emph{Hint.}  In $\R^2$ try part of the curve $y = 1/x$ and the negative $x$-axis.
 \end{enumerate}
\end{prob}


\begin{prob}\label{prob_lin_bij}  Show that a linear bijection $f \colon V \sto W$ between normed
linear spaces is an isometry if and only if it is
 \index{norm!preserving}%
norm preserving (that is, if and only if $\norm{f(x)}_{\ssst W} = \norm x_{\ssst V}$ for all
$x \in V$).
\end{prob}

\begin{defn}  Let $a$ be a vector in a vector space~$V$.  The map
  \[ T_a\colon V \sto V\colon x \mapsto x + a \]
is called
 \index{<@$T_a$ (translation by~$a$)}%
 \index{ta@$T_a$ (translation by~$a$)}%
 \index{translation}%
\df{translation} by~$a$.
\end{defn}

\begin{prob}\label{prob_transl} Show that every translation map on a normed linear space is an
isometry and therefore a homeomorphism.
\end{prob}

\begin{prob}  Let $U$ be a nonempty open set in a normed linear space.  Then $U - U$ contains a
neighborhood of~$0$.  (By $U - U$ we mean $\{u - v \colon u,v \in U\}$.) \emph{Hint.} Consider
the union of all sets of the form $\bigl(T_{-v}\bigr)^\sto (U)$ where $v \in U$. (As in
problem~\ref{prob_transl} $T_{-v}$ is a translation map.)
\end{prob}

\begin{prob}  Show that if $B$ is a closed subset of a normed linear space $V$ and $C$ is a
compact subset of $V$, then $B + C$ is closed. (Recall that part (c) of
problem~\ref{prob_sums_nls} showed that this conclusion cannot be reached by assuming only
that $B$ and $C$ are closed.) \emph{Hint.}  Use the sequential characterization of ``closed''
given in proposition~\ref{scm_closed}.  Let $(a_n)$ be a sequence in $B+C$ which converges to
a point in~$V$.  Write $a_n = b_n + c_n$ where $b_n \in B$ and $c_n \in C$.  Why does $(c_n)$
have a subsequence $\bigl(c_{n_k}\bigr)$ which converges to a point in $C$?  Does
$\bigl(b_{n_k}\bigr)$ converge?
\end{prob}

\begin{prob}  Let $V$ and $W$ be normed linear spaces, $A \subseteq V$, $a \in A'$,
$\alpha \in \R$, and $f,g\colon A \sto W$.  Prove the following:
 \begin{enumerate}
  \item[(a)]  If the limits of $f$ and $g$ exist as $x$ approaches $a$, then so does the limit
of $f + g$ and
    \[ \lim_{x \sto a}(f+g)(x) = \lim_{x \sto a}f(x) + \lim_{x \sto a}g(x)\,. \]
  \item[(b)]  If the limit of $f$ exists as $x$ approaches $a$, then so does the limit of
$\alpha f$ and
    \[ \lim_{x \sto a}(\alpha f)(x) = \alpha \lim_{x \sto a}f(x)\,. \]
 \end{enumerate}
\end{prob}

\begin{prob}  Let $V$ and $W$ be normed linear spaces, $A \subseteq V$, $a \in A'$, and
$f\colon A \sto W$.  Show that
 \begin{enumerate}
  \item[(a)] $\lim_{x \sto a}f(x) = 0 \text{ if and only if } \lim_{x \sto a} \norm{f(x)} = 0$; and
  \item[(b)] $\lim_{h \sto 0}f(a+h) = \lim_{x \sto a}f(x)$.
 \end{enumerate}
\emph{Hint.}  These require only the most trivial modifications of the solutions to
problem~\ref{abs_lim_R} and proposition~\ref{lim_changvar_R}.
\end{prob}

\begin{prob} Let $V$ and $W$ be normed linear spaces, $A \subseteq V$, and $f\colon A \sto W$.
Suppose that $a$ is an accumulation point of $A$ and that $l = \lim_{x \sto a}f(x)$ exists
in~$W$.
 \begin{enumerate}
  \item[(a)]  Show that if the norm on $V$ is replaced by an equivalent one, then $a$ is still
an accumulation point of~$A$.
  \item[(b)]  Show that if both the norm on $V$ and the one on $W$ are replaced by equivalent
ones, then it is still true that $f(x) \sto l$ as $x \sto a$.
 \end{enumerate}
\end{prob}

\begin{prob}  Let $f\colon U \times V \sto W$ where $U$, $V$, and $W$ are normed linear spaces.
If the limit
  \[ l := \lim_{(x,y) \sto (a,b)} f(x,y) \]
exists and if $\lim_{x \sto a}f(x,y)$ and $\lim_{y \sto b}f(x,y)$ exist for all $y \in V$ and
$x \in U$, respectively, then the iterated limits
  \[ \lim_{x \sto a}\bigl(\lim_{y \sto b} f(x,y)\bigr) \quad \text{and}
                  \quad \lim_{y \sto b}\bigl(\lim_{x \sto a} f(x,y)\bigr) \]
exist and are equal to~$l$.
\end{prob}

\begin{prob}\label{prob_norms_Rn}  All norms on $\R^n$ are equivalent.  \emph{Hint.}  It is
enough to show that an arbitrary norm $\trinorm{\hphantom{00}}$ on $R^n$ is equivalent to
$\norm{\quad}_1$ (where $\norm x_1 = \sum_{k=1}^n \abs{x_k}$). Use
proposition~\ref{prop_norm_equiv}.  To find $\alpha > 0$ such that $\trinorm x \le \alpha\norm
x_1$ for all $x$ write $x = \sum_{k=1}^n x_k e^k$ (where $e^1, \dots, e^n$ are the standard
basis vectors on $\R^n$).  To find $\beta > 0$ such that $\norm x_1 \le \beta \trinorm x$ let
$\R_1^n$ be the normed linear space $\R^n$ under the norm $\norm{\quad}_1$.  Show that the
function $x \mapsto \trinorm x$ from $\R_1^n$ into $\R$ is continuous. Show that the unit
sphere $S = \{x \in \R^n: \norm x_1 = 1\}$ is compact in~$\R_1^n$.
\end{prob}











\section{THE SPACE $\fml B(S,V)$}\label{space_bdd}
Throughout this section $S$ will be a nonempty set and $V$ a normed linear space. In the first
section of this chapter we listed $\fml B(S,\R)$ as an example of a normed linear space.  Here
we do little more than observe that the fundamental facts presented in chapter~\ref{unif_conv}
concerning pointwise and uniform convergence in the space $\fml B(S,\R)$ all remain true when
the set $\R$ is replaced by an arbitrary normed linear space.  It is very easy to generalize
these results: replace absolute values by norms.

\begin{defn}  Let $S$ be a set and $V$ be a normed linear space. A function $f \colon S \sto V$ is
 \index{bounded!vector valued function}%
\df{bounded} if there exists a number $M > 0$
such that
  \[ \norm{f(x)} \le M \]
for all $x$ in~$S$.  We denote by
 \index{bounded@$\fml B(S,V)$(bounded vector valued functions on a set)}%
$\fml B(S,V)$ the family of all bounded $V$ valued functions on~$S$.
\end{defn}

\begin{exer}\label{exer_B_vs}  Under the usual pointwise operations $\fml B(S,V)$ is a vector
space.  (Solution~\ref{sol_exer_B_vs}.)
\end{exer}

\begin{defn}\label{def_unif_norm}  Let $S$ be a set and $V$ be a normed linear space.  For every
$f$ in $\fml B(S,V)$ define
  \[ \norm f_u := \sup\{\norm{f(x)} \colon x \in S\}\,. \]
The function $f \mapsto \norm f_u$ is called the
 \index{<@$\norm{f}_u$ (uniform norm on $\fml B(S,V)$)}%
 \index{uniform!norm!on~$\fml B(S,V)$}%
 \index{norm!uniform \underline{\hphantom{OO}} on $\fml B(S,V)$}%
\df{uniform norm} on $\fml B(S,V)$.  This is the
 \index{usual!norm!on $\fml B(S,V)$}%
 \index{norm!usual!on $\fml B(S,V)$}%
usual norm on $\fml B(S,V)$.

In the following problem you are asked to show that this function really is a norm.  The
metric $d_u$ induced by the uniform norm $\norm{\quad}_u$ on $\fml B(S,V)$ is the
 \index{uniform!metric}%
 \index{metric!uniform}%
\df{uniform metric} on~$\fml B(S,V)$.
\end{defn}

\begin{prob} Show that the uniform norm defined in~\ref{def_unif_norm} is in fact a norm
on~$\fml B(S,V)$.
\end{prob}

\begin{defn}  Let $(f_n)$ be a sequence of functions in $\fml F(S,V)$.  If there is a function
$g$ in $\fml F(S,V)$ such that
  \[ \sup\{\norm{f_n(x) - g(x)}\colon x \in S\} \sto 0 \quad \text{as $n \sto \infty$,} \]
then we say that the sequence $(f_n)$
 \index{<@$f_n \sto g$ (unif) (uniform convergence)}%
 \index{uniform!convergence}%
 \index{convergence!uniform}%
\df{converges uniformly} to~$g$ and write $f_n \sto g \text{\,(unif)}$.  The function $g$ is
the
 \index{uniform!limit}%
 \index{limit!uniform}%
\df{uniform limit} of the sequence $(f_n)$.  Notice that if $g$ and all the $f_n$'s belong to
$\fml B(S,V)$, then uniform convergence of $(f_n)$ to $g$ is just convergence of $(f_n)$ to
$g$ with respect to the uniform metric.  Notice also that the preceding repeats verbatim
definition~\ref{def_uc_R}, except that $\R$ has been replaced by $V$ and absolute values by
norms.  We may similarly generalize definition~\ref{def_pc_R}.
\end{defn}

\begin{defn}  Let $(f_n)$ be a sequence in $\fml F(S,V)$.  If there is a function $g$ such that
  \[ f_n(x) \sto g(x) \qquad  \text{for all $x \in S$}, \]
then $(fn)$
 \index{pointwise!convergence}%
 \index{convergence!pointwise}%
\df{converges pointwise} to~$g$.  In this case we write
  \[ f_n \sto g \text{\,(ptws)}. \]
The function $g$ is the
 \index{pointwise!limit}%
 \index{limit!pointwise}%
 \index{<@$f_n \sto g$ (ptws) (pointwise convergence)}%
\df{pointwise limit} of the~$f_n$'s.  Problem~\ref{prob_uc_pwc} repeats
proposition~\ref{uc_vs_pwc}---uniform convergence implies pointwise convergence---except that
it now holds for $V$ valued functions (not just real valued ones).  Problem~\ref{prob_uc_bdd}
generalizes proposition~\ref{unif_lim_bdd}(a).
\end{defn}

\begin{prob}\label{prob_uc_pwc} If a sequence $(f_n)$ in $\fml F(S,V)$ converges uniformly to
a function $g$ in $\fml F(S,V)$, then $f_n \sto g \text{\,(ptws)}$.
\end{prob}

\begin{prob}\label{prob_uc_bdd} Let $f_n)$ be a sequence in $\fml B(S,V)$ and $g$ be a member
of $\fml F(S,V)$.  If $f_n \sto g\text{\,(unif)}$, then $g$ is bounded.
\end{prob}

\begin{prob} Define
 \begin{equation*}
    f(t) =
      \begin{cases}
             (t,0),    &\text{if $0 \le t \le 1$};   \\
             (1,t-1)   &\text{if $1 < t \le 2$};     \\
             (3-t,1)   &\text{if $2 < t \le 3$};     \\
             (0,4-t)   &\text{if $3 < t \le 4$}.
      \end{cases}
 \end{equation*}
Regarding $f$ as a member of the space $\fml B([0,4],\R^2)$ find~$\norm f_u$.
\end{prob}

\begin{exam} If $M$ is a compact metric space, then the family $\fml C(M,V)$ of all continuous
$V$-valued functions on $M$ is a normed linear space.
\end{exam}

\begin{proof} Problem.  \ns  \end{proof}

\begin{prob}\label{prob_clo_alg}  Let $M$ be a compact metric space.  Show that the family
$\fml C(M,\R)$ of all continuous real valued functions on $M$ is a unital algebra and that
$\norm{fg}_u \le \norm f_u\norm g_u$.  Show also that if $A$ is a subalgebra of $\fml
C(M,\R)$, then so is~$\clo A$.
\end{prob}

\begin{prop}\label{prop_unif_lim} If $(f_n)$ is a sequence of continuous $V$ valued functions
on a metric space $M$ and if this sequence converges uniformly to a $V$ valued function $g$ on
$M$, then $g$ is continuous.
\end{prop}

\begin{proof} Problem.  \emph{Hint.}  Modify the proof of proposition~\ref{unif_lim_cont}.   \ns
\end{proof}





\endinput
