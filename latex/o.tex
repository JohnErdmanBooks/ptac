\chapter{FINITE AND INFINITE SETS}

 \setcounter{section}{1}
 \setcounter{thm}{0}

There are a number of ways of comparing the ``sizes'' of sets. In this chapter and the next we
examine perhaps the simplest of these, cardinality.  Roughly speaking, we say that two sets
have the ``same number of elements'' if there is a one-to-one correspondence between the
elements of the sets. In this sense the open intervals $(0,1)$ and $(0,2)$ have the same
number of elements. (The map $x \mapsto 2x$ is a bijection.) Clearly this is only one sense of
the idea of ``size''. It is certainly also reasonable to regard $(0,2)$ as being bigger than
$(0,1)$ because it is twice as long.

We derive only the most basic facts concerning cardinality. In this appendix we discuss some
elementary properties of finite and infinite sets, and in the next we distinguish between
countable and uncountable sets. This is all we will need.

\begin{defn} Two sets $S$ and $T$ are
 \index{<@$\sim$ (cardinal equivalence)}%
 \index{equivalent!cardinally}%
\df{cardinally equivalent} if there exists a bijection from $S$ onto $T$, in which case we
write $S \sim T$. It is easy to see that cardinal equivalence is indeed an equivalence
relation; that is, it is reflexive, symmetric, and transitive.
\end{defn}

\begin{prop}\label{prop_ce_er} Cardinal equivalence is an equivalence relation.  Let $S$, $T$,
and $U$ be sets.  Then
 \begin{enumerate}
  \item[(a)] $S \sim S$;
  \item[(b)] if $S \sim T$, then $T \sim S$; and
  \item[(c)] if $S \sim T$ and $T \sim U$, then $S \sim U$.
 \end{enumerate}
\end{prop}

\begin{proof} Problem.  \ns  \end{proof}

\begin{defn} A set $S$ is
 \index{finite}%
\df{finite} if it is empty or if there exists $n \in \N$ such that $S \sim \{1,\dots,n\}$. A
set is
 \index{infinite}%
\df{infinite} if it is not finite. The next few facts concerning finite and infinite sets
probably appear obvious, but writing down the proofs may in several instances require a bit of
thought.  Here is a question which merits some reflection: if one is unable to explain exactly
why a result is obvious, is it really obvious?
\end{defn}

One ``obvious'' result is that if two initial segments $\{1,\dots,m\}$ and $\{1,\dots,n\}$ of
the set of natural numbers are cardinally equivalent, then $m = n$. We prove this next.

\begin{prop}\label{prop_card_n}  Let $n \in \N$. If $m \in \N$ and $\{1,\dots,m\} \sim
\{1,\dots,n\}$, then $m = n$.
\end{prop}

\begin{proof} Exercise. \emph{Hint.}  Use induction on $n$.
(Solution~\ref{sol_prop_card_n}.)  \ns
\end{proof}

\begin{defn} We define
 \index{cardinalnumber@$\card S$ (cardinal number of $S$)}%
$\card \emptyset$, the cardinal number of the empty set, to be $0$. If $S$ is a nonempty
finite set, then by the preceding proposition there exists only one positive integer $n$ such
that $S \sim \{1,\dots,n\}$.  This integer, $\card S$, is the
 \index{cardinal number}%
\df{cardinal number} of the set $S$, or the \df{number of elements} in $S$.  Notice that if
$S$ is finite with cardinal number $n$ and $T \sim S$, then $T$ is also finite and has
cardinal number~$n$. Thus for finite sets the expressions ``cardinally equivalent'' and ``have
the same cardinal number'' are interchangeable.
\end{defn}

\begin{exam} Let $S = \{a,b,c,d\}$. Then $\card S = 4$, since $\{a,b,c,d\} \sim \{1,2,3,4\}$.
\end{exam}

One simple fact about cardinal numbers is that the number of elements in the union of two
disjoint finite sets is the sum of the numbers of elements in each.

\begin{prop}\label{prop_card_union}  If $S$ and $T$ are disjoint finite sets, then $S \cup T$
is finite and
  \[ \card (S \cup T) = \card S + \card T\,. \]
\end{prop}

\begin{proof} Exercise.  (Solution~\ref{sol_prop_card_union}.)  \ns  \end{proof}

A variant of the preceding result is given in problem \ref{prob_card_diff}.  We now take a
preliminary step toward proving that subsets of finite sets are themselves finite
(proposition~\ref{prop_card_sub}).

\begin{lem}\label{lem_card_sub} If $C \subseteq \{1,\dots,n\}$, then $C$ is finite and
$\card C \le n$.
\end{lem}

\begin{proof} Exercise. \emph{Hint.}  Use mathematical induction.  If $C \subseteq
\{1,\dots,k + 1\}$, then $C \setminus \{k + 1\} \subseteq \{1,\dots,k\}$.  Examine the cases
$k + 1 \notin C$ and $k + 1 \in C$ separately.  (Solution~\ref{sol_lem_card_sub}.)  \ns
\end{proof}

\begin{prop}\label{prop_card_sub}  Let $S \subseteq T$.  If $T$ is finite, then $S$ is finite
and $\card S \le \card T$.
\end{prop}

\begin{proof} Problem.  \emph{Hint.}   The case $T = \emptyset$ is trivial. Suppose $T \ne
\emptyset$.  Let $\iota:S \sto T$ be the inclusion map of $S$ into $T$ (see
chapter~\ref{images}). There exist $n \in \N$ and a bijection $f \colon T \sto \{1,\dots,n\}$.
Let $C = \ran(f \circ\iota)$. The map from $S$ to $C$ defined by $x \mapsto f(x)$ is a
bijection. Use lemma~\ref{lem_card_sub}.) \ns
\end{proof}

The preceding proposition ``subsets of finite sets are finite'' has a useful contrapositive:
``sets which contain infinite sets are themselves infinite.''

\begin{prob}\label{prob_card_diff}  Let $S$ be a set and $T$ be a finite set. Prove that
  \[ \card (T \setminus S) = \card T - \card (T \cap S)\,. \]
\emph{Hint.}  Use problem \ref{prob_sdiff1} and proposition~\ref{prop_card_union}.
\end{prob}

Notice that it is a consequence of the preceding result~\ref{prob_card_diff} that if $S
\subseteq T$ (where $T$ is finite), then
\[\card (T \setminus S)  = \card T - \card S.\]

How do we show that a set $S$ is infinite?  If our only tool were the definition, we would
face the prospect of proving that there does \emph{not} exist a bijection from $S$ onto an
initial segment of the natural numbers.  It would be pleasant to have a more direct approach
than establishing the nonexistence of maps.  This is the point of our next proposition.

\begin{prop}\label{prop_ce_sub}  A set is infinite if and only if it is cardinally equivalent
to a proper subset of itself.
\end{prop}

\begin{proof} Exercise. \emph{Hint.}  Suppose a set $S$ is infinite. Show that it is possible to
choose inductively a sequence of distinct elements $a_1, a_2, a_3,\dots$ in $S$. (Suppose
$a_1,\dots,a_n$ have already been chosen. Can $S \setminus \{a_1,\dots,a_n\}$ be empty?)  Map
each $a_k$ to $a_{k+1}$ and map each member of $S$ which is not an $a_k$ to itself.

For the converse argue by contradiction.  Suppose that $T \sim S$ where $T$ is a proper subset
of $S$, and assume that $S \sim \{1,\dots,n\}$.  Prove that $S \setminus T \sim \{1,\dots,p\}$
for some $p \in \N$. Write $T$ as $S \setminus (S \setminus T)$ and obtain $n = n - p$ by
computing the cardinality of $T$ in two ways. (Make use of problem~\ref{prob_card_diff}.) What
does $n = n - p$ contradict?  (Solution~\ref{sol_prop_ce_sub}.)  \ns
\end{proof}

\begin{exam} The set $\N$ of natural numbers is infinite.
\end{exam}

\begin{proof}  The map $n \mapsto n + 1$ is a bijection from $\N$ onto $\N \setminus \{1\}$,
which is a proper subset of~$\N$.
\end{proof}

\begin{exer}\label{exer_int_inf}  The interval $(0,1)$ is infinite.
(Solution~\ref{sol_exer_int_inf}.)
\end{exer}

\begin{exam}  The set $\R$ of real numbers is infinite.
\end{exam}

The next two results tell us that functions take finite sets to finite sets and that for
injective functions finite sets come from finite sets.

\begin{prop}\label{prop_ran_fin}  If $T$ is a set, $S$ is a finite set, and $f:S \sto T$ is
surjective, then $T$ is finite.
\end{prop}

\begin{proof} Exercise. \emph{Hint.}  Use propositions~\ref{prop_rinv_surj} and
\ref{prop_linv_inj}.  (Solution~\ref{sol_prop_ran_fin}.) \ns
\end{proof}

\begin{prop}\label{prop_preimg_fin} If $S$ is a set, $T$ is a finite set, and $f \colon S \sto T$
is injective, then $S$ is finite.
\end{prop}

\begin{proof}  Exercise.  (Solution~\ref{sol_prop_preimg_fin}.)  \ns  \end{proof}

\begin{prob}\label{prob_union_fin} Let $S$ and $T$ be finite sets.  Prove that $S \cup T$ is
finite and that
  \[ \card (S \cup T)  = \card S + \card T - \card (S \cap T)\,. \]
\emph{Hint.}  Use problem \ref{prob_card_diff}.
\end{prob}

\begin{prob}  If $S$ is a finite set with cardinal number $n$, what is the cardinal number
of~$\sfml P(S)$?
\end{prob}



\endinput
