\chapter{TOPOLOGY OF THE REAL LINE}\label{nbhds_in_r}

It is clear from the definition of ``interior'' that the interior of a set is always contained
in the set.  Those sets for which the reverse inclusion also holds are called \emph{open
sets}.

\section{OPEN SUBSETS OF $\R$}
\begin{defn} A subset $U$ of $\R$ is
 \index{open}%
\df{open} if $U^\circ = U$.  That is, a set is open if and only if every point of the set is
an interior point of the set.  If $U$ is an open subset of $\R$ we
 \index{<@$\open{}{}$ (open subset of)}%
write~$\open{U}{\R}$.

Notice, in particular, that the empty set is open.  This is a consequence of the way
implication is defined in section~\ref{implication}: the condition that each point of
$\emptyset$ be an interior point is \emph{vacuously satisfied} because there \emph{are no
points} in $\emptyset$. (One argues that \emph{if} an element $x$ belongs to the empty
set, \emph{then} it is an interior point of the set.  The hypothesis is false; so the
implication is true.)  Also notice that $\R$ itself is an open subset of~$\R$.
\end{defn}

\begin{exam}\label{open_int_open} Bounded open intervals are open sets.  That is, if $a < b$,
then the open interval $(a,b)$ is an open set.
\end{exam}

\begin{proof} Example \ref{interior_open_int}. \end{proof}

\begin{exam} The interval $(0,\infty)$ is an open set.
\end{exam}

\begin{proof} Problem.  \ns \end{proof}

One way of seeing that a set is open is to verify that each of its points is an interior point
of the set.  That is what the definition says.  Often it is easier to observe that the set can
be written as a union of bounded open intervals.  That this happens exactly when a set is open
is the point of the next proposition.

\begin{prop}\label{union_open_intervals} A nonempty subset of $\R$ is open if and only if it is
a union of bounded open intervals.
\end{prop}

\begin{proof} Let $U \subseteq \R$. First, let us suppose $U$ is a nonempty open subset of~$\R$.
Each point of $U$ is then an interior point of $U$. So for each $x \in U$ we may choose a
bounded open interval $J(x)$ centered at $x$ which is entirely contained in $U$.  Since $x \in
J(x)$ for each $x \in U$, we see that
  \begin{equation}\label{2eqn1} U = \bigcup_{x \in U}\{x\} \subseteq
                \bigcup_{x \in U}J(x).
  \end{equation}
On the other hand, since $J(x) \subseteq U$ for each $x \in U$, we have (see
proposition~\ref{union_of_fam1})
  \begin{equation}\label{2eqn2} \bigcup_{x \in U} J(x) \subseteq U.
  \end{equation}
Together \eqref{2eqn1} and \eqref{2eqn2} show that $U$ is a union of bounded open intervals.

For the converse suppose $U = \bigcup \sfml{J}$ where $\sfml{J}$ is a family of open
bounded intervals.  Let $x$ be an arbitrary point of $U$.  We need only show that $x$ is
an interior point of~$U$. To this end choose an interval $J \in \sfml{J}$ which
contains~$x$. Since $J$ is a bounded open interval we may write $J = (a,b)$ where~$a <
b$.  Choose $\epsilon$ to be the smaller of the numbers $x - a$ and $b - x$.  Then it is
easy to see that $\epsilon > 0$ and that $x \in J_\epsilon(x) = (x - \epsilon, x +
\epsilon) \subseteq (a,b)$. Thus $x$ is an interior point of~$U$.
\end{proof}

\begin{exam}Every interval of the form $(-\infty,a)$ is an open set. So is every interval of
the form $(a,\infty)$. (Notice that this and example \ref{open_int_open} give us the very
comforting result that the things we are accustomed to calling open intervals are indeed open
sets.)
\end{exam}

\begin{proof} Problem. \ns \end{proof}

The study of calculus has two main ingredients: algebra and topology.  Algebra deals with
operations and their properties, and with the resulting structure of groups, fields, vector
spaces, algebras, and the like.  Topology, on the other hand, is concerned with closeness,
$\epsilon$-neighborhoods, open sets, and with the associated structure of metric spaces and
various kinds of topological spaces.  Almost everything in calculus results from the interplay
between algebra and topology.

\begin{defn} The word ``topology'' has a technical meaning.  A family $\sfml{T}$ of subsets
of a set $X$ is a
 \index{topology}%
\df{topology} on $X$ if
 \begin{enumerate}
   \item $\emptyset$ and $X$ belong to $\sfml{T}$;
   \item if $\sfml{S} \subseteq \sfml{T}$ (that is, if $\sfml S$ is a
subfamily of~$\sfml T$), then $\bigcup\sfml{S} \in \sfml{T}$; and
   \item if $\sfml{S}$ is a \emph{finite} subfamily of $\sfml{T}$, then
$\bigcap\sfml{S} \in \sfml{T}$.
 \end{enumerate}
\end{defn}

We can paraphrase this definition by saying that a family of subsets of $X$, which contains
both $\emptyset$ and $X$, is a topology on $X$ if it is closed under arbitrary unions and
finite intersections.  If this definition doesn't make any sense to you at first reading,
don't fret.  This kind of abstract definition, although easy enough to remember, is
irritatingly difficult to understand.  Staring at it doesn't help. It appears that a
bewildering array of entirely different things might turn out to be topologies.  And this is
indeed the case.  An understanding and appreciation of the definition come only gradually.
You will notice as you advance through this material that many important concepts such as
continuity, compactness, and connectedness are defined by (or characterized by) open sets.
Thus theorems which involve these ideas will rely on properties of open sets for their proofs.
This is true not only in the present realm of the real line but in the much wider world of
metric spaces, which we will shortly encounter in all their fascinating variety. You will
notice that two properties of open sets are used over and over: that unions of open sets are
open and that finite intersections of open sets are open.  Nothing else about open sets turns
out to be of much importance.  Gradually one comes to see that these two facts completely
dominate the discussion of continuity, compactness, and so on.  Ultimately it becomes clear
that nearly everything in the proofs goes through in situations where \emph{only} these
properties are available---that is, in topological spaces.

Our goal at the moment is quite modest: we show that the family of all open subsets of $\R$ is
indeed a topology on $\R$.

\begin{prop}\label{union_intersection_open} Let $\sfml S$ be a family of open sets in $\R$.
Then
 \begin{enumerate}
   \item[(a)] the union of $\sfml S$ is an open subset of~$\R$; and
   \item[(b)] if $\sfml S$ is finite, the intersection of $\sfml S$ is
an open subset of~$\R$.
 \end{enumerate}
\end{prop}

\begin{proof} Exercise. (Solution~\ref{sol_union_intersection_open}.)  \ns \end{proof}

\begin{exam} The set $U = \{x \in \R\colon  x < -2\} \cup \{x > 0\colon x^2-x-6 < 0\}$ is an
open subset of~$\R$.
\end{exam}

\begin{proof} Problem. \ns \end{proof}

\begin{exam} The set $\R \setminus \N$ is an open subset of $\R$.
\end{exam}

\begin{proof} Problem. \ns \end{proof}

\begin{exam} The family $\sfml{T}$ of open subsets of $\R$ is not closed under arbitrary
intersections.s (That is, there exists a family $\sfml{S}$ of open subsets of $\R$ such
that $\bigcap \sfml{S}$ is \emph{not} open.)
\end{exam}

\begin{proof} Problem. \ns \end{proof}







\section{CLOSED SUBSETS OF $\R$}
Next we will investigate the closed subsets of $\R$.  These will turn out to be the
complements of open sets.  But initially we will approach them from a different perspective.

\begin{defn}A point $b$ in $\R$ is an
 \index{accumulation point}%
\df{accumulation point} of a set $A \subseteq \R$ if every $\epsilon$-neighborhood of $b$
contains at least one point of $A$ distinct from $b$.  (We do \emph{not} require that $b$
belong to~$A$, although, of course, it may.)  The set of all accumulation points of $A$ is
called the
 \index{derived set}%
\df{derived set} of $A$ and is denoted
 \index{<@$A'$ (derived set of $A$, set of accumulation points)}%
by~$A'$.  The
 \index{closure}%
\df{closure} of~$A$, denoted
 \index{<@$\clo A$ (closure of $A$)}%
by~$\overline A$, is~$A \cup A'$.
\end{defn}

\begin{exam} Let $A = \{1/n\colon n \in \N\}$.  Then $0$ is an accumulation point of $A$.
Furthermore, $\clo A = \{0\} \cup A$.
\end{exam}

\begin{proof} Problem.
%\emph{Hint.} See problem \ref{}  J.OOO
\ns  \end{proof}

\begin{exam} Let $A$ be $(0,1) \cup \{2\} \subseteq \R$. Then $A' = [0,1]$ and $\clo A =
[0,1] \cup \{2\}$.
\end{exam}

\begin{proof} Problem. \ns \end{proof}

\begin{exam}\label{q_dense} Every real number is an accumulation point of the set $\Q$ of
rational numbers (since every open interval in $\R$ contains infinitely many rationals); so
$\overline{\Q}$ is all of~$\R$.
\end{exam}

\begin{exer}\label{exam_clo_rats} Let $A = \Q \cap (0,\infty)$.  Find $A^\circ$, $A'$, and
$\clo A$. (Solution~\ref{sol_exam_clo_rats}.)
\end{exer}

\begin{prob} Let $A = (0,1] \cup \bigl([2,3] \cap \Q \bigr)$. Find:
 \begin{enumerate}
   \item[(a)] $\intr A$;
\vskip 3 true pt
   \item[(b)] $\clo A$;
\vskip 3 true pt
   \item[(c)] $\clo{\intr A}$;
\vskip 3 true pt
   \item[(d)] $\intr{\bigl(\,\clo A\,\bigr)}$;
\vskip 3 true pt
   \item[(e)] $\clo{A^c}$;
\vskip 3 true pt
   \item[(f)] $\intr{\left(\,\clo{A^c}\,\right)}$;
\vskip 3 true pt
   \item[(g)] $\intr{\bigl(A^c\bigr)}$; and
\vskip 3 true pt
   \item[(h)] $\clo{\intr{\bigl(A^c\bigr)}}$.
 \end{enumerate}
\end{prob}

\begin{exam}\label{sup_in_clo} Let $A$ be a nonempty subset of $\R$. If $A$ is bounded above,
then $\sup A$ belongs to the closure of $A$.  Similarly, if $A$ is bounded below, then $\inf
A$ belongs to~$\clo A$.
\end{exam}

\begin{proof} Problem. \ns \end{proof}

\begin{prob} Starting with a set $A$, what is the greatest number of \emph{different} sets you
can get by applying successively the operations of closure, interior, and complement?  Apply
them as many times as you wish and in any order.  For example, starting with the empty set
doesn't produce much.  We get only $\emptyset$ and $\R$.  If we start with the closed interval
$[0,1]$, we get four sets: $[0,1]$, $(0,1)$, $(-\infty,0] \cup [1, \infty)$, and $(-\infty,0)
\cup (1, \infty)$.  By making a more cunning choice of $A$, how many different sets can you
get?
\end{prob}

\begin{prop}\label{intr_vs_clo} Let $A \subseteq \R$.  Then
 \begin{enumerate}
   \item[(a)] $\bigl(\intr A \bigr)^c = \clo{A^c}$; and
   \item[(b)] $\intr{\bigl(A^c\bigr)} = \bigl(\,\clo A\,\bigr)^c$.
 \end{enumerate}
\end{prop}

\begin{proof} Exercise. \emph{Hint.} Part (b) is a very easy consequence of~(a).
(Solution~\ref{sol_intr_vs_clo}.) \ns \end{proof}

\begin{defn} A subset $A$ of $\R$ is
 \index{closed}%
\df{closed} if $\clo A = A$.
\end{defn}

\begin{exam}\label{cl_int_cl} Every closed interval $\bigl($that is, intervals of the form
$[a,b]$ or $(-\infty,a]$ or~$[a,\infty)$ or~$(-\infty,\infty)$ $\bigr)$ are closed.
\end{exam}

\begin{proof} Problem. \ns \end{proof}

\begin{cau} It is a common mistake to treat subsets of $\R$ as if they were doors or windows,
and to conclude, for example, that a set is closed because it is not open, or that it cannot
be closed because it is open. These ``conclusions'' are wrong! A subset of $\R$ may be open
and not closed, or closed and not open, or both open and closed, or neither. For example,
in~$\R$:
 \begin{enumerate}
  \item $(0,1)$ is open but not closed;
  \item $[0,1]$ is closed but not open;
  \item $\R$ is both open and closed; and
  \item $[0,1)$ is neither open nor closed.
 \end{enumerate}
This is not to say, however, that there is no relationship between these properties.  In the
next proposition we discover that the correct relation has to do with complements.
\end{cau}

\begin{prop}\label{open_vs_closed} A subset of $\R$ is open if and only if its complement is
closed.
\end{prop}

\begin{proof} Problem.  \emph{Hint.}  Use proposition \ref{intr_vs_clo}. \ns \end{proof}


\begin{prop} The intersection of an arbitrary family of closed subsets of $\R$ is closed.
\end{prop}

\begin{proof} Let $\sfml A$ be a family of closed subsets of $\R$. By \emph{De Morgan's law}
(see proposition~\ref{comp_of_union}) $\bigcap \sfml{A}$ is the complement of $\bigcup
\{A^c\colon A \in \sfml A\}$. Since each set $A^c$ is open (by \ref{open_vs_closed}), the
union of $\{A^c\colon A \in \sfml A\}$ is open (by \ref{union_intersection_open}(a)); and
its complement $\bigcap \sfml A$ is closed (\ref{open_vs_closed} again).
\end{proof}

\begin{prop} The union of a finite family of closed subsets of $\R$ is closed.
\end{prop}

\begin{proof} Problem. \ns \end{proof}

\begin{prob} Give an example to show that the union of an arbitrary family of closed subsets
of $\R$ need not be closed.
\end{prob}

\begin{defn} Let $a$ be a real number.  Any open subset of $\R$ which contains $a$ is a
 \index{neighborhood!of a point}%
\df{neighborhood} of $a$.  Notice that an $\epsilon$-neighborhood of $a$ is a very special
type of neighborhood: it is an interval and it is symmetric about~$a$.  For most purposes the
extra internal structure possessed by $\epsilon$-neighborhoods is irrelevant to the matter at
hand.  To see that we can operate as easily with general neighborhoods as with
$\epsilon$-neighborhoods do the next problem.
\end{defn}

\begin{prob} Let $A$ be a subset of $\R$.  Prove the following.
 \begin{enumerate}
  \item[(a)] A point $a$ is an interior point of $A$ if and only if some
neighborhood of $a$ lies entirely in~$A$.
  \item[(b)] A point $b$ is an accumulation point of $A$ if and only if every
neighborhood of $b$ contains at least one point of $A$ distinct
from~$b$.
 \end{enumerate}
\end{prob}

\endinput
