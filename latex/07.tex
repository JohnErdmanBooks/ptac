\chapter{LIMITS OF REAL VALUED FUNCTIONS}\label{lim_rvf}

In chapter \ref{seqs_rn} we studied limits of sequences of real
numbers.  In this very short chapter we investigate limits of real
valued functions of a real variable.  Our principal result
(\ref{cnt_vs_lim}) is a characterization of the continuity of a
function $f$ at a point in terms of the limit of $f$ at that point.

Despite the importance of this characterization, there is one
crucial difference between a function being continuous at a point
and having a limit there.  If $f$ is continuous at~$a$, then $a$
must belong to the domain of $f$.  In order for $f$ to have a limit
at~$a$, it is not required that $a$ be in the domain of~$f$.




\section{DEFINITION} To facilitate the definition of ``limit'' we introduce the notion of a
\emph{deleted neighborhood} of a point.

\begin{defn} If $J = (b,c)$ is a neighborhood of a point~$a$ (that is, if $b < a < c$), then
 \index{<@$J^*$ (deleted neighborhood)}%
 \index{j@$J^*$ (deleted neighborhood)}%
$J^*$, the
 \index{deleted!neighborhood}%
 \index{neighborhood!deleted}%
\df{deleted neighborhood} associated with $J$, is just $J$ with the point $a$ deleted.  That
is, $J^* = (b,a) \cup (a,c)$.  In particular, if $J_\delta(a)$ is the $\delta$-neighborhood of
$a$, then $J^*_\delta(a)$ denotes $(a-\delta,a) \cup (a,a+\delta)$.
\end{defn}

\begin{defn} Let $A$ be a subset of $\R$, let $f \colon A \sto \R$, let $a$ be an accumulation
point of $A$, and let $l$ be a real number.  We say that $l$ is the
 \index{limit!of a function}%
\df{limit of $f$ as $x$ approaches $a$} (or the \df{limit of $f$ at $a$}) if: for every
$\epsilon > 0$ there exists $\delta > 0$ such that $f(x)\in J_\epsilon(l)$ whenever $x \in A
\cap J_\delta^*(a)$.

Using slightly different notation we may write this condition as
   \[ (\forall \epsilon > 0)(\exists \delta > 0)(\forall x \in A)\,
                    0 < \abs{x-a} < \delta \implies \abs{f(x) - l} < \epsilon\,. \]
If this condition is satisfied we write
 \index{<@$f(x) \sto l \text{ as } x \sto a$ (limit of a function at a point)}%
      \[ f(x) \sto l \text{ as } x \sto a\]
or
 \index{limit@$\lim_{x \sto a}f(x)$ (limit of a function at a point)}%
   \[ \lim_{x \sto a}f(x) = l\,. \]
(Notice that this last notation is a bit optimistic.  It would not make sense if $f$ could
have two distinct limits as $x$ approaches~$a$.  We will show in proposition \ref{rlim_uniq}
that this cannot happen.)
\end{defn}

The first thing we notice about the preceding definition is that the point $a$ at which we
take the limit need not belong to the domain $A$ of the function~$f$.  Very often in practice
it does not.  Recall the definition in beginning calculus of the derivative of a function $f
\colon \R \sto \R$ at a point $a$.  It is the limit as $h \sto 0$ of the Newton quotient
$\dfrac{f(a+h) - f(a)}h$. This quotient is not defined at the point $h = 0$.  Nevertheless we
may still take the limit as $h$ approaches~$0$.

Here is another example of the same phenomenon.  The function on $(0,\infty)$ defined by $x
\mapsto (1+x)^{1/x}$ is not defined at $x = 0$.  But its limit at $0$ exists: recall from
beginning calculus that $\lim_{x \sto 0}(1+x)^{1/x} = e$.

One last comment about the definition: even if a function $f$ is defined at a point $a$, the
value of $f$ at $a$ is irrelevant to the question of the existence of a limit there.
According to the definition we consider only points $x$ satisfying $0 < \abs{x-a} < \delta$.
The condition $0 < \abs{x-a}$ says just one thing: $x \ne a$.

\begin{prop}\label{rlim_uniq} Let $f \colon A \sto \R$ where $A \subseteq \R$, and let $a$ be an
accumulation point of $A$.  If $f(x) \sto b$ as $x \sto a$, and if $f(x) \sto c$ as $x \sto
a$, then $b = c$.
\end{prop}

\begin{proof} Exercise.  (Solution~\ref{sol_rlim_uniq}.)
  \ns \end{proof}









\section{CONTINUITY AND LIMITS}
There is a close connection between the existence of a limit of a function at a point $a$ and
the continuity of the function at $a$. In proposition \ref{cnt_vs_lim} we state the precise
relationship. But first we give two examples to show that in the absence of additional
hypotheses neither of these implies the other.

\begin{exam} The inclusion function $f \colon  \N \sto \R \colon  n \mapsto n$ is continuous
(because \emph{every} subset of $\N$ is open in $\N$, and thus every function defined on $\N$
is continuous).  But the limit of $f$ exists at no point (because $\N$ has no accumulation
points).
\end{exam}

\begin{exam} Let
   \[ f(x) = \begin{cases}    0,  &\text{ for $x \ne 0$} \\
                              1,  &\text{ for $x=0$\,.}
             \end{cases} \]
Then $\lim_{x \sto 0}f(x)$ exists (and equals $0$), but $f$ is not continuous at $x = 0$.
\end{exam}

We have shown in the two preceding examples that a function $f$ may be continuous at a point
$a$ without having a limit there and that it may have a limit at $a$ without being continuous
there.  If we require the point $a$ to belong to the domain of $f$ and to be an accumulation
point of the domain of $f$ (these conditions are independent!), then a necessary and
sufficient condition for $f$ to be continuous at $a$ is that the limit of $f$ as $x$
approaches $a$ (exist and) be equal to $f(a)$.

\begin{prop}\label{cnt_vs_lim} Let $f \colon  A \sto \R$ where $A \subseteq \R$, and let
$a \in A \cap A'$.  Then $f$ is continuous at $a$ if and only if
   \[ \lim_{x \sto a}f(x) = f(a)\,. \]
\end{prop}

\begin{proof} Exercise.  (Solution~\ref{sol_cnt_vs_lim}.)  \ns \end{proof}

\begin{prop}\label{trnsl_lim} If $f \colon A \sto \R$ where $A \subseteq \R$, and $a \in A'$, then
   \[ \lim_{h \sto 0} f(a + h) = \lim_{x \sto a} f(x) \]
in the sense that if either limit exists, then so does the other and the two limits are equal.
\end{prop}

\begin{proof} Exercise.   (Solution~\ref{sol_trnsl_lim}.)  \ns \end{proof}

\begin{prob}  Let $f(x) = 4 - x$ if $x < 0$ and $f(x) = (2 + x)^2$ if $x > 0$.  Using the
definition of ``limit'' show that $\lim_{x \sto 0}f(x)$ exists.
\end{prob}

\begin{prop}  Let $f \colon A \sto \R$ where $A \subseteq \R$ and let $a \in A'$.  Then
  \[ \lim_{x \sto a}f(x) = 0 \qquad \text{if and only if} \qquad
                                        \lim_{x \sto a}\abs{f(x)} = 0\,. \]
\end{prop}

\begin{proof} Problem.  \ns \end{proof}

\begin{prop}\label{sand_fcn} Let $f$, $g$, $h \colon A \sto \R$ where $A \subseteq \R$, and
let $a \in A'$.  If $f \le g \le h$ and
   \[ \lim_{x \sto a}f(x) = \lim_{x \sto a}h(x) = l, \]
then $\lim_{x \sto a}g(x) = l$.
\end{prop}


\begin{proof} Problem. \emph{Hint.} A slight modification of your proof of proposition
\ref{sand_seq} should do the trick.  \ns
\end{proof}

\begin{prob}[Limits of algebraic combinations of functions] Carefully formulate and prove
the standard results from beginning calculus on the limits of sums, constant multiples,
products, and quotients of functions.
\end{prob}

\begin{prob} Let $A$, $B \subseteq \R$, $a \in A$, $f \colon A \sto B$, and $g \colon B \sto \R$.
 \begin{enumerate}
  \item[(a)] If $l = \lim_{x \sto a}f(x)$ exists and $g$ is continuous at $l$, then
    \[ \lim_{x \sto a}(g \circ f)(x) = g(l)\,. \]
  \item[(b)] Show by example that the following assertion need not be true: If
$l = \lim_{x \sto a}f(x)$ exists and $\lim_{y \sto l}g(y)$ exists, then $\lim_{x \sto a}(g
\circ f)(x)$ exists.
 \end{enumerate}
\end{prob}




\endinput
