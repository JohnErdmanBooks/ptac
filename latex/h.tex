\chapter{ORDER PROPERTIES OF $\R$}\label{order_R}

 \setcounter{section}{1}
 \setcounter{thm}{0}


The second group of axioms are the order axioms. They concern a subset $\Po$ of $\R$ (call
this the set of
 \index{positive!strictly}%
 \index{numbers!strictly positive}%
\df{strictly positive} numbers).

\begin{ax}[V]  The set $\Po$ is closed under addition and multiplication.  (That is, if $x$ and
$y$ belong to $\Po$, so do $x + y$  and $xy$.)
\end{ax}

\begin{ax}[VI]\label{axiom_trichot} For each real number $x$ exactly one of the following is true:
$x = 0$, $x \in \Po$, or  $-x \in \Po$.  This is the
 \index{trichotomy}%
 \index{axiom!of trichotomy}%
\emph{axiom of trichotomy}.
\end{ax}

Define the relation $<$ on $\R$ by
  \[ x < y  \text{ if and only if }  y - x \in \Po\,. \]
Also define $>$ on $\R$ by
  \[ x > y \text{ if and only if } y < x\,. \]
We write $x \le y$ if $x < y$ or $x = y$, and $x \ge y$ if $y \le x$.

\begin{prop}\label{lt_tran} On $\R$ the relation $<$ is transitive (that is, if $x < y$ and
$y < z$, then $x < z$).
\end{prop}

\begin{proof} If $x < y$ and $y < z$, then $y - x$ and $z - y$ belong to $\Po$.  Thus
  \begin{align*}
      z - x &= z +(-x) \\
            &= (z + 0) + (-x) \\
            &= (z + (y + (-y))) + (-x) \\
            &= (z + ((-y) + y)) + (-x) \\
            &= ((z + (-y)) + y) + (-x) \\
            &= (z + (-y)) + (y + (-x)) \\
            &= (z - y) + (y - x) \in \Po.
  \end{align*}
This shows that $x < z$.
\end{proof}

\begin{prob} Justify each of the seven equal signs in the proof of proposition~\ref{lt_tran}.
\end{prob}

\begin{exer}\label{pos} Show that a real number $x$ belongs to the set $\Po$ if and only if
$x > 0$.  (Solution~\ref{sol_pos}.)

\end{exer}

\begin{prop}\label{mult_pos} If $x > 0$ and $y < z$ in $\R$, then $xy < xz$.
\end{prop}

\begin{proof} Exercise.  \emph{Hint.} Use problem \ref{arith_prob1}.
(Solution~\ref{sol_mult_pos}.)  \ns
\end{proof}

\begin{prop} If $x$, $y$, $z \in \R$ and $y < z$, then $x + y < x + z$.
\end{prop}

\begin{proof} Problem.  \emph{Hint.}  Use equation \eqref{arith_eqn4}. \ns
\end{proof}

\begin{prop} If $w < x$ and $y < z$, then $w + y < x + z$.
\end{prop}

\begin{proof} Problem. \ns  \end{proof}

\begin{prob}  Show that $1 > 0$. \emph{Hint.} Keep in mind that $1$ and $0$ are assumed to be
distinct. (Look at the axiom concerning additive and multiplicative identities.)  If $1$ does
not belong to $\Po$, what can you say about the number $-1$? What about $(-1)(-1)$? Use
problem~\ref{arith_prob1}.
\end{prob}

\begin{prop} If $x > 0$, then $x^{-1} > 0$.
\end{prop}

\begin{proof} Problem.  \ns  \end{proof}

\begin{prop}  If $0 < x < y$, then $1/y < 1/x$.
\end{prop}

\begin{proof} Problem.  \ns  \end{proof}

\begin{prop}\label{mult_ineq} If $0 < w < x$ and $0 < y < z$, then $wy < xz$.
\end{prop}

\begin{proof} Exercise. \ns   (Solution~\ref{sol_mult_ineq}.) \end{proof}

\begin{prob} Show that $x < 0$ if and only if $-x > 0$.
\end{prob}

\begin{prob} Show that if $y < z$ and $x < 0$, then $xz < xy$.
\end{prob}

\begin{prob}  Show that $x < y$ if and only if $-y < -x$.
\end{prob}

\begin{prob}\label{prob_sqrt_uniq} Suppose that $x$, $y \ge 0$ and $x^2 = y^2$. Show that $x = y$.
\end{prob}

\begin{prob} Show in considerable detail how the preceding results can be used to solve the
inequality
  \[ \frac5{x + 3} < 2 - \frac1{x - 1}\,. \]
\end{prob}

\begin{prob} Let $\C = \{(a,b) \colon a,b \in \R\}$.  On $\C$ define two binary operations $+$
and $\cdot$ by:
  \[ (a,b) + (c,d) = (a + c, b + d) \]
and
  \[ (a,b) \cdot (c,d) = (ac-bd,ad+bc)\,. \]
Show that $\C$ under these operations is a field. (That is, $\C$ satisfies axioms I--IV.)
This is the field of
 \index{complex numbers}%
\df{complex numbers}.

Determine whether it is possible to make $\C$ into an ordered field.  (That is, determine
whether it is possible to choose a subset $\Po$ of $\C$ which satisfies axioms V and~VI.)
\end{prob}

The axioms presented thus far define an ordered field. To obtain the particular ordered field
$\R$ of real numbers we require one more axiom.  We assume that $\R$ is order complete; that
is, $\R$ satisfies the \emph{least upper bound axiom}.  This axiom will be stated (and
discussed in some detail) in chapter~\ref{lub} (in particular, see~\ref{axiom_lub}).


There is a bit more to the axiomatization of $\R$ than we have indicated in the preceding
discussion. For one thing, how do we know that the axioms are
 \index{consistent}%
 \index{axioms!consistent}%
consistent?  That is, how do we know that they will not yield a contradiction?  For this
purpose one constructs a \emph{model} for $\R$, that is, a concrete mathematical object which
satisfies all the axioms for $\R$. One standard procedure is to define the positive integers
in terms of sets: $0$ is the empty set $\emptyset$, the number $1$ is the set whose only
element is $0$, the number $2$ is the set whose only element is $1$, and so on. Using the
positive integers we construct the set $\Z$ of all integers $\dots, -2, -1 , 0, 1, 2,
\dots\;$.  From these we construct the set  $\Q$ of rational numbers (that is, numbers of the
form $p/q$ where $p$ and $q$ are integers and  $q \ne 0$).
%Later, in chapter~\ref{equiv}, this part of the construction is
%carried out as an exercise in the use of equivalence classes.
Finally the reals are constructed from the rationals.

Another matter that requires attention is the use of the definite article in the expression
``\emph{the} real numbers''. This makes sense only if the axioms are shown to be
 \index{categorical}%
 \index{axioms!categorical}%
categorical; that is, if there is ``essentially'' only one model for the axioms.  This turns
out to be correct about the axioms for $\R$ given an appropriate technical meaning of
``essentially''---but we will not pursue this matter here.  More about both the construction
of the reals and their uniqueness can be found in \cite{Spivak:1967}.



\endinput
