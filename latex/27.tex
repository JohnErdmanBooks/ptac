\chapter{COMPUTATIONS IN $\R^n$}\label{comp_Rn}

In the preceding two chapters we have developed some fundamental facts concerning the
differential calculus in arbitrary Banach spaces. In the present chapter we restrict our
attention to the Euclidean spaces~$\R^n$.  Not only are these spaces very important
historically, but fortunately there are available a variety of powerful yet relatively simple
techniques which make possible explicit computations of many of the concepts introduced in
chapter~\ref{diff_calc}.  The usefulness of these spaces seems to be associated with the
emphasis in classical physics on systems having a finite number of degrees of freedom. The
computational simplicity stems from two facts: first, differentials of functions between
Euclidean spaces are always continuous (see proposition~\ref{prop_fd_cont}); and second, the
usual norm on $\R^n$ is derivable from an inner product. In the first section of this chapter
we derive some standard elementary facts about inner products.  It is important to appreciate
that despite their algebraic appearance, inner products are the source of much of the geometry
in Euclidean spaces.

\section{INNER PRODUCTS}
\begin{defn}  Let $x = (x_1, \dots, x_n)$ and $y = (y_1, \dots,y_n)$ be vectors in $\R^n$.  The
 \index{inner product}%
 \index{product!inner}%
 \index{<@$\langle x,y \rangle$ (dot product, inner product)}%
\df{inner product} (or
 \index{dot product}%
 \index{product!dot}%
\df{dot product}) of $x$ and $y$, denoted by $\langle x,y \rangle$,
is defined by
  \[ \langle x,y \rangle := \sum_{k=1}^n x_ky_k\,. \]
\end{defn}

As a first result we list the most important properties of the inner product.

\begin{prop}\label{prop_prop_ip} Let $x$, $y$, and $z$ be vectors in $\R^n$ and $\alpha$ be a
scalar.  Then
 \begin{enumerate}
  \item[(a)] $\langle x + y , z \rangle = \langle x , z \rangle + \langle y , z \rangle$;
  \item[(b)] $\langle \alpha x , y \rangle = \alpha \langle x , y \rangle$;
  \item[(c)] $\langle x , y \rangle = \langle y , x \rangle$;
  \item[(d)] $\langle x , x \rangle \ge 0$;
  \item[(e)] $\langle x , x \rangle = 0$ only if $x = \vc 0$; and
  \item[(f)] $\norm x = \sqrt{\langle x , x \rangle} $.
 \end{enumerate}
Items (a) and (b) say that the inner product is
 \index{linearity!of the inner product}%
\emph{linear} in its first variable; (c) says it is
 \index{symmetric}%
\emph{symmetric}; and (d) and (e) say that it is
 \index{positive definite}%
\emph{positive definite}.  It is virtually obvious that an inner product is also linear in its
second variable (see exercise~\ref{exer_ip1}).  Thus an inner product may be characterized as
a positive definite, symmetric, bilinear functional on~$\R^n$.
\end{prop}

\begin{proof} Problem.   \ns  \end{proof}

\begin{prop}\label{prop_ip_expan} If $x$ is in $\R^n$, then
  \[ x = \sum_{k=1}^n \langle x , e^k \rangle e^k \,. \]
\end{prop}

This result is used so frequently that it has been stated formally as a proposition.  Its
proof, however, is trivial. [It is clear from the definition of the inner product that
$\langle x , e^k \rangle = x_k$, where, as usual, $\{e^1, \dots, e^n\}$ is the standard basis
for~$\R^n$.]

\begin{exer}\label{exer_ip1}  Use properties (a)--(c) above, but not the definition of
\emph{inner product} to prove that
  \[ \langle x , y + z \rangle =  \langle x , y \rangle + \langle x , z \rangle \]
and
  \[ \langle x , \alpha y \rangle = \alpha \langle x , y \rangle \]
for all $x$, $y$, $z \in \R^n$ and $\alpha \in \R$. (Solution~\ref{sol_exer_ip1}.)
\end{exer}

 \index{parallelogram law}%
\begin{prop}[The Parallelogram Law]  If $x$, $y \in \R^n$, then
  \[ \norm{x + y}^2 + \norm{x - y}^2 = 2\norm x^2 + 2\norm y^2\,. \]
\end{prop}

\begin{proof} Problem.   \ns  \end{proof}

 \index{Schwarz inequality}%
 \index{inequality!Schwarz}%
\begin{prop}[Schwarz's Inequality]\label{Schwarz_ineq}  If $u$, $v \in \R^n$, then
  \[ \abs{\langle u , v \rangle} \le \norm u \norm v\,. \]
\end{prop}

\begin{proof} This has been proved in chapter~\ref{metric}: notice that the left side of the
inequality given in proposition~\ref{Schwarz} is $\bigl(\langle u , v \rangle\bigr)^2$ and the
right side is $\norm u^2  \norm v^2$.
\end{proof}

\begin{defn}  If $x$ and $y$ are nonzero vectors in $\R^n$, define $\measuredangle(x,y)$, the
 \index{<@$\measuredangle(x,y)$ (angle between two vectors)}%
 \index{angle}%
\df{angle} between $x$ and $y$, by
  \[ \measuredangle(x,y)
             :=  \arccos \left(\frac{\langle x , y \rangle}{\norm x \norm y}\right)\,. \]
A version of this formula which is perhaps somewhat more familiar is
  \[ \langle x,y \rangle = \norm x\norm y\cos\measuredangle(x,y)\,. \]
\end{defn}

\begin{exer}\label{exer_angle} How do we know that the preceding definition makes sense?
(What is the domain of the arccosine function?)  (Solution~\ref{sol_exer_angle}.)
\end{exer}

\begin{prob}  Prove the
 \index{law of cosines}%
\emph{law of cosines}: if $x$ and $y$ are nonzero vectors in $\R^n$ and $\theta =
\measuredangle(x,y)$, then
  \[ \norm{x-y}^2 = \norm x^2 + \norm y^2 - 2\norm x\norm y\cos\theta\,. \]
\end{prob}

\begin{exer}\label{exer_angle2} What is the angle between the vectors $(1,0,1)$ and $(0,-1,1)$
in~$\R^3$?   (Solution~\ref{sol_exer_angle2}.)
\end{exer}

\begin{prob} Find the angle between the vectors $(1,0,-1,-2)$ and $(-1,1,0,1)$ in $\R^4$.
\end{prob}

\begin{prob} The angle of intersection of two curves is by definition the angle between the
tangent vectors to the curves at the point of intersection.  Find the angle of
intersection at the point $(1,-2,3)$ of the curves $C_1$ and $C_2$ where
  \[ C_1(t) = (t, t^2 + t - 4, 3 + \ln t) \]
and
  \[ C_2(u) = (u^2 - 8, u^2 - 2u - 5, u^3 - 3u^2 - 3u + 12)\,. \]
\end{prob}

\begin{defn} Two vectors $x$ and $y$ in $\R^n$ are
 \index{perpendicular}%
\df{perpendicular} (or
 \index{orthogonal}%
\df{orthogonal}) if $\langle x,y \rangle = 0$.  In this case we write
 \index{<@$x \perp y$ (perpendicular, orthogonal)}%
$x \perp y$.  Notice that the relationship between perpendicularity and angle is what we
expect: if $x$ and $y$ are nonzero vectors then $x \perp y$ if and only if
$\measuredangle(x,y) = \pi/2$.  The zero vector is perpendicular to all vectors but the angle
it makes with another vector is not defined.
\end{defn}

\begin{prob}  Find a linear combination of the vectors $(1,0,2)$ and $(2,-1,1)$ which is
perpendicular to the vector $(2,2,1)$ in~$\R^3$.
\end{prob}

\begin{prob}  Prove the
 \index{Pythagorean theorem}%
\emph{Pythagorean theorem}: if $x \perp y$ in $\R^n$, then
  \[ \norm{x + y}^2 = \norm x^2 + \norm y^2\,. \]
Does the converse hold?
\end{prob}

\begin{notn}  Let $f \colon U \sto \R^n$ and $g \colon V \sto \R^n$ where $U$ and $V$ are subsets
of a normed linear space which are not disjoint.  Then we denote by $\langle f,g \rangle$ the
real valued function on $U \cap V$ whose value at a point $x$ in $U \cap V$ is $\langle
f(x),g(x) \rangle$.  That is,
  \[ \langle f,g \rangle \colon U \cap V \sto \R \colon x \mapsto \langle f(x),g(x) \rangle\,. \]
The scalar field $\langle f,g \rangle$ is the
 \index{inner product!of functions}%
\df{inner product} (or
 \index{dot product!of functions}%
\df{dot product} of $f$ and~$g$.
\end{notn}

\begin{prop}\label{prop_ip_diff}  Suppose that the functions $f \colon U \sto \R^n$ and
$g\colon V \sto \R^n$, defined on subsets of a normed linear space $W$, are differentiable at
a point $a$ in the interior of $U \cap V$.  Then $\langle f,g \rangle$ is differentiable at
$a$ and
  \[ d\langle f,g \rangle_a = \langle f(a),dg_a \rangle + \langle df_a,g(a) \rangle\,. \]
\end{prop}

\begin{proof} Problem.  \emph{Hint.}  Use propositions~\ref{prop_Leib_rul}
and~\ref{prop_diff_coord}.  \ns
\end{proof}

\begin{cor}\label{cor_ip_der} If $f$ and $g$ are curves at a point $a$ in $\R^n$ and are
differentiable, then
  \[ D\langle f,g \rangle (a) = \langle f(a),Dg(a) \rangle + \langle Df(a), g(a) \rangle\,. \]
\end{cor}

\begin{proof} Use \ref{prop_ip_diff} and \ref{prop_diff_der}.
  \begin{align*}
     D \langle f,g \rangle (a) &= d\langle f,g \rangle_a(1) \\
              &= \langle f(a),dg_a(1) \rangle + \langle df_a(1),g(a) \rangle \\
              &= \langle f(a),Dg(a) \rangle + \langle Df(a),g(a) \rangle
  \end{align*}
\end{proof}

\begin{prob} Let $f = (f^1,f^2,f^3)$ where
  \begin{align*}
        f^1(t) &= t^3 + 2t^2 - 4t + 1  \\
        f^2(t) &= t^4 - 2t^3 + t^2 + 3 \\
        f^3(t) &= t^3 - t^2 + t - 2
  \end{align*}
and let $g(t) = \norm{f(t)}^2$ for all $t$ in $\R$.  Find $Dg(1)$.
\end{prob}

\begin{prob}  Let $c$ be a differentiable curve in $\R^n$.  Show that the point $c(t)$ moves
on the surface of a sphere centered at the origin if and only if the tangent vector $Dc(t)$ at
$t$ is perpendicular to the position vector $c(t)$ at each~$t$. \emph{Hint.}  Use
corollary~\ref{cor_ip_der}.
\end{prob}









\section{THE GRADIENT}  In beginning calculus texts the gradient of a real valued function of
$n$ variables is usually defined to be an $n$-tuple of partial derivatives.  This definition,
although convenient for computation, disguises the highly geometric nature of the gradient.
Here we adopt a different definition: the gradient of a scalar field is the vector which
represents, in a sense to be made precise below, the differential of the function. First we
look at an important example of a bounded linear functional on~$\R^n$.

\begin{exam}\label{exam_blf}  Let $b \in \R^n$.  Define
  \[ \sbsb{\psi}b\colon \R^n \sto \R\colon x \mapsto \langle x,b \rangle\,. \]
Then $\sbsb{\psi}b$ is a bounded linear functional on $\R^n$ and $\norm{\sbsb{\psi}b} =
\norm b$.
\end{exam}

\begin{proof} Exercise.  (Solution~\ref{sol_exam_blf}.)  \ns  \end{proof}

The reason for the importance of the preceding example is that functions of the form
$\sbsb{\psi}b$ turn out to be the \emph{only} bounded linear functionals on~$\R^n$. Since
on $\R^n$ every linear functional is bounded (see propositions \ref{prop_fd_cont}
and~\ref{prop_equiv_cont}), the functions $\sbsb{\psi}b$ are in fact the only real valued
linear maps on~$\R^n$.  Thus we say that every linear functional on $\R^n$ can be
represented in the form $\sbsb{\psi}b$ for some vector $b$ in $\R^n$.  Furthermore, this
representation is unique.  These assertions are stated formally in the next theorem.

\begin{thm}[Riesz-Fr\'echet Theorem]\label{thm_RF} If $f \in (\R^n)^*$, then there exists
 \index{Riesz-Fr\'echet theorem}%
a unique vector $b$ in $\R^n$ such that
  \[ f(x) = \langle x,b \rangle \]
for all $x$ in $\R^n$.
\end{thm}

\begin{proof} Problem.  \emph{Hint.}  For the existence part, (a) write $\sum_{k=1}^n x_ke^k$
for $x$ in the expression $f(x)$, and use the linearity of~$f$.  Then, (b) write $\langle x,b
\rangle$ as a sum.  Comparing the results of (a) and (b), guess the identity of the desired
vector~$b$.  The uniqueness part is easy: suppose $f(x) = \langle x,a \rangle = \langle x,b
\rangle$ for all $x$ in~$\R^n$. Show $a = b$.    \ns
\end{proof}

\begin{defn}  If a map $T \colon V \sto W$ between two normed linear spaces is both an isometry
and a vector space isomorphism, we say that it is an
 \index{isometric isomorphism}%
 \index{isomorphism!isometric}%
\df{isometric isomorphism} and that the spaces $V$ and $W$ are \df{isometrically isomorphic}.
\end{defn}

\begin{prop}  Each Euclidean space $\R^n$ is isometrically isomorphic to its dual~$(\R^n)^*$.
\end{prop}

\begin{proof} Problem.   \emph{Hint.}  Consider the map
  \[ \psi\colon \R^n \sto (\R^n)^*\colon b \mapsto \psi_b\,. \]
One thing that must be established is that $\psi$ is linear; don't confuse this with showing
that each $\psi_b$ is linear---a task already accomplished in example~\ref{exam_blf}.  Use the
Riesz-Fr\'echet theorem~\ref{thm_RF} and problem~\ref{prob_lin_bij}.    \ns
\end{proof}

The Riesz-Fr\'echet theorem \ref{thm_RF} is the crucial ingredient in our definition of the
\emph{gradient} of a scalar field.

\begin{defn}  Let $U \subseteq \R^n$ and $\phi\colon U \sto \R$ be a scalar field.  If $\phi$
is differentiable at a point $a$ in $\intr U$, then its differential $d\phi_a$ is a bounded
linear map from $\R^n$ into~$\R$.  That is, $d\phi_a \in (\R^n)^*$.  Thus according to the
Riesz-Fr\'echet theorem~\ref{thm_RF} there exists a unique vector, which we denote by
$\nabla\phi(a)$, representing the linear functional $d\phi_a$.  That is,
 \index{<@$\nabla\phi(a)$ (gradient of $\phi$ at~$a$)}%
$\nabla\phi(a)$ is the unique vector in $\R^n$ such that
  \[ d\phi_a(x)  =  \langle x,\nabla\phi(a) \rangle \]
for all $x$ in $\R^n$.  The vector $\nabla\phi(a)$ is the
 \index{gradient!at a point}%
\df{gradient} of $\phi$ at~$a$.  If $U$ is an open subset of $\R^n$ and $\phi$ is
differentiable at each point of~$U$, then the function
  \[ \nabla\phi\colon U \sto \R^n\colon u \mapsto \nabla\phi(u) \]
is the
 \index{gradient!on an open set}%
\df{gradient} of~$\phi$.  Notice two things: first, the gradient of a scalar field is a vector
field; and second, the differential $d\phi_a$ is the zero linear functional if and only if the
gradient at~$a$, $\nabla\phi(a)$, is the zero vector in~$\R^n$.
\end{defn}

Perhaps the most useful fact about the gradient of a scalar field $\phi$ at a point $a$ in
$\R^n$ is that it is the vector at $a$ which points in the direction of the most rapid
increase of~$\phi$.

\begin{prop}\label{prop_max_dd}  Let $\phi\colon U \sto \R$ be a scalar field on a subset $U$
of~$\R^n$.  If $\phi$ is differentiable at a point $a$ in $U$ and $d\phi_a$ is not the zero
functional, then the maximum value of the directional derivative $D_u\phi(a)$, taken over all
unit vectors $u$ in $\R^n$, is achieved when $u$ points in the direction.of the gradient
$\nabla\phi(a)$.  The minimum value is achieved when $u$ points in the opposite direction
$-\nabla\phi(a)$.
\end{prop}

\begin{proof} Exercise. \emph{Hint.} Use proposition~\ref{prop_dd_diff} and recall that
$\langle x,y \rangle = \norm x \norm y \cos\measuredangle(x,y)$.
(Solution~\ref{sol_prop_max_dd}.) \ns
\end{proof}

When a curve $c$ is composed with a scalar field $\phi$ we obtain a real valued function of a
single variable.  An easy but useful special case of the \emph{chain rule} says that the
derivative of the composite $\phi\circ c$ is the dot product of the derivative of $c$ with the
gradient of~$\phi$.

\begin{prop}\label{prop_cr_curv}  Suppose that $c$ is a curve in $\R^n$ which is differentiable
at a point $t$ in $\R$ and that $\phi$ belongs to $\fml D_{c(t)}(\R^n,\R)$.  Then $\phi\circ
c$ is differentiable at $t$ and
  \[ D(\phi \circ c)(t) = \langle Dc(t),(\nabla\phi)(c(t))\rangle\,. \]
\end{prop}

\begin{proof} Problem.  \emph{Hint.}  Use proposition~\ref{prop_diff_der} and the \emph{chain
rule}~\ref{thm_ch_rul}.    \ns
\end{proof}

\begin{prob} If $\phi \colon \R^n \sto \R\colon x \mapsto \norm x^2$, then $\phi$ is
differentiable at each point $b$ of $\R^n$ and
  \[ d\phi_b = 2\psi_b \]
Furthermore, $\nabla\phi = 2I$ (where $I$ is the identity function on~$\R^n$).  \emph{Hint.}
The definition of $\psi_b$ is given in~\ref{exam_blf}.  Write $\phi = \langle I,I \rangle$ and
use~\ref{prop_ip_diff}.
\end{prob}

\begin{prob}\label{prob_dc_mvt} For the function $\phi$ given in the preceding problem verify
by direct computation the formula for the \emph{mean value
theorem}~(proposition~\ref{prop_mvt_Bs}).
\end{prob}

\begin{defn}  A linear transformation $T\colon \R^n \sto \R^n$ is
 \index{self-adjoint}%
\df{self-adjoint} if $\langle Tx,y \rangle = \langle x,Ty \rangle$ for all $x$, $y \in \R^n$.
\end{defn}

\begin{prob}\label{prob_salt}  Let $T \in \ofml B(\R^n,\R^n)$ be self-adjoint and
 \[\mu\colon \R^n \sto \R\colon  x \mapsto \langle Tx,x \rangle.\]
 \begin{enumerate}
  \item[(a)]  Show that $\mu$ is differentiable at each point $b$ in $\R^n$ and find $d\mu_b$.
  \item[(b)] Find $\nabla\mu$.
 \end{enumerate}
\end{prob}

\begin{prob}  Repeat problem~\ref{prob_dc_mvt}, this time using the function $\mu$ given in
problem~\ref{prob_salt}.
\end{prob}

 \index{conservation of energy}%
\begin{exer}[Conservation of Energy]\label{cons_energ} Consider a
particle $P$ moving in $\R^3$ under the influence of a force~$F$. Suppose that the position of
$P$ at time $t$ is $x(t)$ where $x\colon \R \sto \R^3$ is at least twice differentiable.  Let
$v := Dx$ be the velocity of $P$ and $a := Dv$ be its acceleration. Assume
 \index{Newton's second law}%
\emph{Newton's second law}: $F \circ x = ma$, where $F\colon \R^3 \sto \R^3$ is the force
acting on $P$ and $m$ is the mass of~$P$. Suppose further that the force field $F$ is
 \index{conservative}%
\df{conservative}; that is, there exists a scalar field $\phi\colon \R^3 \sto \R$ such that $F
= -\nabla\phi$. (Such a scalar field is a
 \index{potential!function}%
\df{potential function} for~$F$.)  The
 \index{kinetic energy}%
\df{kinetic energy} of $P$ is defined by
  \[ KE := \tfrac12 m \norm v^2\,, \]
its
 \index{potential!energy}%
\df{potential energy} by
  \[ PE := \phi \circ x\,, \]
and its
 \index{total!energy}%
\df{total energy} by
  \[ TE := KE + PE\,. \]
Prove, for this situation, the law of \emph{conservation of energy}:
  \[ TE \text{ is constant.} \]
\emph{Hint.}  Use propositions \ref{prop_sc_const}, \ref{prop_ip_diff},
and~\ref{prop_cr_curv}.   (Solution~\ref{sol_cons_energ}.)
\end{exer}

In most circumstances the simplest way of computing the gradient of a scalar field $\phi$ on
$\R^n$ is to calculate the $n$ partial derivatives of~$\phi$.  The $n$-tuple of these
derivatives is the gradient.  In most beginning calculus texts this is the \emph{definition}
of the gradient.

\begin{prop}\label{prop_grad_pd} If $\phi$ is a scalar field on a subset of $\R^n$ and is
differentiable at a point~$a$, then
  \[ \nabla\phi(a) = \sum_{k=1}^n\phi_k(a) e^k\,. \]
\end{prop}

\begin{proof}  Exercise.  \emph{Hint.}  Substitute $\nabla\phi(a)$ for $x$
in~\ref{prop_ip_expan}.  Use~\ref{prop_dd_diff}.  (Solution~\ref{sol_prop_grad_pd}.) \ns
\end{proof}

\begin{exer}\label{exer_comp_dd}  Let $\phi(w,x,y,z) = wz - xy$, $u = (\frac12, -\frac12,
\frac12, -\frac12)$, and $a = (1,2,3,4)$. Find the directional derivative $D_u\phi(a)$.
\emph{Hint.}  Use \ref{prop_dd_diff}, the definition of \emph{gradient},
and~\ref{prop_grad_pd}.   (Solution~\ref{sol_exer_comp_dd}.)
\end{exer}

 \index{method of steepest descent}%
 \index{steepest descent, method of}%
\begin{exer}[Method of Steepest Descent]\label{mthd_st_desc}  Let $\phi(x,y) = 2x^2 + 6y^2$
and $a = (2,-1)$.  Find the steepest downhill path on the surface $z = \phi(x,y)$ starting at
the point $a$ and ending at the minimum point on the surface.  \emph{Hints.} (1)~It is enough
to find the equation of the projection of the curve onto the $xy$-plane; every curve $t
\mapsto \bigl(x(t),y(t)\bigr)$ in the $xy$-plane is the projection along the $z$-axis of a
unique curve $t \mapsto \bigl(x(t),y(t),\phi(x(t),y(t))\bigr)$ on the surface $z = \phi(x,y)$.
(2)~If $c\colon t \mapsto \bigl(x(t),y(t)\bigr)$ is the desired curve and we set $c(0) = a$,
then according to proposition~\ref{prop_max_dd} the unit vector $u$ which minimizes the
directional derivative $D_u\phi(b)$ at a point $b$ in $\R^2$ is the one obtained by choosing
$u$ to point in the direction of~$-\nabla\phi(b)$.  Thus in order for the curve to point in
the direction of the most rapid decrease of $\phi$ at each point~$c(t)$, the tangent vector to
the curve at $c(t)$ must be some positive multiple $p(t)$ of~$-(\nabla\phi)\bigl(c(t)\bigr)$.
The function $p$ will govern the speed of descent; since this is irrelevant in the present
problem, set $p(t) = 1$ for all~$t$. (3)~Recall from beginning calculus that on an interval
the only nonzero solution to an equation of the form $Dx(t) = kx(t)$ is of the form $x(t) =
x(0)e^{kt}$.  (4)~The parameter $t$ which we have introduced is artificial.  Eliminate it to
obtain an equation of the form $y = f(x)$.  (Solution~\ref{sol_mthd_st_desc}.)
\end{exer}

 \index{mean value theorem!for scalar fields}%
\begin{prop}[A Mean Value Theorem for Scalar Fields]\label{prop_mvt_sf} Let $\phi$ be a
differentiable scalar field on an open convex subset $U$ of $\R^n$ and suppose that $a$ and
$b$ are distinct points belonging to~$U$.  Then there exists a point $c$ in the closed segment
$[a,b]$ such that
  \[ \phi(b) - \phi(a) = \langle b - a,\nabla\phi(c) \rangle\,. \]
\end{prop}

\begin{proof} Problem.   \emph{Hint.}   Let $l(t) = (1 - t)a + tb$ for $0 \le t \le 1$.  Apply
the \emph{mean value theorem} for a real valued function of a single variable (\ref{mvthm}) to
the function $\phi\circ l$.  Use proposition~\ref{prop_cr_curv}. \ns
\end{proof}

\begin{prob}  Let $c(t) = (\cos t,\sin t,t)$ and $\phi(x,y,z) = x^2y - 3yz$.  Find
$D(\phi\circ c)(\pi/6)$.  \emph{Hint.} Use \ref{prop_cr_curv} and~\ref{prop_grad_pd}.
\end{prob}

\begin{prob}  Let $\phi(x,y,z) = xz - 4y$, $u = (\tfrac12,0,\tfrac12\sqrt3)$, and $a =
(1,0,-\tfrac{\pi}2)$.  Find the directional derivative~$D_u\phi(a)$.
\end{prob}

\begin{prob} Show that if $V$ is a normed linear space, $f \in \fml D_a(\R^n,V)$, and $v$ is a
nonzero vector in~$\R^n$, then
  \[ D_vf(a) = \sum_{k=1}^nv_kf_k(a)\,. \]
\emph{Hint.}  Use proposition \ref{prop_dd_diff}.
\end{prob}

\begin{prob}  Let $f \colon \R^2 \sto \R^2\colon x \mapsto ({x_1}^2 - {x_2}^2, 3x_1x_2)$,
$a = (2,1)$, and $v = (-1,2)$.  Use the preceding problem to find~$D_vf(a)$.
\end{prob}

\begin{prob}\label{prob_sd}  Find the path of steepest descent on the surface $z = x^6 + 12y^4$
starting at the point whose $x$-coordinate is $1$ and whose $y$-coordinate is~$\frac12$.
\end{prob}

\begin{prob}  Suppose that the temperature $\phi(x,y)$ at points $(x,y)$ on a flat surface is
given by the formula
  \[ \phi(x,y) = x^2 - y^2\,. \]
Starting at a point $(a,b)$ on the surface, what path should be followed so that the
temperature will increase as rapidly as possible?
\end{prob}

 \index{method of steepest descent}%
 \index{steepest descent, method of}%
\begin{prob}  This (like exercise \ref{mthd_st_desc} and problem~\ref{prob_sd}) is a steepest
descent problem; but here, we suppose that for some reason we are unable to solve explicitly
the resulting differential equations.  Instead we invoke an approximation technique.  Let
  \[ \phi(x) = 13{x_1}^2 - 42x_1 + 13{x_2}^2 + 6x_2 + 10x_1x_2 + 9 \]
for all $x$ in~$\R^2$.  The goal is to approximate the path of steepest descent.  Start at an
arbitrary point $x^0$ in $\R^2$ and choose a number $h > 0$.  At $x^0$ compute the gradient of
$\phi$, take $u^0$ to be the unit vector pointing in the direction of $-\nabla\phi(x^0)$, and
then move $h$ units in the direction of $u^0$ arriving at a point~$x^1$.  Repeat the
procedure: find the unit vector $u^1$ in the direction of $-\nabla\phi(x^1)$, then from $x^1$
move $h$ units along $u^1$ to a point~$x^2$.  Continue in this fashion.  In other words, $x^0
\in \R^2$ and $h > 0$ are arbitrary, and for $n \ge 0$
  \[ x^{n+1} = x^n + hu^n \]
where $u^n = -\norm{\nabla\phi(x^n)}^{-1}\nabla\phi(x^n)$.
 \begin{enumerate}
  \item[(a)] Start at the origin $x^0 = (0,0)$ and choose $h = 1$. Compute 25 or 30 values of
$x^n$.  Explain geometrically what is happening here.  Why is $h$ ``too large''?  \emph{Hint.}
Don't attempt to do this by hand.  Write a program for a computer or a programmable
calculator. In writing your program don't ignore the possibility that $\nabla\phi(x^n)$  may
be zero for some~$n$. (Keep in mind when you write this up that your reader probably has no
idea how to read the language in which you write your program. Document it well enough that
the reader can easily understand what you are doing at each step.)
  \item[(b)] Describe what happens when $h$ is ``too small''.  Again start at the origin, take
$h = 0.001$ and compute 25 or 30 values of~$x^n$.
  \item[(c)] By altering the values of $h$ at appropriate times, find a succession of points $x^0,
\dots, x^n$ (starting with $x^0 = (0,0)\,)$ such that the distance between $x^n$ and the point
where $\phi$ assumes its minimum value is less than~$0.001$.  (By examining the points $x^0,
\dots, x^n$ you should be able to guess, for this particular function, the exact location of
the minimum.)
  \item[(d)] Alter the program in part (a) to eliminate division by $\norm{\nabla\phi(x^n)}$.
(That is, let $x^{n+1} = x^n - h\nabla\phi(x^n)$.)  Explain what happens in this case when $h$
is ``too large'' (say $h = 1$).  Explain why the altered program works better (provided that
$h$ is chosen appropriately) than the program in (a) for the present function~$\phi$.
 \end{enumerate}
\end{prob}

\begin{prob}  Is it possible to find a differentiable scalar field $\phi$ on $\R^n$ and a point
$a$ in $\R^n$ such that $D_u\phi(a) > 0$ for every nonzero $u$ in~$\R^n$?
\end{prob}

\begin{prob}  Is it possible to find a differentiable scalar field $\phi$ on $\R^n$ and a nonzero
vector $u$ in $\R^n$ such that $D_u\phi(a) > 0$ for every $a$ in~$\R^n$?
\end{prob}









\section{THE JACOBIAN MATRIX}
\begin{defn} Let $\open U{\R^n}$ and $f\colon U \sto \R^m$ be a differentiable function.  Recall
that the components $f^1, \dots, f^m$ of $f$ satisfy
  \[ f(x) = \bigl(f^1(x), \dots, f^m(x)\bigr) = \sum_{j=1}^m f^j(x) e^j \]
for all $x$ in~$U$.  More briefly we may write
  \[ f = (f^1, \dots, f^m) = \sum_{j=1}^m f^j e^j\,. \]
Recall also (from proposition~\ref{prop_pd_vvf}) that
  \[ (f^j)_k(a)  =  (f_k)^j(a) \]
whenever $1 \le j \le m$, $1 \le k \le n$, and $a \in U$. Consequently the notation $f_k^j(a)$
is unambiguous; from now on we use it.  As the differential of $f$ at a point $a$ in $U$ is a
(bounded) linear map from $\R^n$ into $\R^m$, it may be represented by an $m \times n$ matrix.
This is called the
 \index{<@$\bigl[f_k^j(a)\bigr]$ (Jacobian matrix)}%
 \index{Jacobian!matrix}%
 \index{matrix!Jacobian}%
\df{Jacobian matrix} of $f$ at~$a$.  The entry in the $j^{\text{th}}$ row and $k^{\text{th}}$
column of this matrix is~$f_k^j(a)$.
\end{defn}

\begin{prop}\label{prop_jm_pd} If $f \in \fml D_a(\R^n,\R^m)$, then
  \[ [df_a] = [f_k^j(a)]\,. \]
\end{prop}

\begin{proof} Exercise. \emph{Hint.} It helps to distinguish notationally between the standard
basis vectors in $\R^n$ and those in $\R^m$.  Denote the ones in $\R^n$ by $e^1, \dots, e^n$
and those in $\R^m$ by $\hat e^1, \dots, \hat e^m$. Use~\ref{prop_mrlt_bas}.
(Solution~\ref{sol_prop_jm_pd}.)  \ns
\end{proof}

Note that the $j^{\text{th}}$ row of the Jacobian matrix is $f_1^j, \dots, f_n^j$.  Thought of
as a vector in $\R^n$ this is just the gradient of the scalar field~$f^j$.  Thus we may think
of the Jacobian matrix $[df_a]$ as being in the form
  \[ \begin{bmatrix}   \nabla f^1(a)  \\
                         \vdots       \\
                       \nabla f^m(a)
     \end{bmatrix}\,. \]

\begin{exer}\label{exer_jm_comp}  Let $f\colon \R^4 \sto \R^3\colon  (w,x,y,z) \mapsto
(wxz, x^2 + 2y^2 + 3z^2, wy\arctan z)$, let $a = (1,1,1,1)$, and let $v = (0,2,-3,1)$.
 \begin{enumerate}
  \item[(a)] Find $[df_a]$.
  \item[(b)] Find $df_a(v)$.
 \end{enumerate}
(Solution~\ref{sol_exer_jm_comp}.)
\end{exer}

\begin{prob} Let $f \colon \R^2 \sto \R^2\colon  x \mapsto ({x_1}^2 - {x_2}^2, 3x_1x_2)$ and
$a = (2,1)$.
 \begin{enumerate}
  \item[(a)] Find $[df_a]$.
  \item[(b)] Use part (a) to find $df_a(-1,3)$.
 \end{enumerate}
\end{prob}

\begin{prob}  Let $f \colon \R^3 \sto \R^4 \colon (x,y,z) \mapsto (xy, y - z^2, 2xz, y + 3z)$,
let $a = (1,-2,3)$, and let $v = (2,1,-1)$.
 \begin{enumerate}
  \item[(a)] Find $[df_a]$.
  \item[(b)] Use part (a) to calculate $D_vf(a)$.
 \end{enumerate}
\end{prob}

\begin{prob}  Let $f \colon \R^2 \sto \R^2 \colon (x,y) \mapsto (x^2y, 2xy^2)$, let $a = (2,-1)$,
and let $u = (\frac35, \frac45)$.  Compute $D_uf(a)$ in three ways:
 \begin{enumerate}
  \item[(a)] Use the definition of \emph{directional derivative}.
  \item[(b)] Use proposition~\ref{prop_diff_dd}.
  \item[(c)] Use proposition~\ref{prop_dd_diff}.
 \end{enumerate}
\end{prob}

\begin{prob}  Suppose that $f \in \fml D_a(\R^3,\R^4)$ and that the Jacobian matrix of $f$ at
$a$ is
  \[ \begin{bmatrix}  b & c & e \\
                      g & h & i \\
                      j & k & l \\
                      m & n & p
     \end{bmatrix} \]
Find $f_1(a)$, $f_2(a)$, $f_3(a)$, $\nabla f^1(a)$, $\nabla f^2(a)$, $\nabla f^3(a)$, and
$\nabla f^4(a)$.
\end{prob}

\begin{prob}  Let $f \in \fml D_a(\R^n,\R^m)$ and $v \in \R^n$.  Show that
 \begin{enumerate}
  \item[(a)] $df_a(v) = \sum_{j=1}^m \langle \nabla f^j(a),v \rangle e^j$, and
  \item[(b)] $\norm{df_a} \le \sum_{j=1}^m \norm{\nabla f^j(a)}$.
 \end{enumerate}
\end{prob}






\section{THE CHAIN RULE}  In some respects it is convenient for scientists  to work with variables
rather than functions.  Variables denote the physical quantities in which a scientist is
ultimately interested.  (In thermodynamics, for example, $T$ is temperature, $P$ pressure, $S$
entropy, and so on.)  Functions usually have no such standard associations.  Furthermore, a
problem which deals with only a small number of variables may turn out to involve a dauntingly
large number of functions if they are specified. The simplification provided by the use of
variables may, however, be more apparent than real, and the price paid in increased ambiguity
for their suggestiveness is often substantial. Below are a few examples of ambiguities
produced by the combined effects of excessive reliance on variables, inadequate (if
conventional) notation, and the unfortunate mannerism of using the same name for a function
and a dependent variable (``Suppose $x = x(s,t)$\dots'').
 \begin{enumerate}
  \item[(A)]  If $z = f(x,y)$, what does $\pd{}{x}z(y,x)$ mean? (Perhaps $f_1(y,x)$?
Possibly $f_2(y,x)$?)
  \item[(B)]  If $z = f(x,t)$ where $x = x(t)$, then what is $\pd zt$?  (Is it $f_2(t)$?  Perhaps
the derivative of $t \mapsto f(x(t),t)$ is intended?)
  \item[(C)]  Let $f(x,y)$ be a function of two variables. Does the expression $z = f(tx,ty)$ have
three partial derivatives $\pd zx$, $\pd zy$, and $\pd zt$? Do $\pd zx$ and $\pd fx$ mean the
same thing?
  \item[(D)]  Let $w = w(x,y,t)$ where $x = x(s,t)$ and $y = y(s,t)$.  A direct application of the
chain rule (as stated in most beginning calculus texts) produces
    \[ \pd wt = \pd wx \pd xt + \pd wy \pd yt + \pd wt\,. \]
Is this correct ? Do the terms of the form $\pd wt$ cancel?
  \item[(E)]  Let $z = f(x,y) = g(r,\theta)$ where $x = r\cos\theta$ and $y = r\sin\theta$.  Do
$\pd zr$ and $\pd z\theta$ make sense? Do $\pd zx$ and $\pd zy$?  How about $z_1$ and $z_2$?
Are any of these equal?
  \item[(F)]  The formulas for changing polar to rectangular coordinates are $x = r\cos\theta$ and $y =
r\sin\theta$. So if we compute the partial derivative of the variable $r$ with respect to the
variable~$x$ we get
 \[ \pd rx = \dfrac{\partial }{\partial x}\sqrt{x^2 + y^2}
             = \dfrac x{\sqrt{x^2 + y^2}} =  \dfrac xr = \cos\theta\,. \]
On the other hand, since $r = \dfrac x{\cos\theta}$, we use the \emph{chain rule} to get
  \[ \pd rx = \dfrac1{\cos\theta} = \sec\theta\,. \]
Do you suppose something is wrong here? What?
 \end{enumerate}

The principal goal of the present section is to provide a reliable formalism for dealing with
partial derivatives of functions of several variables in such a way that questions like
(A)--(F) can be avoided.  The basic strategy is quite simple: when in doubt give names to the
relevant functions (especially composite ones!) and then use the chain rule.  Perhaps it
should be remarked that one need not make a fetish of avoiding variables. Many problems stated
in terms of variables can be solved quite simply without the intrusion of the names of
functions. (\emph{E.g.\ }What is $\pd zx$ if $z = x^3y^2$?)  This section is intended as a
guide for the perplexed.  Although its techniques are often useful in dissipating confusion
generated by inadequate notation, it is neither necessary nor even particularly convenient to
apply them routinely to every problem which arises. Let us start by writing the chain rule for
functions between Euclidean spaces in terms of partial derivatives.  Suppose that $f \in \fml
D_a(\R^p,\R^n)$ and $g \in \fml D_{f(a)}(\R^n,\R^m)$. Then according to
theorem~\ref{thm_ch_rul}
  \[ d(g \circ f)_a = dg_{f(a)} \circ df_a\,. \]
Replacing these linear transformations by their matrix representations and using
proposition~\ref{prop_comp_mrlt} we obtain
 \begin{equation}\label{eqn_cr_sv1}
    \bigl[d(g \circ f)_a\bigr] = \bigl[dg_{f(a)}\bigr]\bigl[df_a\bigr]\,.
 \end{equation}

 \index{chain rule}%
\begin{prop}\label{prop_cr_pd}  If $f \in \fml D_a(\R^p,\R^n)$ and $g \in \fml
D_{f(a)}(\R^n,\R^m)$, then $\bigl[d(g \circ f)_a\bigr]$ is the $m \times p$ matrix whose entry
in the $j^{\text{th}}$ row and $k^{\text{th}}$ column is $\sum_{i=1}^n g_i^j(f(a))f_k^i(a)$.
That is,
  \[ (g \circ f)_k^j(a) = \sum_{i=1}^n\bigl(g_i^j \circ f\bigr)(a) f_k^i(a) \]
for $1 \le j \le m$ and $1 \le k \le p$.
\end{prop}

\begin{proof} Multiply the two matrices on the right hand side of~\eqref{eqn_cr_sv1} and use
proposition~\ref{prop_jm_pd}.
\end{proof}


It is occasionally useful to restate proposition~\ref{prop_cr_pd} in the following (clearly
equivalent) way.

\begin{cor} If $f \in \fml D_a(\R^p,\R^n)$ and $g \in \fml D_{f(a)}(\R^n,\R^m)$, then
  \[ \bigl[d(g \circ f)_a\bigr] = \bigl[\langle\nabla g^j(f(a)) ,
                         f_k(a)\rangle\bigr]_{j=1}^m{\bigl.\bigr.}_{k=1}^p\,. \]
\end{cor}

\begin{exer}\label{exer_crpd_not}  This is an exercise in translation of notation.  Suppose
$y = y(u,v,w,x)$ and $z = z(u,v,w,x)$ where $u = u(s,t)$, $v = v(s,t)$, $w = w(s,t)$, and $x =
x(s,t)$.  Show that (under suitable hypotheses)
  \[ \pd zt = \pd zu\pd ut + \pd zv\pd vt + \pd zw\pd wt + \pd zx\pd xt\,. \]
(Solution~\ref{sol_exer_crpd_not}.)
\end{exer}

\begin{prob}  Suppose that the variables $x$, $y$, and $z$ are differentiable functions of the
variables $\alpha$, $\beta$, $\gamma$, $\delta$, and $\epsilon$, which in turn depend in a
differentiable fashion on the variables $r$, $s$, and~$t$.  As in exercise~\ref{exer_crpd_not}
use proposition~\ref{prop_cr_pd} to write $\pd zr$ in terms of quantities such as $\pd
z\alpha$, $\pd \delta r$, \emph{etc.}
\end{prob}

\begin{exer}\label{exer_crpd_xmp}  Let $f(x,y,z) = (xy^2, 3x - z^2, xyz, x^2 + y^2, 4xz + 5)$,
$g(s,t,u,v,w) = (s^2 + u^2 + v^2, s^2v - 2tw^2)$, and $a = (1,0,-1)$.  Use the \emph{chain
rule} to find $\bigl[d(g \circ f)_a\bigr]$. (Solution~\ref{sol_exer_crpd_xmp}.)
\end{exer}

\begin{prob}  Let $f(x,y,z) = (x^3y^2\sin z, x^2 + y\cos z)$, $g(u,v) = (\sqrt uv, v^3)$,
$k = g \circ f$, $a = (1,-2,\pi/2)$, and $h = (1,-1,2)$.  Use the \emph{chain rule} to find
$dk_a(h)$.
\end{prob}

\begin{prob}  Let $f(x,y,z) = (x^2y + y^2z, xyz)$, $g(x,y) = (x^2y, 3xy, x - 2y, x^2 + 3)$,
and $a = (1,-1,2)$.  Use the \emph{chain rule} to find $\bigl[d(g \circ f)_a\bigr]$.
\end{prob}

We now consider a slightly more complicated problem.  Suppose that $w = w(x,y,t)$ where $x =
x(s,t)$ and $y = y(s,t)$ and that all the functions mentioned are differentiable.  (This is
problem (D) at the beginning of this section.)  It is perhaps tempting to write
 \begin{equation}\label{eqns_dwdt}
  \begin{aligned}
   \pd wt &= \pd wx\pd xt + \pd wy\pd yt + \pd wt\pd tt \\
          &= \pd wx\pd xt + \pd wy\pd yt + \pd wt
  \end{aligned}
 \end{equation}
(since $\pd tt = 1$).  The trouble with this is that the $\pd wt$ on the left is not the same
as the one on the right.  The $\pd tt$ on the right refers only to the rate of change of $w$
with respect to $t$ insofar as $t$ appears \emph{explicitly} in the formula for~$w$; the one
on the left takes into account the fact that in addition $w$ depends \emph{implicitly} on $t$
\emph{via} the variables $x$ and~$y$.  What to do?  Use functions.  Relate the variables by
functions as follows.
 \begin{equation}\label{pic_vars1}
  \begin{CD}
   \begin{matrix} s \\ t \end{matrix}  @>f>>
   \begin{matrix} x  \\ y \\ t \end{matrix} @>g>> w
  \end{CD}
 \end{equation}
Also let $h = g \circ f$.  Notice that $f^3 = \pi_2$ (that is, $f^3(s,t) = t$).  Then
according to the \emph{chain rule}
  \[ h_2 = \sum_{k=1}^3 (g_k \circ f)\, f_2^k\,. \]
But $f_2^3 = 1$ (that is, $\pd tt = 1$).  So
 \begin{equation}\label{eqn_dwdt2}
    h_2 = (g_1 \circ f)f_2^1 + (g_2 \circ f)f_2^2 + g_3 \circ f.
 \end{equation}
The ambiguity of \eqref{eqns_dwdt} has been eliminated in~\eqref{eqn_dwdt2}.  The $\pd wt$ on
the left is seen to be the derivative with respect to $t$ of the composite $h = g \circ f$,
whereas the $\pd wt$ on the right is just the derivative with respect to $t$ of the
function~$g$.

One last point.  Many scientific workers adamantly refuse to give names to
functions.  What do they do?  Look back at diagram~\eqref{pic_vars1} and remove
the names of the functions.
 \begin{equation}\label{pic_vars2}
  \begin{CD}
   \begin{matrix} s \\ t \end{matrix}  @>>>
   \begin{matrix} x  \\ y \\ t \end{matrix} @>>> w
  \end{CD}
 \end{equation}
The problem is that the symbol ``$t$'' occurs twice.  To specify differentiation of the
composite function (our $h$) with respect to~$t$, indicate that the ``$t$'' you are interested
in is the one in the left column of~\eqref{pic_vars2}.  This may be done by listing everything
else that appears in that column. That is, specify which variables are held constant.  This
specification conventionally appears as a subscript outside parentheses.  Thus the $\pd wt$ on
the left of~\eqref{eqns_dwdt} (our $h_2$) is written as $\cpd wts$ (and is read, ``$\pd wt$
with $s$ held constant'').  Similarly, the $\pd wt$ on the right of~\eqref{eqns_dwdt}) (our
$g_3$) involves differentiation with respect to $t$ while $x$ and $y$ are fixed.  So it is
written $\cpd wt{x,y}$ (and is read, ``$\pd wt$ with $x$ and $y$ held constant).
Thus~\eqref{eqns_dwdt} becomes
  \begin{equation}\label{eqn_dwdt3}
     \cpd wts =  \pd wx\pd xt + \pd wy\pd yt + \cpd wt{x,y}
  \end{equation}

It is not necessary to write, for example, an expression such as $\cpd wx{t,y}$ because there
is no ambiguity; the symbol ``$x$'' occurs only once in~\eqref{pic_vars2}.  If you choose to
use the convention just presented, it is best to use it only to avoid confusion; use it
because you \emph{must}, not because you \emph{can}.

\begin{exer}\label{exer_crpd_xmp2}  Let $w = t^3 + 2yx^{-1}$ where $x = s^2 + t^2$ and $y =
s \arctan t$.  Use the \emph{chain rule} to find $\cpd wts$ at the point where $s = t = 1$.
(Solution~\ref{sol_exer_crpd_xmp2}.)
\end{exer}

We conclude this section with two more exercises on the use of the \emph{chain rule}.  Part of
the difficulty here and in the problems at the end of the section is to interpret correctly
what the problem says.  The suggested solutions may seem longwinded, and they are.
Nevertheless these techniques prove valuable in situations complicated enough to be confusing.
With practice it is easy to do many of the indicated steps mentally.

\begin{exer}\label{exer_crpd_xmp3}  Show that if $z = xy + x\phi(yx^{-1})$, then $x \pd zx +
y \pd zy = xy + z$. \emph{Hint.} Start by restating the exercise in terms of functions.  Add
suitable hypotheses.  In particular, suppose that $\phi \colon \R \sto \R$ is differentiable.
Let
  \[ j(x,y) = xy + x\phi(yx^{-1}) \]
for $x$, $y \in \R$, $x \ne 0$.  Then for each such $x$ and $y$
  \begin{equation}\label{eqn_crpd_xmp3}
      xj_1(x,y) + yj_2(x,y) = xy + j(x,y)\,.
  \end{equation}
To prove this assertion proceed as follows.
 \begin{enumerate}
  \item[(a)] Let $g(x,y) = yx^{-1}$.  Find $\bigl[dg_{(x,y)}\bigr]$.
  \item[(b)] Find $\bigl[d(\phi \circ g)_{(x,y)}\bigr]$.
  \item[(c)] Let $G(x,y) = \bigl(x,\phi(yx^{-1})\bigr)$.  Use (b) to find
$\bigl[dG_{(x,y)}\bigr]$.
  \item[(d)] Let $m(x,y) = xy$. Find $\bigl[dm_{(x,y)}\bigr]$.
  \item[(e)] Let $h(x,y) = x\phi(yx^{-1})$.  Use (c) and (d) to find $\bigl[dh_{(x,y)}\bigr]$.
  \item[(f)] Use (d) and (e) to find $\bigl[dj_{(x,y)}\bigr]$.
  \item[(g)] Use (f) to prove~\eqref{eqn_crpd_xmp3}.
 \end{enumerate}
(Solution~\ref{sol_exer_crpd_xmp3}.)
\end{exer}

\begin{exer}\label{exer_crpd_xmp4}  Show that if $f(u,v) = g(x,y)$ where $f$ is a differentiable
real valued function, $u = x^2 - y^2$, and $v = 2xy$, then
  \begin{equation}\label{eqn_crpd_xmp4a}
      y \pd gx - x \pd gy = 2v \pd fu - 2u \pd fv
  \end{equation}
\emph{Hint.}  The equations $u = x^2 - y^2$ and $v = 2xy$ give $u$ and $v$ in terms of $x$
and~$y$.  Think of the function $h\colon (x,y) \mapsto (u,v)$ as a change of variables
in~$\R^2$.  That is, define
  \[ h \colon \R^2 \sto \R^2 \colon (x,y) \mapsto (x^2 - y^2, 2xy)\,. \]

Then on the $uv$-plane (that is, the codomain of~$h$) the function $f$ is real valued and
differentiable.  The equation $f(u,v) = g(x,y)$ serves only to fix notation.  It indicates
that $g$ is the composite function $f \circ h$.  We may visualize the situation thus.
 \begin{equation}
  \begin{CD}
   \begin{matrix} x \\ y \end{matrix}  @>h>>
   \begin{matrix} u \\ v \end{matrix} @>f>> w
  \end{CD}
 \end{equation}
where $g = f \circ h$.

Now what are we trying to prove?  The conclusion~\eqref{eqn_crpd_xmp4a} is clear enough if we
evaluate the partial derivatives at the right place.  Recalling that we have defined $h$ so
that $u = h^1(x,y)$ and $v = h^2(x,y)$, we may write~\eqref{eqn_crpd_xmp4a} in the following
form.
  \begin{equation}\label{eqn_crpd_xmp4b}
      y g_1(x,y) - x g_2(x,y) = 2h^2(x,y)f_1(h(x,y)) - 2h^1(x,y)f_2(h(x,y))\,.
  \end{equation}
Alternatively we may write
  \[ \pi_2 g_1 - \pi_1 g_2 = 2 h^2f_1 - 2h^1f_2 \]
(where $\pi_1$ and $\pi_2$ are the usual coordinate projections).  To
verify~\eqref{eqn_crpd_xmp4b} use the \emph{chain rule} to find $\bigl[dg_{(x,y)}\bigr]$.
(Solution~\ref{sol_exer_crpd_xmp4}.)
\end{exer}

\begin{prob}  Let $w = \frac12x^2y + \arctan(tx)$ where $x = t^2 - 3u^2$ and $y = 2tu$.  Find
$\cpd wtu$ when $t = 2$ and $u = -1$.
\end{prob}

\begin{prob}  Let $z = \frac1{16}uw^2xy$ where $w = t^2 - u^2 + v^2$, $x = 2tu + tv$, and $y = 3uv$.
Find $\cpd zu{t,v}$ when $t = 1$, $u = -1$, and $v = -2$.
\end{prob}

\begin{prob}  If $z = f\biggl(\frac{x-y}y\biggr)$, then $x \pd zx + y \pd zy = 0$.  State this
precisely and prove it.
\end{prob}

\begin{prob}  If $\phi$ is a differentiable function on an open subset of $\R^2$ and $w =
\phi(u^2 - t^2, t^2 - u^2)$, then $t \pd wu + u \pd wt = 0$.  \emph{Hint.}  Let $h(t,u) = (u^2
- t^2, t^2 - u^2)$ and $w = \psi(t,u)$ where $\psi = \phi \circ h$.  Compute
$\bigl[dh_{(t,u)}\bigr]$.  Use the \emph{chain rule} to find $\bigl[d\psi_{(t, u)}\bigr]$.
Then simplify $t \psi_2(t,u) + u \psi_1(t,u)$.
\end{prob}

\begin{prob}  If $f(u,v) = g(x,y)$ where $f$ is a differentiable real valued function on $\R^2$
and if $u = x^3 + y^3$ and $v = xy$, then
  \[ x \pd gx + y \pd gy = 3u \pd fu + 2v \pd fv\,. \]
\end{prob}

\begin{prob}  Let $f(x,y) = g(r,\theta)$ where $(x,y)$ are Cartesian coordinates and $(r,\theta)$
are polar coordinates in the plane.  Suppose that $f$ is differentiable at all $(x,y)$
in~$\R^2$.
 \begin{enumerate}
  \item[(a)]  Show that except at the origin
      \[ \pd fx  = (\cos \theta)\pd gr - \frac1r(\sin \theta)\pd g\theta\,. \]
  \item[(b)] Find a similar expression for $\pd fy$.
 \end{enumerate}
\emph{Hint.}  Recall that Cartesian and polar coordinates are related by $x = r\cos\theta$ and
$y = r\sin\theta$.
\end{prob}

\begin{prob} Let $n$ be a fixed positive integer.  A function $f \colon \R^2 \sto \R^2$ is
 \index{homogeneous}%
\df{homogeneous of degree} $n$ if $f(tx,ty) = t^nf(x,y)$ for all $t$, $x$, $y \in \R$.  If
such a function $f$ is differentiable, then
  \begin{equation}\label{eqn_crpd_homog}
       x \pd fx + y \pd fy = nf\,.
  \end{equation}
\emph{Hint.}  Try the following:
 \begin{enumerate}
  \item[(a)] Let $G(x,y,t) = (tx,ty)$.  Find $\bigl[dG_{(t,x,y)}\bigr]$.
  \item[(b)] Let $h = f \circ G$.  Find $\bigl[dh_{(x,y,t)}\bigr]$.
  \item[(c)] Let $H(x,y,t) = (t^n, f(x,y))$.  Find $\bigl[dH_{(x,y,t)}\bigr]$.
  \item[(d)] Let $k = m \circ H$ (where $m(u,v) = uv$).  Find $\bigl[dk_{(x,y,t)}\bigr]$.
  \item[(e)] By hypothesis $h = k$; so the answers to (b) and (d) must be the same.  Use this
fact to derive~\eqref{eqn_crpd_homog}.
 \end{enumerate}
\end{prob}





\endinput
