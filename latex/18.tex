\chapter{COMPLETE SPACES}

\section{CAUCHY SEQUENCES}
\begin{defn} A sequence $(x_n)$ in a metric space is a
 \index{Cauchy!sequence}%
 \index{sequence!Cauchy}%
\df{Cauchy sequence} if for every $\epsilon > 0$ there exists $n_0 \in \N$ such that
$d(x_m,x_n) < \epsilon$ whenever $m$, $n \ge n_0$. This condition is often abbreviated as
follows: $d(x_m, x_n) \sto 0$ as $m$, $n \sto \infty$ (or $\lim_{m,n \sto \infty}d(x_m,x_n) =
0$).
\end{defn}

\begin{exam} In the metric space $\R$ the sequence $(1/n)$ is Cauchy.
\end{exam}

\begin{proof} Given $\epsilon > 0$ choose $n_0 > 2/\epsilon$. If $m$, $n > n_0$, then $d(1/m, 1/n)
= \abs{(1/m) - (1/n)}  \le (1/m) + (1/n) \le 2/n_0 < \epsilon$.  Notice that in $\R$ this
sequence is also convergent.
\end{proof}

\begin{exam}\label{compl_exm1} In the metric space $\R \setminus \{0\}$ the sequence $(1/n)$
is Cauchy. (The proof is exactly the same as in the preceding example.)  Notice, however, that
this sequence does not converge in $\R \setminus \{0\}$.
\end{exam}

\begin{prop}\label{conv_cau} In a metric space every convergent sequence is Cauchy.
\end{prop}

\begin{proof} Exercise.  (Solution~\ref{sol_conv_cau}.)
  \ns  \end{proof}

\begin{prop}\label{cs_css} Every Cauchy sequence which has a convergent subsequence is itself
convergent (and to the same limit as the subsequence).
\end{prop}

\begin{proof} Exercise.   (Solution~\ref{sol_cs_css}.)  \ns  \end{proof}

\begin{prop}\label{cau_bdd} Every Cauchy sequence is bounded.
\end{prop}

\begin{proof} Exercise.   (Solution~\ref{sol_cau_bdd}.)  \ns  \end{proof}

Although every convergent sequence is Cauchy (proposition~\ref{conv_cau}), the converse need
not be true (example~\ref{compl_exm1}). Those spaces for which the converse is true are said
to be \emph{complete}.







\section{COMPLETENESS}
\begin{defn} A metric space $M$ is
 \index{complete}%
\df{complete} if every Cauchy sequence in $M$ converges to a point of~$M$.
\end{defn}

\begin{exam}\label{r_compl} The metric space $\R$ is complete.
\end{exam}

\begin{proof} Let $(x_n)$ be a Cauchy sequence in~$\R$.  By proposition~\ref{cau_bdd}
the sequence $(x_n)$ is bounded; by corollary~\ref{bdd_convss} it has a convergent
subsequence; and so by proposition~\ref{cs_css} it converges.
\end{proof}

\begin{exam} If $(x_n)$ and $(y_n)$ are Cauchy sequences in a metric space, then
$\left(d\bigl(x_n,y_n\bigr)\right)_{n=1}^\infty$ is a Cauchy sequence in~$\R$.
\end{exam}

\begin{proof}  Problem. \emph{Hint.}  Proposition~\ref{ms_ineq}.
\ns \end{proof}

\begin{exam} The set $\Q$ of rational numbers (regarded as a subspace of $\R$) is not complete.
\end{exam}

\begin{proof} Problem.  \ns  \end{proof}

\begin{prop}\label{prop_cpt_impl_cmpl} Every compact metric space is complete.
\end{prop}

\begin{proof} Problem. \emph{Hint.} Theorem~\ref{cpt_scpt} and
proposition~\ref{cs_css}.  \ns  \end{proof}

\begin{prob} Let $M$ be a metric space with the discrete metric.
 \begin{enumerate}
  \item[(a)] Which sequences in $M$ are Cauchy?
  \item[(b)] Show that $M$ is complete.
 \end{enumerate}
\end{prob}

\begin{prob} Show that completeness is not a topological property.
\end{prob}

\begin{prop}\label{prop_clss_ms} Let $M$ be a complete metric space and $M_0$ be a subspace
of~$M$.  Then $M_0$ is complete if and only if it is a closed subset of~$M$.
\end{prop}

\begin{proof} Problem.  \ns  \end{proof}

\begin{prop}\label{prod_compl} The product of two complete metric spaces is complete.
\end{prop}

\begin{proof} Exercise. (Solution~\ref{sol_prod_compl}.)  \ns  \end{proof}

\begin{prop}\label{eq_compl} If $d$ and $\rho$ are strongly equivalent metrics on a
set~$M$, then the space $(M,d)$ is complete if and only if $(M,\rho)$ is.
\end{prop}

\begin{proof} Exercise. (Solution~\ref{sol_eq_compl}.)  \ns  \end{proof}

\begin{exam}\label{Rn_compl} With its usual metric the space $\R^n$
is complete.
\end{exam}

\begin{proof} Since $\R$ is complete (\ref{r_compl}), proposition~\ref{prod_compl} and
induction show that $\R^n$ is complete under the metric $d_1$ (defined in \ref{taxicab}).
Since the usual metric is strongly equivalent to $d_1$, we may conclude from
proposition~\ref{eq_compl} that $\R^n$ is complete under its usual metric.
\end{proof}

Here is one more example of a complete metric space.

\begin{exam}\label{bdd_compl} If $S$ is a set, then the metric space $\fml B(S,\R)$ is complete.
\end{exam}

\begin{proof} Exercise.   (Solution~\ref{sol_bdd_compl}.) \end{proof}

\begin{exam}\label{CMR_cmpl} If $M$ is a compact metric space, then $\fml C(M,\R)$ is a
complete metric space.
\end{exam}

\begin{proof} Problem.  \ns   \end{proof}

\begin{prob} Give examples of metric spaces $M$ and $N$, a homeomorphism $f \colon M \sto N$,
and a Cauchy sequence $(x_n)$ in $M$ such that the sequence $\bigl(f(x_n)\bigr)$ is \emph{not}
Cauchy in~$N$.
\end{prob}

\begin{prob} Show that if $D$ is a dense subset of a metric space $M$ and every Cauchy sequence
in $D$ converges to a point of~$M$, then $M$ is complete.
\end{prob}









\section{COMPLETENESS VS. COMPACTNESS}

In proposition~\ref{prop_cpt_impl_cmpl} we saw that every compact metric space is complete.
The converse of this is not true without additional assumptions. (Think of the reals.) In the
remainder of this chapter we show that adding total boundedness to completeness will suffice.
For the next problem we require the notion of the ``diameter'' of a set in a metric space.

\begin{defn} The
 \index{diameter@$\diam A$ (diameter of a set)}%
\df{diameter} of a subset $A$ of a metric space is defined by
   \[ \diam A := \sup\{d(x,y) \colon x,y \in A\} \]
if this supremum exists.  Otherwise $\diam A := \infty$.
\end{defn}

\begin{prob}  Show that $\operatorname{diam} A = \diam \clo A$ for every subset $A$ of a metric space.
\end{prob}

\begin{prop} In a metric space $M$ the following are equivalent:
 \begin{enumerate}
  \item[(1)] $M$ is complete.
  \item[(2)] Every nested sequence of nonempty closed sets in $M$ whose diameters approach
$0$ has nonempty intersection.
 \end{enumerate}
\end{prop}

\begin{proof} Problem. \emph{Hint.} For the definition of ``nested'' see~\ref{nested}.
To show that (1) implies (2), let $(F_k)$ be a nested sequence of nonempty closed subsets
of~$M$. For each $k$ choose $x_k \in F_k$.  Show that the sequence $(x_k)$ is Cauchy.  To show
that (2) implies (1), let $(x_k)$ be a Cauchy sequence in $M$.  Define $A_n = \{x_k \colon k
\ge n\}$ and $F_n = \clo{A_n}$.  Show that $(F_n)$ is a nested sequence of closed sets whose
diameters approach~$0$ (see the preceding problem).  Choose a point $a$ in $\cap F_n$.  Find a
subsequence $\left(x_{n_k}\right)$ of $(x_n)$ such that $d\left(a,x_{n_k}\right) < 2^{-k}$.
\ns
\end{proof}

\begin{prob} Since $\R$ is complete, the preceding problem tells us that every nested sequence
of nonempty closed subsets of $\R$ whose diameters approach $0$ has nonempty intersection.
 \begin{enumerate}
  \item[(a)] Show that this statement is no longer correct if the words ``whose diameters
approach~$0$'' are deleted.
  \item[(b)] Show that the statement is no longer correct if the word ``closed'' is deleted.
 \end{enumerate}
\end{prob}

\begin{prop} In a totally bounded metric space every sequence has a Cauchy subsequence.
\end{prop}

\begin{proof} Problem.  \emph{Hint.}  Let $(x_n)$ be a sequence in a totally bounded metric
space~$M$.  For every $n \in \N$ the space $M$ can be covered by a finite collection of open
balls of radius $1/n$.  Thus, in particular, there is an open ball of radius $1$ which
contains infinitely many of the terms of the sequence~$(x_n)$.  Show that it is possible
inductively to choose subsets $N_1, N_2, \dots$ of $\N$ such that for every $m,n \in \N$
 \begin{enumerate}
  \item[(i)] $n > m$ implies $N_n \subseteq N_m$,
  \item[(ii)] $N_n$ is infinite, and
  \item[(iii)] $\{x_k: k \in N_n\}$ is contained in some open ball of
radius $1/n$.
 \end{enumerate}
Then show that we may choose (again, inductively) $n_1, n_2, \dots$
in $\N$ such that for every $j,k \in \N$
 \begin{enumerate}
  \item[(iv)] $k > j$ implies $n_k > n_j$, and
  \item[(v)] $n_k \in N_k$.
 \end{enumerate}
Finally, show that the sequence $(x_n)$ is Cauchy.   \ns
\end{proof}

\begin{prop} A metric space is compact if and only if it is complete and totally bounded.
\end{prop}

\begin{proof} Problem.   \ns   \end{proof}

\begin{prob}  Let $(x_n)$ be a sequence of real numbers with the property that each term of
the sequence (from the third term on) is the average of the two preceding terms.  Show that
the sequence converges and find its limit.  \emph{Hint.}  Proceed as follows.

(a) Compute the distance between $x_{n+1}$ and $x_n$ in terms of the distance between $x_n$
and~$x_{n-1}$.

(b) Show (inductively) that
   \[ \abs{x_{n+1} - x_n} = 2^{1-n}\abs{x_2 - x_1}\,. \]

(c) Prove that $(x_n)$ has a limit by showing that for $m < n$
   \[ \abs{x_n - x_m} \le 2^{2-m}\abs{x_2 - x_1}\,. \]

(d)  Show (again inductively) that $2x_{n+1} + x_n = 2x_2 + x_1$.
\end{prob}

\begin{prob} Show that if $(x_n)$ is a sequence lying in the
interval $[-1,1]$ which satisfies
   \[ \abs{x_{n+1} - x_n} \le \tfrac14 \abs{x_n^2 - x_{n-1}^2} \qquad\text{ for $n \ge 2$}\,, \]
then $(x_n)$ converges.
\end{prob}





\endinput
