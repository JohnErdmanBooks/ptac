\chapter{CONTINUOUS FUNCTIONS FROM $\R$ TO $\R$}\label{cont_on_R}


The continuous functions are perhaps the most important single class of functions studied in
calculus.  Very roughly, a function $f\colon\R \sto \R$ is continuous at a point $a$ in $\R$
if $f(x)$ can be made arbitrarily close to $f(a)$ by insisting that $x$ be sufficiently close
to $a$.  In this chapter we define continuity for real valued functions of a real variable and
derive some useful necessary and sufficient conditions for such a function to be continuous.
Also we will show that composites of continuous functions are themselves continuous.  We will
postpone until the next chapter proofs that other combinations (sums, products, and so on) of
continuous functions are continuous.  The first applications of continuity will come in
chapters \ref{ivt} and \ref{evt} on connectedness and compactness.





\section{CONTINUITY---AS A LOCAL PROPERTY}
The definition of continuity uses the notion of the inverse image of a set under a function.
It is a good idea to look at appendices \ref{images} and \ref{inverses}, at least to fix
notation.

\begin{defn}\label{df_continuity} A function $f\colon\R \sto \R$ is
 \index{continuous!at a point}%
\df{continuous at} a point $a$ in $\R$ if $f^\gets(V)$ contains a neighborhood of $a$ whenever
$V$ is a neighborhood of~$f(a)$. Here is another way of saying exactly the same thing: $f$ is
continuous at $a$ if every neighborhood of $f(a)$ contains the image under $f$ of a
neighborhood of~$a$.  (If it is not entirely clear that these two assertions are equivalent,
use propositions \ref{prop_f_finv}\,(a) and \ref{prop_finv_f}\,(a) to prove that $U \subseteq
f^\gets(V)$ if and only if $f^\sto(U) \subseteq V$.)
\end{defn}

As we saw in chapter \ref{nbhds_in_r}, it seldom matters whether we work with general
neighborhoods (as in the preceding definition) or with the more restricted
$\epsilon$-neighborhoods (as in the next proposition).

\begin{prop}\label{cond1_cont} A function $f\colon\R \sto \R$ is continuous at $a \in \R$
if and only if for every $\epsilon > 0$ there exists $\delta > 0$ such that
 \begin{equation}\label{continuity1}
        J_\delta(a) \subseteq
                   f^\gets \bigl(J_\epsilon(f(a))\bigr)
 \end{equation}
\end{prop}

Before starting on a proof it is always a good idea to be certain that the meaning of the
proposition is entirely clear.  In the present case the ``if and only if'' tells us that we
are being given a condition \emph{characterizing} continuity at the point~$a$; that is, a
condition which is both necessary and sufficient in order for the function to be continuous
at~$a$. The condition states that no matter what positive number $\epsilon$ we are given, we
can find a corresponding positive number $\delta$ such that property \eqref{continuity1}
holds. This property is the heart of the matter and should be thought of conceptually rather
than in terms of symbols.  Learning mathematics is easiest when we regard the content of
mathematics to be ideas; it is hardest when mathematics is thought of as a game played with
symbols.  Thus property \eqref{continuity1} says that if $x$ is any number in the open
interval $(a-\delta, a+\delta)$, then the corresponding value $f(x)$ lies in the open interval
between $f(a)-\epsilon$ and $f(a)+\epsilon$.  Once we are clear about what it is that we wish
to establish, it is time to turn to the proof.

\begin{proof} Suppose $f$ is continuous at $a$.  If $\epsilon > 0$, then $J_\epsilon(f(a))$
is a neighborhood of $f(a)$ and therefore $f^\gets\bigl(J_\epsilon(f(a))\bigr)$ contains a
neighborhood $U$ of $a$. Since $a$ is an interior point of $U$, there exists $\delta
> 0$ such that $J_\delta(a) \subseteq U$.  Then
  \[J_\delta(a) \subseteq U \subseteq f^\gets\bigl(J_\epsilon(f(a))\bigr)\,.\]

Conversely, suppose that for every $\epsilon > 0$ there exists $\delta > 0$ such that
$J_\delta(a) \subseteq f^\gets\bigl(J_\epsilon(f(a))\bigr)$.  Let $V$ be a neighborhood of
$f(a)$.  Then $J_\epsilon(f(a)) \subseteq V$ for some $\epsilon > 0$.  By hypothesis, there
exists $\delta > 0$ such that $J_\delta(a) \subseteq f^\gets\bigl(J_\epsilon(f(a))\bigr)$.
Then $J_\delta(a)$ is a neighborhood of $a$ and $J_\delta(a) \subseteq f^\gets(V)$; so $f$ is
continuous at~$a$.
\end{proof}

We may use the remarks preceding the proof of \ref{cond1_cont} to give a second
characterization of continuity at a point.  Even though the condition given is algebraic in
form, it is best to think of it geometrically.  Think of it in terms of distance.  In the next
corollary read ``$\abs{x-a}<\delta$'' as ``$x$ is within $\delta$ units of $a$'' (not as ``the
absolute value of $x$ minus $a$ is less than $\delta$'').  If you follow this advice,
statements about continuity, when we get to metric spaces, will sound familiar.  Otherwise,
everything will appear new and strange.


\begin{cor}\label{cond2_cont} A function $f\colon\R \sto \R$ is continuous at $a$ if and
only if for every $\epsilon > 0$ there exists $\delta > 0$ such that
 \begin{equation}\label{continuity2}
   \abs{x - a} < \delta  \implies \abs{f(x) - f(a)} < \epsilon\,.
 \end{equation}
\end{cor}

\begin{proof} This is just a restatement of proposition \ref{cond1_cont} because condition
\eqref{continuity1} holds if and only if
 \[ x \in J_\delta(a) \implies  f(x) \in J_\epsilon(f(a))\,. \]
But $x$ belongs to the interval $J_\delta(a)$ if and only if $\abs{x - a} < \delta$, and
$f(x)$ belongs to $J_\epsilon(f(a))$ if and only if $\abs{f(x) - f(a)} < \epsilon$. Thus
\eqref{continuity1} and \eqref{continuity2} say exactly the same thing.
\end{proof}

Technically, \ref{df_continuity} is the definition of continuity at a point, while
\ref{cond1_cont} and \ref{cond2_cont} are characterizations of this property.
Nevertheless, it is not altogether wrong-headed to think of them as alternative (but
equivalent) definitions of continuity.  It really doesn't matter which one we choose to
be \emph{the} definition. Each of them has its uses.  For example, consider the result:
if $f$ is continuous at $a$ and $g$ is continuous at $f(a)$, then $g \circ f$ is
continuous at $a$ (see~\ref{comp_cont_loc}). The simplest proof of this uses
\ref{df_continuity}.  On the other hand, when we wish to verify that some particular
function is continuous at a point (see, for example, \ref{cont_exam1}), then, usually,
\ref{cond2_cont} is best.  There are other characterizations of continuity (see
\ref{cond3_cont}--\ref{cond7_cont}).  Before embarking on a particular problem involving
this concept, it is wise to take a few moments to reflect on which choice among the
assorted characterizations is likely to produce the simplest and most direct proof. This
is a favor both to your reader and to yourself.











\section{CONTINUITY---AS A GLOBAL PROPERTY}
\begin{defn} A function $f\colon\R \sto \R$ is
 \index{continuous}%
\df{continuous} if it is continuous at every point in its domain.
\end{defn}

The purpose of the next few items is to give you practice in showing that particular functions
are (or are not) continuous.  If you did a lot of this in beginning calculus, you may wish to
skip to the paragraph preceding proposition~\ref{cond3_cont}.

\begin{exam}\label{cont_exam1} The function $f\colon\R \sto \R\colon x \mapsto -2x + 3$ is
continuous.
\end{exam}

\begin{proof} We use corollary \ref{cond2_cont}.  Let $a \in \R$. Given $\epsilon > 0$ choose
$\delta = \frac12 \epsilon$. If $\abs{x - a} <  \delta$,then
  \begin{align*} \abs{f(x)-f(a)} &= \abs{(-2x + 3) - (-2a + 3)} \\
                                 &= 2\abs{x - a} < 2\delta = \epsilon.
  \end{align*}
\end{proof}

\begin{exam} The function $f\colon\R \sto \R$ defined by
 \[f(x) = \begin{cases} 0  &\text{for $x \le 0$}, \\
                       1   &\text{for $x > 0$}  \end{cases}\]
is not continuous at $a = 0$.
\end{exam}

\begin{proof} Use proposition \ref{cond1_cont}; the denial of the condition given there is
that there exists a number $\epsilon$ such that for all $\delta > 0$ property
\ref{continuity1} fails. (See example~\ref{exam_not_cont}.)  Let $\epsilon = \frac12$. Then
$f^\gets\bigl(J_{1/2}(f(0))\bigr) = f^\gets(J_{1/2}(0)) = f^\gets(-\frac12, \frac12) =
(-\infty, 0]$.  Clearly this contains no $\delta$-neighborhood of $0$.  Thus \ref{continuity1}
is violated, and $f$ is not continuous at~$0$.
\end{proof}

\begin{exam}\label{cont_exam2} The function $f\colon\R \sto \R\colon x \mapsto 5x - 8$ is continuous.
\end{exam}

\begin{proof} Exercise. (Solution~\ref{sol_cont_exam2}.)  \ns \end{proof}

\begin{exam}\label{cont_exam3} The function $f\colon x \mapsto x^3$ is continuous at the
point~$a~=~-1$.
\end{exam}

\begin{proof} Exercise. (Solution~\ref{sol_cont_exam3}.)  \ns \end{proof}

\begin{exam}\label{cont_exam4} The function $f\colon x \mapsto 2x^2 - 5$ is continuous.
\end{exam}

\begin{proof} Exercise.  (Solution~\ref{sol_cont_exam4}.)
  \ns \end{proof}

\begin{exam} Let $f(x) = 7 - 5x$ for all $x \in \R$. Then $f$ is continuous at the point~$a = -2$.
\end{exam}

\begin{proof} Problem.  \ns \end{proof}

\begin{exam} Let $f(x) = \sqrt{|x + 2|}$ for all $x \in \R$.  Then $f$ is continuous at the 
point~$a = 0$.
\end{exam}

\begin{proof} Problem. \ns \end{proof}

\begin{exam} If $f(x) = 3x - 5$ for all $x \in \R$, then $f$ is continuous.
\end{exam}

\begin{proof} Problem.  \ns \end{proof}

\begin{exam} The function $f$ defined by
 \[f(x) = \begin{cases}  -x^2        &\text{for $x < 0$}, \\
                    x+\frac1{10}    &\text{for $x \ge 0$}
          \end{cases}\]
is not continuous.
\end{exam}

\begin{proof} Problem.  \ns \end{proof}

\begin{exam} For $x > 0$ sketch the functions $x \mapsto \sin\frac1x$ and
$x \mapsto x\sin\frac1x$.  Then verify the following.
 \begin{enumerate}
  \item[(a)] The function $f$ defined by
   \[f(x) = \begin{cases}        0    &\text{for $x \le 0$}, \\
                      \sin\frac1x,    &\text{for $x > 0$}
            \end{cases}\]
is not continuous.
  \item[(b)]   The function $f$ defined by
   \[f(x) = \begin{cases}        0    &\text{for $x \le 0$}, \\
                    x \sin\frac1x,    &\text{for $x > 0$}
            \end{cases}\]
is continuous at $0$.
 \end{enumerate}
\end{exam}

\begin{proof} Problem.  \ns \end{proof}

The continuity of a function $f$ at a point is a
 \index{local!property}%
\df{local} property; that is, it is entirely determined by the behavior of $f$ in arbitrarily
small neighborhoods of the point. The continuity of $f$, on the other hand, is a
 \index{global!property}%
\df{global} property; it can be determined only if we know how $f$ behaves everywhere on its
domain.  In \ref{df_continuity}--\ref{cond2_cont} we gave three equivalent conditions for
local continuity.  In \ref{cond3_cont}--\ref{cond7_cont} we give equivalent conditions for the
corresponding global concept.  The next proposition gives the most useful of these conditions;
it is the one that becomes the \emph{definition} of continuity in arbitrary topological
spaces. It says: a necessary and sufficient condition for $f$ to be continuous is that the
inverse image under $f$ of open sets be open.  This shows that continuity is a purely
 \index{topological!property}%
\df{topological property}; that is, it is entirely determined by the topologies (families of
all open sets) of the domain and the codomain of the function.

\begin{prop}\label{cond3_cont} A function $f\colon\R \sto \R$ is continuous if and only if
$f^\gets(U)$ is open whenever $U$ is open in $\R$.
\end{prop}

\begin{proof} Exercise.  (Solution~\ref{sol_cond3_cont}.)
  \ns \end{proof}
























\begin{prop}\label{cond4_cont} A function $f\colon\R \sto \R$ is continuous if and only if
$f^\gets(C)$ is a closed set whenever $C$ is a closed subset of~$\R$.
\end{prop}

\begin{proof} Problem. \ns \end{proof}

\begin{prop} A function $f\colon\R \sto \R$ is continuous if and only if
 \[f^\gets(\intr B) \subseteq \intr{(f^\gets(B))}\]
for all $B \subseteq \R$.
\end{prop}


\begin{proof} Problem.  \ns \end{proof}

\begin{prop} A function $f\colon\R \sto \R$ is continuous if and only if
 \[f^\sto(\,\clo A\,) \subseteq \clo{f^\sto (A)}\]
for all $A \subseteq \R$.
\end{prop}

\begin{proof} Problem.  \emph{Hint.}  This isn't so easy.  Use problem \ref{cond4_cont}
and the fact (see propositions \ref{prop_f_finv} and \ref{prop_finv_f}) that for any sets $A$
and $B$
 \[f^\sto(f^\gets(B)) \subseteq B \quad\text{ and }
                     \quad A \subseteq f^\gets(f^\sto(A)).\]
Show that if $f$ is continuous, then $\clo A \subseteq
f^\gets\bigl(\,\clo{f^\sto(A)}\,\bigr)$. Then apply $f^\sto$.  For the converse, apply the
hypothesis to the set $f^\gets(C)$ where $C$ is a closed subset of $\R$.  Then apply
$f^\gets$.
\end{proof}

\begin{prop}\label{cond7_cont} A function $f\colon\R \sto \R$ is continuous if and only if
 \[\clo{f^\gets(B)} \subseteq f^\gets(\,\clo B\,)\]
for all $B \subseteq \R$.
\end{prop}

\begin{proof} Problem.  \ns \end{proof}

\begin{prop}\label{comp_cont_loc} Let $f,g\colon\R \sto \R$. If $f$ is continuous at $a$ and
$g$ is continuous at $f(a)$, then the composite function $g \circ f$ is continuous at $a$.
\end{prop}

\begin{proof} Let $W$ be a neighborhood of $g(f(a))$.  We wish to show that the inverse image
of $W$ under $g \circ f$ contains a neighborhood of~$a$.  Since $g$ is continuous at $f(a)$,
the set $g^\gets(W)$ contains a neighborhood $V$ of $f(a)$.  And since $f$ is continuous at
$a$, the set $f^\gets(V)$ contains a neighborhood $U$ of~$a$.  Then
 \begin{align*}
   (g \circ f)^\gets(W) &= f^\gets(g^\gets(W)) \\
                             &\supseteq f^\gets(V) \\
                             &\supseteq U\,,
 \end{align*}
which is what we wanted.
\end{proof}

This important result has an equally important but entirely obvious consequence.

\begin{cor}\label{comp_cont} The composite of two continuous functions is continuous.
\end{cor}

\begin{prob} Give a direct proof of corollary \ref{comp_cont}.  (That is, give a proof which
does not rely on proposition~\ref{comp_cont_loc}.)
\end{prob}









\section{FUNCTIONS DEFINED ON SUBSETS OF $\R$}
The remainder of this chapter is devoted to a small but important technical problem.  Thus far
the definitions and propositions concerning continuity have dealt with functions whose domain
is $\R$.  What do we do about functions whose domain is a proper subset of $\R$?  After all,
many old friends---the functions $x \mapsto \sqrt x$, $x \mapsto \frac1x$, and $x \mapsto \tan
x$, for example---have domains which are not all of~$\R$.  The difficulty is that if we were
to attempt to apply proposition \ref{cond1_cont} to the square root function $f\colon x
\mapsto \sqrt x$ (which would of course be improper since the hypothesis $f\colon\R \sto \R$
is not satisfied), we would come to the unwelcome conclusion that $f$ is not continuous at
$0$: if $\epsilon > 0$ then the set $f^\gets\bigl(J_\epsilon(f(0))\bigr) =
f^\gets\bigl(J_\epsilon(0)\bigr) = f^\gets(-\epsilon,\epsilon) = [0,{\epsilon}^2)$ contains no
neighborhood of $0$ in~$\R$.

Now this can't be right.  What we must do is to provide an appropriate definition for the
continuity of functions whose domains are proper subsets of $\R$.  And we wish to do it in
such a way that we make as few changes as possible in the resulting propositions.

The source of our difficulty is the demand (in definition \ref{df_continuity}) that
$f^\gets(V)$ contain a neighborhood of the point $a$---and neighborhoods have been defined
only in $\R$. But why~$\R$?  That is not the domain of our function; the set $A = [0,\infty)$
is.  We \emph{should} be talking about neighborhoods in~$A$.  So the question we now face is:
how should we define neighborhoods in (and open subsets of) proper subsets of~$\R$?  The best
answer is astonishingly simple.  An open subset of $A$ is the intersection of an open subset
of $\R$ with~$A$.

\begin{defn}\label{rel_open} Let $A \subseteq \R$. A set $U$ contained in $A$ is
 \index{open!in a subset}%
\df{open in} $A$ if there exists an open subset $V$ of $\R$ such that $U = V \cap A$.
Briefly, the open subsets of $A$ are restrictions to $A$ of open subsets of $\R$.  If $U$ is
an open subset of $A$ we write $\open UA$.  A \df{neighborhood of $a$ in $A$} is an open
subset of $A$ which contains~$a$.
\end{defn}

\begin{exam} The set $[0,1)$ is an open subset of~$[0,\infty)$.
\end{exam}

\begin{proof} Let $V = (-1,1)$. Then $\open V\R$ and $[0,1) = V
\cap [0,\infty)$; so $\open{[0,1)}{[0,\infty)}$.
\end{proof}

Since, as we have just seen, $[0,1)$ is open in $[0,\infty)$ but is not open in $\R$, there is
a possibility for confusion.  Openness is not an intrinsic property of a set.  When we say
that a set is open, the answer to the question ``open in what?'' must be either clear from
context or else specified.  Since the topology (that is, collection of open subsets) which a
set $A$ inherits from $\R$ is often called the
 \index{relative!topology}%
 \index{topology!relative}%
\df{relative topology} on $A$, emphasis may be achieved by saying
that a subset $B$ of $A$ is
 \index{relatively open}%
\emph{relatively open} in~$A$. Thus, for example, we may say that
$[0,1)$ is relatively open in $[0,\infty)$; or we may say that
$[0,1)$ is a
 \index{relative!neighborhood}%
 \index{neighborhood!relative}%
relative neighborhood of~$0$ (or any other point in the interval).
The question here is emphasis and clarity, not logic.

\begin{exam} The set $\{1\}$ is an open subset of~$\N$.
\end{exam}

\begin{proof} Problem.  \ns \end{proof}

\begin{exam} The set of all rational numbers $x$ such that $x^2 \le 2$ is an open subset of~$\Q$.
\end{exam}

\begin{proof} Problem.  \ns \end{proof}

\begin{exam} The set of all rational numbers $x$ such that $x^2 \le 4$ is \emph{not} an
open subset of~$\Q$.
\end{exam}

\begin{proof} Problem.  \ns \end{proof}

\begin{defn} Let $A \subseteq \R$, $a \in A$, and $\epsilon > 0$.  The
 \index{relative!$\epsilon$-neighborhood}%
$\epsilon$-\df{neighborhood of $a$ in $A$} is $(a-\epsilon, a+\epsilon) \cap A$.
\end{defn}

\begin{exam} In $\N$ the $\frac12$\,-\,neighborhood of $1$ is $\{1\}$.
\end{exam}

\begin{proof} Since $(1-\frac12, 1+\frac12) \cap \N = (\frac12, \frac32) \cap \N = \{1\}$,
we conclude that the $\epsilon$-neighborhood of $1$ is~$\{1\}$.
\end{proof}

Now is the time to show that the family of relatively open subsets of $A$ (that is, the
relative topology on $A$) is in fact a topology on~$A$.  In particular, we must show that this
family is closed under unions and finite intersections.

\begin{prop} Let $A \subseteq \R$.  Then
 \begin{enumerate}
  \item[(i)] $\emptyset$ and $A$ are relatively open in $A$;
  \item[(ii)] if $\sfml U$ is a family of relatively open sets in $A$, then $\bigcup\sfml U$
is relatively open in~$A$; and
  \item[(iii)] if $\sfml U$ is a finite family of relatively open sets in $A$, then
$\bigcap\sfml U$ is relatively open in~$A$.
 \end{enumerate}
\end{prop}

\begin{proof} Problem. \ns \end{proof}

\begin{exam}Every subset of $\N$ is relatively open in~$\N$.
\end{exam}

\begin{proof} Problem. \ns \end{proof}

Now we are in a position to consider the continuity of functions defined on proper subsets
of~$\R$.

\begin{defn} Let $a \in A \subseteq \R$ and $f\colon A \sto \R$.  The function $f$ is
 \index{continuous!at a point}%
\df{continuous at} $a$ if $f^\gets(V)$ contains a neighborhood of $a$ in $A$ whenever $V$ is a
neighborhood of $f(a)$. The function $f$ is
 \index{continuous}%
\df{continuous} if it is continuous at each point in its domain.
\end{defn}

It is important to notice that we discuss the continuity of a function only at points where it
is defined.  We will not, for example, make the claim found in so many beginning calculus
texts that the function $x \mapsto 1/x$ is discontinuous at zero. Nor will we try to decide
whether the sine function is continuous at the Bronx zoo.

The next proposition tells us that the crucial characterization of continuity in terms of
inverse images of open sets (see~\ref{cond3_cont}) still holds under the definition we have
just given.  Furthermore codomains don't matter; that is, it doesn't matter whether we start
with open subsets of $\R$ or with sets open in the range of the function.

\begin{prop}\label{cond3a_cont} Let $A$ be a subset of~$\R$.  A function $f\colon A \sto \R$
is continuous if and only if $f^\gets(V)$ is open in $A$ whenever $V$ is open in $\ran f$.
\end{prop}

\begin{proof} Problem.  \emph{Hint.} Notice that if $\open W{\R}$ and $V = W \cap \ran f$,
then $f^\gets(V) = f^\gets(W)$.  \ns
\end{proof}

\begin{prob}\label{cont_rel} Discuss the changes (if any) that must be made in \ref{cond1_cont},
\ref{cond2_cont}, \ref{comp_cont_loc}, and \ref{comp_cont}, in order to accommodate functions
whose domain is not all of~$\R$.
\end{prob}

\begin{prob} Let $A \subseteq \R$.  Explain what ``(relatively) closed subset of~$A$'' should mean.
Suppose further that $B \subseteq A$.  Decide how to define ``the closure of $B$ in~$A$'' and
``the interior of $B$ with respect to~$A$''.  Explain what changes these definitions will
require in propositions \ref{cond4_cont}--\ref{cond7_cont} so that the results hold for
functions whose domains are proper subsets of~$\R$.
\end{prob}

\begin{prob} Let $A = (0,1) \cup (1,2)$.  Define $f\colon A \sto \R$ by
  \[f(x) = \begin{cases}   0   &\text{ for $0<x<1$}, \\
                           1   &\text{ for $1<x<2$}\,.
\end{cases}\]
Is $f$ a continuous function?
\end{prob}

\begin{exam} The function $f$ defined by
  \[f(x) = \frac1x \qquad\text{ for } x \ne 0\]
is continuous.
\end{exam}

\begin{proof} Problem.  \ns \end{proof}

\begin{exam} The function $f$ defined by
  \[f(x) = \sqrt{x+2} \qquad\text{ for } x \ge -2\]
is continuous at $x=-2$.
\end{exam}

\begin{proof} Problem.  \ns \end{proof}

\begin{exam} The square root function is continuous.
\end{exam}

\begin{proof} Problem.  \ns \end{proof}

\begin{prob} Define $f\colon (0,1) \sto \R$ by setting $f(x) =0$ if $x$ is irrational and
$f(x) = 1/n$ if $x = m/n$ where $m$ and $n$ are natural numbers with no common factors.  Where
is $f$ continuous?
\end{prob}

\begin{exam}\label{cont_exam5} Inclusion mappings between subsets of $\R$ are continuous.
That is, if $A \subseteq B \subseteq \R$, then the inclusion mapping $\iota\colon A \sto
B\colon a \mapsto a$ is continuous.  (See definition~\ref{def_incl_map}.)
\end{exam}

\begin{proof} Let $\open UB$.  By the definition of the relative topology on $B$ (see~\ref{rel_open}),
there exists an open subset $V$ of $\R$ such that $U = V \cap B$.  Then $\iota^\gets(U) =
\iota^\gets(V \cap B) = V \cap B \cap A = \open{V \cap A}A$.  Since the inverse image under
$\iota$ of each open set is open, $\iota$ is continuous.
\end{proof}

It is amusing to note how easy it is with the preceding example in hand to show that
restrictions of continuous functions are continuous.

\begin{prop} Let $A \subseteq B \subseteq \R$.  If $f\colon B \sto \R$ is continuous,
then $f|_A$, the restriction of $f$ to $A$, is continuous.
\end{prop}

\begin{proof} Recall (see appendix~\ref{images} section~\ref{sec_res_ext}) that $f|_A =
f \circ \iota$ where $\iota$ is the inclusion mapping of $A$ into~$B$.  Since $f$ is
continuous (by hypothesis) and $\iota$ is continuous (by example~\ref{cont_exam5}), their
composite $f|_A$ is also continuous (by the generalization of \ref{comp_cont}
in~\ref{cont_rel}).
\end{proof}

We conclude this chapter with the observation that if a continuous function is positive at a
point, it is positive nearby.

\begin{prop}\label{pos_at_pt} Let $A \subseteq \R$ and $f\colon A \sto \R$ be continuous at
the point~$a$.  If $f(a)>0$, then there exists a neighborhood $J$ of $a$ in $A$ such that
$f(x) > \frac12f(a)$ for all $x \in J$.
\end{prop}

\begin{proof} Problem. \ns \end{proof}




\endinput
