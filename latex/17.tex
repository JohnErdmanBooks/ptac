\chapter{CONNECTEDNESS}\label{conn}
In chapter~\ref{ivt} we discussed connected subsets of the real line.  Although they are
easily characterized (they are just the intervals), they possess important properties, most
notably the \emph{intermediate value property}.  Connected subsets of arbitrary metric spaces
can be somewhat more complicated, but they are no less important.






\section{CONNECTED SPACES}
\begin{defn}\label{def_conn} A metric space $M$ is
 \index{disconnected!metric space}%
\df{disconnected} if there exist disjoint nonempty open sets $U$ and $V$ whose union is~$M$.
In this case we say that the sets $U$ and $V$ \df{disconnect}~$M$.  A metric space is
 \index{connected!metric space}%
\df{connected} if it is not disconnected.  A subset of a metric space M is
 \index{connected!subset}%
\df{connected} (respectively,
 \index{disconnected!subset}%
\df{disconnected}) if it is connected (respectively, disconnected) as a subspace of~$M$.  Thus
a subset $N$ of $M$ is disconnected if there exist nonempty disjoint sets $U$ and $V$ open in
the \emph{relative} topology on $N$ whose union is~$N$.
\end{defn}

\begin{exam}  Every discrete metric space with more than one point is disconnected.
\end{exam}

\begin{exam} The set $\Q^2$ of points in $\R^2$ both of whose coordinates are rational is a
disconnected subset of~$\R^2$.
\end{exam}

\begin{proof}  The subspace $\Q^2$ is disconnected by the sets $\{(x,y) \in \Q^2 \colon x < \pi\}$
and $\{(x,y) \in \Q^2 \colon x > \pi\}$.  (Why are these sets open in the relative topology
on~$\Q^2$?)
\end{proof}

\begin{exam} The following subset of $\R^2$ is not connected.
  \[ \{(x,x^{-1}) \colon x > 0\} \cup \{(0,y) \colon y \in \R\} \]
\end{exam}

\begin{proof} Problem.  \ns  \end{proof}

\begin{prop}\label{prop_dconn_2ptimg} A metric space $M$ is disconnected if and only if there
exists a continuous surjection from $M$ onto a two element discrete space, say $\{0,1\}$.  A
metric space $M$ is connected if and only if every continuous function $f$ from $M$ into a two
element discrete space is constant.
\end{prop}

\begin{proof} Problem.  \ns  \end{proof}

Proposition~\ref{prop_nasc_conn} remains true for arbitrary metric spaces; and the same proof
works.

\begin{prop}\label{prop_conn_oc} A metric space $M$ is connected
if and only if the only subsets of $M$ which are both open and
closed are the null set and $M$ itself.
\end{prop}

\begin{proof} Exercise.  (Solution~\ref{sol_prop_conn_oc}.)    \ns  \end{proof}

Just as in chapter~\ref{ivt}, dealing with the relative topology on a subset of a metric space
can sometimes be a nuisance.  The remedy used there is available here: work with mutually
separated sets.

\begin{defn} Two nonempty subsets $C$ and $D$ of a metric space $M$ are said to be
 \index{mutually separated}%
 \index{separated!mutually}%
\df{mutually separated} if
  \[ C \cap \clo D = \clo C \cap D = \emptyset\,. \]
\end{defn}

\begin{prop}\label{prop_mut_sep2} A subset $N$ of a metric space $M$ is disconnected if and
only if it is the union of two nonempty sets mutually separated in~$M$.
\end{prop}

\begin{proof} Exercise.  \emph{Hint.}  Make a few changes in the
proof of proposition~\ref{prop_mut_sep}.    (Solution~\ref{sol_prop_mut_sep2}.)  \ns
\end{proof}

\begin{prop}\label{prop_clconn_conn} If $A$ is a connected subset of a metric space, then any
set $B$ satisfying $A \subseteq B \subseteq \clo A$ is also connected.
\end{prop}

\begin{proof} Problem. \emph{Hint.} Use proposition~\ref{prop_mut_sep2}  \ns   \end{proof}

If a metric space is disconnected, it is often a rather simple job to demonstrate this fact.
All one has to do is track down two subsets which disconnect the space.  If the space is
connected, however, one is confronted with the unenviable task of showing that \emph{every}
pair of subsets fails for some reason to disconnect the space.  How, for example, does one go
about showing that the unit square $[0,1] \times [0,1]$ is connected?  Or the unit circle? Or
the curve $y = \sin x$?  In~\ref{thm_contimg_conn2}, \ref{prop_un_conn},
and~\ref{prop_arcw_conn} we give sufficient conditions for a metric space to be connected.
The first of these is that the space be the continuous image of a connected space .

\begin{thm}\label{thm_contimg_conn2} A metric space $N$ is connected if there exist a connected
metric space $M$ and a continuous surjection from $M$ onto~$N$.
\end{thm}

\begin{proof} Change ``$\R$'' to ``$M$'' in the proof of theorem~\ref{thm_contimg_conn}.
\end{proof}

\begin{exam}\label{exam_sin_conn} The graph of the curve $y = \sin x$ is a connected subset
of~$\R^2$.
\end{exam}

\begin{proof} Exercise.   (Solution~\ref{sol_exam_sin_conn}.) \ns  \end{proof}

\begin{exam} The unit circle $\{(x,y) \in \R^2 \colon x^2 + y^2 = 1\}$ is a connected subset of
the plane.
\end{exam}

\begin{proof} Problem.  \ns   \end{proof}

\begin{prop}\label{prop_un_conn} A metric space is connected if it is the union of a family of
connected subsets with nonempty intersection.
\end{prop}

\begin{proof} Exercise.  (Argue by contradiction.  Use definition~\ref{def_conn}.)
(Solution~\ref{sol_prop_un_conn}.) \ns
\end{proof}

\begin{prob} Use proposition~\ref{prop_dconn_2ptimg} to give a second proof of
proposition~\ref{prop_un_conn}.
\end{prob}

\begin{exam}\label{exam_sqr_conn} The closed unit square $[0,1] \times [0,1]$ in $\R^2$ is connected.
\end{exam}

\begin{proof} Exercise.   (Solution~\ref{sol_exam_sqr_conn}.)  \ns   \end{proof}








\section{ARCWISE CONNECTED SPACES}
A concept closely related to (but stronger than) connectedness is \emph{arcwise
connectedness}.

\begin{defn}  A metric space is
 \index{arcwise connected}%
 \index{connected!arcwise}%
\df{arcwise connected} (or
 \index{path connected}%
 \index{connected!path}%
\df{path connected}) if for every $x$, $y \in M$ there exists a continuous map $f \colon [0,1]
\sto M$ such that $f(0) = x$ and $f(1) = y$. Such a function $f$ is an
 \index{arc}%
\df{arc} (or
 \index{path}%
\df{path} , or
 \index{curve}%
\df{curve}) connecting $x$ to~$y$. It is very easy to prove that arcwise connected spaces are
connected (proposition~\ref{prop_arcw_conn}).  The converse is false
(example~\ref{exam_conn_notarcw}).  If, however, we restrict our attention to open subsets of
$\R^n$, then the converse does hold (proposition~\ref{prop_opconn_arcw}).
\end{defn}

\begin{prop}\label{prop_arcw_conn}  If a metric space is arcwise connected, then it is connected.
\end{prop}

\begin{proof} Problem.  \ns  \end{proof}

\begin{exam} The following subset of $\R^2$ is not connected.
\[ \begin{split}
     \{(x,y) \colon (x-1)^2 + (y-1)^2 &< 4\} \cup \{(x,y) \colon x < 0\} \\
                     &\cup  \{(x,y) \colon (x-10)^2 + (y-1)^2 < 49\}
   \end{split}\]
\end{exam}

\begin{proof} Problem.  \ns  \end{proof}

\begin{exam} The following subset of $\R^2$ is connected.
 \[ \begin{split}
       \{(x,y) \colon (x-1)^2 + (y-1)^2 &< 4\} \cup \{(x,y) \colon y < 0\} \\
         &\cup  \{(x,y) \colon (x-10)^2 + (y-1)^2 < 49\}
    \end{split}\]
\end{exam}

\begin{proof} Problem.  \ns  \end{proof}

\begin{exam} The following subset of $\R^2$ is connected.
  \[ \{(x,x^3 + 2x) \colon x \in \R\} \cup \{(x,x^2+56) \colon x \in \R\} \]
\end{exam}

\begin{proof} Problem.  \ns  \end{proof}

\begin{exam} Every open ball in $\R^n$ is connected.  So is every closed ball.
\end{exam}

\begin{proof} Problem.  \ns   \end{proof}

\begin{exam}\label{exam_conn_notarcw} Let $B = \{(x,\sin x^{-1}) \colon 0 < x \le 1\}$.  Then
$\clo B$ is a connected subset of $\R^2$ but is not arcwise connected.
\end{exam}

\begin{proof} Exercise. \emph{Hint.}  Let $M = \clo B$.  To show that $M$ is not arcwise
connected argue by contradiction.  Assume there exists a continuous function $f \colon [0,1]
\sto M$ such that $f(1) \in B$ and $f(0) \notin B$.  Prove that $f^2 = \pi_2 \circ f$ is
discontinuous at the point $t_0 = \sup f^\gets(M \setminus B)$.  To this end show that $t_0
\in f^\gets(M \setminus B)$.  Then, given $\delta > 0$, choose $t_1$ in $[0,1]$ so that $t_0 <
t_1 < t_0 + \delta$.  Without loss of generality one may suppose that $f^2(t_0) \le 0$.  Show
that $\bigl(f^1\bigr)^\sto[t_0,t_1]$ is an interval containing $0$ and $f^1(t_1)$ (where $f^1
= \pi_1 \circ f$).  Find a point $t$ in $[t_0,t_1]$ such that $0 < f^1(t) < f^1(t_1)$ and
$f^2(t) = 1$. (Solution~\ref{sol_exam_conn_notarcw}.) \ns
\end{proof}

\begin{prop}\label{prop_opconn_arcw} Every connected open subset of $\R^n$ is arcwise connected.
\end{prop}

\begin{proof} Exercise. \emph{Hint.}  Let $A$ be a connected open subset of $\R^n$ and
$a \in A$.  Let $U$ be the set of all points in $A$ which can be joined to $a$ by an arc
in~$A$.  Show that $A \setminus U$ is empty by showing that $U$ and $A \setminus U$
disconnect~$A$.   (Solution~\ref{sol_prop_opconn_arcw}.) \ns
\end{proof}

\begin{prob} Does there exist a continuous bijection from a closed disk in $\R^2$ to its
circumference?  Does there exist a continuous bijection from the interval $[0,1]$ to the
circumference of a disk in $\R^2$?
\end{prob}

\begin{prob}  Let $x$ be a point in a metric space~$M$.  Define the
 \index{components!of a metric space}%
\df{component} of $M$ containing $x$ to be the largest connected subset of $M$ which
contains~$x$.  Discover as much about components of metric spaces as you can.  First, of
course, you must make sure that the definition just given makes sense.  (How do we know that
there really \emph{is} a ``largest'' connected set containing~$x$?)

Here are some more things to think about.
 \begin{enumerate}
  \item The components of a metric space are a disjoint family whose union is the whole space.
  \item It is fairly clear that the components of a discrete metric space are the points of
the space.  If the components are points must the space be discrete?
  \item Components of a metric space are closed sets;  must they be open?
  \item Distinct components of a metric space are mutually separated.
  \item If a metric space $M$ is the union of two mutually separated sets $C$ and $D$ and if
points $x$ and $y$ belong to the same component of $M$, then both points are in $C$ or both
are in~$D$. What about the converse?  Suppose $x$ and $y$ are points in $M$ such that whenever
$M$ is written as the union of two mutually separated sets $C$ and $D$, both points lie in $C$
or both lie in~$D$.  Must $x$ and $y$ lie in the same component?
 \end{enumerate}
\end{prob}

\begin{prob} A function $f$ in $\fml C(M,\R)$, where $M$ is a metric space, is
 \index{idempotent}%
\df{idempotent} if $(f(x))^2 = f(x)$ for all $x \in M$.  The constant functions $0$ and $1$
are the \df{trivial} idempotents of $\fml C(M,\R)$.  Show that $\fml C(M,\R)$ possesses a
nontrivial idempotent if and only if the underlying metric space is disconnected. (This is one
of a large number of results which link algebraic properties of $\fml C(M,\R)$ to topological
properties of the underlying space~$M$.)
\end{prob}




\endinput
