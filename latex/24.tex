\chapter{THE CAUCHY INTEGRAL}\label{C_int}

In this chapter we develop a theory of integration for vector valued functions.  In one way
our integral, the Cauchy integral, will be more general and in another way slightly less
general than the classical Riemann integral, which is presented in beginning calculus.  The
Riemann integral is problematic: the derivation of its properties is considerably more
complicated than the corresponding  derivation for the Cauchy integral.  But very little added
generality is obtained as a reward for the extra work.  On the other hand the Riemann integral
is not nearly general enough for advanced work in analysis; there the full power of the
Lebesgue integral is needed.  Before starting our discussion of integration we derive some
standard facts concerning uniform continuity.

\section{UNIFORM CONTINUITY}

\begin{defn}  A function $f \colon M_1 \sto M_2$ between metric spaces is
 \index{uniform!continuity}%
 \index{continuous!uniformly}%
\df{uniformly continuous} if for every $\epsilon > 0$ there exists $\delta > 0$ such that
$d(f(x),f(y)) < \epsilon$ whenever $d(x,y) < \delta$.
\end{defn}

Compare the definitions of ``continuity'' and ``uniform continuity''.  A function $f\colon M_1
\sto M_2$ is continuous if
  \[ \forall a \in M_1\,   \forall \epsilon > 0\,
             \exists \delta > 0 \, \forall x \in M_1 \,
                     d(x,a) < \delta \implies d(f(x),f(a)) < \epsilon\,. \]
We may just as well write this reversing the order of the first two (universal) quantifiers.
  \[ \forall \epsilon > 0\, \forall a \in M_1\,
             \exists \delta > 0 \,  \forall x \in M_1\,
                     d(x,a) < \delta \implies d(f(x),f(a)) < \epsilon\,. \]
The function $f$ is uniformly continuous if
  \[ \forall \epsilon > 0\,  \exists \delta > 0 \,
             \forall a \in M_1\,     \forall x \in M_1\,
                     d(x,a) < \delta  \implies d(f(x),f(a)) < \epsilon\,. \]
Thus the difference between continuity and uniform continuity is the order of two quantifiers.
This makes the following result obvious.

\begin{prop}\label{prop_uc_cont}  Every uniformly continuous function between metric spaces is
continuous.
\end{prop}

\begin{exam}  The function $f \colon \R \sto \R \colon x \mapsto 3x - 4$ is uniformly continuous.
\end{exam}

\begin{proof}  Given $\epsilon > 0$, choose $\delta = \epsilon/3$. If $\abs{x-y} < \delta$, then
$\abs{f(x) - f(y)} = \abs{(3x - 4) - (3y - 4)} = 3\abs{x - y} < 3\delta = \epsilon$.
\end{proof}


\begin{exam}\label{exam_ucont}  The function $f \colon [1,\infty) \sto \R \colon x \mapsto x^{-1}$
is uniformly continuous.
\end{exam}

\begin{proof} Exercise. (Solution~\ref{sol_exam_ucont}.)   \ns  \end{proof}


\begin{exam}\label{exam_not_ucont}  The function $g \colon (0,1] \sto \R \colon x \mapsto x^{-1}$
is not uniformly continuous.
\end{exam}

\begin{proof} Exercise.  (Solution~\ref{sol_exam_not_ucont}.)  \ns  \end{proof}

\begin{prob}  Let  $M$  be an arbitrary positive number.  The function
  \[ f\colon [0, M] \sto \R \colon x \mapsto x^2 \]
is uniformly continuous.  Prove this assertion using only the definition of  ``uniform
continuity''.
\end{prob}

\begin{prob}   The function
  \[ g \colon [0,\infty) \sto \R \colon x \mapsto x^2 \]
is \emph{not} uniformly continuous.
\end{prob}

\begin{prop} Norms on vector spaces are always uniformly continuous.
\end{prop}

\begin{proof}  Problem.   \ns  \end{proof}


We have already seen in proposition~\ref{prop_uc_cont} that uniform continuity implies
continuity.  Example~\ref{exam_not_ucont} shows that the converse is not true in general.
There are however two special (and important!) cases where the concepts coincide.  One is
linear maps between normed linear spaces, and the other is functions defined on compact metric
spaces.

\begin{prop}\label{prop_lin_ucont}  A linear transformation between normed linear spaces is
continuous if and only if it is uniformly continuous.
\end{prop}

\begin{proof} Problem.  \ns  \end{proof}

Of course, the preceding result does not hold in general metric spaces (where ``linearity''
makes no sense).  The next proposition, for which we give a preparatory lemma, is a metric
space result.

\begin{lem}\label{lem_conv_subseq}  Let $(x_n)$ and $(y_n)$ be sequences in a compact metric
space.  If $d(x_n,y_n) \sto 0$ as $n \sto \infty$, then there exist convergent subsequences of
$(x_n)$ and $(y_n)$ which have the same limit.
\end{lem}

\begin{proof} Exercise.  (Solution~\ref{sol_lem_conv_subseq}.) \ns  \end{proof}

\begin{prop}\label{prop_cont_ucont}  Let $M_1$ be a compact metric space and $M_2$ be an arbitrary
metric space.  Every continuous function $f \colon M_1 \sto M_2$ is uniformly continuous.
\end{prop}

\begin{proof} Exercise.  (Solution~\ref{sol_prop_cont_ucont}.) \ns   \end{proof}


In section~\ref{Cauchy_int} of this chapter, where we define the Cauchy integral, an important
step in the development is the extension of the integral from an exceedingly simple class of
functions, the step functions, to a class of functions large enough to contain all the
continuous functions.  The two basic ingredients of this extension are the density of the step
functions in the large class and the uniform continuity of the integral.
Theorem~\ref{thm_ext_uc} below is the crucial result which allows this extension.  First, two
preliminary results.

\begin{prop}\label{prop_ucont_Cauchy}  If $f \colon M_1 \sto M_2$ is a uniformly continuous map
between two metric spaces and $(x_n)$ is a Cauchy sequence in $M_1$, then $\bigl(f(x_n)\bigr)$
is a Cauchy sequence in~$M_2$.
\end{prop}

\begin{proof} Problem.  \ns   \end{proof}

\begin{prob}  Show by example that proposition~\ref{prop_ucont_Cauchy} is no longer true if
the word ``uniformly'' is deleted .
\end{prob}


\begin{lem}\label{lem_ucont_lim}  Let $M_1$ and $M_2$ be metric spaces, $S \subseteq M_1$, and
$f \colon S \sto M_2$ be uniformly continuous.  If two sequences $(x_n)$ and $(y_n)$ in $S$
converge to the same limit in $M_1$ and if the sequence $\bigl(f(x_n)\bigr)$ converges, then
the sequence $\bigl(f(y_n)\bigr)$ converges and $\lim f(x_n) = \lim f(y_n)$.
\end{lem}

\begin{proof}  Exercise.  \emph{Hint.}  Consider the ``interlaced'' sequence
  \[ (x_1, y_1, x_2, y_2, x_3, y_3, \dots )\,. \]
(Solution~\ref{sol_lem_ucont_lim}.) \ns
\end{proof}

We are now in a position to show that a uniformly continuous map $f$ from a subset of a metric
space into a complete metric space can be extended in a unique fashion to a continuous
function on the closure of the domain of~$f$.

\begin{thm}\label{thm_ext_uc}  Let $M_1$ and $M_2$ be metric spaces, $S$ a subset of $M_1$, and
$f \colon S \sto M_2$.  If $f$ is uniformly continuous and $M_2$ is complete, then there
exists a unique continuous extension of $f$ to~$\clo S$. Furthermore, this extension is
uniformly continuous.
\end{thm}

\begin{proof} Problem. \emph{Hint.} Define $g\colon \clo S \sto M_2$ by $g(a) = \lim f(x_n)$
where $(x_n)$ is a sequence in  $S$ converging to~$a$.  First show that $g$ is well defined.
To this end you must show that
 \begin{enumerate}
  \item [(i)] $\lim f(x_n)$ does exist, and
  \item [(ii)] the value assigned to $g$ at $a$ does not depend on the particular sequence
$(x_n)$ chosen.  That is, if $x_n \sto a$ and $y_n \sto a$, then $\lim f(x_n) = \lim f(y_n)$.
 \end{enumerate}
Next show that $g$ is an extension of~$f$.

To establish the uniform continuity of $g$, let $a$ and $b$ be points in~$\clo S$.  If $(x_n)$
is a sequence in $S$ converging to $a$, then $f(x_n) \sto g(a)$.  This implies that both
$d(x_j,a)$ and $d\bigl(f(x_j),g(a)\bigr)$ can be made as small as we please by choosing $j$
sufficiently large.  A similar remark holds for a sequence $(y_n)$ in $S$ which converges
to~$b$.  From this show that $x_j$ is arbitrarily close to $y_k$ (for large $j$ and $k$)
provided we assume that a is sufficiently close to~$b$.  Use this in turn to show that $g(a)$
is arbitrarily close to~$g(b)$ when $a$ and $b$ are sufficiently close.

The uniqueness argument is very easy.  \ns
\end{proof}

\begin{prop}  Let $f\colon M \sto N$ be a continuous bijection between metric spaces.  If $M$
is complete and $f^{-1}$ is uniformly continuous, then $N$ is complete.
\end{prop}

\begin{proof}  Problem.  \ns  \end{proof}










\section{THE INTEGRAL OF STEP FUNCTIONS}

\vskip .1 in

\begin{center}
\fbox{\textbf{Throughout this section $E$ is a Banach space.}}
\end{center}

\vskip .2 in

\begin{defn}   An $(n+1)$-tuple $(t_0,t_1, \dots, t_n)$ of real numbers is a
 \index{<@$(t_0,t_1,\dots,t_n)$ (partition of an interval)}%
 \index{partition}%
\df{partition} of the interval $[a,b]$ in $\R$ provided that
 \begin{enumerate}
  \item[(i)] $t_0 = a$,
  \item[(ii)] $t_n = b$, and
  \item[(iii)] $t_{k-1} < t_k$ for $1 \le k \le n$.
 \end{enumerate}
If $P = (s_0, \dots, s_m)$ and $Q = (t_0, \dots, t_n)$ are
partitions of the same interval $[a,b]$ and if
  \[ \{s_0, \dots, s_m\} \subseteq \{t_0, \dots, t_n\} \]
then we say that $Q$ is a
 \index{refinement}%
\df{refinement} of $P$ and we write $P \preceq Q$.

Let $P = (s_0, \dots, s_m)$, $Q = (t_0, \dots, t_n)$, and $R = (u_0, \dots, u_p)$ be
partitions of $[a,b] \subseteq \R$.  If
  \[ \{u_0,\dots,u_p\} = \{s_0,\dots,s_m\} \cup \{t_0,\dots,t_n\} \]
then $R$ is the
 \index{refinement!smallest common}%
 \index{smallest!common refinement}%
 \index{<@$P \lor Q$ (smallest common refinement of partitions)}%
\df{smallest common refinement} of $P$ and $Q$ and is denoted by $P \lor Q$.  It is clear that
$P \lor Q$ is the partition with fewest points which is a refinement of both $P$ and~$Q$.
\end{defn}


\begin{exer}\label{exer_partition} Consider partitions $P = \bigl(0, \frac14, \frac13, \frac12,
\frac23, \frac34, 1\bigr)$ and $Q = \bigl(0, \frac15, \frac13, \frac23, \frac56, 1\bigr)$
of~$[0,1]$.  Find $P \lor Q$. (Solution~\ref{sol_exer_partition}.)
\end{exer}

\begin{defn} Let $S$ be a set and $A$ be a subset of~$S$.  We define
$\chi_{{}_{\sst A}}\colon S \sto \R$, the
 \index{<@$\chi_{{}_{\sst A}}$ (characteristic function of a set)}%
 \index{characteristic function}%
 \index{function!characteristic}%
\df{characteristic function} of~$A$, by
  \[ \chi_{{}_{\sst A}}(x) =
      \begin{cases}
                 1,  \qquad &\text{if $x \in A$} \\
                 0,  \qquad &\text{if $x \in A^c$}.
      \end{cases} \]
If $E$ is a Banach space, then a function $\sigma\colon [a,b] \sto E$ is an
 \index{e@$E$ valued step function}%
 \index{step function}%
 \index{function!$E$ valued step}%
\df{$E$ valued step function} on the interval $[a,b]$ if
 \begin{enumerate}
  \item[(i)] $\ran \sigma$ is finite and
  \item[(ii)] for every $x \in \ran\sigma$ the set $\sigma^\gets(\{x\})$ is the union of finitely
many subintervals of~$[a,b]$.
 \end{enumerate}
We denote by $\fml S([a,b],E)$ the family of all $E$ valued step functions defined on~$[a,b]$.
Notice that $\fml S([a,b],E)$ is a vector subspace of $\fml B([a,b],E)$.
\end{defn}

It is not difficult to see that $\sigma\colon [a,b] \sto E$ is a step function if and only if
there exists a partition $(t_0, \dots, t_n)$ of $[a,b]$ such that $\sigma$ is constant on each
of the open subintervals $(t_{k-1},t_k)$.  If, in addition, we insist that $\sigma$ be
discontinuous at each of the points $t_1, \dots, t_{n-1}$, then this partition is unique.
Thus we speak of the
 \index{partition!associated with a step function}%
\df{partition associated with} (or \df{induced by}) a step
function~$\sigma$.

\begin{notn} Let $\sigma$ be a step function on $[a,b]$ and $P = (t_0, \dots, t_n)$ be a partition
of $[a,b]$ which is a refinement of the partition associated with~$\sigma$.  We define
 \[ \sigma_P = (x_1, \dots, x_n) \]
where $x_k$ is the value of $\sigma$ on the open interval $(t_{k-1},t_k)$ for $1 \le k \le n$.
\end{notn}


\begin{exer}\label{exer_int_stepf}  Define $\sigma\colon [0,5] \sto \R$ by
  \[ \sigma = \chi_{{}_\sst{[1,4]}}
            - \chi_{{}_\sst{(2,5]}}
            - \chi_{{}_\sst{\{4\}}}
           -2 \chi_{{}_\sst{[2,3)}}
            - \chi_{{}_\sst{[1,2)}}
            + \chi_{{}_\sst{[4,5)}}\,. \]
 \begin{enumerate}
  \item[(a)] Find the partition $P$ associated with~$\sigma$.
  \item[(b)] Find $\sigma_Q$ where $Q = (0,1,2,3,4,5)$.
 \end{enumerate}
(Solution~\ref{sol_exer_int_stepf}.)
\end{exer}


\begin{defn}  Let $\sigma$ be an $E$ valued step function on $[a,b]$ and $(t_0, \dots, t_n)$
be the partition associated with~$\sigma$.  For $1 \le k \le n$ let $\Delta t_k = t_k -
t_{k- 1}$ and $x_k$ be the value of $\sigma$ on the subinterval $(t_{k-1},t_k)$.  Define
 \[ \int \sigma := \sum_{k=1}^n (\Delta t_k) x_k\,. \]
The vector $\int\sigma$ is the
 \index{integral!of a step function}%
 \index{step function!integral of}%
\df{integral} of $\sigma$ over~$[a,b]$.  Other standard notations for $\int\sigma$ are
 \index{<@$\int_a^b \sigma$ (integral of a step function)}%
$\int_a^b \sigma$ and~$\int_a^b \sigma(t)\,dt$.
\end{defn}

\begin{exer}\label{exer_int_stepf2} Find $\int_0^5 \sigma$ where $\sigma$ is the step function
given in exercise~\ref{exer_int_stepf}.  (Solution~\ref{sol_exer_int_stepf2}.)
\end{exer}

\begin{prob}  Let $\sigma\colon [0,10] \sto \R$  be defined by
  \[ \sigma =  2\chi_{{}_\sst{[1,5)}}
              -3\chi_{{}_\sst{[2,8)}}
              -5\chi_{{}_\sst{\{6\}}}
              + \chi_{{}_\sst{[4,10]}}
             +4 \chi_{{}_\sst{[9,10]}}\,. \]
 \begin{enumerate}
  \item[(a)] Find the partition associated with $\sigma$.
  \item[(b)] If $Q = (0,1,2,3,4,5,6,7,8,9,10)$, what is $\sigma_Q$?
  \item[(c)] Find $\int \sigma$.
 \end{enumerate}
\end{prob}

The next lemma is essentially obvious, but it is good practice to write out a proof anyway.
It says that in computing the integral of a step function $\sigma$ it doesn't matter whether
we work with the partition induced by $\sigma$ or with a refinement of that partition.

\begin{lem}\label{lem_int_stepf}  Let $\sigma$ be an $E$ valued step function on $[a,b]$.
If $Q = (u_0, \dots, u_m)$ is a refinement of the partition associated with $\sigma$ and if
$\sigma_Q = (y_1, \dots, y_m)$, then
  \[ \int\sigma = \sum_{k=1}^n(\Delta u_k)y_k\,. \]
\end{lem}

\begin{proof} Exercise.  (Solution~\ref{sol_lem_int_stepf}.)  \ns  \end{proof}

It follows easily from the preceding lemma that changing the value of a step function at a
finite number of points does not affect the value of its integral.  Next we show that for a
given interval the integral is a bounded linear transformation on the family of all $E$ valued
step functions on the interval.

\begin{prop}\label{prop_intsf_lin}  The map
  \[ \int \colon \fml S([a,b],E) \sto E \]
is bounded and linear with $\norm \int = b - a$.  Furthermore,
  \[ \biggl\|\int\sigma(t)\,dt\biggr\| \le \int \norm{\sigma(t)}\,dt \]
for every $E$ valued step function $\sigma$ on~$[a,b]$.
\end{prop}

\begin{proof} Problem. \emph{Hint.} To show that $\int(\sigma + \tau) = \int\sigma + \int\tau$,
let $P$ and $Q$ be the partitions associated with $\sigma$ and $\tau$, respectively.  Define
the partition $R = (t_0, \dots, t_n)$ to be $P \lor Q$.  Clearly $R$ is a refinement of the
partition associated with~$\sigma + \tau$. Suppose $\sigma_R = (x_1, \dots, x_n)$ and $\tau_R
= (y_1, \dots, y_n)$.  Use lemma~\ref{lem_int_stepf} to compute $\int(\sigma + \tau)$.  To
find $\norm\int$ use the definition of the norm of a linear map and
lemma~\ref{lem_equiv_norm}.  \ns
\end{proof}

We now show that in the case of real valued step functions the integral is a positive linear
functional; that is, it takes positive functions to positive numbers.

\begin{prop}\label{prop_int_pos} If $\sigma$ is a real valued step function on $[a,b]$ and if
$\sigma(t) \ge 0$ for all $t$ in $[a,b]$, then $\int\sigma \ge 0$.
\end{prop}

\begin{proof} Problem.  \ns  \end{proof}

\begin{cor}  If $\sigma$ and $\tau$ are real valued step functions on $[a,b]$ and if
$\sigma(t) \le \tau(t)$ for all $t$ in~$[a,b]$, then $\int\sigma \le \int\tau$.
\end{cor}

\begin{proof}  Apply the preceding proposition to $\tau - \sigma$. Then (by~\ref{prop_intsf_lin})
$\int\tau - \int\sigma = \int(\tau~-~\sigma)~\ge~0$.
\end{proof}

Finally we prepare the ground for piecing together and integrating two functions on adjoining
intervals.

\begin{prop}\label{prop_int_adj_int} Let $c$ be an interior point of the interval~$[a,b]$.  If
$\tau$ and $\rho$ are $E$ valued step functions on the intervals $[a,c]$ and $[c,b]$,
respectively, define a function $\sigma\colon [a,b] \sto E$ by
  \[ \sigma(t) =
        \begin{cases}
             \tau(t),  \qquad &\text{if $a \le t \le c$} \\
             \rho(t),  \qquad &\text{if $c < t \le b$}.
        \end{cases}\]
Then $\sigma$ is an $E$ valued step function on $[a,b]$ and
  \[ \int_a^b\sigma = \int_a^c\tau + \int_c^b\rho\,. \]
\end{prop}

\begin{proof} Exercise.  (Solution~\ref{sol_prop_int_adj_int}.)
 \ns  \end{proof}

Notice that if $\sigma$ is a step function on $[a,b]$, then $\tau := \sigma|_{[a,c]}$ and
$\rho := \sigma|_{[c,b]}$ are step functions and by the preceding proposition
  \[ \label{sum_int_step} \int_a^b\sigma = \int_a^c\tau + \int_c^b\rho\,. \]
In this context one seldom distinguishes notationally between a function on an interval and
the restriction of that function to a subinterval.  Thus~\eqref{sum_int_step} is usually
written
  \[ \int_a^b\sigma = \int_a^c\sigma + \int_c^b\sigma. \]


\begin{prob}\label{prob_int_compop}  Let $\sigma\colon [a,b] \sto E$ be a step function and
$T\colon E \sto F$ be a bounded linear transformation from $E$ into another Banach space~$F$.
Then $T \circ \sigma$ is an $F$-valued step function on $[a,b]$ and
  \[ \int(T\circ \sigma) =  T\bigl(\int\sigma\bigr)\,. \]
\end{prob}













\section{THE CAUCHY INTEGRAL}\label{Cauchy_int}   We are now ready to extend the integral from
the rather limited family of step functions to a class of functions large enough to contain
all continuous functions (in fact, all piecewise continuous functions).

Following Dieudonn\'e\cite{Dieudonne:1962} we will call members of this larger class
\emph{regulated functions}.


\vskip .1 in

\begin{center}
\fbox{\textbf{In this section $E$ will be a Banach space, and $a$ and $b$ real numbers with $a
< b$.}}
\end{center}

\vskip .2 in

\begin{defn}  Recall that the family $\fml S = \fml S([a,b],E)$ of $E$ valued step functions on
$[a,b]$ is a subspace of the normed linear space $\fml B = \fml B([a,b],E)$ of bounded
$E$-valued functions on $[a,b]$.  The closure $\clo{\fml S}$ of $\fml S$ in $\fml B$ is the
family of
 \index{regulated functions}%
 \index{functions!regulated}%
\df{regulated functions} on~$[a,b]$.
\end{defn}

It is an interesting fact that the regulated functions on an interval turn out to be exactly
those functions which have one-sided limits at every point of the interval.  We will not need
this fact, but a proof may be found in Dieudonn\'e\cite{Dieudonne:1962}.

According to problem\ref{prob_clo_subsp}(b) the set $\clo{\fml S}$ is a vector subspace of the
Banach space~$\fml B$.  Since it is closed in $\fml B$ it is itself a Banach space.  It is on
this Banach space that we define the Cauchy integral.  The Cauchy integral is not as general
as the Riemann integral because the set $\clo{\fml S}$ is not quite as large as the set of
functions on which the Riemann integral is defined.  We do not prove this; nor do we prove the
fact that when both integrals are defined, they agree. What we are interested in proving is
that every continuous function is regulated; that is, every continuous function on $[a,b]$
belongs to~$\fml S$.

\begin{prop}\label{prop_cont_reg}  Every continuous $E$-valued function on $[a,b]$ is regulated.
\end{prop}

\begin{proof} Exercise. \emph{Hint.} Use proposition~\ref{prop_cont_ucont}.
(Solution~\ref{sol_prop_cont_reg}.)   \ns
\end{proof}

It is not difficult to modify the preceding proof to show that every piecewise continuous
function on $[a,b]$ is regulated. (Definition: A function $f\colon[a,b] \sto E$ is
 \index{piecewise continuous}%
 \index{continuous!piecewise}%
\df{piecewise continuous} if there exists a partition $(t_0, \dots, t_n)$ of $[a,b]$ such that
$f$ is continuous on each subinterval $(t_{k-1},t_k)$.)

\begin{cor}  Every continuous $E$ valued function on $[a,b]$ is the uniform limit of a sequence
of step functions.
\end{cor}

\begin{proof}  According to problem~\ref{prob_clo_seq} a function $f$ belongs to the
closure of $S$ if and only if there is a sequence of step functions which converges
(uniformly) to~$f$.
\end{proof}

Now we are ready to define the Cauchy integral of a regulated function.

\begin{defn} Recall that $\fml S = \fml S([a,b],E)$ is a subset of the Banach space
$\fml B([a,b],E)$. In the preceding section we defined the integral of a step function.
The map
 \[ \int \colon \fml S \sto E \]
was shown to be bounded and linear [proposition~\ref{prop_intsf_lin}]; therefore it is
uniformly continuous [proposition~\ref{prop_lin_ucont}].  Thus [theorem~\ref{thm_ext_uc}] it
has a unique continuous extension to~$\clo{\fml S}$.  This extension, which we denote also by
$\int$, and which is, in fact, uniformly continuous, is the
 \index{<@$\int_a^b f$ (integral of a regulated function)}%
 \index{Cauchy!integral}%
 \index{integral!Cauchy}%
\df{Cauchy} (or \df{Cauchy-Bochner}) \df{integral}.  For $f$ in $\clo{\fml S}$ we call $\int
f$ the \df{integral of} $f$ (over $[a,b]$). As with step functions we may wish to emphasize
the domain of $f$ or the role of a particular variable, in which case we may write $\int_a^bf$
or $\int_a^bf(t)\,dt$ for $\int f$.
\end{defn}

\begin{prob}  Use the \emph{definition} of the Cauchy integral to show that
$\int_0^1x^2\,dx = 1/3$. \emph{Hint.}  Start by finding a sequence of step functions
which converges uniformly to the function $x \mapsto x^2$.  The \emph{proof} of
proposition~\ref{prop_cont_reg} may help; so also may problem~\ref{prob_sum_sqrs}.
\end{prob}


Most of the properties of the Cauchy integral are derived from the corresponding
properties of the integral of step functions by taking limits.  In the remainder of this
section it is well to keep in mind one aspect of theorem~\ref{thm_ext_uc} (and its
proof): When a uniformly continuous function $f$ is extended from a set $\fml S$ to a
function $g$ on its closure $\clo{\fml S}$, the value of $g$ at a point $a$ in $\clo{\fml
S}$ is the limit of the values $f(x_n)$ where $(x_n)$ is any sequence in $\fml S$ which
converges to~$a$.  What does this say in the present context?  If $h$ is a regulated
function, then there exists a sequence $(\sigma_n)$ of step functions converging
uniformly to $h$ and furthermore
  \[ \int h = \lim_{n\sto\infty} \int\sigma_n\,. \]
We use this fact repeatedly without explicit reference.

One more simple fact is worthy of notice.  Following lemma~\ref{lem_int_stepf} we remarked
that changing the value of a step function at finitely many points does not affect the value
of its integral.  The same is true of regulated functions. [Proof. Certainly it suffices to
show that changing the value of a regulated function $f$ at a single point $c$ does not alter
the value of~$\int f$.  Suppose $(\sigma_n)$ is a sequence of step functions converging
uniformly to $f$.  Also suppose that $g$ differs from $f$ only at~$c$.  Replace each step
function $\sigma_n$ by a function $\tau_n$ which is equal to $\sigma_n$ at each point other
than $c$ and whose value at $c$ is~$g(c)$.  Then $\tau_n \sto g \text{\,(unif)}$ and (by the
comment in the preceding paragraph) $\int g = \lim \int \tau_n = \lim \int \sigma_n = \int
f.$]

The following theorem (and proposition~\ref{prop_norm_Cint}) generalize
proposition~\ref{prop_intsf_lin}.

\begin{thm}\label{thm_Cint_blt}  The Cauchy integral is a bounded linear transformation which
maps the space of $E$ valued regulated functions on $[a,b]$ into the space~$E$.
\end{thm}

\begin{proof} Exercise. (Solution~\ref{sol_thm_Cint_blt}.)  \ns   \end{proof}

Next we show that for real valued functions the Cauchy integral is a positive linear
functional.  This generalizes proposition~\ref{prop_int_pos}.

\begin{prop}  Let $f$ be a regulated real valued function on~$[a,b]$.  If $f(t) \ge 0$ for all
$t$ in $[a,b]$, then $\int f \ge 0$.
\end{prop}

\begin{proof} Problem.  \emph{Hint.} Suppose that $f(t) \ge 0$ for all $t$ in $[a,b]$ and that
$(\sigma_n)$ is a sequence of real valued step functions converging uniformly to~$f$.  For
each $n \in N$ define
  \[ \sigma_n^+\colon [a,b] \sto \R\colon t \mapsto \max\{\sigma_n(t),0\}\,. \]
Show that $(\sigma_n^+)$ is a sequence of step functions converging uniformly to~$f$.  Then
use proposition~\ref{prop_int_pos}.  \ns
\end{proof}

\begin{cor}  If $f$ and $g$ are regulated real valued functions on $[a,b]$ and if $f(t) \le g(t)$
for all $t$ in $[a,b]$, then $\int f \le \int g$.
\end{cor}

\begin{proof} Problem.  \ns  \end{proof}

\begin{prop}  Let $f\colon [a,b] \sto E$ be a regulated function and $(\sigma_n)$ be a sequence of
$E$-valued step functions converging uniformly to~$f$.  Then
  \[ \bignorm{\int f} = \lim\bignorm{\int\sigma_n}\,. \]
Furthermore, if $g(t) = \norm{f(t)}$ for all $t$ in $[a,b]$ and $\tau_n(t) =
\norm{\sigma_n(t)}$ for all $n$ in $\N$ and $t$ in $[a,b]$, then $(\tau_n)$ is a sequence of
real valued step functions which converges uniformly to~$g$ and $\int g = \lim\int\tau_n$.
\end{prop}

\begin{proof} Problem.  \ns  \end{proof}

\begin{prop}\label{prop_norm_Cint} Let $f$ be a regulated $E$ valued function on~$[a,b]$.  Then
  \[ \bignorm{\int f} \le \int\norm{f(t)}\,dt \]
and therefore
  \[ \bignorm{\int f} \le (b-a)\norm f_u\,. \]
Thus
  \[ \bignorm{\,\int\,} = b-a \]
where $\int\colon \clo{\fml S} \sto E$ is the Cauchy integral and $\clo{\fml S}$ is the family
of regulated $E$ valued functions on the interval~$[a,b]$.
\end{prop}

\begin{proof} Problem.   \ns   \end{proof}

\begin{prob}  Explain in one or two brief sentences why the following is obvious: If $(f_n)$ is
a sequence of $E$-valued regulated functions on $[a,b]$ which converges uniformly to~$g$, then
$g$ is regulated and $\int g = \lim \int f_n$.
\end{prob}

\begin{prob} Let $\sigma, \tau\colon [0,3] \sto \R$ be the step functions defined by
  \[ \sigma = \chi_{{}_\sst{[0,2]}} \qquad \text{ and }
         \qquad  \tau = \chi_{{}_\sst{[0,2]}} + 2\chi_{{}_\sst{(2,3]}}\,. \]
Recall from appendix~\ref{prods} that the function $(\sigma,\tau)\colon [0,3] \sto \R^2$ is
defined by $(\sigma,\tau)(t) = \bigl(\sigma(t),\tau(t)\bigr)$ for $0 \le t \le 3$.  It is
clear that $(\sigma,\tau)$ is a step function.
 \begin{enumerate}
  \item[(a)] Find the partition $R$ associated with~$(\sigma,\tau)$. Find $(\sigma,\tau)_R$.
Make a careful sketch of~$(\sigma,\tau)$.
  \item[(b)] Find $\int(\sigma,\tau)$.
 \end{enumerate}
\end{prob}

\begin{prob} Same as preceding problem except let $\sigma = \chi_{{}_\sst{[0,1]}}$.
\end{prob}

We now generalize proposition~\ref{prop_int_adj_int} to regulated functions on adjoining
intervals.

\begin{prop} Let $c$ be an interior point of the interval $[a,b]$.  If $g$ and $h$ are regulated
$E$-valued functions on the intervals $[a,c]$ and $[c,b]$, respectively, define a function
$f\colon [a,b] \sto E$ by
  \[ f(t) =
       \begin{cases}
              g(t), \qquad &\text{if $a \le c \le c$} \\
              h(t), \qquad &\text{if $c < t \le b.$}
       \end{cases} \]
Then $f$ is a regulated function on $[a,b]$ and
  \[ \int_a^b f = \int_a^c g + \int_c^b h\,. \]
\end{prop}

\begin{proof} Problem.  \ns  \end{proof}

As was remarked after proposition~\ref{prop_int_adj_int} it is the usual practice to use the
same name for a function and for its restriction to subintervals. Thus the notation of the
next corollary.

\begin{cor}\label{cor_int_adj_int} If $f\colon [a,b] \sto E$ is a regulated function and $c$
is an interior point of the interval $[a,b]$, then
 \begin{equation}\label{eqn_adj_int}
    \int_a^b f = \int_a^c f + \int_c^b f\,.
 \end{equation}
\end{cor}

\begin{proof} Let $g = f|_{[a,c]}$ and $h = f|_{[c,b]}$.  Notice that $g$ and $h$ are regulated.
[If, for example, $(\sigma_n)$ is a  sequence of step functions converging uniformly on
$[a,b]$ to $f$, then the step functions $\sigma_n|_{[a,c]}$ converge uniformly on $[a,c]$
to~$g$.]  Then apply the preceding proposition.
\end{proof}

It is convenient for formula~\eqref{eqn_adj_int} to be correct when $a$, $b$, and $c$ are not
in increasing order or, for that matter, even necessarily distinct.  This can be achieved by
means of a simple notational convention.

\begin{defn} For a regulated function $f$ on $[a,b]$ (where $a < b$) define
  \[ \int_b^a f := -\int_a^b f\,. \]
Furthermore, if $g$ is any function whose domain contains the point $a$, then
  \[ \int_a^a g := 0\,. \]
\end{defn}

\begin{cor}  If $f$ is an $E$ valued regulated function whose domain contains an interval to which
the points $a$, $b$, and $c$ belong, then
  \[ \int_a^b f = \int_a^c f + \int_c^b f\,. \]
\end{cor}

\begin{proof}  The result is obvious if any two of the points $a$, $b$, and $c$ coincide; so we
suppose that they are distinct. There are six possible orderings.  We check one of these.
Suppose $c < a < b$.  By corollary~\ref{cor_int_adj_int}
  \[ \int_c^b f = \int_c^a f + \int_a^b f\,. \]
Thus
  \[ \int_a^b f = -\int_c^a f + \int_c^b f = \int_a^c f + \int_c^b f\,. \]
The remaining five cases are similar.
\end{proof}

Suppose $f$ is an $E$-valued regulated function and $T$ is a bounded linear map from $E$ into
another Banach space~$F$.  It is interesting and useful to know that integration of the
composite function $T \circ f$ can be achieved simply by integrating $f$ and then
applying~$T$.  This fact can be expressed by means of the following commutative diagram.
Notation: $\clo{\fml S}_E$ and $\clo{\fml S}_F$ denote, respectively, the $E$ valued and $F$
valued regulated functions on $[a,b]$; and $C_T$ is the bounded linear transformation
discussed in problem~\ref{prob_comp_op}. ($C_T(f) = T \circ f$ for every~$f$.)
  \[ \xy
        \square[\clo{\fml S}_E`E`\clo{\fml S}_F`F;\int`C_T`T`\int]
     \endxy \]

Alternatively it may be expressed by a formula, as in the next proposition.

\begin{prop}\label{prop_int_compop}  Let $T \colon E \sto F$ be a bounded linear map between Banach
spaces and $f$ be a regulated $E$ valued function on the interval~$[a,b]$.  Then $T \circ f$
is a regulated $F$ valued function on $[a,b]$ and
  \[ \int(T \circ f) = T \biggl(\int f\biggr)\,. \]
\end{prop}


\begin{proof} Exercise.  \emph{Hint.}  Use problem~\ref{prob_int_compop}.
(Solution~\ref{sol_prop_int_compop}.)  \ns
\end{proof}

\begin{cor}\label{cor_1param_fam}  If $E$ and $F$ are Banach spaces and
  \[ T\colon [a,b] \sto \ofml B(E,F)\colon  t \mapsto T_t \]
is continuous, then for every $x$ in $E$
  \[ \int  T_t(x)\,dt = \biggl(\int T\biggr)(x)\,. \]
\end{cor}

\begin{proof} Problem.   \ns \emph{Hint.}  For $x$ in $E$ let $E_x\colon \ofml B(E,F) \sto F$
be the map (evaluation at~$x$) defined in problem~\ref{prob_eval_op}.  Write $T_t(x)$ as $(E_x
\circ T)(t)$ and apply proposition~\ref{prop_int_compop}.
\end{proof}

\begin{prop}\label{prop_int_const_vec}  Let $f\colon [a,b] \sto \R$ be a regulated function and
$x \in E$.  For all $t$ in $[a,b]$ let $g(t) = f(t)\,x$.  Then $g$ is a regulated $E$-valued
function and $\int g = \bigl(\int f\bigr)x$.
\end{prop}

\begin{proof} Problem.  \emph{Hint.}  Prove the result first for the case $f$ is a step function.
Then take limits.
\end{proof}

\begin{prop}  If $f\colon [a,b] \sto E$ and $g\colon [a,b] \sto F$ are regulated functions whose
ranges lie in Banach spaces, then the function
  \[ (f,g)\colon [a,b] \sto E \times F\colon  t \mapsto \bigl(f(t),g(t)\bigr) \]
is regulated and
  \[ \int (f,g) = \biggl(\int f, \int g\biggr)\,. \]
\end{prop}

\begin{proof} Problem. \emph{Hint.} Write $\int f$ as $\int \bigl(\pi_1\circ(f,g)\bigr)$ where
$\pi_1\colon E \times F \sto E$ is the usual coordinate projection.  Write $\int g$ in a
similar fashion. Use proposition~\ref{prop_int_compop}.  Keep in mind that if $p$ is a point
in the product $E \times F$, then $p = \bigl(\pi_1(p),\pi_2(p)\bigr)$.  \ns
\end{proof}

\begin{prop}  Suppose $h\colon [a,b] \sto E \times F$ is a regulated function from $[a,b]$ into the
product of two Banach spaces.  Then the components $h^1$ and $h^2$ are regulated functions and
  \[ \int h = \biggl(\int h_1, \int h_2\biggr)\,. \]
\end{prop}

\begin{proof} Problem.  \ns  \end{proof}


\begin{prob}\label{prob_int_Rnval}  Suppose $f \colon [a,b] \sto \R^n$ is a regulated function.
Express the integral of $f$ in terms of the integrals of its components $f^1, \dots, f^n$.
Justify your answer carefully.  \emph{Hint.}  When $\R^n$ appears (without further
qualification) its norm is assumed to be the usual Euclidean norm.  This is not the product
norm.  What needs to be done to ensure that the results of the preceding problem will continue
to be true if the product norm on $E \times F$ is replaced by an equivalent norm?
\end{prob}

\begin{prob} Suppose $T \colon [0,1] \sto \ofml B(\R^2,\R^2)\colon  t \mapsto T_t)$ is continuous,
and suppose that for each $t$ in $[0,1]$ the matrix representation of $T_t$  is given by
  \[ \bigl[T_t\bigr] = \begin{bmatrix} 1 & t \\ t^2 & t^3 \end{bmatrix}\,. \]
Find $\bigl[\int T\bigr]$ the matrix representation of~$\int T$. \emph{Hint.} Use
\ref{cor_1param_fam}, \ref{prop_int_const_vec}, and~\ref{prob_int_Rnval}.
\end{prob}

\begin{prob}  For every $f$ in $\fml C([a,b],E)$ define
  \[ \norm f_1 = \int_a^b \norm{f(t)}\,dt\,. \]
 \begin{enumerate}
  \item[(a)] Show that $\norm{\quad}_1$ is a norm on $\fml C([a,b],E)$. \emph{Hint.}  In showing
that $\norm f_1 = 0$ implies $f = 0$, proposition~\ref{pos_at_pt} may help.
  \item[(b)] Are the norms $\norm{\quad}_1$ and $\norm{\quad}_u$ on $\fml C([a,b],E)$ equivalent?
  \item[(c)]  Let $\fml C_1$ be the vector space $\fml C([a,b],E)$ under the norm $\norm{\quad}_1$
and $\fml C_u$ be the same vector space under its usual uniform norm.  Does convergence of a
sequence in $\fml C_u$ imply convergence in~$C_1$?  What about the converse?
 \end{enumerate}
\end{prob}

\begin{prob}  Show that if $f \in \fml C([0,1],\R)$ and $\int_0^1 x^nf(x)\,dx = 0$ for
$n = 0,1,2,\dots$, then $f = 0$.
\end{prob}

\begin{prob}  Find $\lim\frac{\int_0^1 x^nf(x)\,dx}{\int_0^1 x^n\,dx}$ when $f$ is a continuous
real valued function on~$[0,1]$. \emph{Hint.}  For each $n$ in $\N$ let $L_n(f) =
\frac{\int_0^1 x^nf(x)\,dx}{\int_0^1 x^n\,dx}$.  Show that $L_n$ is a continuous linear
functional on the space $\fml C([0,1],\R)$ of continuous real valued functions
on~$[0,1]$. What is $\lim_{n \sto \infty} L_n(p)$ when $p$ is a polynomial? Use the
 \index{Weierstrass approximation theorem}%
 \index{approximation theorem!Weierstrass}%
\emph{Weierstrass approximation theorem}~\ref{Wat}.
\end{prob}

\endinput
