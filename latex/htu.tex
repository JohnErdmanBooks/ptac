\chapter*{FOR STUDENTS: HOW TO USE THIS PROBLEMTEXT}

Some years ago at a national meeting of mathematicians many of the conventioneers went about
wearing pins which proclaimed, ``Mathematics is Not a Spectator Sport''.  It is hard to
overemphasize the importance of this observation. The idea behind it has been said in many
ways by many people; perhaps it was said best by Paul Halmos~\cite{Halmos:1967}: \emph{The
only way to learn mathematics is to do mathematics.}

In most respects learning mathematics is more like learning to play tennis than learning
history.  It is principally an activity, and only secondarily a body of knowledge.  Although
no one would try to learn tennis just by watching others play the game and perhaps reading a
book or two about it, thousands of mathematics students in American universities every year
attempt to master mathematical subjects by reading textbooks and passively watching their
instructors do mathematics on a blackboard.  There are, of course, reasons why this is so, but
it is unfortunate nevertheless.  This book is designed to encourage \emph{you} to do
mathematics.

When you sit down to work it is important to have a sizeable block of time at your disposal
during which you will not be interrupted. As you read pay especially close attention to
definitions.  (After all, before you can think about a mathematical concept you must know what
it means.)  Read until you arrive at a result (results are labeled ``theorem'',
``proposition'', ``example'', ``problem'', ``lemma'', \emph{etc.}).  Every result requires
justification.  The proof of a result may appear in the body of the text, or it may be left to
you as an \emph{exercise} or a \emph{problem}.

When you reach a result stop and try to prove it.  Make a serious attempt.  If a hint appears
after the statement of the result, at first do not read it.  Do not try to find the result
elsewhere; and do not ask for help. Halmos~\cite{Halmos:1967} points out: ``To the passive
reader a routine computation and a miracle of ingenuity come with equal ease, and later, when
he must depend on himself, he will find that they went as easily as they came.''  Of course,
it is true that sometimes, even after considerable effort, you will not have discovered a
proof.  What then?

If a hint is given, and if you have tried seriously but unsuccessfully to derive the result,
then (and only then) should you read the hint. Now try again. Seriously.

What if the hint fails to help, or if there is no hint?  If you are stuck on a result whose
proof is labeled ``exercise'', then follow the link to the solution.  Turning to the solution
should be regarded as a last resort.  Even then do not read the whole proof; read just the
first line or two, enough to get you started.  Now try to complete the proof on your own. If
you can do a few more steps, fine. If you get stuck again in midstream, read some more of the
proof. Use as little of the printed proof as possible.

If you are stuck on a result whose proof is a ``problem'', you will not find a solution in the
text.  After a really serious attempt to solve the problem, go on.  You can't bring your
mathematical education to a halt because of one refractory problem. Work on the next result.
After a day or two go back and try again. Problems often ``solve themselves''; frequently an
intractably murky result, after having been allowed to ``rest'' for a few days, will suddenly,
and inexplicably become entirely clear.  In the worst case, if repeated attempts fail to
produce a solution, you may have to discuss the problem with someone else---instructor,
friend, mother, \dots .

A question that students frequently ask is, ``When I'm stuck and I have no idea at all
what to do next, how can I continue to work on a problem?''  I know of only one really
good answer. It is advice due to Polya. \emph{If you can't solve a problem, then there is
an easier problem you can't solve: find it.}

Consider examples.  After all, mathematical theorems are usually generalizations of things
that happen in interesting special cases. Try to prove the result in some concrete cases.  If
you succeed, try to generalize your argument.  Are you stuck on a theorem about general metric
spaces?  Try to prove it for Euclidean $n$-space. No luck?  How about the plane?  Can you get
the result for the real line?  The unit interval?

Add hypotheses. If you can't prove the result as stated, can you prove it under more
restrictive assumptions?  If you are having no success with a theorem concerning general
matrices, can you prove it for symmetric ones?  How about diagonal ones?  What about the $2
\times 2$ case?

Finally, one way or another, you succeed in producing a proof of the stated result.  Is
that the end of the story?  By no means. Now you should look carefully at the hypotheses.
Can any of them be weakened?  Eliminated altogether?  If not, construct counterexamples
to show that each of the hypotheses is necessary. Is the conclusion true in a more
general setting?  If so, prove it. If not, give a counterexample.  Is the converse true?
Again, prove or disprove.  Can you find applications of the result?  Does it add anything
to what you already know about other mathematical objects and facts?

Once you have worked your way through several results (a section or a chapter, say) it is a
good idea to consider the organization of the material.  Is the order in which the
definitions, theorems, \emph{etc.} occur a good one?  Or can you think of a more perspicuous
ordering?  Rephrase definitions (being careful, of course, not to change meanings!), recast
theorems, reorganize material, add examples.  Do anything you can to make the results into a
clear and coherent body of material.  In effect you should end up writing your own advanced
calculus text.

The fruit of this labor is understanding. After serious work on the foregoing items you will
begin to feel that you ``understand'' the body of material in question.  This quest for
understanding, by the way, is pretty much what mathematicians do with their lives.

If you don't enjoy the activities outlined above, you probably don't very much like
mathematics.

Like most things that are worth doing, learning advanced calculus involves a substantial
commitment of time and energy; but as one gradually becomes more and more proficient, the
whole process of learning begins to give one a great sense of accomplishment, and, best of
all, turns out to be lots of fun.


\endinput
