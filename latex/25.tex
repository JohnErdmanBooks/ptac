\chapter{DIFFERENTIAL CALCULUS}\label{diff_calc}

In chapter~\ref{dorf} we studied the calculus of real valued functions of a real variable.  We
now extend that study to vector valued functions of a vector variable.  The first three
sections of this chapter repeat almost exactly the material in chapter~\ref{dorf}.  Absolute
values are replaced by norms, and linear maps by continuous linear maps.  Other than that
there are few differences.  In fact, if you have done chapter~\ref{dorf} carefully you may
wish just to glance over the first three sections of this chapter and move on to
section~\ref{diff_curv}.

\vspace{.2in}

\begin{center}
\fbox{\parbox{3.8in}{\textbf{Throughout this chapter $V$, $W$, and $X$ (with or without
subscripts) will be normed linear spaces.}}}
\end{center}

 \vspace{.2in}

\section{$\lobo O$ AND $\lobo o$ FUNCTIONS}

\begin{notn}  Let $a \in V$.  We denote
 \index{f@$\fml F_a(V,W)$ (functions defined in a neighborhood of~$a$)}%
by~$\fml F_a(V,W)$ the family of all functions defined on a neighborhood of $a$ taking values
in~$W$.  That is, $f$ belongs to $\fml F_a(V,W)$ if there exists a set $U$ such that $a \in
\open U {\dom f} \subseteq V$ and if the image of $f$ is contained in~$W$.  We shorten $\fml
F_a(V,W)$ to $\fml F_a$ when no confusion will result.  Notice that for each $a \in V$, the
set $\fml F_a$ is closed under addition and scalar multiplication. (As usual, we define the
sum of two functions $f$ and $g$ in $\fml F_a$ to be the function $f+g$ whose value at $x$ is
$f(x)+g(x)$ whenever $x$ belongs to $\dom f \,\cap\, \dom g$.)  Despite the closure of $\fml
F_a$ under these operations, $\fml F_a$ is \emph{not} a vector space.
\end{notn}

\begin{prob}  Let $a \in V \ne \{\vc 0\}$. Prove that $\fml F_a(V,W)$ is not a vector space.
\end{prob}

Among the functions defined on a neighborhood of the zero vector in $V$ are two
subfamilies of crucial importance; they are $\lobo O(V,W)$ (the family of
\emph{``big-oh''} functions) and $\lobo o(V,W)$ (the family of \emph{``little-oh''}
functions).

\begin{defn}  A function $f$ in $\fml F_{\vc 0}(V,W)$ belongs to
 \index{ohbig@$\lobo O(V,W)$ functions}%
$\lobo O(V,W)$ if there exist numbers $c > 0$ and $\delta > 0$ such that
 \[\norm{f(x)} \le c\,\norm x\]
whenever $\norm x < \delta$.

A function $f$ in $\fml F_{\vc 0}(V,W)$ belongs to
 \index{ohlittle@$\lobo o(V,W)$ functions}%
$\lobo o(V,W)$ if for every $c > 0$ there exists $\delta > 0$ such that
  \[ \norm{f(x)} \le c\,\norm x \]
whenever $\norm x < \delta$.  Notice that $f$ belongs to $\lobo o(V,W)$ if and only if $f(\vc
0) = \vc 0$ and
  \[ \lim_{h \sto \vc 0}\frac{\norm{f(h)}}{\norm h} = \vc 0\,. \]
When no confusion seems likely we will shorten $\lobo O(V,W)$ to $\lobo O$ and $\lobo o(V,W)$
to $\lobo o$.
\end{defn}


The properties of the families $\lobo O$ and $\lobo o$ are given in
propositions~\ref{prop_o_in_O}--\ref{prop_comp_Oo}, \ref{prop_o_x_vect},
and~\ref{prop_O_x_fcn}.

\begin{prop}\label{prop_o_in_O} Every member of $\lobo o(V,W)$ belongs to $\lobo O(V,W)$; so does
every member of $\lobo B(V,W)$. Every member of $\lobo O(V,W)$ is continuous at~$\vc 0$.
\end{prop}

\begin{proof}  Obvious from the definitions.   \end{proof}

\begin{prop}\label{prop_bdd_not_o} Other than the zero transformation, no bounded linear
transformation belongs to~$\lobo o$.
\end{prop}

\begin{proof} Exercise. (Solution~\ref{sol_prop_bdd_not_o}.)  \ns  \end{proof}

\begin{prop}\label{prop_O_cl_add} The family $\lobo O$ is closed under addition and scalar
multiplication.
\end{prop}

\begin{proof} Exercise. (Solution~\ref{sol_prop_O_cl_add}.) \ns   \end{proof}

\begin{prop}\label{prop_o_cl_add}  The family $\lobo o$ is closed under addition and scalar
multiplication.
\end{prop}

\begin{proof} Problem.  \ns  \end{proof}

The next two propositions say that the composite of a function in $\lobo O$ with one in $\lobo
o$ (in either order) ends up in $\lobo o$.

\begin{prop} If $g \in \lobo O(V,W)$ and $f \in \lobo o(W,X)$, then $f \circ g \in \lobo o(V,X)$.
\end{prop}

\begin{proof} Problem.  \ns   \end{proof}

\begin{prop}\label{prop_comp_Oo} If $g \in \lobo o(V,W)$ and $f \in \lobo O(W,X)$, then
$f \circ g \in \lobo o(V,X)$.
\end{prop}

\begin{proof} Exercise. (Solution~\ref{sol_prop_comp_Oo}.)
  \ns   \end{proof}

\begin{notn}\label{notn_fcns_mult}  In the sequel it will be convenient to multiply vectors
not only by scalars but also by scalar valued functions.  If $\phi$ is a function in $\fml
F_a(V,\R)$ and $w \in W$, we define the function $\phi w$ by
  \[ (\phi w)(x) = \phi (x) \cdot w \]
for all $x$ in the domain of $\phi$.  Clearly, $\phi w$ belongs to $\fml F_a(V,W)$.

Similarly, it is useful to multiply vector valued functions by scalar valued functions.  If
$\phi \in \fml F_a(V,\R)$ and $f \in \fml F_a(V,W)$, we define the function $\phi f$ by
  \[ (\phi f)(x) = \phi (x) \cdot f(x) \]
for all $x$ in $\dom \phi\, \cap\, \dom f$. Then $\phi f$ belongs to $\fml F_a(V,W)$.
\end{notn}

\begin{prop}\label{prop_o_x_vect}  If $\phi \in \lobo o(V,\R)$ and $w \in W$, then
$\phi w \in \lobo o(V,W)$.
\end{prop}

\begin{proof} Exercise. (Solution~\ref{sol_prop_o_x_vect}.)  \ns   \end{proof}

\begin{prop}\label{prop_O_x_fcn}  If $\phi \in \lobo O(V,\R)$ and $f \in \lobo O(V,W)$,
then $\phi f \in \lobo o(V,W)$.
\end{prop}

\begin{proof} Exercise. (Solution~\ref{sol_prop_O_x_fcn}.)  \ns   \end{proof}

\begin{rem}  The list summarizing these facts is almost the same as the one in~\ref{rem_oO_R}.
In (1) and (2) the linear maps $\ofml L$ have been replaced by continuous linear maps $\ofml
B$; and (7) is new.  (As before,
 \index{continuous@$\fml C_{\vc 0}$ (functions continuous at~$\vc 0$)}%
$\fml C_{\vc 0}$ is the set of all functions in $\fml F_{\vc 0}$ which are continuous at~$\vc
0$.)
  \begin{align*}
    &(1)\qquad \lobo B \cup \lobo o \subseteq \lobo O \subseteq \fml C_{\vc 0} \, .\\
    &(2)\qquad \lobo B \cap \lobo o = \{\vc 0\}\,. \\
    &(3)\qquad \lobo O + \lobo O \subseteq \lobo O\,; \qquad \alpha\,\lobo O \subseteq \lobo O\,. \\
    &(4)\qquad \lobo o + \lobo o \subseteq \lobo o\,; \qquad \alpha\,\lobo o \subseteq \lobo o\,. \\
    &(5)\qquad \lobo o \circ \lobo O \subseteq \lobo o\,. \\
    &(6)\qquad \lobo O \circ \lobo o \subseteq \lobo o\,. \\
    &(7)\qquad \lobo o(V,\R)\cdot W\subseteq \lobo o(V,W)\,.\\
    &(8)\qquad \lobo O(V,\R)\cdot \lobo O(V,W) \subseteq  \lobo o(V,W)\,.
  \end{align*}
\end{rem}

\begin{prob}  Find a function in $\fml C_{\vc 0}(\R^2, \R) \setminus \lobo O(\R^2, \R)$.  Also,
find a function in $\lobo O(\R^2, \R) \setminus \lobo o(\R^2, \R)$.
\end{prob}

\begin{prob}\label{prob_O_comp_O}  Show that $\lobo O \circ \lobo O\subseteq\lobo O$.  That is,
if $g \in \lobo O(V,W)$ and $f \in \lobo O(W,X)$, then $f \circ g \in \lobo O(V,X)$.  (As
usual, the domain of $f \circ g$ is taken to be $\{x \in V \colon g(x) \in \dom f\}$.)
\end{prob}

\begin{prob} If $f^1 \in \lobo O(V_1,W)$ and $f^2 \in \lobo O(V_2,W)$, then the function $g$
defined on $\dom f^1 \times \dom f^2$ by
  \[ g(v^1,v^2) = f^1(v^1) + f^2(v^2) \]
belongs to $\lobo O(V_1 \times V_2, W)$. \emph{Hint.}  The simplest proof never mentions the
domain elements $v^1$ and $v^2$. Instead write $g$ in terms of the projection maps $\pi_1$ and
$\pi_2$ on $V_1 \times V_2$, and apply problem~\ref{prob_O_comp_O}.
\end{prob}

\begin{prob} If $\phi \in \lobo O(V_1,\R)$ and $f \in \lobo O(V_2,W)$, then the function $h$
defined on $\dom \phi \times \dom f$ by
  \[ h(v^1,v^2) = \phi(v^1)\,f(v^2) \]
belongs to $\lobo o(V_1 \times V_2,W)$. (Use the hint given in the preceding problem.)
\end{prob}

\begin{prob} Show that the membership of the families $\lobo O(V,W)$ and $\lobo o(V,W)$ is not
changed when the norms on the spaces $V$ and $W$ are replaced by equivalent norms.
\end{prob}







\section{TANGENCY}

\begin{defn} Two functions $f$ and $g$ in $\fml F_{\vc 0}(V,W)$ are
 \index{tangent!at zero}%
\df{tangent at zero}, in which case we write
 \index{<@$\simeq$ (tangency)}%
$f \simeq g$, if $f - g \in \lobo o(V,W)$.
\end{defn}

\begin{prop}\label{prop_tan_equiv} The relation ``tangency at zero'' is an equivalence relation
on~$\fml F_{\vc 0}$.
\end{prop}

\begin{proof} Exercise. (Solution~\ref{sol_prop_tan_equiv}.)  \ns   \end{proof}


The next result shows that at most one bounded linear transformation can be tangent at zero to
a given function.

\begin{prop}\label{prop_tan_lin}  Let $S,T \in \ofml B$ and $f \in \fml F_0$.  If $S \simeq f$
and $T \simeq f$, then $S = T$.
\end{prop}

\begin{proof} Exercise.  (Solution~\ref{sol_prop_tan_lin}.) \ns   \end{proof}

\begin{prop}\label{prop_tan_sum}  If $f \simeq g$ and $j \simeq k$, then $f + j \simeq g + k$,
and furthermore, $\alpha f \simeq \alpha g$ for all $\alpha \in \R$.
\end{prop}

\begin{proof} Problem.  \ns   \end{proof}

In the next proposition we see that if two real valued functions are tangent at zero,
multiplication by a vector does not disrupt this relationship.  (For notation
see~\ref{notn_fcns_mult}.)

\begin{prop}\label{prop_tan_x_vec}  Let $\phi, \psi \in \fml F_0(V,\R)$ and $w \in W$.  If
$\phi \simeq \psi$, then $\phi w \simeq \psi w$.
\end{prop}

\begin{proof} Exercise.  (Solution~\ref{sol_prop_tan_x_vec}.)  \ns   \end{proof}

\begin{prop}\label{prop_lin_comp_tan}  Let $f$, $g \in \fml F_{\vc 0}(V,W)$ and $T \in
\ofml B(W,X)$.  If $f \simeq g$, then $T \circ f \simeq T \circ g$.
\end{prop}

\begin{proof} Problem.  \ns   \end{proof}

\begin{prop}\label{prop_tan_comp_O}  Let $h \in \lobo O(V,W)$ and $f$, $g \in \fml F_0(W,X)$.
If $f \simeq g$, then $f \circ h \simeq g \circ h$.
\end{prop}

\begin{proof} Problem.  \ns   \end{proof}

\begin{prob}  Fix a vector $x$ in $V$. Define a function $M_x \colon \R \sto V$ by
  \[ M_x(\alpha)=\alpha x \qquad \text{for all $\alpha \in \R$.} \]
Notice that $M_x \in \lobo B(\R,V)$.  If $f \in \fml F_0(\R,V)$ and $f \simeq M_x$, then
 \begin{enumerate}
  \item[(a)] $\dfrac{f(\alpha)}{\alpha} \sto x$ as $\alpha \sto 0$;
and
  \item[(b)] $f(\alpha) \sto 0$ as $\alpha \sto 0$.
 \end{enumerate}
\end{prob}

\begin{prob} Each of the following is an abbreviated version of a proposition.  Formulate precisely
and prove.
 \begin{enumerate}
  \item[(a)] $\fml C_{\vc 0} + \lobo O \subseteq \fml C_{\vc 0}$.
  \item[(b)] $\fml C_{\vc 0} + \lobo o \subseteq \fml C_{\vc 0}$.
  \item[(c)] $\lobo O + \lobo o \subseteq \lobo O$.
 \end{enumerate}
\end{prob}

\begin{prob}  Suppose that $f \simeq g$. Then the following hold.
 \begin{enumerate}
  \item[(a)] If $g$ is continuous at $\vc 0$, so is $f$.
  \item[(b)] If $g$ belongs to $\lobo O$, so does $f$.
  \item[(c)] If $g$ belongs to $\lobo o$, so does $f$.
 \end{enumerate}
\end{prob}






\section{DIFFERENTIATION}

\begin{defn}  Let $f \in \fml F_a(V,W)$. Define the function
 \index{<@$\Delta F_a$}%
 \index{delta@$\Delta F_a$}%
$\Delta f_a$ by
  \[ \Delta f_a(h) \equiv f(a+h) - f(a) \]
for all $h$ such that $a + h$ is in the domain of $f$.  Notice that since $f$ is defined in a
neighborhood of $a$, the function $\Delta f_a$ is defined in a neighborhood of $\vc 0$; that
is, $\Delta f_a$ belongs to $\fml F_{\vc 0}(V,W)$. Notice also that $\Delta f_a(\vc 0) = \vc
0$.
\end{defn}

\begin{prop}\label{prop_del_sm} If $f \in \fml F_a(V,W)$ and $\alpha \in \R$, then
  \[ \Delta(\alpha f)_a = \alpha\,\Delta f_a\,. \]
\end{prop}

\begin{proof} No changes need to be made in the proof of~\ref{del_sm}.   \end{proof}

\begin{prop}\label{prop_del_sum} If $f$, $g \in \fml F_a(V,W)$, then
  \[ \Delta(f+g)_a = \Delta f_a + \Delta g_a\,. \]
\end{prop}

\begin{proof} The proof given (in the \emph{solutions to exercises}) for
proposition~\ref{del_sum} needs no alteration.
\end{proof}

\begin{prop}\label{prop_del_mult}  If $\phi \in \fml F_a(V,\R)$ and $f \in \fml F_a(V,W)$, then
  \[ \Delta(\phi f)_a = \phi(a)\cdot\Delta f_a \,+\, \Delta\phi_a\cdot f(a)
                                            \,+\, \Delta\phi_a\cdot\Delta f_a\,. \]
\end{prop}

\begin{proof} Problem.  \ns   \end{proof}

\begin{prop}\label{prop_del_comp} If $f \in \fml F_a(V,W)$, $g \in \fml F_{f(a)}(W,X)$, and
$g \circ f \in \fml F_a(V,X)$, then
  \[ \Delta(g \circ f)_a = \Delta g_{{}_{\sst{f(a)}}} \circ \Delta f_a\,. \]
\end{prop}

\begin{proof} Use the proof given (in the \emph{solutions to exercises}) for
proposition~\ref{del_comp}.
\end{proof}

\begin{prop}  A function $f \colon V \sto W$ is continuous at the point $a$ in $V$  if and only if
$\Delta f_a$ is continuous at $\vc 0$.
\end{prop}

\begin{proof} Problem.  \ns   \end{proof}

\begin{prop}\label{prop_del_inv}  If $f \colon U \sto U_1$ is a bijection between subsets of
arbitrary vector spaces, then for each $a$ in $U$ the function $\Delta f_a \colon U - a \sto
U_1 - f(a)$ is invertible and
  \[ \bigl(\Delta f_a\bigr)^{-1}=\Delta\bigl(f^{-1}\bigr)_{f(a)}\,. \]
\end{prop}

\begin{proof} Problem.   \ns    \end{proof}

\begin{defn}  Let $f \in \fml F_a(V,W)$.  We say that $f$ is
 \index{differentiable!at a point}%
\df{differentiable at} $a$ if there exists a bounded linear map which is tangent at $\vc 0$ to
$\Delta f_a$. If such a map exists, it is called the
 \index{differential}%
\df{differential} of $f$ at $a$ and is denoted
 \index{dfa@$df_a$ (differential of $f$ at~$a$)}%
by~$df_a$. Thus $df_a$ is just a member of $\ofml B(V,W)$ which satisfies $df_a \simeq \Delta
f_a$.  We denote by $\fml D_a(V,W)$ the family of all functions in $\fml F_a(V,W)$ which are
differentiable at~$a$. We often shorten this to~$\fml D_a$.
\end{defn}

\begin{prop}\label{prop_diff_uniq}  Let $f \in \fml F_a(V,W)$. If $f$ is differentiable at $a$,
then its differential is unique. (That is, there is at most one bounded linear map tangent at
$0$ to $\Delta f_a$.)
\end{prop}

\begin{proof} Proposition~\ref{prop_tan_lin}.  \end{proof}

\begin{rem}\label{rem_prop_diff}  If $f$ is a function in $\fml F_a(V,W)$ which is differentiable
at $a$, its differential $df_a$ has three important properties:
 \begin{enumerate}
  \item[(i)] it is linear;
  \item[(ii)] it is continuous (that is, bounded as a linear map);
  \item[(iii)] $\D\lim_{h \sto \vc 0} \frac{\Delta f_a(h) - df_a(h)}{\norm h} = \vc 0$\,.
 \end{enumerate}
(An expression of the form $\dfrac{\Delta f_a(h) - df_a(h)}{\norm h}$ is called a
 \index{Newton quotient}%
 \index{quotient!Newton}%
\df{Newton quotient}.)
\end{rem}

\begin{exer}\label{exer_diff_R2}  Let $f(x,y) = 3x^2 - xy + 4y^2$. Show that $df_{(1,-1)}(x,y) =
7x - 9y$.  Interpret $df_{(1,-1)}$ geometrically.  (Solution~\ref{sol_exer_diff_R2}.)
\end{exer}

\begin{exer}\label{exer_diff_R3R2}  Let
  \[ f \colon \R^3 \sto \R^2 \colon (x,y,z) \mapsto (x^2y - 7, 3xz + 4y) \]
and $a = (1,-1,0)$.  Use the \emph{definition} of ``differential'' to find $df_a$.
\emph{Hint.}  Work with the matrix representation of $df_a$.  Since the differential must
belong to $\ofml B(\R^3,\R^2)$, its matrix representation is a $2 \times 3$ matrix
$M = \begin{bmatrix}
          r  &  s  &  t \\
          u  &  v  &  w
     \end{bmatrix}$\,.  Use the requirement---condition (iii) of the preceding remark---that
$\norm h^{-1}\,\norm{\Delta f_a(h) - Mh} \sto \vc 0$ as $h \sto \vc 0$ to discover the
identity of the entries in~$M$.   (Solution~\ref{sol_exer_diff_R3R2}.)
\end{exer}

\begin{prop}\label{prop_diff_O}  If $f \in \fml D_a$, then $\Delta f_a \in \lobo O$.
\end{prop}

\begin{proof} Use the proof given (in the \emph{solutions to exercises}) for
proposition~\ref{del_in_O}.  Since we are working with bounded linear maps between normed
spaces instead of linear maps on~$\R$, we must change $\ofml L$ to~$\ofml B$.
\end{proof}

\begin{cor}\label{cor_diff_cont} Every function which is differentiable at a point is continuous
there.
\end{cor}

\begin{proof} Use the proof given (in the \emph{solutions to exercises}) for
corollary~\ref{diff_cont}.  \end{proof}

\begin{prop}\label{prop_diff_sm}  If $f$ is differentiable at $a$ and $\alpha \in \R$, then
$\alpha f$ is differentiable at $a$ and
  \[ d(\alpha f)_a = \alpha\,df_a\,. \]
\end{prop}

\begin{proof} The proof given (in the \emph{solutions to exercises}) for
proposition~\ref{diff_sm} works, but three references need to be changed.  (What are the
correct references?)
\end{proof}

\begin{prop}\label{prop_diff_sum} If $f$ and $g$ are differentiable at $a$, then $f + g$ is
differentiable at $a$ and
  \[ d(f+g)_a = df_a + dg_a\,. \]
\end{prop}

\begin{proof} Problem.  \ns  \end{proof}

 \index{Leibniz's rule}%
\begin{prop}[Leibniz's Rule]\label{prop_Leib_rul}
 \index{Leibniz's rule}%
If $\phi \in \fml D_a(V,\R)$ and $f \in \fml D_a(V,W)$, then $\phi f \in \fml D_a(V,W)$ and
  \[ d(\phi f)_a = d\phi_a \cdot f(a) \,+\, \phi(a)\,df_a\,. \]
\end{prop}

\begin{proof} Exercise.  (Solution~\ref{sol_prop_Leib_rul}.)  \ns   \end{proof}

In the chapters of beginning calculus texts devoted to the differential calculus of several
variables, the expression ``chain rule'' refers most frequently to a potentially infinite
collection of related results concerning the differentiation of composite functions.  Several
examples such as the following are usually given:

\begin{quote}
Let $w = f(x,y,z)$, $x = x(u,v)$, $y = y(u,v)$, and $z = z(u,v)$. If these functions are all
differentiable, then the function
  \[ w = f(x(u,v)\,,\,y(u,v)\,,\,z(u,v)) \]
is differentiable and
 \begin{equation*}\tag{$\ast$}
  \begin{split}\label{eqn_pd_cr}
    \frac{\partial w}{\partial u}
     &= \frac{\partial w}{\partial x}\frac{\partial x}{\partial u}
     + \frac{\partial w}{\partial y}\frac{\partial y}{\partial u}
     + \frac{\partial w}{\partial z}\frac{\partial z}{\partial u}\\
   \frac{\partial w}{\partial v}
     &= \frac{\partial w}{\partial x}\frac{\partial x}{\partial v}
     + \frac{\partial w}{\partial y}\frac{\partial y}{\partial v}
     + \frac{\partial w}{\partial z}\frac{\partial z}{\partial v}\\
  \end{split}
 \end{equation*}
\end{quote}

\noindent Then the reader is encouraged to invent new ``chain rules'' for functions having
different numbers of variables. Formulations such as \eqref{eqn_pd_cr} have many shortcomings
the most serious of which is a nearly complete lack of discernible geometric content.  The
version of the \emph{chain rule} which we will prove says that (after a suitable translation)
the best linear approximation to the composite of two functions is the composite of the best
linear approximations.  In other words, the differential of the composite is the composite of
the differentials.  Notice that theorem~\ref{thm_ch_rul}, where this is stated formally, is
simpler than \eqref{eqn_pd_cr}; it has obvious geometric content; it is a single ``rule''
rather than a family of them; and it holds in arbitrary infinite dimensional normed linear
spaces as well as in finite dimensional ones.

\begin{prop}[The Chain Rule.]\label{thm_ch_rul}
 \index{chain rule}%
If $f \in \fml D_a(V,W)$ and $g \in \fml D_{f(a)}(W,X)$, then $g \circ f \in \fml D_a(V,X)$
and
  \[ d(g \circ f)_a = dg_{{}_{\scriptstyle{f(a)}}} \circ df_a\,. \]
\end{prop}

\begin{proof} Exercise.  (Solution~\ref{sol_thm_ch_rul}.)  \ns   \end{proof}

Each of the preceding propositions concerning differentiation is a direct consequence of a
similar result concerning the map $\Delta$. In particular, the linearity of the map $f \mapsto
df_a$ (propositions~\ref{prop_diff_sm} and~\ref{prop_diff_sum}) follows from the fact that the
function $f \mapsto \Delta f_a$ is linear (propositions~\ref{prop_del_sm}
and~\ref{prop_del_sum}); \emph{Leibniz's rule}~\ref{prop_Leib_rul} is a consequence of
proposition~\ref{prop_del_mult}; and the proof of the \emph{chain rule}~\ref{thm_ch_rul} makes
use of proposition~\ref{prop_del_comp}.  It is reasonable to hope that the result given in
proposition~\ref{prop_del_inv} concerning $\Delta\bigl(f^{-1}\bigr)_{f(a)}$ will lead to
useful information concerning the differential of the inverse function. This is indeed the
case; but obtaining information about $d\bigl(f^{-1}\bigr)_{f(a)}$ from~\ref{prop_del_inv} is
a rather involved undertaking.  It turns out, happily enough, that (under mild hypotheses) the
differential of $f^{-1}$ is the inverse of the differential of $f$.  This result is known as
the \emph{inverse function theorem}, and the whole of chapter~\ref{IFT} is devoted to
detailing its proof and to examining some of its consequences.

\begin{prob} Show that proposition~\ref{prop_diff_sm} is actually a special case of
\emph{Leibniz's rule}~\ref{prop_Leib_rul}.  Also, suppose that $\phi \in \fml D_a(V,\R)$ and
$w \in W$.  Prove that $\phi w \in \fml D_a(V,W)$ and that $d(\phi w)_a = d\phi_a \cdot w$.
\end{prob}

\begin{prob}  Let $f \colon \R^2 \sto \R^2$ be defined by $f(x) = (3x_1 - x_2 + 7, x_1 + 4x_2)$.
Show that $\bigl[df_{(1,0)}\bigr] =
 \begin{bmatrix}
     3  &  -1 \\
     1  &  4
 \end{bmatrix}$.  \emph{Hint.}  Let
 $M = \begin{bmatrix}
            3  &  -1 \\
            1  &  4
      \end{bmatrix}$ and $a = (1,0)$.  Show that the Newton quotient
$\dfrac{\mathstrut\Delta^{\mathstrut} f_a(h) - Mh}{\norm h}$ approaches $\vc 0$ as $h \sto \vc
0$. Use the uniqueness of differentials (proposition~\ref{prop_diff_uniq}) to conclude that
$\bigl[df_a\bigr] = M$.
\end{prob}

\begin{prob}  Let $F \colon \R^2 \sto \R^4$ be defined by
  \[ F(x) = (x_2\,,\,{x_1}^2\,,\,4 - x_1x_2\,,\,7x_1) \]
and let $a = (1,1)$. Use the \emph{definition} of ``differential'' to show that
  \[\bigl[dF_a\bigr] =
     \begin{bmatrix}
         0  &  1 \\
         2  &  0 \\
        -1  & -1 \\
         7  &  0
     \end{bmatrix}\,. \]
\end{prob}

\begin{prob}  Let $F \colon \R^3 \sto \R^4\colon  x \mapsto (x_1 + 2x_3, x_2 - x_3, 4x_2,
2x_1 - 5x_2)$ and $a = (1,2,-5)$. Use the \emph{definition} of ``differential'' to find
$\bigl[dF_a\bigr]$.  \emph{Hint.} Use the technique suggested in
exercise~\ref{exer_diff_R3R2}.
\end{prob}

\begin{prob}  Let $F \colon \R^3 \sto \R^2$ be defined by $F(x,y,z) = (xy - 3,y + 2z^2)$.
Use the \emph{definition} of ``differential'' to find $\bigl[dF_{(1,-1,2)}\bigr]$.
\end{prob}

\begin{prob}  Let $F \colon \R^3 \sto \R^3: x \mapsto (x_1x_2,x_2 - {x_3}^2,2x_1x_3)$.  Use the
\emph{definition} of ``differential'' to find $\bigl[dF_a\bigr]$ at $a$ in~$\R^3$.
\end{prob}

\begin{prob}\label{prob_diff_lt} Let $T \in \ofml B(V,W)$ and $a \in V$. Find $dT_a$.
\end{prob}








\section{DIFFERENTIATION OF CURVES}\label{diff_curv} In the preceding section we have discussed
the differentiability of functions mapping one normed linear space into another. Here we
briefly consider the important and interesting special case which occurs when the domains of
the functions are one dimensional.
\begin{defn}  A
 \index{curve}%
\df{curve} in a normed linear space $V$ is a continuous mapping from an interval in $\R$
into~$V$.  If $c \colon J \sto V$ is a curve, if $0$ belongs to the interval $J$, and if $c(0)
= a \in V$, then $c$ is a
 \index{curve!at a point}%
\df{curve at}~$a$.
\end{defn}

In classical terminology a curve $c \colon J \sto V \colon t \mapsto c(t)$ is usually referred
to as a
 \index{curve!parametrized}%
 \index{parametrized!curve}%
\df{parametrized curve} in~$V$.  The interval $J$ is the
 \index{parameter!interval}%
\df{parameter interval} and the variable $t$ belonging to $J$ is the
 \index{parameter}%
\df{parameter}.  If you start with a subset $A$ of $V$ and find a continuous function $c$ from
an interval $J$ onto $A$, then $c$ is called a
 \index{parametrization}%
\df{parametrization} of~$A$.

\begin{exam}  Let
 \begin{align*}
      c_1 &\colon [0,2\pi] \sto \R^2 \colon t \mapsto (\cos t,\sin t) \\
   \intertext{and}
      c_2 &\colon [0,3\pi] \sto \R^2 \colon t \mapsto (\sin(2t + 1), \cos(2t + 1)).
 \end{align*}
Then $c_1$ and $c_2$ are two different parametrizations of the unit circle
  \[ \Sp^1 := \{(x,y) \in \R^2 \colon x^2 + y^2 = 1\}\,. \]
Parameters need have no physical significance, but it is quite common to think of a curve in
$\R^2$ or $\R^3$ as representing the motion of a particle in that  space: the parameter $t$ is
taken to be time and the range value $c(t)$ is the position of the particle at time~$t$.  With
this interpretation we may view $c_1$ and $c_2$ as representing the motion of particles
traveling around the unit circle.  In the first case the particle starts (when $t=0$) at the
point $(1,0)$ and makes one complete trip around the circle traveling counterclockwise.  In
the second case, the particle starts at the point $(\sin 1,\cos 1)$ and traverses $\Sp^1$
three times moving clockwise.
\end{exam}

\begin{exam}  Let $a$ and $u \ne \vc 0$ be vectors in $V$. The curve
  \[ \ell\colon \R \sto V\colon t \mapsto a + tu \]
is the
 \index{parametrized!line}%
 \index{line!parametrized}%
\df{parametrized line} through $a$ in the direction of~$u$.  Of course, infinitely many other
parametrizations of the range of $\ell$ are possible, but this is the standard one and we
adopt it.
\end{exam}

\begin{prob}  Find a parametrization of the unit square
  \[ A := \{(x,y) \in \R^2\colon  d_u\bigl((x,y)\, , \,({\tfrac12},
                       {\tfrac12}) \bigr) = {\tfrac12}\}\]
which starts at $(0,0)$ and traverses $A$ once in a counterclockwise direction.
\end{prob}

\begin{defn} Let $c\colon J \sto V$ be a curve in $V$ and suppose that $a$ is a point in the
interior of the interval~$J$.  Then
 \index{dc@$Dc$ (derivative of a curve $c$)}%
$Dc(a)$, the
 \index{derivative!of a curve}%
 \index{curve!derivative of}%
\df{derivative} of $c$ at $a$, is defined by the formula
  \[ Dc(a) := \lim_{h \sto \vc 0}\frac{\Delta c_a(h)}h \]
if the indicated limit exists.  Notice that this is just the definition of ``derivative''
given in beginning calculus.  The derivative at $a$ is also called the
 \index{tangent!vector}%
\df{tangent vector} to $c$ at $a$ or, in case we are thinking of the motion of a particle, the
 \index{velocity}%
\df{velocity} of $c$ at~$a$.  If $Dc(a)$ exists and is not zero, then the parametrized line
through $c(a)$ in the direction of $Dc(a)$ is the
 \index{tangent!line}%
 \index{line!tangent}%
\df{tangent line} to the image of $c$ at the point~$c(a)$.
\end{defn}

\begin{exer}\label{exer_tang_curv}  Let $c\colon [0,2\pi] \sto \R^2\colon  t \mapsto (\cos t,\sin t)$.
 \begin{enumerate}
  \item[(a)] Find the tangent vector to $c$ at $t=\frac{\pi}3$.
  \item[(b)] Find the tangent line to the range of $c$ at the point $(\frac12,\frac{\sqrt 3}2)$.
  \item[(c)] Write the equation of the range in $\R^2$ of the tangent line found in~(b).
 \end{enumerate}
(Solution~\ref{sol_exer_tang_curv}.)
\end{exer}

\begin{prop} \label{prop_diff_der} If a curve $c\colon J \sto V$ is differentiable at a point $a$
in the interior of the interval $J$, then it has a derivative at $a$ and
  \[ Dc(a) = dc_a(1)\,. \]
\end{prop}

\begin{proof} Exercise.  \emph{Hint.}  Start with the Newton quotient $\dfrac{\Delta c_a(h)}h$.
Subtract and add $\dfrac{dc_a(h)}h$.)  (Solution~\ref{sol_prop_diff_der}.)    \ns
\end{proof}


The converse of the preceding proposition is also true. Every curve possessing a derivative is
differentiable. This is our next proposition.

\begin{prop}\label{prop_der_diff} If a curve $c\colon J \sto V$ has a derivative at a point $a$
in $\intr J$, then it is differentiable at $a$ and
  \[ dc_a(h) = h\,Dc(a) \qquad \text{ for all $h \in \R$}. \]
\end{prop}

\begin{proof} Problem.  \emph{Hint.}  Define $T \colon \R \sto V\colon  h \mapsto h\,Dc(a)$.
Show that $\Delta c_a \simeq T$.  \ns
\end{proof}

\begin{prob} Suppose that curves $c_1$ and $c_2$ in a normed linear space are defined and
differentiable in some neighborhood of $a \in \R$.  Then
 \begin{enumerate}
  \item[(a)] $D(c_1 + c_2)(a) = Dc_1(a) + Dc_2(a)$.
  \item[(b)] $D(\alpha c_1)(a) = \alpha\,Dc_1(a)$ for all $\alpha \in \R$.
 \end{enumerate}
\end{prob}

\begin{prob}\label{prob_nasc_tan_curves} Let $V$ be a normed linear space and $a \in V$.
 \begin{enumerate}
  \item[(a)] Suppose $c$ is a differentiable curve at the zero vector in~$V$.  Then $c \simeq \vc 0$
if and only if $Dc(0) = \vc 0$.
  \item[(b)] Suppose $c_1$ and $c_2$ are differentiable curves at~$a$.  Then $c_1 \simeq c_2$ if
and only if $Dc_1(0) = Dc_2(0)$.
 \end{enumerate}
\end{prob}

\begin{prop}\label{prob_der_curv1} If $c \in \fml D_t(\R, V)$ and $f \in \fml D_a(V,W)$, where
$a = c(t)$, then $f \circ c \in \fml D_t(\R,W)$ and
  \[ D(f \circ c)(t) = df_a(Dc(t))\,. \]
\end{prop}

\begin{proof} Problem.  \ns  \end{proof}

Thus far integration and differentiation have been treated as if they belong to separate
worlds. In the next theorem, known as the \emph{fundamental theorem of calculus}, we derive
the most important link between these two topics.

 \index{fundamental theorem of calculus}%
\begin{thm}[Fundamental Theorem Of Calculus]\label{thm_ftc}  Let $a$ belong to an open interval
$J$ in the real line, $E$ be a Banach space, and $f \colon J \sto E$ be a regulated curve.
Define $F(x) = \int_a^xf$ for all $x \in J$.  If $f$ is continuous at $c \in J$, then $F$ is
differentiable at $c$ and $DF(c) = f(c)$.
\end{thm}

\begin{proof}  Exercise. \emph{Hint.}   For every $\epsilon > 0$ there exists $\delta > 0$
such that $c + h \in J$ and $\norm{\Delta f_c(h)} < \epsilon$ whenever $\abs h < \delta$.
(Why?)  Use the (obvious) fact that $h\,f(c) = \int_c^{c + h}f(c)\,dt$ to show that
$\norm{\Delta F_c(h) - h\,f(c)} < \epsilon\,\abs h$ whenever $0 < \abs h < \delta$. From this
conclude that $\lim\limits_{h \sto 0}\dfrac 1h \Delta F_c(h) = f(c)$.
(Solution~\ref{sol_thm_ftc}.)  \ns
\end{proof}




\section{DIRECTIONAL DERIVATIVES} We now return to the study of maps between arbitrary normed
linear spaces.  Closely related to differentiability is the concept of \emph{directional
derivative}, an examination of which provides some technical advantages and also throws light
on the geometric aspect of differentiation.

\begin{defn}  Let $f$ be a member of $\fml F_a(V,W)$ and $v$ be a nonzero vector in~$V$.  Then
$D_vf(a)$, the
 \index{dvf@$D_vf$ (directional derivative)}%
 \index{derivative!directional}%
 \index{directional derivative}%
\df{derivative of $f$ at $a$ in the direction of $v$}, is defined by
  \[ D_vf(a) := \lim_{t \sto 0} \frac1t\,\Delta f_a(tv)\]
if this limit exists.  This directional derivative is also called the
 \index{G\^ateaux!differential}%
 \index{differential!G\^ateaux}%
\df{G\^ateaux differential} (or
 \index{G\^ateaux!variation}%
 \index{variation!G\^ateaux}%
\df{G\^ateaux variation}) of $f$, and is sometimes denoted by $\delta f(a;v)$. Many authors
require that in the preceding definition $v$ be a unit vector.  We will \emph{not} adopt this
convention.
\end{defn}

Recall that for $\vc 0 \ne v \in V$ the curve $\ell \colon \R \sto V$ defined by $\ell(t) = a
+ tv$ is the parametrized line through $a$ in the direction of~$v$.  In the following
proposition, which helps illuminate our use of the adjective ``directional'', we understand
the domain of $f \circ \ell$ to be the set of all numbers $t$ for which the expression
$f(\ell(t))$ makes sense; that is,
  \[ \dom(f \circ \ell) = \{t \in \R \colon  a + tv \in \dom f\}\,. \]
Since $a$ is an interior point of the domain of $f$, the domain of $f \circ \ell$ contains an
open interval about~$0$.

\begin{prop}\label{prop_diff_dd}  If $f \in \fml D_a(V,W)$ and $\vc 0 \ne v \in V$, then the
directional derivative $D_vf(a)$ exists and is the tangent vector to the curve $f \circ \ell$
at $0$ (where $\ell$ is the parametrized line through $a$ in the direction of~$v$).  That is,
  \[ D_vf(a) = D(f \circ \ell)(0)\,. \]
\end{prop}

\begin{proof} Exercise.  (Solution~\ref{sol_prop_diff_dd}.)  \ns   \end{proof}

\begin{exer}\label{exer_dd1} Let $f(x,y) = \ln\bigl(x^2 + y^2\bigr)^{\frac12}$.  Find $D_vf(a)$
when $a=(1,1)$ and $v=(\frac35,\frac45)$.  (Solution~\ref{sol_exer_dd1}.)
\end{exer}

\begin{notn}  For $a < b$ let $\fml C^1([a,b],\R)$ be the family of all functions $f$ differentiable
on some open subset of $\R$ containing $[a,b]$ whose derivative $Df$ is continuous on~$[a,b]$.
\end{notn}

\begin{exer}\label{exer_dd2}  For all $f$ in $\fml C^1([0,\frac{\pi}2],\R)$ define
  \[ \phi(f) = \int_0^{\frac{\pi}2}(\cos x + Df(x))^2\,dx\,. \]
Compute $D_v\phi(a)$ when $a(x) = 1 + \sin x$ and $v(x) = 2 - \cos x$ for $0 \le x \le
\frac{\pi}2$.   (Solution~\ref{sol_exer_dd2}.)
\end{exer}

\begin{prob} Let $f(x,y) = e^{x^2 + y^2}$, $a = (1,-1)$, and $v = (-1,2)$.  Find $D_vf(a)$.
\end{prob}

\begin{prob} Let $f(x,y) = (2xy,y^2 - x^2)$, $a = (-1,2)$, and $v =
\bigl(\frac1{\sqrt2},\frac1{\sqrt2}\bigr)$.  Find $D_vf(a)$.
\end{prob}

\begin{prob} Let $\phi\colon \fml C([0,1],\R) \sto \R\colon  f \mapsto
\int_0^1\bigl(\sin^6\pi x + (f(x))^2\bigr)\,dx$, and for $0 \le x \le 1$ let $a(x) = e^{-x} -
x + 3$ and $v(x) = e^x$.  Find $D_v\phi(a)$.
\end{prob}

According to proposition~\ref{prop_diff_dd}, differentiability implies the existence of
directional derivatives in all directions. (In problem~\ref{prob_dd3} you are asked to show
that the converse is not true.)  The next proposition makes explicit the relationship between
differentials and directional derivatives for differentiable functions.

\begin{prop}\label{prop_dd_diff} If $f \in \fml D_a(V,W)$, then for every nonzero $v$ in $V$
  \[ D_vf(a) = df_a(v)\,. \]
\end{prop}

\begin{proof} Exercise.  \emph{Hint.} Use problem~\ref{prob_der_curv1}
(Solution~\ref{sol_prop_dd_diff}.)  \ns
\end{proof}

It is worth noticing that even though the domain of a curve is $1$-dimensional, it is still
possible to take directional derivatives.  The relationship between derivatives and
directional derivatives is very simple: the derivative of a curve $c$ at a point $t$ is just
the directional derivative at $t$ in the direction of the unit vector $1$ in~$\R$.  Proof:
 \begin{alignat*}{2}
    Dc(t) &= dc_t(1) & &\quad \text{(by
                           proposition~\ref{prop_diff_der})} \\
          &= D_1c(t) & &\quad \text{(by
                           proposition~\ref{prop_dd_diff})}
 \end{alignat*}

\begin{prob}  Let $\phi \colon \fml C([0,1],\R) \sto \fml C^1([0,1],\R)$ be defined by $\phi f(x)
= \int_0^x f(s)\,ds$ for all $f$ in $\fml C([0,1],\R)$ and $x$ in~$[0,1]$. For arbitrary
functions $a$ and $v \ne 0$ in $\fml C([0,1],\R)$ compute $D_v\phi(a)$ using
 \begin{enumerate}
  \item[(a)] the definition of ``directional derivative''.
  \item[(b)] proposition~\ref{prop_diff_dd}.
  \item[(c)] proposition~\ref{prop_dd_diff}.
 \end{enumerate}
\end{prob}

\begin{prob}\label{prob_dd3}  Show that the converse of proposition~\ref{prop_diff_dd} need not hold.
A function may have directional derivatives in all directions but fail to be differentiable.
\emph{Hint.}  Consider the function defined by $f(x,y) = \smash[b]{\dfrac{x^3}{x^2 + y^2}}$
for $(x,y) \ne (0,0)$ and $f(0,0) = 0$.
\end{prob}










\section{FUNCTIONS MAPPING INTO PRODUCT SPACES} In this short section we examine the
differentiability of functions which map into the product of normed linear spaces. This turns
out to be a very simple matter: a necessary and sufficient condition for a function whose
codomain is a product to be differentiable is that each of its components be differentiable.
The somewhat more complicated topic of differentiation of functions whose \emph{domain} lies
in a product of normed linear spaces is deferred until the next chapter when we have the
\emph{mean value theorem} at our disposal.

\begin{prop}\label{prop_diff_prodsp} If $f^1 \in \fml D_a(V,W_1)$ and $f^2 \in \fml D_a(V,W_2)$,
then the function $f = \bigl(f^1,f^2\bigr)$ belongs to $\fml D_a(V,W_1 \times W_2)$ and $df_a
= \bigl(d(f^1)_a,d(f^2)_a\bigr)$.
\end{prop}

\begin{proof}    Exercise. \emph{Hint.}  The function $f$ is defined by $f(x) =
\bigl(f^1(x),f^2(x)\bigr)$ for all $x \in \dom f^1 \cap \dom f^2$.  Check that $f =
j_{{}_{\sst 1}} \circ f^1 \,+\, j_{{}_{\sst 2}} \circ f^2$ where $j_{{}_{\sst 1}}\colon  W_1
\sto W_1 \times W_2\colon  u \mapsto (u,0)$ and $j_{{}_{\sst 2}}\colon W_2 \sto W_1 \times
W_2\colon  v \mapsto (0,v)$.  Use proposition \ref{prop_diff_sum}, theorem \ref{thm_ch_rul},
and problem~\ref{prob_diff_lt}.  (Solution~\ref{sol_prop_diff_prodsp}.)  \ns
\end{proof}

The preceding proposition says that a function is differentiable if its components are. The
converse, which is the next proposition, is also true: the components of a differentiable
function are differentiable.

\begin{prop}\label{prop_diff_coord}  If $f$ belongs to $\fml D_a(V, W_1 \times W_2)$, then its
components $f^1$ and $f^2$ belong to $\fml D_a(V,W_1)$ and $\fml D_a(V,W_2)$, respectively,
and
  \[ df_a = \bigl(d(f^1)_a, d(f^2)_a\bigr)\,. \]
\end{prop}

\begin{proof} Problem.  \emph{Hint.}  For $k = 1$, $2$ write $f^k = \pi_k \circ f$ where, as usual,
$\pi_k(x_1,x_2) = x_k$. Then use the \emph{chain rule}~\ref{thm_ch_rul} and
problem~\ref{prob_diff_lt}.   \ns  \end{proof}

An easy consequence of the two preceding propositions is that curves in product spaces have
derivatives if and only if their components do.

\begin{cor}\label{cor_diff_prodcurv}  Let $c^1$ and $c^2$ be curves in $W_1$ and $W_2$,
respectively.  If $c^1$ and $c^2$ are differentiable at $t$, then so is the curve $c =
(c^1,c^2)$ in $W_1 \times W_2$ and $Dc(t) = \bigl(Dc^1(t), Dc^2(t)\bigr)$. Conversely, if $c$
is a differentiable curve in the product $W_1 \times W_2$, then its components $c^1$ and $c^2$
are differentiable and $Dc = \bigl(Dc^1,Dc^2\bigr)$.
\end{cor}

\begin{proof} Exercise.  (Solution~\ref{sol_cor_diff_prodcurv}.)  \ns  \end{proof}

It is easy to see how to generalize the three preceding results to functions whose codomains
are the products of any finite collection of normed linear spaces.  Differentiation is done
componentwise.

\begin{exam}  Consider the helix
  \[ c \colon \R \sto \R^3 \colon t \mapsto (\cos t,\sin t,t)\,. \]
Its derivative at $t$ is found by differentiating each component separately. That is, $Dc(t) =
(-\sin t,\cos t,1)$ for all $t \in \R$.
\end{exam}

\begin{prob}  Let $f(x,y) = (\ln(xy), y^2 - x^2)$, $a = (1,1)$, and $v = (\frac35,\frac45)$.
Find the directional derivative $D_vf(a)$.
\end{prob}
