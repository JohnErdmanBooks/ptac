\chapter{COUNTABLE AND UNCOUNTABLE SETS}


 \setcounter{section}{1}
 \setcounter{thm}{0}


There are many sizes of infinite sets---infinitely many in fact. In our subsequent work we
need only distinguish between countably infinite and uncountable sets. A set is countably
infinite if it is in one-to-one correspondence with the set of positive integers; if it is
neither finite nor countably infinite, it is uncountable. In this section we present some
basic facts about and examples of both countable and uncountable sets. This is all we will
need. Except for problem~\ref{prob_unc_notr}, which is presented for general interest, we
ignore the many intriguing questions which arise concerning various sizes of uncountable sets.
For a very readable introduction to such matters see \cite{Kaplansky:1977}, chapter~2.

\begin{defn}  A set is
 \index{countably infinite}%
 \index{infinite!countably}%
\df{countably infinite} (or
\index{denumerable}%
\df{denumerable}) if it is cardinally equivalent to the set $\N$ of natural numbers. A
bijection from $\N$ onto a countably infinite set $S$ is an
 \index{enumeration}%
\df{enumeration} of the elements of~$S$. A set is
 \index{countable}%
\df{countable} if it is either finite or countably infinite.  If a
set is not countable it is
 \index{uncountable}%
\df{uncountable}.
\end{defn}

\begin{exam} The set $\E$ of even integers in $\N$  is countable.
\end{exam}

\begin{proof}  The map $n \mapsto 2n$ is a bijection from $\N$ onto~$\E$.
\end{proof}

The first proposition of this section establishes the fact that the ``smallest'' infinite sets
are the countable ones.

\begin{prop}  Every infinite set contains a countably infinite subset.
\end{prop}


\begin{proof} Problem. \emph{Hint.}  Review the proof of proposition~\ref{prop_ce_sub}. \ns
\end{proof}

If we are given a set $S$ which we believe to be countable, it may be extremely difficult to
prove this by exhibiting an explicit bijection between $\N$ and~$S$.  Thus it is of great
value to know that certain constructions performed with countable sets result in countable
sets.  The next five propositions provide us with ways of generating new countable sets from
old ones.  In particular, we show that each of the following is countable.
 \begin{enumerate}
  \item Any subset of a countable set.
  \item The range of a surjection with countable domain.
  \item The domain of an injection with countable codomain.
  \item The product of any finite collection of countable sets.
  \item The union of a countable family of countable sets.
 \end{enumerate}

\begin{prop}\label{prop_sub_cntbl}  If $S \subseteq T$ where $T$ is countable, then $S$ is countable.
\end{prop}

\begin{proof} Exercise. \emph{Hint.} Show first that every subset of $\N$ is countable.
(Solution~\ref{sol_prop_sub_cntbl}.)   \ns
\end{proof}

The preceding has an obvious corollary: if $S \subseteq T$ and $S$ is uncountable, then so
is~$T$.

\begin{prop}\label{prop_preimg_cntbl} If $f \colon S \sto T$ is injective and $T$ is countable,
then $S$ is countable.
\end{prop}

\begin{proof} Problem.  \emph{Hint.}  Adapt the proof of proposition~\ref{prop_preimg_fin}.  \ns
\end{proof}

\begin{prop}\label{prop_ran_cntbl}  If $f \colon S \sto T$ is surjective and $S$ is countable,
then $T$ is countable.
\end{prop}

\begin{proof} Problem.  \emph{Hint.}  Adapt the proof of proposition~\ref{prop_ran_fin}.  \ns
\end{proof}

\begin{lem}\label{lem_nxn_cntbl}  The set $\N \times \N$ is countable.
\end{lem}

\begin{proof} Exercise. \emph{Hint.}   Consider the map $(m,n) \mapsto 2^{m-1}(2n - 1)$.
(Solution~\ref{sol_lem_nxn_cntbl}.) \ns
\end{proof}

\begin{exam}\label{exam_qplus_cntbl} The set $\Q^+ \setminus \{0\} = \{x \in \Q \colon x > 0\}$
is countable.
\end{exam}

\begin{proof} Suppose that the rational number $m/n$ is written in lowest terms. (That is, $m$
and $n$ have no common factors greater than~1.) Define $f(m/n) = (m,n)$. It is easy to see
that the map $f \colon \Q^+\setminus\{0\} \sto \N \times \N$ is injective. By proposition
\ref{prop_preimg_cntbl} and the preceding lemma, $Q^+$ is countable.
\end{proof}

\begin{prop}\label{prop_prod_cntbl} If $S$ and $T$ are countable sets, then so is $S \times T$.
\end{prop}

\begin{proof} Problem.  \emph{Hint.}  Either $S \sim \{1,\dots,n\}$ or else $S \sim \N$. In
either case there exists an injective map $f \colon S \sto \N$. Similarly there exists an
injection $g \colon T \sto \N$. Define the function $f \times g \colon S \times T \sto \N
\times \N$ by $(f \times g)(x,y) = (f(x),g(y))$.   \ns
\end{proof}

\begin{cor}  If $S_1, \dots, S_n$ are countable sets, then $S_1 \times\dots\times S_n$ is
countable.
\end{cor}

\begin{proof} Proposition \ref{prop_prod_cntbl} and induction.
\end{proof}

Finally we show that a countable union of countable sets is countable.

\begin{prop}\label{prop_union_cntbl}  Suppose that $\sfml A$ is a countable family of sets and
that each member of $\sfml A$ is itself countable. Then $\bigcup\sfml A$ is countable.
\end{prop}

\begin{proof} Exercise. \emph{Hint.}  Use lemma \ref{lem_nxn_cntbl} and
proposition~\ref{prop_ran_cntbl}.   (Solution~\ref{sol_prop_union_cntbl}.) \ns
\end{proof}

\begin{exam}  The set $\Q$ of rational numbers is countable.
\end{exam}

\begin{proof} Let $A = \Q^+\setminus\{0\}$ and $B = -A = \{x \in \Q \colon x < 0\}$. Then $\Q
= A \cup B \cup \{0\}$.  The set $A$ is countable by example~\ref{exam_qplus_cntbl}.  Clearly
$A \sim B$  (the map $x \mapsto -x$ is a bijection); so $B$ is countable. Since $Q$ is the
union of three countable sets, it is itself countable by the preceding proposition.
\end{proof}

By virtue of \ref{prop_sub_cntbl}--\ref{prop_union_cntbl} we have a plentiful supply of
countable sets. We now look at an important example of a set which is not countable.

\begin{exam}\label{exam_r_uncntbl}  The set $\R$ of real numbers is uncountable.
\end{exam}

\begin{proof} We take it to be known that if we exclude decimal expansions which end in an
infinite string of 9's, then every real number has a unique decimal expansion. (For an
excellent and thorough discussion of this matter see Stromberg's beautiful text on
\emph{classical real analysis}~\cite{Stromberg:1981}, especially Theorem~2.57.)  By (the
corollary to) proposition~\ref{prop_sub_cntbl} it will suffice to show that the open unit
interval $(0,1)$ is uncountable. Argue by contradiction: assume that $(0,1)$ is countably
infinite. (We know, of course, from exercise~\ref{exer_int_inf} that it is not finite.) Let
$r_1, r_2, r_3, \dots$ be an enumeration of $(0,1)$. For each $j \in \N$ the number $r_j$ has
a unique decimal expansion
  \[ 0.r_{j1}\,r_{j2}\,r_{j3}\,\dots\,. \]
Construct another number $x = 0.x_1\,x_2\,x_3\,\dots$ as follows. For each $k$ choose $x_k= 1$
if $r_{kk} \ne 1$ and $x_k = 2$ if $r_{kk} = 1$. Then $x$ is a real number between $0$ and
$1$, and it cannot be any of the numbers $r_k$ in our enumeration (since it differs from $r_k$
at the $k^{\text{th}}$ decimal place). But this contradicts the assertion that $r_1, r_2, r_3,
\dots$ is an enumeration of $(0,1)$.
\end{proof}

\begin{prob} Prove that the set of irrational numbers is uncountable.
\end{prob}

\begin{prob} Show that if $S$ is countable and $T$ is uncountable, then $T \setminus S  \sim T$.
\end{prob}

\begin{prob}  Let $\epsilon$ be an arbitrary number greater than zero. Show that the rationals
in $[0,1]$ can be covered by a countable family of open intervals the sum of whose
lengths is no greater than $\epsilon$. (Recall that a family $\sfml U$ of sets is said to
\df{cover} a set $A$ if $A \subseteq \bigcup \sfml U$.) Is it possible to cover the set
$\Q$ of all rationals in $\R$ by such a family?  \emph{Hint.}  $\sum_{k=1}^\infty
2^{-k}~=~1$.
\end{prob}

\begin{prob} (Definition: The \df{open disk} in $\R^2$ with radius $r > 0$ and center $(p,q)$ is
defined to be the set of all points $(x,y)$ in $\R^2$ such that $(x - p)^2 + (y - q)^2 <
r^2$.)  Prove that the family of all open disks in the plane whose centers have rational
coordinates and whose radii are rational is countable.
\end{prob}

\begin{prob} (Definition: A real number is
 \index{algebraic number}%
\df{algebraic} if it is a root of some polynomial of degree greater than $0$ with integer
coefficients. A real number which is not algebraic is \df{transcendental}.  It can be shown
that the numbers $\pi$ and $e$, for example, are transcendental.) Show that the set of all
transcendental numbers in $\R$ is uncountable.  \emph{Hint.}  Start by showing that the set of
polynomials with integer coefficients is countable.
\end{prob}

\begin{prob} Prove that the set of all sequences whose terms consist of only 0's  and 1's is
uncountable.  \emph{Hint.} Something like the argument in example~\ref{exam_r_uncntbl} works.
\end{prob}

\begin{prob} Let $\sfml J$ be a disjoint family of intervals in $\R$ each with length greater
than~0.  Show that $\sfml J$ is countable.
\end{prob}

\begin{prob}\label{prob_unc_notr} Find an uncountable set which is not cardinally equivalent
to~$\R$. \emph{Hint.}  Let $\fml F = \fml F(\R,\R)$. Assume there exists a bijection $\phi
\colon \R \sto \fml F$. What about the function $f$ defined by
  \[ f(x) = 1 + \bigl(\phi(x)\bigr)(x) \]
for all $x \in \R$?
\end{prob}





\endinput
