\chapter{METRIC SPACES}\label{metric}

Underlying the definition of the principal objects of study in calculus---derivatives,
integrals, and infinite series---is the notion of ``limit''. What we mean when we write
     \[\lim_{x \sto a}f(x) = L\]
is that $f(x)$ can be made arbitrarily close to $L$ by choosing $x$ sufficiently close to~$a$.
To say what we mean by ``closeness'' we need the notion of the distance between two points. In
this chapter we study ``distance functions'', also known as ``metrics''.

In the preceding chapters we have looked at such topics as limits, continuity, connectedness,
and compactness from the point of view of a single example, the real line $\R$, where the
distance between two points is the absolute value of their difference.  There are other
familiar distance functions (the Euclidean metric in the plane $\R^2$ or in three-space
$\R^3$, for example, where the distance between points is usually taken to be the square root
of the sum of the squares of the differences of their coordinates), and there are many less
familiar ones which are also useful in analysis. Each of these has its own distinctive
properties which merit investigation. But that would be a poor place to start. It is easier to
study first those properties they have in common. We list four conditions we may reasonably
expect any distance function to satisfy.  If $x$ and $y$ are points, then the distance between
$x$ and $y$ should be:
 \begin{enumerate}
  \item[(i)] greater than or equal to zero;
  \item[(ii)] greater than zero if $x$ and $y$ are distinct;
  \item[(iii)] the same as the distance between $y$ and $x$; and
  \item[(iv)] no larger than the sum of the distances produced by taking a detour through a point~$z$.
 \end{enumerate}
We formalize these conditions to define a ``metric'' on a set.




\section{DEFINITIONS}
\begin{defn}\label{def_met} Let $M$ be a nonempty set. A function $d \colon M \times M \sto \R$ is a
 \index{metric}%
\df{metric} (or
 \index{distance!function}%
 \index{dxy@$d(x,y)$ (distance between two points)}%
\df{distance function}) on $M$ if for all $x$, $y$, $z \in M$
 \begin{align*}
     &\text{(1)}\quad d(x,y) = d(y,x) \\
     &\text{(2)}\quad d(x,y) \le d(x,z) + d(z,y) \\
     &\text{(3)}\quad d(x,y) = 0 \text{ if and only if } x = y.
 \end{align*}
If $d$ is a metric on a set $M$, then we say that the pair $(M,d)$ is a
 \index{metric!space}%
 \index{space!metric}%
\df{metric space}.
\end{defn}

It is standard practice---although certainly an abuse of language---to refer to ``the metric
space $M$'' when in fact we mean ``the metric space $(M,d)$''.  We will adopt this convention
when it appears that no confusion will result.  We must keep in mind, however, that there are
situations where it is clearly inappropriate; if, for example, we are considering two
different metrics on the same set $M$, a reference to ``the metric space $M$'' would be
ambiguous.

In our formal definition of ``metric'', what happened to condition (i) above, which requires a
metric to be nonnegative? It is an easy exercise to show that it is implied by the remaining
conditions.

\begin{prop}\label{met_pos} If $d$ is a metric on a set $M$, then $d(x,y) \ge 0$ for all $x$,
$y \in M$.
\end{prop}

\begin{proof} Exercise. (Solution~\ref{sol_met_pos}.)  \ns  \end{proof}

\begin{defn} For each point $a$ in a metric space $(M,d)$ and each number $r > 0$ we define
\index{ball@$B_r(a)$ (open ball of radius $r$ about~$a$)}%
$B_r(a)$, the
 \index{open!ball}%
 \index{ball!open}%
\df{open ball} about $a$ of radius~$r$, to be the set of all those points whose distance from
$a$ is less than $r$. That is,
  \[ B_r(a) := \{x \in M \colon d(x,a) < r\}\,. \]
\end{defn}



\section{EXAMPLES}
\begin{exam}\label{met_r} The absolute value of the difference of two numbers is a metric on~$\R$.
We will call this the usual metric on $\R$.  Notice that in this metric the open ball about
$a$ of radius $r$ is just the open interval $(a - r,a + r)$. (Proof: $x \in B_r(a)$ if and
only if $d(x,a) <r$ if and only if $\abs{x - a} < r$ if and only if $a - r < x < a + r$.)
\end{exam}

\begin{prob} Define $d(x,y) = \abs{\arctan x - \arctan y}$ for all $x$, $y \in \R$.
 \begin{enumerate}
  \item[(a)] Show that $d$ is a metric on $\R$.
  \item[(b)] Find $d(-1,\sqrt 3)$.
  \item[(c)] Solve the equation $d(x,0) = d(x,\sqrt 3)$.
 \end{enumerate}
\end{prob}

\begin{prob} Let $f(x) = \dfrac1{\mathstrut1+x}$ for all $x \ge 0$. Define a metric $d$ on
$[0,\infty)$ by $d(x,y) = \abs{f(x) - f(y)}$. Find a point $z \ne 1$ in this space whose
distance from 2 is equal to the distance between 1 and~2.
\end{prob}

\begin{exam} Define $d(x,y) = \abs{x^2 - y^2}$ for all $x$, $y \in \R$.  Then $d$ is
\emph{not} a metric on~$\R$.
\end{exam}

\begin{proof} Problem. \ns  \end{proof}

\begin{exam} Let $f(x) = \dfrac x{\mathstrut1+x^2}$ for $x \ge 0$. Define a function $d$
on $[0,\infty) \times [0,\infty)$ by $d(x,y) = \abs{f(x)- f(y)}$.  Then $d$ is \emph{not} a
metric on~$[0,\infty)$.
\end{exam}

\begin{proof} Problem.  \ns  \end{proof}

For our next example we make use of a (special case of) an important fact known as
\emph{Minkowski's inequality}. This we derive from another standard result, \emph{Schwarz's
inequality}.
 \index{Schwarz inequality}%
 \index{inequality!Schwarz}%
\begin{prop}[Schwarz's Inequality]\label{Schwarz} Let $u_1,\dots,u_n, v_1, \dots, v_n \in \R$.
Then
  \[ \biggl(\sum_{k=1}^n u_kv_k \biggr)^2
          \le \biggl(\sum_{k=1}^n{u_k}^2 \biggr)\biggl(\sum_{k=1}^n {v_k}^2\biggr)\,. \]
\end{prop}

\begin{proof}  To simplify notation make some abbreviations: let $a = \sum\limits_{k=1}^n {u_k}^2$,
$b = \sum\limits_{k=1}^n {v_k}^2$, and $c = \sum\limits_{k=1}^n u_kv_k$. Then
 \begin{align*}
    0 &\le \sum_{k=1}^n\left(\sqrt b \,\,u_k -
                           \frac c{\sqrt b}\,v_k \right)^2 \\
      &= ab - 2c^2 + c^2 \\
      &= ab - c^2 \,.
 \end{align*}
\end{proof}

 \index{Minkowski inequality}%
 \index{inequality!Minkowski}%
\begin{prop}[Minkowski's Inequality]\label{Minkow} Let $u_1, \dots, u_n, v_1, \dots, v_n \in \R$. Then
  \[ \biggl(\sum_{k=1}^n \bigl(u_k + v_k\bigr)^2\biggr)^{\frac12} \le
          \biggl(\sum_{k=1}^n {u_k}^2\biggr)^{\frac12}
                   + \biggl(\sum_{k=1}^n {v_k}^2\biggr)^{\frac12}\,. \]
\end{prop}

\begin{proof}  Let $a$, $b$, and $c$ be as in \ref{Schwarz}. Then
 \begin{align*}
    \sum_{k=1}^n (u_k + v_k)^2 &= a + 2c + b \\
                               &\le a + 2\abs{c} + b \\
                               &\le a + 2\sqrt{ab} + b \qquad \text{(by \ref{Schwarz})} \\
                               &= \bigl(\sqrt a + \sqrt b \bigr)^2 \,.
 \end{align*}
\end{proof}


\begin{exam}\label{Eucl_met} For points $x = (x_1,\dots,x_n)$ and $y = (y_1,\dots,y_n)$ in $\R^n$ let
  \[ d(x,y) := \biggl(\sum_{k=1}^n(x_k - y_k )^2\biggr)^{\frac12}\,. \]
Then $d$ is the
 \index{usual!metric on $\R^n$}%
 \index{metric!usual}%
\df{usual} (or
 \index{Euclidean metric}%
 \index{metric!Euclidean}%
\df{Euclidean}) \df{metric} on $\R^n$. The only nontrivial part of the proof that $d$ is a
metric is the verification of the \emph{triangle inequality} (that is, condition (2) of the
definition):
   \[d(x,y) \le d(x,z) + d(z,y)\,.\]
To accomplish this let $x = (x_1,\dots,x_n)$, $y = (y_1,\dots,y_n)$, and $z = (z_1,\dots,z_n)$
be points in $\R^n$. Apply \emph{Minkowski's inequality}~(\ref{Minkow}) with $u_k = x_k - z_k$
and $v_k = z_k - y_k$ for $1 \le k \le n$ to obtain
 \begin{align*}
    d(x,y) &= \biggl(\sum_{k=1}^n (x_k - y_k)^2\biggr)^{\frac12} \\
           &= \biggl(\sum_{k=1}^n \bigl((x_k-z_k)  +
                     (z_k-y_k)\bigr)^2\biggr)^{\frac12} \\
           &\le \biggl(\sum_{k=1}^n (x_k - z_k)^2\biggr)^{\frac12}
                 + \biggl(\sum_{k=1}^n (z_k - y_k)^2\biggr)^{\frac12} \\
          &= d(x,z) + d(z,y) \,.
 \end{align*}
\end{exam}

\begin{prob} Let $d$ be the usual metric on $\R^2$.
 \begin{enumerate}
  \item[(a)] Find $d(x,y)$ when $x = (3,-2)$ and $y = (-3,1)$.
  \item[(b)] Let $x = (5,-1)$ and $y = (-3,-5)$. Find a point $z$ in $\R^2$ such that $d(x,y)
= d(y,z) = d(x,z)$.
  \item[(c)] Sketch $B_r(a)$ when $a = (0,0)$ and $r = 1$.
 \end{enumerate}
\end{prob}

The Euclidean metric is by no means the only metric on $\R^n$ which is useful.  Two more
examples follow (\ref{taxicab} and~\ref{unif_rn}).

\begin{exam}\label{taxicab} For points $x = (x_1,\dots,x_n)$ and $y = (y_1,\dots,y_n)$ in $\R^n$ let
 \index{done@$d_1$ (taxicab metric, product metric)}%
   \[ d_1(x,y) := \sum_{k=1}^n \abs{x_k - y_k}\,. \]
It is easy to see that $d_1$ is a metric on $\R^n$. When $n = 2$ this is frequently called the
 \index{taxicab metric}%
 \index{metric!taxicab}%
\df{taxicab metric}. (Why?)
\end{exam}

\begin{prob} Let $d_1$ be the taxicab metric on $\R^2$ (see~\ref{taxicab}).
 \begin{enumerate}
  \item[(a)] Find $d_1(x,y)$ where $x=(3,-2)$ and $y=(-3,1)$.
  \item[(b)] Let $x=(5,-1)$ and $y=(-3,-5)$. Find a point $z$ in $\R^2$  such that $d_1(x,y)
= d_1(y,z) = d_1(x,z)$.
  \item[(c)] Sketch $B_r(a)$ for the metric $d_1$ when $a=(0,0)$ and $r = 1$.
 \end{enumerate}
\end{prob}

\begin{exam}\label{unif_rn} For $x = (x_1,\dots,x_n)$ and $y = (y_1,\dots,y_n)$ in $\R^n$ let
 \index{du@$d_u$ (uniform metric)}%
   \[ d_u(x,y) := \max\{\abs{x_k - y_k} \colon 1 \le k \le n\}\,. \]
Then $d_u$ is a metric on $\R^n$. The triangle inequality is verified as follows:
 \begin{align*}
   \abs{x_k - y_k} &\le \abs{x_k - z_k} + \abs{z_k - y_k} \\
                   &\le \max\{\abs{x_i - z_i}: 1 \le i \le n\} +
                                   \max\{\abs{z_i - y_i}: 1 \le i \le n\} \\
                   &= d_u(x,z) + d_u(z,y)
 \end{align*}
whenever $1 \le k \le n$. Thus
 \begin{align*}
     d_u(x,y) &= \max\{\abs{x_k - y_k} \colon 1 \le k \le n\} \\
              &\le d_u(x,z) + d_u(z,y)\,.
 \end{align*}
The metric $d_u$ is called the
 \index{uniform!metric!on $\R^n$}%
 \index{metric!uniform!on $\R^n$}%
\df{uniform metric}. The reason for this name will become clear later.
\end{exam}

Notice that on the real line the three immediately preceding metrics agree; the distance
between points is just the absolute value of their difference.  That is, when $n = 1$ the
metrics given in \ref{Eucl_met}, \ref{taxicab}, and \ref{unif_rn} reduce to the one given
in~\ref{met_r}.

\begin{prob} This problem concerns the metric $d_u$ (defined in example~\ref{unif_rn}) on~$\R^2$.
 \begin{enumerate}
  \item[(a)] Find $d_u(x,y)$ when $x = (3,-2)$ and $y=(-3,1)$.
  \item[(b)] Let $x=(5,-1)$ and $y=(-3,-5)$. Find a point $z$ in $\R^2$ such that $d_u(x,y)
= d_u(y,z) = d_u(x,z)$.
  \item[(c)] Sketch $B_r(a)$ for the metric $d_u$ when $a=(0,0)$ and $r = 1$.
 \end{enumerate}
\end{prob}

\begin{exam}\label{discrete} Let $M$ be any nonempty set. For $x$, $y \in M$ define
   \[ d(x,y) = \begin{cases} 1 &\text{ if } x\ne y \\
                             0 &\text{ if } x=y\,.
               \end{cases} \]
It is easy to see that $d$ is a metric; this is the
 \index{discrete metric}%
 \index{metric!discrete}%
\df{discrete metric} on~$M$. Although the discrete metric is rather trivial it proves quite
useful in constructing counterexamples.
\end{exam}

\begin{exam}\label{Gam} Let $d$ be the usual metric on $\R^2$ and 0 be the origin. Define a
function $\rho$ on $\R^2$ as follows:
 \index{<@$\rho$ (Greek airline metric)}%
   \[\rho (x,y) := \begin{cases}
                      d(x,y), &\text{if $x$ and $y$ are collinear with 0, and} \\
             d(x,0) + d(0,y), &\text{otherwise.}
                   \end{cases} \]
The function $\rho$ is a metric on~$\R^2$.  This metric is sometimes called the
 \index{Greek airline metric}%
 \index{metric!Greek airline}%
\df{Greek airline metric}.
\end{exam}

\begin{proof} Problem. \ns  \end{proof}

\begin{prob} Let $\rho$ be the Greek airline metric on $\R^2$.
 \begin{enumerate}
  \item[(a)] Let $x = (-1,2)$, $y = (-3,6)$, and $z = (-3,4)$. Find $\rho(x,y)$ and $\rho(x,z)$.
Which point, $y$ or $z$, is closer to $x$ with respect to~$\rho$?
  \item[(b)] Let $r = 1$. Sketch $B_r(a)$ for the metric $\rho$ when $a = (0,0)$, $a = (\frac14,0)$,
$a = (\frac12,0)$, $a = (\frac34,0)$, $a = (1,0)$, and $a = (3,0)$.
 \end{enumerate}
\end{prob}

\begin{prop}\label{ms_ineq} Let $(M,d)$ be a metric space and $x$, $y$, $z \in M$. Then
   \[ \abs{d(x,z) - d(y,z)} \le d(x,y)\,. \]
\end{prop}

\begin{proof} Problem.  \ns  \end{proof}

\begin{prop} If $a$ and $b$ are distinct points in a metric space, then there exists a number
$r > 0$ such that $B_r(a)$ and $B_r(b)$ are disjoint.
\end{prop}

\begin{proof} Problem. \ns  \end{proof}

\begin{prop}\label{ball_int} Let $a$ and $b$ be points in a metric space and $r$, $s > 0$.
If $c$ belongs to $B_r(a) \cap B_s(b)$, then there exists a number $t > 0$ such that $B_t(c)
\subseteq B_r(a) \cap B_s(b)$.
\end{prop}

\begin{proof} Problem. \ns  \end{proof}

\begin{prob} Let $f(x) = \dfrac 1{1+x^2}$ for all $x \ge 0$, and define a metric $d$ on the
interval $[0,\infty)$ by
  \[ d(x,y) = \abs{f(x) - f(y)}\,. \]
 \begin{enumerate}
  \item[(a)] With respect to this metric find the point halfway between $1$ and~$2$.
  \item[(b)] Find the open ball $B_{\frac3{10}}(1)$.
 \end{enumerate}
\end{prob}







\section{STRONGLY EQUIVALENT METRICS}
Ahead of us lie many situations in which it will be possible to replace a computationally
complicated metric on some space by a simpler one without affecting the fundamental
properties of the space. As it turns out, a sufficient condition for this process to be
legitimate is that the two metrics be ``strongly equivalent''.  For the moment we content
ourselves with the \emph{definition} of this term and an example; applications will be
discussed later when we introduce the weaker notion of ``equivalent metrics''
(see~\ref{def_equiv_metrics}).

\begin{defn} Two metrics $d_1$ and $d_2$ on a set $M$ are
 \index{equivalent!strongly}%
 \index{metrics!equivalent!strongly}%
 \index{strongly equivalent metrics}%
\df{strongly equivalent} if there exist numbers $\alpha$, $\beta > 0$ such that
 \begin{align*}
     d_1(x,y) &\le \alpha\, d_2(x,y) \text{\quad and} \\
     d_2(x,y) &\le \beta\, d_1(x,y)
 \end{align*}
for all $x$ and $y$ in $M$.
\end{defn}

\begin{prop}\label{3equiv} On $\R^2$ the three metrics $d$, $d_1$, and $d_u$, defined in
examples~\ref{Eucl_met}, \ref{taxicab}, and~\ref{unif_rn}, are strongly equivalent.
\end{prop}

\begin{proof} Exercise. \emph{Hint.}  First prove that if $a$, $b \ge 0$, then
   \[ \max\{a,b\} \le a+b \le \sqrt 2 \,\sqrt{a^2+b^2} \le 2 \max \{a,b\}\,.\]
(Solution~\ref{sol_3equiv}.)   \ns
\end{proof}

\begin{prob} Let $d$ and $\rho$ be strongly equivalent metrics on a set~$M$. Then every
open ball in the space $(M,d)$ contains an open ball of the space $(M,\rho)$ (and
\emph{vice versa}).
\end{prob}

\begin{prob} Let $a$, $b$, $c$, $d \in \R$.  Establish each of the following:
 \begin{enumerate}
  \item[(a)] $(\frac13a + \frac23b)^2 \le \frac13a^2 + \frac23b^2\,.$

\vskip 2 pt

  \item[(b)] $(\frac12a + \frac13b + \frac16c)^2
                    \le \frac12a^2 + \frac13b^2 + \frac16c^2\,.$

\vskip 2 pt

  \item[(c)] $(\frac5{12}a + \frac13b + \frac16c +\frac1{12}d)^2
                \le \frac5{12}a^2 + \frac13b^2 + \frac16c^2 + \frac1{12}d^2\,.$
 \end{enumerate}

\vskip 2 pt \noindent \emph{Hint.} If (a), (b), and (c) are all special cases of some general
result, it may be easier to give one proof (of the general theorem) rather than three proofs
(of the special cases). In each case what can you say about the numbers multiplying $a$, $b$,
$c$, and $d$? Notice that if $x > 0$, then $xy = \sqrt x\,(\sqrt x\,y)$. Use Schwarz's
inequality~\ref{Schwarz}.
\end{prob}

\begin{prop}\label{bdd_met} Let $(M,d)$ be a metric space. The function $\rho$ defined on
$M \times M$ by
  \[ \rho(x,y) = \frac{d(x,y)}{1 + d(x,y)} \]
is a metric on~$M$.
\end{prop}

\begin{proof} Problem. \emph{Hint.}  Show first that  $\dfrac u{1+u} \le \dfrac v{1+v}$
whenever $0 \le u \le v$.  \ns
\end{proof}

\begin{prob}\label{d_ne_rho} In problem \ref{bdd_met} take $M$ to be the real line $\R$ and
$d$ to be the usual metric on $\R$ (see~\ref{met_r}).
 \begin{enumerate}
  \item[(a)] Find the open ball $B_{\frac35}(1)$ in the metric space $(\R, \rho)$.
  \item[(b)] Show that the metrics $d$ and $\rho$ are \emph{not} strongly equivalent on~$\R$.
 \end{enumerate}
\end{prob}



\endinput
