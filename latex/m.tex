\chapter{FUNCTIONS WHICH HAVE INVERSES}\label{inverses}

\section{INJECTIONS, SURJECTIONS, AND BIJECTIONS}
A function $f$ is
 \index{injective}%
\df{injective} (or
 \index{one-to-one}%
\df{one-to-one}) if $x = y$ whenever $f(x) = f(y)$.  That is, $f$ is injective if no two
distinct elements in its domain have the same image.  For a real valued function of a real
variable this condition may be interpreted graphically: A function is one-to-one if and only
if each horizontal line intersects the graph of the function at most once. An injective map is
called an
 \index{injection}%
\df{injection}.

\begin{exam}  The sine function is not injective (since, for example, $\sin 0 = \sin \pi$).
\end{exam}

\begin{exam}\label{exam_inj_fcn} Let $f(x) = \dfrac{x+2}{3x-5}$.  The function $f$ is injective.
\end{exam}

\begin{proof} Exercise.  (See~\ref{sol_exam_inj_fcn}.)  \ns  \end{proof}

\begin{exer}\label{exer_inj_q}  Find an injective mapping from $\{x \in \Q\colon  x > 0\}$
into~$\N$. (Solution~\ref{sol_exer_inj_q}.)
\end{exer}

\begin{prob} Let $f(x) = \dfrac{2x-5}{3x+4}$. Show that $f$ is injective.
\end{prob}

\begin{prob}  Let $f(x) = 2x^2 - x - 15$. Show that $f$ is \emph{not} injective.
\end{prob}

\begin{prob} Let $f(x) = x^3 - 2x^2 + x - 3$. Is $f$ injective?
\end{prob}

\begin{prob} Let $f(x) = x^3 - x^2 + x - 3$. Is $f$ injective?
\end{prob}

\begin{defn}  A function is
 \index{surjective}%
\df{surjective} (or
 \index{onto}%
\df{onto}) if its range is equal to its codomain.
\end{defn}

\begin{exam}  Define $f\colon \R \sto \R$ by $f(x) = x^2$ and $g\colon \R \sto [0, \infty)$
by $g(x) = x^2$.  Then the function $g$ is surjective while $f$ is not (even though the two
functions have the same graph!).
\end{exam}

\begin{exer}\label{exer_surj}  Find a surjection (that is, a surjective map) from $[0,1]$
onto $[0, \infty)$.
(Solution~\ref{sol_exer_surj}.)
\end{exer}

\begin{defn}  A function is
 \index{bijective}%
\df{bijective} if it is both injective and surjective. A bijective
map is called a
 \index{bijection}%
\df{bijection} or a
 \index{one-to-one correspondence}%
\df{one-to-one correspondence}.
\end{defn}

\begin{exer}\label{exer_bij1}  Give an explicit formula for a bijection between $\Z$ and~$\N$.
(Solution~\ref{sol_exer_bij1}.)
\end{exer}

\begin{exer}\label{exer_bij2} Give an explicit formula for a bijection between $\R$ and the open
interval~$(0,1)$.
(Solution~\ref{sol_exer_bij2}.)
\end{exer}

\begin{exer}\label{exer_bij3}  Give a formula for a bijection between the interval $[0,1)$ and the
unit circle $x^2 + y^2 = 1$ in the plane.
(Solution~\ref{sol_exer_bij3}.)
\end{exer}

\begin{exer}\label{exer_bij4} Let $f\colon [1,2] \sto [0,3]$ be defined by $f(x) = 1/x$. Find an
extension $g$ of $f$ to the interval $[0,3]$ such that $g\colon [0, 3] \sto [0,3]$ is a
bijection. (Solution~\ref{sol_exer_bij4}.)
\end{exer}

\begin{exer}\label{exer_bij5}  Let $f\colon R \sto [-1,1]$ be defined by $f(x) = \sin x$. Find a set
$A \subseteq R$ such that the restriction of $f$ to $A$ is a bijection from $A$ onto $[-1,1]$.
(Solution~\ref{sol_exer_bij5}.)
\end{exer}

\begin{prob}  Let $f\colon S \sto T$ and $g\colon T \sto U$.  Prove the following.
 \begin{enumerate}
  \item[(a)] If $f$ and $g$ are injective, so is $g \circ f$.
  \item[(b)] If $f$ and $g$ are surjective, so is $g \circ f$.
  \item[(c)] If $f$ and $g$ are bijective, so is $g \circ f$.
 \end{enumerate}
\end{prob}

\begin{prob}  Find a bijection between the open intervals $(0,1)$ and $(-8,5)$. Prove that the
function you use really \emph{is} a bijection between the two intervals.
\end{prob}

\begin{prob}  Find a bijection between the open intervals $(0,1)$ and $(3,\infty)$. (Proof not
required.)
\end{prob}

\begin{prob}   Find a bijection between the interval $(0,1)$ and the parabola $y = x^2$ in the
plane. (Proof not required.)
\end{prob}

\begin{prob}  Let $f\colon [1,2] \sto [0,11]$ be defined by $f(x) = 3x^2 - 1$. Find an extension
$g$ of $f$ to $[0,3]$ which is a bijection from $[0,3]$ onto~$[0,11]$. (Proof not required.)
\end{prob}

It is important for us to know how the image $f^{\sto}$ and the inverse image $f^\gets$ of a
function $f$ behave with respect to unions, intersections, and complements of sets. The basic
facts are given in the next 10 propositions. Although these results are quite elementary, we
make extensive use of them in studying continuity.

\begin{prop}\label{prop_f_finv} Let $f\colon S \sto T$ and $B \subseteq T$.
 \begin{enumerate}
   \item[(a)] $f^{\sto}(f^{\gets}(B)) \subseteq B$.
   \item[(b)] Equality need not hold in (a).
   \item[(c)] Equality does hold in (a) if $f$ is surjective.
 \end{enumerate}
\end{prop}

\begin{proof} Exercise.  (Solution~\ref{sol_prop_f_finv}.) \ns   \end{proof}

\begin{prop}\label{prop_finv_f}  Let $f\colon S \sto T$ and $A \subseteq
S$.
\begin{enumerate}
\item[(a)] $A \subseteq f^{\gets}(f^{\sto}(A))$.
\item[(b)] Equality need not hold in (a).
\item[(c)] Equality does hold in (a) if $f$ is injective.
\end{enumerate}
\end{prop}

\begin{proof} Problem.  \ns  \end{proof}

\begin{prob} Prove the converse of proposition~\ref{prop_finv_f}.  That is, show that if
$f \colon S \sto T$ and $f^{\gets}(f^{\sto}(A)) = A$ for all $A \subseteq S$, then $f$ is
injective. \emph{Hint.}  Suppose $f(x) = f(y)$. Let $A = \{x\}$. Show that $y \in
f^\gets(f^{\sto}(A))$. \ns
\end{prob}

\begin{prop}\label{prop_f_union}  Let $f\colon S \sto T$ and $A,B \subseteq S$. Then
   \[f^{\sto}(A \cup B) = f^{\sto}(A) \cup f^{\sto}(B).\]
\end{prop}

\begin{proof} Exercise. (Solution~\ref{sol_prop_f_union}.) \ns  \end{proof}

\begin{prop}\label{prop_finv_union}  Let $f\colon S \sto T$ and $C$, $D \subseteq T$. Then
   \[ f^{\gets}(C \cup D) = f^{\gets}(C) \cup f^{\gets}(D)\,. \]
\end{prop}

\begin{proof} Problem.  \ns  \end{proof}

\begin{prop}\label{prop_finv_intrs}  Let $f\colon S \sto T$ and $C $, $D \subseteq T$. Then
   \[ f^\gets(C \cap D) = f^\gets(C) \cap f^\gets(D)\,. \]
\end{prop}

\begin{proof} Exercise. (Solution~\ref{sol_prop_finv_intrs}.) \ns  \end{proof}

\begin{prop}  Let $f\colon S \sto T$ and $A$, $B \subseteq S$.
 \begin{enumerate}
   \item[(a)] $f^{\sto}(A \cap B) \subseteq f^{\sto}(A) \cap f^{\sto}(B)$.
   \item[(b)] Equality need not hold in (a).
   \item[(c)] Equality does hold in (a) if $f$ is injective.
 \end{enumerate}
\end{prop}

\begin{proof} Problem.  \ns  \end{proof}

\begin{prop}\label{prop_finv_comp}  Let $f\colon S \sto T$ and $D \subseteq T$. Then
   \[ f^\gets(D^c) = \bigl(f^\gets(D)\bigr)^c\,. \]
\end{prop}

\begin{proof} Problem.  \ns  \end{proof}

\begin{prop}\label{prop_f_comp}  Let $f\colon S \sto T$ and $A \subseteq S$.
 \begin{enumerate}
   \item[(a)] If $f$ is injective, then $f^{\sto}(A^c) \subseteq \bigl(f^{\sto}(A)\bigr)^c$.
   \item[(b)] If $f$ is surjective, then $f^{\sto}(A^c) \supseteq \bigl(f^{\sto}(A)\bigr)^c$.
   \item[(c)] If $f$ is bijective, then $f^{\sto}(A^c) = \bigl(f^{\sto}(A)\bigr)^c$.
 \end{enumerate}
\end{prop}

\begin{proof} Problem.  \emph{Hints.}  For part (a), let $y \in f^{\sto}(A^c)$. To prove that
$y \in \bigl(f^{\sto}(A)\bigr)^c$, assume to the contrary that $y \in f^{\sto}(A)$ and derive
a contradiction. For part (b), let $y \in \bigl(f^{\sto}(A)\bigr)^c$. Since $f$ is surjective,
there exists $x \in S$ such that $y = f(x)$. Can $x$ belong to~$A$?.  \ns
\end{proof}

\begin{prop}\label{prop_f_fam}  Let $f\colon S \sto T$ and $\sfml A \subseteq \sfml P(S)$.
 \begin{enumerate}
   \item[(a)] $f^{\sto}(\bigcap\sfml A) \subseteq \bigcap\{f^{\sto}(A) \colon A \in \sfml A\}$.
   \item[(b)] If $f$ is injective, equality holds in (a).
   \item[(c)] $f^{\sto}(\bigcup\sfml A) = \bigcup\{f^{\sto}(A) \colon A \in \sfml A\}$.
 \end{enumerate}
\end{prop}

\begin{proof} Exercise. (Solution~\ref{sol_prop_f_fam}.)  \ns  \end{proof}

\begin{prop}\label{prop_finv_fam}  Let $f\colon S \sto T$ and $\sfml B \subseteq \sfml P(T)$.
 \begin{enumerate}
   \item[(a)] $f^\gets(\bigcap \sfml B) = \bigcap\{f^\gets(B)\colon B \in \sfml B\}$.
   \item[(b)] $f^\gets(\bigcup \sfml B) = \bigcup\{f^\gets(B)\colon B \in \sfml B\}$.
 \end{enumerate}
\end{prop}

\begin{proof} Problem.  \ns  \end{proof}








\section{INVERSE FUNCTIONS}\label{sec_inv_fcns}   Let $f\colon S \sto T$
and $g\colon T \sto S$.  If $g\circ f = I_S$, then $g$ is a
 \index{left inverse}%
 \index{inverse!left}%
\df{left inverse} of $f$ and, equivalently, $f$ is a
 \index{right inverse}%
 \index{inverse!right}%
\df{right inverse} of $g$. We say that  $f$ is
 \index{invertible}%
\df{invertible} if there exists a function from $T$ into  $S$ which is both a left and a right
inverse for~$f$.  Such a function is denoted by $f^{-1}$ and is called the
 \index{<@$f^{-1}$ (inverse of a function $f$)}%
\df{inverse} of~$f$.  (Notice that the last ``the'' in the preceding sentence requires
justification.  See proposition~\ref{prop_finv_unique} below.) A function is
 \index{invertible}%
\df{invertible} if it has an inverse. According to the definition just given, the inverse
$f^{-1}$ of a function $f$ must satisfy
  \[ f \circ f^{-1} = I_T \qquad\text{and}\qquad f^{-1}\circ f = I_S\,. \]
A simple, but important, consequence of this is that for an invertible function, $y = f(x)$ if
and only if $x = f^{-1}(y)$. [Proof: if $y = f(x)$, then $f^{-1}(y) = f^{-1}(f(x)) = I_S(x) =
x$. Conversely, if $x = f^{-1}(y)$, then $f(x) = f(f^{-1}(y)) = I_T(y) = y$.]

\begin{prop}\label{prop_finv_unique}  A function can have at most one inverse.
\end{prop}

\begin{proof} Exercise. (Solution~\ref{sol_prop_finv_unique}.)  \ns  \end{proof}

\begin{prop}  If a function has both a left inverse and a right inverse, then the left and right
inverses are equal (and therefore the function is invertible).
\end{prop}

\begin{proof} Problem.  \ns  \end{proof}

\begin{exer}\label{exer_inv_trig}  The \emph{arcsine} function is defined to be the inverse
of what function? (\emph{Hint.}  The answer is not \emph{sine}.) What about \emph{arccosine}?
\emph{arctangent}? (Solution~\ref{sol_exer_inv_trig}.)
\end{exer}

The next two propositions tell us that a necessary and sufficient condition for a function to
have right inverse is that it be surjective and that a necessary and sufficient condition for
a function to have a left inverse is that it be injective. Thus, in particular, a function is
invertible if and only if it is bijective. In other words, the invertible members of $\fml
F(S,T)$ are the bijections.

\begin{prop}\label{prop_rinv_surj}  Let $S \ne \emptyset$. A function $f\colon S \sto T$ has a
right inverse if and only if it is surjective.
\end{prop}

\begin{proof} Exercise. (Solution~\ref{sol_prop_rinv_surj}.)  \ns  \end{proof}

\begin{prop}\label{prop_linv_inj}  Let $S \ne \emptyset$. A function $f\colon S \sto T$ has a left
inverse if and only if it is injective.
\end{prop}

\begin{proof}  Problem.  \ns   \end{proof}

\begin{prob} Prove: if a function $f$ is bijective, then $\bigl(f^{-1}\bigr)^\gets = f^{\sto}$.
\end{prob}

\begin{prob}  Let $ f(x) = \dfrac{ax + b}{cx + d}$ where $a$, $b$, $c$, $d \in \R$ and not both
$c$ and $d$ are zero.
 \begin{enumerate}
   \item[(a)] Under what conditions on the constants $a, b, c, \text{ and } d$ is $f$ injective?
   \item[(b)] Under what conditions on the constants $a, b, c, \text{ and } d$ is $f$ its own inverse?
 \end{enumerate}
\end{prob}


\endinput
