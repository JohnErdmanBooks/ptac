\chapter{MORE ON CONTINUITY AND LIMITS}

\section{CONTINUOUS FUNCTIONS}
As is the case with real valued functions of a real variable, a function $f\colon M_1 \sto
M_2$ between two metric spaces is continuous at a point $a$ in $M_1$ if $f(x)$ can be made
arbitrarily close to $f(a)$ by insisting that $x$ be sufficiently close to~$a$.

\begin{defn} Let $(M_1,d_1)$ and $(M_2,d_2)$ be metric spaces.  A function $f\colon M_1 \sto M_2$ is
 \index{continuous!at a point}%
\df{continuous at} a point $a$ in $M_1$ if every neighborhood of $f(a)$ contains the image under
$f$ of a neighborhood of~$a$. Since every neighborhood of a point contains an open ball about the
point and since every open ball about a point \emph{is} a neighborhood of that point, we may
restate the definition as follows. The function $f$ is \df{continuous at} $a$ if every open ball
about $f(a)$ contains the image under $f$ of an open ball about~$a$; that is, if the following
condition is satisfied: for every $\epsilon > 0$ there exists $\delta > 0$ such that
  \begin{equation}\label{clms_eq1} f^{\sto} (B_\delta(a))
                      \subseteq B_\epsilon(f(a)).
  \end{equation}
There are many equivalent ways of expressing \eqref{clms_eq1}.
Here are three:
 \begin{gather*}
    B_\delta(a) \subseteq f^\gets(B_\epsilon(f(a))) \\
    x \in B_\delta(a) \text{ implies } f(x) \in B_\epsilon(f(a)) \\
    d_1(x,a) < \delta \text{ implies } d_2(f(x),f(a)) < \epsilon
 \end{gather*}
\end{defn}

Notice that if $f$ is a real valued function of a real variable, then the definition above agrees
with the one given at the beginning of chapter~\ref{cont_on_R}.

\begin{defn} A function $f\colon M_1 \sto M_2$ between two metric spaces is
 \index{continuous}%
\df{continuous} if it is continuous at each point of $M_1$.
\end{defn}

In proving propositions concerning continuity, one should not slavishly insist on specifying
the radii of open balls when these particular numbers are of no interest. As an illustration,
the next proposition, concerning the composite of continuous functions, is given two
proofs---one with the radii of open balls specified, and a smoother one in which they are
suppressed.

\begin{prop} Let $f\colon M_1 \sto M_2$ and $g\colon M_2 \sto M_3$ be functions between
metric spaces. If $f$ is continuous at $a$ in $M_1$ and $g$ is continuous at $f(a)$
in~$M_2$, then the composite function $g \circ f$ is continuous at~$a$.
\end{prop}

\begin{proof}[Proof 1] Let $\eta > 0$.  We wish to show that there
exists $\delta > 0$ such that
  \[B_\delta(a) \subseteq (g \circ f)^\gets(B_\eta(g(f(a)))). \]
Since $g$ is continuous at $f(a)$ there exists $\epsilon > 0$ such
that
  \[B_\epsilon(f(a)) \subseteq g^\gets(B_\eta(g(f(a)))). \]
Since $f$ is continuous at $a$ there exists $\delta > 0$ such that
  \[B_\delta(a) \subseteq  f^\gets(B_\epsilon(f(a))).\]
Thus we have
  \begin{align*} (g \circ f)^\gets(B_\eta(g(f(a))))
                  &= f^\gets(g^\gets(B_\eta(g(f(a))))) \\
                  &\supseteq f^\gets(B_\epsilon(f(a))) \\
                  &\supseteq B_\delta(a).  \qedhere
  \end{align*}
\end{proof}

\begin{proof}[Proof 2] Let $B_3$ be an arbitrary open ball about
$g(f(a))$. We wish to show that the inverse image of $B_3$ under $g
\circ f$ contains an open ball about~$a$.  Since $g$ is continuous
at $f(a)$, the set $g^\gets(B_3)$ contains an open ball $B_2$ about
$f(a)$. And since $f$ is continuous at $a$, the set $f^\gets(B_2)$
contains an open ball $B_1$ about $a$. Thus we have
  \begin{align*} (g \circ f)^\gets(B_3)
              &= f^\gets(g^\gets(B_3)) \\
              &\supseteq f^\gets(B_2) \\
              &\supseteq B_1.   \qedhere
  \end{align*}
\end{proof}

The two preceding proofs are essentially the same. The only difference is that the first proof
suffers from a severe case of clutter. It certainly is not more rigorous; it is just harder to
read. It is good practice to relieve proofs (and their readers) of extraneous detail. The following
corollary is an obvious consequence of the proposition we have just proved.

\begin{cor}\label{cmp_cnt} The composite of two continuous functions is continuous.
\end{cor}

Next we prove a result emphasizing that continuity is a topological notion; that is, it can be
expressed in terms of open sets. A necessary and sufficient condition for a function to be
continuous is that the inverse image of every open set be open.

\begin{prop}\label{top_ch_cont} A function $f\colon M_1 \sto M_2$ between metric spaces is
continuous if and only if $f^\gets(U)$ is an open subset of $M_1$ whenever $U$ is open in~$M_2$.
\end{prop}

\begin{proof} Exercise.  (Solution~\ref{sol_top_ch_cont}.)  \ns \end{proof}

\begin{exam}\label{exam1_cont} As an application of the preceding proposition we show that the
function
  \[ f\colon \R^2 \sto \R\colon  (x,y) \mapsto 2x - 5y \]
is continuous. One approach to this problem is to find, given a point $(a,b)$ in $\R^2$
and $\epsilon > 0$, a number $\delta > 0$ sufficiently small that $\sqrt{(x - a)^2 + (y -
b)^2} < \delta$ implies $\abs{(2x - 5y) - (2a - 5b)} < \epsilon$.  This is not
excessively difficult, but it is made somewhat awkward by the appearance of squares and a
square root in the definition of the usual metric on~$\R^2$.  A simpler approach is
possible.  We wish to prove continuity of $f$ with respect to the usual metric $d$ on
$\R^2$ (defined in~\ref{Eucl_met}).

We know that the metric $d_1$ (defined in \ref{taxicab}) on $\R^2$ is (strongly)
equivalent to $d$ (\ref{3equiv}) and that equivalent metrics produce identical
topologies. Thus $(\R^2, d_1)$ and $(\R^2, d)$ have the same topologies. Since the
continuity of a function is a topological concept (this was the point
of~\ref{top_ch_cont}), we know that $f$ will be continuous with respect to $d$ if and
only if it is continuous with respect to~$d_1$. Since the metric $d_1$ is algebraically
simpler, we prove continuity with respect to $d_1$.  To this end, let $(a,b) \in \R^2$
and $\epsilon > 0$.  Choose $\delta = \epsilon/5$.  If $d_1((x,y),(a,b)) = \abs{x - a} +
\abs{y - b} < \delta$, then
 \begin{align*}  \abs{f(x,y) - f(a,b)}
                       &= \abs{2(x - a) - 5(y - b)} \\
                       &\le 5(\abs{x - a} + \abs{y - b}) \\
                       &< 5\delta \\
                       &= \epsilon.
 \end{align*}
\end{exam}

\begin{exam} The function
  \[ f \colon \R^2 \sto \R \colon  (x,y) \mapsto 5x - 7y \]
is continuous.
\end{exam}

\begin{proof} Problem.  \ns  \end{proof}


The principle used in example \ref{exam1_cont} works generally: replacing a metric on the
domain of a function by an equivalent metric does not affect the continuity of the function.
The same assertion is true for the codomain of a function as well.  We state this formally.

\begin{prop}\label{cont_eqmtr} Let $f\colon M_1 \sto M_2$ be a continuous function between two
metric spaces $(M_1,d_1)$ and $(M_2,d_2)$.  If $\rho_1$ is a metric on $M_1$ equivalent to $d_1$
and $\rho_2$ is equivalent to $d_2$, then $f$ considered as a function from the space
$(M_1,\rho_1)$ to the space $(M_2,\rho_2)$ is still continuous.
\end{prop}


\begin{proof} This is an immediate consequence of propositions \ref{mtr_vs_top} and~\ref{top_ch_cont}.
\end{proof}

\begin{exam}\label{cont_mult} Multiplication is a continuous function on~$\R$.  That is,
if we define
  \[ M \colon \R^2 \sto \R \colon (x,y) \mapsto xy\,, \]
then the function $M$ is continuous.
\end{exam}

\begin{proof} Exercise.  \emph{Hint.} Use \ref{cont_eqmtr}.   (Solution~\ref{sol_cont_mult}.)  \ns
\end{proof}

\begin{exam}\label{cont_add} Addition is a continuous function on $\R$.  That is, if we define
  \[ A \colon \R^2 \sto \R \colon (x,y) \mapsto x+y\,, \]
then the function $A$ is continuous.
\end{exam}

\begin{proof} Problem.    \ns  \end{proof}

\begin{prob}  Let $d$ be the usual metric on $\R^2$, let $\rho$ be the metric on $\R^2$ defined in
example \ref{Gam}, and let $f\colon (\R^2, d) \sto (\R^2,\rho)$ be the identity function on $\R^2$.
Show that $f$ is \emph{not} continuous.
\end{prob}

\begin{prob} Let $\R_d$ be the set of real numbers with the discrete metric, and let $M$ be an
arbitrary metric space. Describe the family of all continuous functions $f\colon \R_d \sto M$.
\end{prob}

\begin{prop}\label{prop_cont_closed} Let $f\colon M_1 \sto M_2$ where $M_1$ and $M_2$ are metric
spaces.  Then $f$ is continuous if and only if $f^\gets(C)$ is a closed subset of $M_1$ whenever
$C$ is a closed subset of~$M_2$.
\end{prop}

\begin{proof} Problem.  \ns  \end{proof}

\begin{prop} Let $f\colon M_1 \sto M_2$ be a function between metric spaces.  Then $f$ is
continuous if and only if
  \[ f^\gets(\intr B) \subseteq \intr{\bigl(f^\gets(B)\bigr)} \]
for all $B \subseteq M_2$.
\end{prop}

\begin{proof} Problem.  \ns  \end{proof}

\begin{prop} Let $f \colon M_1 \sto M_2$ be a function between metric spaces.  Then $f$ is
continuous if and only if
  \[ f^{\sto}\bigl(\,\clo A\,\bigr) \subseteq \clo{f^{\sto}(A)} \]
for every $A \subseteq M_1$.
\end{prop}

\begin{proof} Problem.  \ns  \end{proof}

\begin{prop} Let $f\colon M_1 \sto M_2$ be a function between metric spaces.  Then $f$ is
continuous if and only if
  \[ \clo{f^\gets(B)}  \subseteq f^\gets\bigl(\,\clo B\,\bigr) \]
for every $B \subseteq M_2$.
\end{prop}

\begin{proof} Problem.  \ns  \end{proof}

\begin{prop} Let $f$ be a real valued function on a metric space $M$ and $a \in M$.  If $f$ is
continuous at $a$ and $f(a) > 0$, then there exists a neighborhood $B$ of $a$ such that $f(x) >
\frac12 f(a)$ for all $x \in B$.
\end{prop}

\begin{proof} Problem.   \ns  \end{proof}

\begin{prop} Let $N$ be a metric space and $M$ be a subspace of~$N$.
 \begin{enumerate}
  \item[(a)] The inclusion map $\iota\colon M \sto N\colon x \mapsto x$ is continuous.
  \item[(b)] Restrictions of continuous functions are continuous. That is, if $f\colon M_1 \sto M_2$
is a continuous mapping between metric spaces and $A \subseteq M_1$, then $f|_A$ is continuous.
 \end{enumerate}
\end{prop}

\begin{proof} Problem.   \ns  \end{proof}

\begin{prob}  Show that alteration of the codomain of a continuous function does not affect its
continuity.  Precisely: If $f\colon M_0 \sto M$ and $g\colon  M_0 \sto N$ are functions between
metric spaces such that $f(x) = g(x)$ for all $x \in M_0$ and if their common image is a metric
subspace of both $M$ and $N$, then $f$ is continuous if and only if $g$ is.
\end{prob}

In the next proposition we show that if two continuous functions agree on a dense subset of a
metric space, they agree everywhere on that space.

\begin{prop}\label{agree_dense} If $f$, $g \colon M \sto N$ are continuous functions between metric
spaces, if $D$ is a dense subset of $M$, and if $f|_D = g|_D$, then $f = g$.
\end{prop}

\begin{proof} Problem.  \emph{Hint.} Suppose that there is a point $a$ where $f$ and $g$ differ.
Consider the inverse images under $f$ and $g$, respectively, of disjoint neighborhoods of $f(a)$
and~$g(a)$.  Use proposition~\ref{cnd_dns}. \ns
\end{proof}

\begin{prob} Suppose that $M$ is a metric space and that $f\colon M \sto M$ is continuous but
is \emph{not} the identity map.  Show that there exists a proper closed set $C \subseteq M$ such
that
  \[ C \cup f^\gets(C) = M\,. \]
\emph{Hint.} Choose $x$ so that $x \ne f(x)$.  Look at the complement of $U \cap f^\gets(V)$ where
$U$ and $V$ are disjoint neighborhoods of $x$ and $f(x)$, respectively.
\end{prob}

There are two ways in which metric spaces may be regarded as ``essentially the same'': They may be
isometric (having essentially the same distance function); or they may be topologically equivalent
(having essentially the same open sets).

\begin{defn} Let$(M,d)$ and $(N,\rho)$ be metric spaces.  A bijection $f \colon M \sto N$ is an
 \index{isometry}%
\df{isometry} if
  \[ \rho(f(x),f(y)) = d(x,y) \]
for all $x$, $y \in M$. If an isometry exists between two metric spaces, the spaces are said
to be
 \index{isometric spaces}%
\df{isometric}.
\end{defn}

\begin{defn} A bijection $g\colon M\sto N$ between metric spaces is a
 \index{homeomorphism}%
\df{homeomorphism} if both $g$ and $g^{-1}$ are continuous.  Notice that if $g$ is a
homeomorphism, then $g^{\sto}$ establishes a one-to-one correspondence between the family
of open subsets of $M$ and the family of open subsets of~$N$.  For this reason two metric
spaces are said to be
 \index{topological!equivalence}%
 \index{equivalent!topologically}%
\df{(topologically) equivalent} or
 \index{homeomorphic}%
\df{homeomorphic} if there exists a homeomorphism between them.  Since the open sets of a
space are determined by its metric, it is clear that every isometry is automatically a
homeomorphism. The converse, however, is not correct (see example \ref{exam2_cont}
below).
\end{defn}


\begin{prob} Give an example of a bijection between metric spaces which is continuous but is not
a homeomorphism.
\end{prob}

\begin{exam}\label{exam2_cont} The open interval $(0,1)$ and the real line $\R$ (with their usual
metrics) are homeomorphic but not isometric.
\end{exam}

\begin{proof} Problem.  \ns  \end{proof}

We have seen (in chapter \ref{seqs_msps}) that  certain properties of sets in metric spaces can be
characterized by means of sequences.  Continuity of functions between metric spaces also has a
simple and useful sequential characterization.

\begin{prop}\label{seq_ch_cont} Let $f \colon M_1 \sto M_2$ be a function between metric spaces
and $a$ be a point in~$M_1$.  The function $f$ is continuous at a if and only if $f(x_n) \sto f(a)$
whenever $x_n \sto a$.
\end{prop}

\begin{proof} Exercise. (Solution~\ref{sol_seq_ch_cont}.)
  \ns \end{proof}

\begin{prob} Give a second solution to proposition~\ref{agree_dense}, this time making use of
propositions~\ref{scm_dense} and~\ref{seq_ch_cont}.
\end{prob}

\begin{prob} Use examples \ref{cont_add} and \ref{cont_mult} to show that if $x_n \sto a$ and
$y_n \sto b$ in~$\R$, then $x_n + y_n \sto a + b$ and $x_ny_n \sto ab$. (Do not give an
``$\epsilon$-$\delta$ proof''.)
\end{prob}

\begin{prob} Let $c$ be a point in a metric space $M$.  Show that the function
  \[ f \colon M \sto \R \colon  x \mapsto d(x,c) \]
is continuous. \emph{Hint.} Use problem \ref{ms_ineq}.
\end{prob}

\begin{prob}\label{dist_sets_cont}  Let $C$ be a nonempty subset of a metric space.  Then the
function
  \[ g \colon M \sto \R \colon  x \mapsto d(x,C) \]
is continuous.  (See \ref{dist_sets} for the definition of $d(x,C)$.)
\end{prob}

 \index{Urysohn's lemma}%
\begin{prop}[Urysohn's lemma]\label{prob_urys_lem} Let $A$ and $B$ be nonempty disjoint closed
subsets of a metric space~$M$.  Then there exists a continuous function $f \colon M \sto \R$
such that $\ran f \subseteq [0,1]$, $f^{\sto}(A) = \{0\}$, and $f^{\sto}(B) = \{1\}$.
\end{prop}

\begin{proof} Problem.  \emph{Hint.} Consider $\dfrac{d(x,A)}{d(x,A) + d(x,B)}$.  Use
problems~\ref{dist_sets_prob}(c) and~\ref{dist_sets_cont}.  \ns
\end{proof}

\begin{prob} (Definition. Disjoint sets $A$ and $B$ in a metric space $M$ are said to be
 \index{separated!by open sets}%
\df{separated by open sets} if there exist $\open{U,\,V}M$ such that $U \cap V = \emptyset$,
$A \subseteq U$, and $B~\subseteq~V$.) Show that in a metric space every pair of disjoint
closed sets can be separated by open sets.
\end{prob}

\begin{prob} If $f$ and $g$ are continuous real valued functions on a metric space~$M$, then
$\{x \in M\colon  f(x) \ne g(x)\}$ is an open subset of~$M$.
\end{prob}

\begin{prob} Show that if $f$ is a continuous real valued function on a metric space,
then $\abs{f}$ is continuous.  (We denote by $\abs{f}$ the function $x \mapsto \abs{f(x)}$.)
\end{prob}

\begin{prob} Show that metrics are continuous functions. That is, show that if $M$ is a set and
$d\colon M \times M \sto \R$ is a metric, then $d$ is continuous. Conclude from this that if $x_n
\sto a$ and $y_n \sto b$ in a metric space, then $d(x_n,y_n) \sto d(a,b)$.
\end{prob}

\begin{prob} Show that if $f$ and $g$ are continuous real valued functions on a metric space~$M$
and $f(a) = g(a)$ at some point $a \in M$, then for every $\epsilon > 0$ there exists a
neighborhood $U$ of $a$ such that $f(x) < g(x) + \epsilon$ for all $x \in U$.
\end{prob}







\section{MAPS INTO AND FROM PRODUCTS}
Let $(M_1,\rho_1)$ and $(M_2,\rho_2)$ be metric spaces.  Define the
 \index{coordinate projections}%
 \index{projections!coordinate}%
\df{coordinate projections} $\pi_1$ and $\pi_2$ on the product $M_1\times M_2$ by
  \[\pi_k\colon M_1 \times M_2 \sto M_k\colon (x_1,x_2) \mapsto x_k
                     \qquad\text{for $k=1,2$.}\]
If $M_1 \times M_2$ has the product metric $d_1$ (see~\ref{3prods} and~\ref{prod_met}),
then the coordinate projections turn out to be continuous functions.

\begin{exer}\label{coord_projs} Prove the assertion made in the preceding sentence.
(Solution~\ref{sol_coord_projs}.)
\end{exer}

\begin{notn} Let $S_1$, $S_2$, and $T$ be sets.  If $f\colon  T \sto S_1 \times S_2$, then we
define functions
 \index{<@$f^k$ ($k^{\text{th}}$ component of a function)}%
$f^1 := \pi_1 \circ f$ and $f^2 := \pi_2 \circ f$.  These are the
 \index{components!of a function}%
\df{components} of $f$.

If, on the other hand, functions $g\colon T \sto S_1$ and $h\colon T \sto S_2$ are given, we define
the function $(g,h)$ by
  \[(g,h)\colon T \sto S_1 \times S_2\colon x \mapsto (g(x),h(x))\,.\]
Thus it is clear that whenever $f\colon T \sto S_1 \times S_2$, we have
  \[f = (f^1,f^2)\,.\]
\end{notn}

\begin{prop}\label{components} Let $M_1$, $M_2$, and $N$ be metric spaces and $f\colon N \sto M_1
\times M_2$. The function $f$ is continuous if and only if its components $f^1$ and $f^2$ are.
\end{prop}

\begin{proof} Exercise.   (Solution~\ref{sol_components}.) \ns  \end{proof}

\begin{prop}\label{prod_cont} Let $f$ and $g$ be continuous real valued functions on a metric space.
 \begin{enumerate}
  \item[(a)] The product $fg$ is a continuous function.
  \item[(b)] For every real number $\alpha$ the function $\alpha g\colon x \mapsto \alpha g(x)$
is continuous.
 \end{enumerate}
\end{prop}

\begin{proof} Exercise.   (Solution~\ref{sol_prod_cont}.)  \ns  \end{proof}

\begin{prob} Let $f$ and $g$ be continuous real valued functions on a metric space and suppose that
$g$ is never zero.  Show that the function $f/g$ is continuous.
\end{prob}

\begin{prop} If $f$ and $g$ are continuous real valued functions on a metric space, then $f+g$ is
continuous.
\end{prop}

\begin{proof} Problem.  \ns  \end{proof}

\begin{prob} Show that every polynomial function on $\R$ is continuous.  \emph{Hint.} An induction
on the degree of the polynomial works nicely.
\end{prob}

\begin{defn} Let $S$ be a set and $f$, $g\colon S \sto \R$.  Then $f \lor g$, the
 \index{<@$f \lor g$ (supremum of two functions)}%
 \index{supremum!of two functions}%
\df{supremum} (or
 \index{maximum!of two functions}%
\df{maximum}) of $f$ and $g$, is defined by
  \[(f \lor g)(x) := \max\{f(x),g(x)\}\]
for every $x \in S$.  Similarly, $f \land g$, the
 \index{<@$f \land g$ (infimum of two functions)}%
 \index{infimum!of two functions}%
\df{infimum} (or
 \index{minimum!of two functions}%
\df{minimum}) of $f$ and $g$, is defined by
  \[ (f \land g)(x) := \min\{f(x),g(x)\} \]
for every $x \in S$.
\end{defn}

\begin{prob} Let $f(x)= \sin x$ and $g(x) = \cos x$ for $0 \le x \le 2\pi$.  Make a careful
sketch of $f \lor g$ and $f \land g$.
\end{prob}

\begin{prob}\label{prob_cont_sup}  Show that if $f$ and $g$ are continuous real valued functions
on a metric space, then $f \lor g$ and $f \land g$ are continuous.  \emph{Hint.} Consider
things like $f + g + \abs{f - g}$.
\end{prob}

In proposition \ref{components} we have dealt with the continuity of functions which map
\emph{into} products of metric spaces.  We now turn to functions which map \emph{from}
products; that is, to functions of several variables.

\begin{notn} Let $S_1$, $S_2$, and $T$ be sets and $f\colon S_1 \times S_2 \sto T$.  For each
$a \in S_1$ we define the function
 \index{<@$f(a,\,\cdot\,)$, $f(\,\cdot\,,b)$}%
  \[f(a,\,\cdot\,)\colon S_2 \sto T\colon y \mapsto f(a,y)\]
and for each $b \in S_2$ we define the function
  \[f(\,\cdot\,,b)\colon S_1 \sto T\colon  x \mapsto f(x,b)\,.\]
Loosely speaking, $f(a,\,\cdot\,)$ is the result of regarding $f$ as a function of only its
second variable; $f(\,\cdot\,,b)$ results from thinking of $f$ as depending on only its first
variable.
\end{notn}

\begin{prop}\label{jc_sc} Let $M_1$, $M_2$, and $N$ be metric spaces and $f\colon M_1 \times
M_2 \sto N$.  If $f$ is continuous, then so are $f(a,\,\cdot\,)$ and $f(\,\cdot\,,b)$ for all $a
\in M_1$ and $b \in M_2$.
\end{prop}

\begin{proof} Exercise.   (Solution~\ref{sol_jc_sc}.)  \ns  \end{proof}

This proposition is sometimes paraphrased as follows: Joint continuity implies separate continuity.
The converse is \emph{not} true. (See problem~\ref{sc_n_jc}.)

\begin{rem} It should be clear how to extend the results of
propositions~\ref{components} and~\ref{jc_sc} to products of any finite number of metric spaces.
\end{rem}

\begin{prob}\label{sc_n_jc} Let $f\colon \R^2 \sto \R$ be defined by
  \[f(x,y) = \begin{cases} xy(x^2+y^2)^{-1},
                    &\text{for $(x,y) \ne (0,0)$} \\
                 0, &\text{ for $x = y = 0$.}
             \end{cases}\]
 \begin{enumerate}
  \item[(a)] Show that $f$ is continuous at each point of $\R^2$ except at $(0,0)$, where it is
\emph{not} continuous.
  \item[(b)] Show that the converse of proposition \ref{jc_sc} is not true.
 \end{enumerate}
\end{prob}

\begin{notn} Let $M$ and $N$ be metric spaces.  We denote by
 \index{continuous@$\fml C(M,N)$!family of continuous functions from $M$ into $N$}%
$\fml C(M,N)$ the family of all continuous functions $f$ taking $M$ into~$N$.
\end{notn}

In proposition \ref{unif_lim_bdd} we showed that the uniform limit of a sequence of bounded real
valued functions is bounded.  We now prove an analogous result for continuous real valued
functions: The uniform limit of a sequence of continuous real valued functions is continuous.

\begin{prop}\label{unif_lim_cont} If $(f_n)$ is a sequence of continuous real valued functions
on a metric space $M$ which converges uniformly to a real valued function $g$ on $M$, then $g$ is
continuous.
\end{prop}

\begin{proof} Exercise. (Solution~\ref{sol_unif_lim_cont}.)  \ns  \end{proof}

\begin{prob} If the word ``pointwise'' is substituted for ``uniformly'' in proposition
\ref{unif_lim_cont}, the conclusion no longer follows.  In particular, find an example of a
sequence $(f_n)$ of continuous functions on $[0,1]$ which converges pointwise to a function $g$ on
$[0,1]$ which is \emph{not} continuous.
\end{prob}









\section{LIMITS}
We now generalize to metric spaces the results of chapter \ref{lim_rvf} on limits of real
valued functions.  Most of this generalization is accomplished quite simply: just replace open
intervals on the real line with open balls in metric spaces.

\begin{defn} If $B_r(a)$ is the open ball of radius $r$ about a point $a$ in a metric space~$M$,
then $B_r^*(a)$, the
 \index{ball@$B_r^*(a)$ (deleted open ball)}%
 \index{deleted!open ball}%
 \index{ball!deleted}%
\df{deleted open ball} of radius $r$ about $a$, is just $B_r(a)$ with the point $a$ deleted.  That
is, $B_r^*(a) = \{x \in M\colon  0 < d(x,a) < r\}$.
\end{defn}

\begin{defn} Let $(M,d)$ and $(N,\rho)$ be metric spaces, $A \subseteq M$, $f\colon A \sto N$,
$a$ be an accumulation point of~$A$, and $l~\in~N$.  We say that $l$ is the
 \index{limit!of a function}%
\df{limit of} $f$ \df{as} $x$ \df{approaches}~$a$ (of the \df{limit of} $f$ \df{at}~$a$) if:
for every $\epsilon > 0$ there exists $\delta > 0$ such that $f(x) \in B_\epsilon(l)$ whenever
$x \in A \cap B_\delta^*(a)$.  In slightly different notation:
  \[(\forall \epsilon > 0)(\exists \delta > 0)(\forall x \in A)\;
        0 < d(x,a) < \delta \implies \rho(f(x),l) < \epsilon.\]
When this condition is satisfied we write
 \index{<@$f(x) \sto l \text{ as } x \sto a$ (limit of a
function at a point)}%
  \[f(x) \sto l \quad \text{ as } x \sto a\]
or
 \index{limit@$\lim_{x \sto a}f(x)$ (limit of a function at a point)}%
  \[\lim_{x \sto a}f(x) = l.\]
As in chapter~\ref{lim_rvf} this notation is optimistic.  We will show in the next proposition that
limits, if they exist, are unique.
\end{defn}

\begin{prop}\label{mslim_uniq} Let $f\colon A \sto N$ where $M$ and $N$ are metric spaces and
$A \subseteq M$, and let $a \in A'$.  If $f(x) \sto b$ as $x \sto a$ and $f(x) \sto c$ as $x \sto
a$, then $b = c$.
\end{prop}

\begin{proof} Exercise.    (Solution~\ref{sol_mslim_uniq}.)   \ns  \end{proof}

For a function between metric spaces the relationship between its continuity at a point and its
limit there is exactly the same as in the case of real valued functions.  (See proposition
\ref{cnt_vs_lim} and the two examples which precede~it.)

\begin{prop}\label{cont_vs_lim} Let $M$ and $N$ be metric spaces, let $f\colon A \sto N$ where
$A \subseteq M$, and let $a \in A \cap A'$. Then $f$ is continuous at $a$ if and only if
   \[\lim_{x \sto a}f(x) = f(a).\]
\end{prop}

\begin{proof} Problem.  \emph{Hint.}  Modify the proof of \ref{cnt_vs_lim}.  \ns
\end{proof}

\begin{prop}\label{lim_changvar_R} If $M$ is a metric space, $f\colon A \sto M$ where $A \subseteq \R$,
and $a \in A'$, then
   \[ \lim_{h \sto 0}f(a + h) = \lim_{x \sto a}f(x) \]
in the sense that if either limit exists, then so does the other and the two limits are equal.
\end{prop}

\begin{proof} Problem.  \emph{Hint.}  Modify the proof of \ref{trnsl_lim}.  \ns
\end{proof}

We conclude this chapter by examining the relationship between ``double'' and ``iterated'' limits
of real valued functions of two real variables.  A limit of the form
  \[ \lim_{(x,y) \sto (a,b)}f(x,y) \]
is a
 \index{limit@$\lim_{(x,y) \sto (a,b)}f(x,y)$ (double limits)}%
 \index{double limit}%
 \index{limit!double}%
\df{double limit}; limits of the form
   \[\lim_{x \sto a}\bigl(\lim_{y \sto b}f(x, y)\bigr) \quad
         \text{ and } \quad \lim_{y \sto b}
         \bigl(\lim_{x \sto a}f(x,y)\bigr)\]
are
 \index{limit@$\lim_{x \sto a}\bigl(\lim_{y \sto b}f(x, y)\bigr)$ (iterated limits)}%
 \index{iterated limits}%
 \index{limit!iterated}%
\df{iterated limits}.  The meaning of the expression $\lim_{x \sto a}\bigl(\lim_{y \sto b}
f(x,y)\bigr)$ should be clear: it is $\lim_{x \sto a}h(x)$ where $h$ is the function defined by
$h(x) = \lim_{y \sto b}f(x,y)$.

\begin{exam} Let $f(x,y) = x\sin(1 + x^2y^2)$ for all $x,y \in \R$.  Then
$\lim_{(x,y)\sto(0,0)}f(x,y) = 0$.
\end{exam}

\begin{proof} The function $f$ maps $\R^2$ into $\R$.  We take the usual Euclidean metric on both
of these spaces.  Given $\epsilon > 0$, choose $\delta = \epsilon$.  If $(x,y) \in
B_\delta^*(0,0)$, then
  \[\abs{f(x,y) - 0} = \abs x \abs{\sin(1 + x^2y^2)}  \le \abs x \\
         \le \sqrt{x^2 + y^2} = d\bigl((x,y),(0,0)\bigr) <  \delta = \epsilon\,. \qedhere   \]
\end{proof}

\begin{exam} Let $f(x,y) = x\sin(1 + x^2y^2)$ for all $x$, $y \in \R$. Then
$\lim_{x\sto 0}\bigl(\lim_{y \sto 0}f(x,y)\bigr)~=~0$.
\end{exam}

\begin{proof} Compute the inner limit first: $\lim_{x\sto 0}\bigl(\lim_{y \sto 0}
\bigl(x\sin(1+x^2y^2)\bigr) = \lim_{x \sto 0}(x\sin 1)~=~0$.
\end{proof}

Because of the intimate relationship between continuity and limits
(proposition~\ref{cont_vs_lim}) and because of the fact that joint continuity implies separate
continuity (proposition~\ref{jc_sc}), many persons wrongly conclude that the existence of a
double limit implies the existence of the corresponding iterated limits. One of the last
problems in this chapter will provide you with an example of a function having a double limit
at the origin but failing to have one of the corresponding iterated limits.  In the next
proposition we prove that if in addition to the existence of the double limit we assume that
$\lim_{x \sto a}f(x,y)$ and $\lim_{y \sto b}f(x,y)$ always exist, then both iterated limits
exist and are equal.

\begin{prop}\label{dbl_vs_iter}  Let $f$ be a real valued function of two real variables.  If the limit
  \[l = \lim_{(x,y) \sto (a,b)}f(x,y)\]
exists and if $\lim_{x \sto a}f(x,y)$ and $\lim_{y \sto b}f(x,y)$ exist for all $y$ and $x$,
respectively, then the iterated limits
   \[\lim_{x \sto a}\bigl(\lim_{y \sto b}f(x, y)\bigr) \quad \text{
               and } \quad \lim_{y \sto b}\bigl(\lim_{x \sto a}f(x,y)\bigr)\]
both exist and are equal to~$l$.
\end{prop}

\begin{proof} Exercise.  \emph{Hint.}  Let $g(x) = \lim_{y \sto b}f(x,y)$ for all $x \in \R$.  We
wish to show that $\lim_{x \sto a}g(x) = l$.  The quantity $\abs{g(x) - l}$ is small whenever both
$\abs{g(x) - f(x,y)}$ and $\abs{f(x,y) - l}$ are.  Since $\lim_{(x,y) \sto (a,b)}f(x,y) = l$ it is
easy to make $\abs{f(x,y) - l}$ small: insist that $(x,y)$ lie in some sufficiently small open ball
about $(a,b)$ of radius, say, $\delta$.  This can be accomplished by requiring, for example, that
 \begin{equation}\label{mslim_eq1}
          \abs{x - a} < \delta/2
 \end{equation}
and that
 \begin{equation}\label{mslim_eq2}
          \abs{y - b} < \delta/2.
 \end{equation}
Since $g(x) = \lim_{y \sto b}f(x,y)$ for every $x$, we can make $\abs{g(x) - f(x,y)}$ small (for
fixed $x$) by supposing that
 \begin{equation}\label{mslim_eq3}
          \abs{y - b} < \eta
 \end{equation}
for some sufficiently small~$\eta$.  So the proof is straightforward: require $x$ to satisfy
\eqref{mslim_eq1} and for such $x$ require $y$ to satisfy \eqref{mslim_eq2} and
\eqref{mslim_eq3}. (Solution~\ref{sol_dbl_vs_iter}.) \ns
\end{proof}

It is sometimes necessary to show that certain limits do \emph{not} exist. There is a rather simple
technique which is frequently useful for showing that the limit of a given real valued function
does not exist at a point~$a$.  Suppose we can find two numbers $\alpha \ne \beta$ such that in
\emph{every} neighborhood of $a$ the function $f$ assumes (at points other than $a$) both the
values $\alpha$ and~$\beta$.  (That is, for every $\delta > 0$ there exist points $u$ and $v$ in
$B_\delta(a)$ distinct from $a$ such that $f(u) = \alpha$ and $f(v) = \beta$.)  Then it is easy to
see that $f$ cannot have a limit as $x$ approaches~$a$.  Argue by contradiction: suppose $\lim_{x
\sto a}f(x) = l$.  Let  $\epsilon = \abs{\alpha - \beta}$.  Then $\epsilon > 0$; so there exists
$\delta > 0$ such that $\abs{f(x) - l} < \epsilon/2$ whenever $0 < d(x,a) < \delta$.  Let $u$ and
$v$ be points in $B_\delta^*(a)$ satisfying $f(u) = \alpha$ and $f(v) = \beta$.  Since $\abs{f(u) -
l} < \epsilon/2$ and $\abs{f(v) - l} < \epsilon/2$, it follows that
 \begin{align*}
              \epsilon  &= \abs{\alpha - \beta}   \\
                        &= \abs{f(u) - f(v)}      \\
                        &\le \abs{f(u) - l} + \abs{l - f(v)}  \\
                        &< \epsilon
 \end{align*}
which is an obvious contradiction.

\begin{exam}\label{lim_nex} Let $f(x,y) = \dfrac{x^2y^2}{x^2y^2 + (x + y)^4}$ if $(x,y) \ne (0,0)$.
Then $\lim_{(x,y) \sto (0,0)}f(x,y)$ does not exist.
\end{exam}

\begin{proof} Exercise.   (Solution~\ref{sol_lim_nex}.) \ns  \end{proof}

\begin{exam} The limit as $(x,y) \sto (0,0)$ of $\dfrac{x^3y^3}{x^{12} + y^4}$ does not exist.
\end{exam}

\begin{proof} Problem.  \ns  \end{proof}

\begin{prob}\label{abs_lim_R} Let $f\colon A \sto \R$ where $A \subseteq M$ and $M$ is a metric
space, and let $a \in A'$.  Show that
    \[ \lim_{x \sto a}f(x) = 0 \quad \text{ if and only if }
                  \quad \lim_{x \sto a}\abs{f(x)} = 0\,. \]
\end{prob}

\begin{prob} Let $f$, $g$, $h\colon A \sto \R$ where $A \subseteq M$ and $M$ is a metric space,
and let $a \in A'$.  Show that if $f \le g \le h$ and
   \[ \lim_{x \sto a}f(x) = \lim_{x \sto a}h(x) = l\,, \]
then $\lim_{x \sto a}g(x) = l$.
\end{prob}

\begin{prob} Let $M$, $N$, and $P$ be metric spaces, $a \in A \subseteq M$, $f\colon A \sto N$,
and $g\colon N \sto P$.
 \begin{enumerate}
  \item[(a)] Show that if $l = \lim_{x \sto a}f(x)$ exists and $g$ is continuous at $l$, then
$\lim_{x \sto a}(g \circ f)(x)$ exists and is equal to $g(l)$.
  \item[(b)] Show by example that the following assertion need not be true: If $l =
\lim_{x \sto a}f(x)$ exists and $\lim_{y \sto l}g(y)$ exists, then $\lim_{x \sto a}(g \circ f)(x)$
exists.
 \end{enumerate}
\end{prob}

\begin{prob} Let $a$ be a point in a metric space. Show that
   \[ \lim_{x \sto a}d(x,a) = 0\,. \]
\end{prob}

\begin{prob}\label{prob_lim_Rn} In this problem $\R^n$ has its usual metric; in particular,
   \[ d(x,0) = \biggl(\sum_{k=1}^n x_k^2\biggr)^{1/2} \]
for all $x = (x_1,\dots,x_n) \in \R^n$.
 \begin{enumerate}
  \item[(a)] Show that
   \[ \lim_{x \sto 0} \frac{x_jx_k}{d(x,0)} = 0 \]
whenever $1 \le j \le n$ and $1 \le k \le n$.
  \item[(b)] For $1 \le k \le n$ show that $\lim_{x \sto 0}\dfrac{x_k}{d(x,0)}$ does not exist.
 \end{enumerate}
\end{prob}

\begin{prob}\label{lim_exam1} Let $f(x,y) = \dfrac{x^2 - y^2}{x^2 + y^2}$ for $(x,y) \ne (0,0)$.
Find the following limits, if they exist.
 \begin{enumerate}
  \item[(a)] $\lim_{(x,y) \sto (0,0)}f(x,y)$
  \item[(b)] $\lim_{x \sto 0}\bigl(\lim_{y \sto 0}f(x,y)\bigr)$
  \item[(c)] $\lim_{y \sto 0}\bigl(\lim_{x \sto 0}f(x,y)\bigr)$
 \end{enumerate}
\end{prob}

\begin{prob} Same as problem \ref{lim_exam1}, but $f(x,y) = \dfrac{xy}{x^2 + y^2}$.
\end{prob}

\begin{prob} Same as problem \ref{lim_exam1}, but $f(x,y) = \dfrac{x^2y^2}{x^2 + y^2}$.
\end{prob}

\begin{prob} Same as problem \ref{lim_exam1}, but $f(x,y) = y\sin(1/x)$ if $x \ne 0$ and
$f(x,y) = 0$ if $x = 0$.
\end{prob}




\endinput
