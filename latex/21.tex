\chapter{LINEARITY}


\section{LINEAR TRANSFORMATIONS}  Linear transformations are central to our study of calculus.
Functions are differentiable, for example, if they are smooth enough to admit decent
approximation by (translates of) linear transformations. Thus before tackling differentiation
(in chapter \ref{diff_calc}) we familiarize ourselves with some elementary facts about
linearity.

\begin{defn}  A function $T \colon V \sto W$ between vector spaces is
 \index{linear}%
 \index{linear!function}%
 \index{function!linear}%
\df{linear} if
 \begin{equation}\label{lt_cond1}
      T(x + y) = Tx + Ty \qquad \text{for all $x,y \in V$}
 \end{equation}
and
 \begin{equation}\label{lt_cond2}
     T(\alpha x) = \alpha Tx \qquad \text{for all $x \in V$ and
           $\alpha \in \R$.}
 \end{equation}
\end{defn}

A linear function is most commonly called a
 \index{linear!transformation}%
 \index{transformation!linear}%
\df{linear transformation},
sometimes a
 \index{linear!mapping}%
 \index{mapping!linear}%
\df{linear mapping}. If the domain and codomain of a linear transformation are the same vector
space, then it is often called a
 \index{linear!operator}%
 \index{operator!linear}%
\df{linear operator}, and occasionally a
 \index{vector!space!endomorphism}%
\df{vector space endomorphism}.  The family of all linear transformations from $V$ into $W$ is
denoted
 \index{l@$\ofml L(V,W)$ (linear maps between vector spaces)}%
by~$\ofml L(V,W)$.  Two oddities of notation concerning linear transformations deserve
comment.  First, the value of $T$ at x is usually written $Tx$ rather than~$T(x)$.  Naturally
the parentheses are used whenever their omission would create ambiguity.  For example, in
\eqref{lt_cond1} above $Tx + y$ is not an acceptable substitute for $T(x + y)$.  Second, the
symbol for composition of two linear transformations is ordinarily omitted.  If $S \in \ofml
L(U,V)$ and $T \in \ofml L(V,W)$, then the composite of $T$ and $S$ is denoted
 \index{<@$TS$ (composite of linear maps)}%
 \index{composite!of linear maps}%
by~$TS$ (rather than by $T \circ S$).  This will cause no confusion since we will define no
other ``multiplication'' of linear maps. As a consequence of this convention if $T$ is a
linear operator, then $T \circ T$ is written as $T^2$, $T \circ T \circ T$ as $T^3$, and so
on. One may think of condition \eqref{lt_cond1} in the definition of linearity in the
following fashion. Let $T \times T$ be the mapping from $V \times V$ into $W \times W$ defined
by
 \[(T \times T)(x,y) = (Tx,Ty).\]
Then condition \eqref{lt_cond1} holds if and only if the diagram
 \[ \xy
     \square<800,500>[V \times V`W \times W`V`W;T \times T`+`+`T]
    \endxy \]
commutes.  (The vertical maps are addition in $V$ and in $W$.)

Condition \eqref{lt_cond2} of the definition can similarly be thought of in terms of a
diagram.  For each scalar a define the function $M_\alpha$, \df{multiplication by} $\alpha$,
from a vector space into itself by
 \[ M_\alpha(x) = \alpha x\,. \]
(We use the same symbol for multiplication by $\alpha$ in both of the spaces $V$ and~$W$.)
Then condition \eqref{lt_cond2} holds if and only if for every scalar $\alpha$ the following
diagram commutes.
 \[ \xy
     \square[V`W`V`W;T`M_\alpha`M_\alpha`T]
    \endxy \]

\begin{exam}\label{lt_exam1}  If $T \colon \R^3 \sto \R^2 \colon  x \mapsto (x_1
+ x_3, x_1 - 2x_2)$, then $T$ is linear.
\end{exam}

\begin{proof} Exercise. (Solution~\ref{sol_lt_exam1}.)  \ns  \end{proof}

\begin{prob}  Let $T \colon  \R^2 \sto \R^4$ be defined by
  \[ T(x,y) = (x + 2y, 3x - y, -2x, -x + y). \]
Show that T is linear.
\end{prob}

\begin{exer}\label{lt_exer1}  Let $T \colon  \R^3 \sto \R^3$ be a linear transformation which
satisfies $T(e^1) = (1, 0, 1)$, $T(e^2) = (0, 2, -1)$, and $T(e^3) = (-4, -1, 3)$ (where
$e^1$, $e^2$, and $e^3$ are the standard basis vectors for $\R^3$ defined in example
\ref{basis_vect}).  Find $T(2, 1, 5)$.  \emph{Hint.}  Use problem \ref{vs_prob3}.
(Solution~\ref{sol_lt_exer1}.)
\end{exer}

\begin{prob}  Suppose that $T \in \ofml L(\R^3,\R^3)$ satisfies
 \begin{align*}
          T e^1 &= (1,2,-3) \\
          T e^2 &= (1,-1,0) \\
          T e^3 &= (-2,0,1)
 \end{align*}
Find $T(3,-2,1)$.
\end{prob}

\begin{prop}\label{lt_prop1} Let $T \colon  V \sto W$ be a linear transformation between two vector
spaces. Then
 \begin{enumerate}
  \item[(a)] $T(0) = 0$.
  \item[(b)] $T(x - y) = Tx - Ty$ for all $x$, $y \in V$.
 \end{enumerate}
\end{prop}

\begin{proof} Exercise.  (Solution~\ref{sol_lt_prop1}.)  \ns  \end{proof}

\begin{exam}  The identity map from a vector space into itself is linear.
\end{exam}

\begin{proof} Obvious.     \end{proof}

\begin{exam}  Each coordinate projection defined on $\R^n$
 \[ \pi_k \colon  \R^n \sto \R \colon  x \mapsto x_k \]
is linear.
\end{exam}

\begin{proof} For $1 \le k \le n$, we have $\pi_k(x + y) = \pi_k(x_1 + y_1, \dots, x_n + y_n) =
x_k + y_k = \pi_k(x) + \pi_k(y)$ and $\pi_k(\alpha x) = \pi_k(\alpha x_1, \dots, \alpha x_n) =
\alpha x_k = \alpha \pi_k(x)$.
\end{proof}

\begin{exam}\label{lt_exam2}  Let $\fml F = \fml F((a,b),\R)$ be the family of all real valued
functions defined on the open interval $(a,b)$ and let $\fml D = \fml D((a,b),\R)$ be the set
of all members of $\fml F$ which are differentiable at each point of~$(a,b)$.  Then $\fml D$
is a vector subspace of $\fml F$ and that the differentiation operator
  \[ D \colon \fml D \sto \fml F \colon  f \mapsto f' \]
(where $f'$ is the derivative of $f$) is linear.
\end{exam}

\begin{proof} We know from example \ref{vs_exam1} that $\fml F$ is a vector space.  To show that
$\fml D$ is a vector subspace of~$\fml F$ use proposition~\ref{vs_prop1}.  That $\fml D$ is
nonempty is clear since constant functions are differentiable. That the space $\fml D$ of
differentiable functions is closed under addition and scalar multiplication and that the
operation $D$ of differentiation is linear (that is, $D(\alpha f) = \alpha Df$ and $D(f + g) =
Df + Dg$) are immediate consequences of propositions \ref{diff_sm} and~\ref{diff_sum}.
\end{proof}

\begin{exam} We have not yet discussed integration of continuous functions, but recalling a few
basic facts about integration from beginning calculus shows us that this is another example of
a linear transformation.  Let $\fml C = \fml C([a,b],\R)$ be the family of all members of
$\fml F = \fml F([a,b],\R)$ which are continuous.  We know from beginning calculus that any
continuous function on a closed and bounded interval is (Riemann) integrable, that
 \[ \int_a^b \alpha f(x)\,dx = \alpha \int_a^b f(x)\,dx \]
where $\alpha \in \R$ and $f \in \fml C$, and that
 \[\int_a^b (f(x) + g(x))\,dx = \int_a^b f(x)\,dx + \int_a^b g(x)\,dx \]
where $f,g \in \fml C$.  It then follows easily (again from \ref{vs_prop1}) that $\fml C$ is a
vector subspace of $\fml F$ and that the function $K \colon \fml C \sto \R \colon  f \mapsto
\int_a^b f(x)\,dx$ is linear.
\end{exam}


An important observation is that the composite of two linear transformations is linear.

\begin{prop}\label{prop_comp_lt} Let $U$, $V$, and $W$ be vector spaces. If $S \in \ofml L(U,V)$
and $T \in \ofml L(V,W)$ then $TS \in \ofml L(U,W)$.
\end{prop}

\begin{proof}  Problem. \ns  \end{proof}

\begin{defn}  If $T \colon  V \sto W$ is a linear transformation between vector spaces, then the
 \index{kernel}%
\df{kernel} (or
 \index{null space}%
\df{null space}) of~$T$, denoted
 \index{kernel@$\ker T$ (kernel of a linear map)}%
by~$\ker T$, is $T^\gets\{\vc 0\}$. That is,
 \[ \ker T := \{x \in V \colon  Tx = \vc 0\}. \]
\end{defn}

\begin{exer}\label{lt_exer2}  Let $T \in \ofml L(\R^3,\R^3)$
satisfy
 \begin{align*}
                        T e^1 &= (1,-2,3) \\
                        T e^2 &= (0,0,0)  \\
                        T e^3 &= (-2,4,-6)
 \end{align*}
where $e^1$, $e^2$, and $e^3$ are the standard basis vectors in $\R^3$.  Find and describe
geometrically both the kernel of $T$ and the range of~$T$. (Solution~\ref{sol_lt_exer2}.)
\end{exer}

\begin{prob} Let $T$ be the linear transformation of example \ref{lt_exam1}.  Find and describe
geometrically the kernel and the range of~$T$.
\end{prob}

\begin{prob} Let $D$ be the linear transformation defined in \ref{lt_exam2}. What is the kernel
of~$D$?
\end{prob}

It is useful to know that the kernel of a linear transformation is always a vector subspace of
its domain, that its range is a vector subspace of its codomain, and that a necessary and
sufficient condition for a linear transformation to be injective is that its kernel contain
only the zero vector.

\begin{prop}\label{prop_ker_vsubsp} If $T \colon  V \sto W$ is a linear transformation between vector
spaces, then $\ker T$ is a subspace of~$V$.
\end{prop}

\begin{proof} Problem.  \emph{Hint.} Use proposition \ref{vs_prop1}. \ns
\end{proof}

\begin{prop}\label{lt_prop2} If $T \colon  V \sto W$ is a linear transformation between vector spaces,
then $\ran T$ is a subspace of~$W$.
\end{prop}

\begin{proof} Exercise.  \emph{Hint.} Use proposition \ref{vs_prop1}.
(Solution~\ref{sol_lt_prop2}.) \ns
\end{proof}

\begin{prop}\label{lt_prop3}  A linear transformation $T$ is injective if and only if $\ker~T~=~\{0\}$.
\end{prop}

\begin{proof} Problem.  \emph{Hint.} First show that if $T$ is injective and $x \in \ker T$, then $x=0$.
For the converse, suppose that $\ker T = \{\vc 0\}$ and that $Tx = Ty$.  Show that $x = y$.
\ns
\end{proof}

\begin{prob}\label{prob_lt_det_bas}  Let $W$ be a vector space.
 \begin{enumerate}
  \item[(a)] Show that a linear transformation $T \colon  \R^n \sto W$ is
completely determined by its values on the standard basis vectors
$e^1, \dots e^n$ of~$\R^n$.  \emph{Hint.} Use problem
\ref{vs_prob3}(b).
  \item[(b)]  Show that if $S,T \in \ofml L(\R^n, W)$ and $Se^k =
Te^k$ for $1 \le k \le n$, then $S = T$.
  \item[(c)]  Let $w^1, \dots, w^n \in W$.  Show that there exists a
unique $T \in \ofml L(\R^n)$ such that $Te^k = w^k$ for $1 \le k
\le n$.
 \end{enumerate}
\end{prob}

Injective linear mappings take linearly independent sets to linearly independent sets

\begin{prop}  If $T \in \ofml L(V,W)$ is injective and $A$ is a linearly independent subset of $V$,
then $T^\sto(A)$ is a linearly independent set in~$W$.
\end{prop}

\begin{proof} Problem.  \emph{Hint.}  Start with vectors $y^1, \dots y^n$ in $T^\sto(A)$ and suppose
that some linear combination of them $\sum_{k=1}^n \alpha_kx^k$ is~$0$.  Show that all the
scalars $\alpha_k$ are~$0$.  Use proposition~\ref{lt_prop3}.  \ns
\end{proof}

Linear mappings take convex sets to convex sets.

\begin{prop}  If $V$ and $W$ are vector spaces, $C$ is a convex subset of $V$, and $T \colon
V \sto W$ is linear, then $T^\sto(C)$ is a convex subset of~$W$.
\end{prop}

\begin{proof}  Problem.  \emph{Hint.}  Let $u$, $v \in T^\sto(C)$ and $0 \le t \le 1$.  Show
that $(1-t)u + tv \in T^\sto(C)$.    \ns
\end{proof}

\begin{prob} Let $T \in \ofml L(\R^3,\R^3)$ satisfy :
 \begin{align*}
         Te^1 &= (0,1,0) \\
         Te^2 &= (0,0,1) \\
         Te^3 &= (3,-2,0)
 \end{align*}
Show that $T$ is bijective. \emph{Hint.} To show that $T$ is injective use
proposition~\ref{lt_prop3}.
\end{prob}

\begin{prob}\label{prob_intop_c1}  Let $\fml C^1 = \fml C^1([a,b],\R)$ be the set of all functions
$f$ in $\fml F = \fml F([a,b],\R)$ such that $f'$ exists on $[a,b]$ in the usual sense of
having one-sided derivatives at $a$ and $b$) and is continuous.  (A function belonging to
$\fml C^1$ is said to be
 \index{continuously differentiable}%
 \index{differentiable!continuously}%
\df{continuously differentiable}.)  It is easy to see that the set of all $C^1$ functions is a
vector subspace of~$\fml F$.  Let $\fml C = \fml C([a,b],\R)$ be the family of all continuous
members of~$\fml F$.  For every $f \in \fml C$ and every $x \in [a,b]$ let
  \[ (Jf)(x) := \int_a^x f(t)\,dt\,. \]
 \begin{enumerate}
  \item[(a)]  Why does $Jf$ belong to $\fml C^1$?
  \item[(b)]  Show that the map $J \colon \fml C \sto \fml C^1 \colon  f \mapsto Jf$ is linear.
 \end{enumerate}
\end{prob}

\begin{prob} (Products) Recall (see Appendix \ref{prods}) that for every pair of functions
$f^1 \colon T \sto S_1$ and $f^2 \colon T \sto S_2$ having the same domain there exists a
unique map, namely $f = (f^1, f^2)$, mapping $T$ into the product space $S^1 \times S^2$ which
satisfies $\pi_1 \circ f = f^1$ and $\pi_2 \circ f = f^2$.  (See in particular exercise
\ref{prod_exer}.)  Now suppose that $T$, $S_1$, and $S_2$ are vector spaces and that $f^1$ and
$f^2$ are linear. Then $S_1 \times S_2$ is a vector space (see example \ref{prod_vs}).  Show
that the function $f = (f^1,f^2)$ is linear.
\end{prob}











\section{THE ALGEBRA OF LINEAR TRANSFORMATIONS}  The set $\ofml L(V,W)$ of linear transformations
between two vector spaces is contained in the vector space $\fml F(V,W)$ of \emph{all}
$W$-valued functions whose domain is~$V$.  (That $\fml F$ is a vector space was proved in
example~\ref{vs_exam1}.)  It is easy to show that $\ofml L(V,W)$ is a vector space; just show
that it is a subspace of~$\fml F(V,W)$.

\begin{prop}\label{endo_vs}  Let $V$ and $W$ be vector spaces. Then $\ofml L(V,W)$ with pointwise
operations of addition and scalar multiplication is a vector space.
\end{prop}

\begin{proof} Exercise. \emph{Hint.}  Use example \ref{vs_exam1} and proposition~\ref{vs_prop1}.
(Solution~\ref{sol_endo_vs}.)   \ns
\end{proof}

Let $T \colon V \sto W$ be a function between two sets.  We say that $T$ is
 \index{invertible!function}%
\df{invertible} if there exists a function $T^{-1}$ mapping $W$ to $V$ such that $T \circ
T^{-1}$ is the identity function on $V$ and $T^{-1} \circ T$ is the identity function on~$W$.
(For details about this see appendix~\ref{inverses}.)   Now, suppose that $V$ and $W$ are
vector spaces and that $T \colon V\sto W$ is a linear transformation.  In this context what do
we mean when we say that $T$ is \emph{invertible}?  For a linear transformation to be
invertible we will require two things: the transformation must possess an inverse function,
and this function must itself be linear. It is a pleasant fact about linear transformations
that the second condition is automatically satisfied whenever the first is.

\begin{prop}\label{prop_Tinv_lin} If $T \in \ofml L(V,W)$ is bijective, then its inverse
$T^{-1} \colon W \sto V$ is linear.
\end{prop}

\begin{proof} Exercise   (Solution~\ref{sol_prop_Tinv_lin}.) \ns    \end{proof}

\begin{defn}
A linear transformation $T \colon V \sto W$ is
 \index{invertible!linear mapping}%
 \index{linear!mapping!invertible}%
 \index{mapping!linear!invertible}%
\df{invertible} (or is an
 \index{isomorphism}%
\df{isomorphism}) if there exists a linear transformation $T^{-1}$ such that $T^{-1}\circ T =
I_V$ and $T \circ T^{-1} = I_W$.
\end{defn}

The point of the preceding proposition is that this definition is somewhat redundant. In
particular, the following are just different ways of saying the same thing about a linear
transformation~$T$.
 \begin{enumerate}
  \item[(a)] $T$ is invertible.
  \item[(b)] $T$ is an isomorphism.
  \item[(c)] As a function $T$ has an inverse.
  \item[(d)] $T$ is bijective.
 \end{enumerate}

\begin{defn} Vector spaces $V$ and $W$ are
 \index{isomorphic}%
\df{isomorphic} if there exists an isomorphism from $V$ onto~$W$.
\end{defn}

\begin{prob} Let $s$ be the set of all sequences of real numbers. Regard $s$ as a vector space
under pointwise operations. That is,
 \begin{align*}
          x + y &:= (x_1 + y_1, x_2 + y_2, \dots)  \\
       \alpha x &:= (\alpha x_1, \alpha x_2, \dots)
 \end{align*}
whenever $x = (x_1, x_2,  \dots)$ and $y = (y_1, y_2, \dots)$ belong to $s$ and $\alpha$ is a
scalar.  Define the
 \index{unilateral shift}%
 \index{shift!unilateral}%
 \index{operator!unilateral shift}%
\df{unilateral shift operator} $U \colon  s \sto s$ by  \[U(x_1,x_2,x_3,\dots) := (0,x_1,x_2,
\dots).\]
 \begin{enumerate}
  \item[(a)] Show that $U \in \ofml L(s,s)$.
  \item[(b)] Does $U$ have a right inverse?  If so, what is it?
  \item[(c)] Does $U$ have a left inverse?  If so, what is it?
 \end{enumerate}
\end{prob}

\begin{defn}\label{def_alg} Suppose that a vector space $V$ is equipped with an additional operation
$(x,y) \mapsto xy$ from $V \times V$ into~$V$ (we will call it \emph{multiplication}) which
satisfies
 \begin{enumerate}
  \item[(a)] $x(y + z) = xy + xz$,
  \item[(b)] $(x + y)z = xz + yz$,
  \item[(c)] $(xy)z = x(yz)$, and
  \item[(d)] $\alpha(xy) = x(\alpha  y)$
 \end{enumerate}
whenever  $x,y,z \in V$ and $\alpha \in \R$.  Then $V$ is an
 \index{algebra}%
\df{algebra}.  (Sometimes it is called a \df{linear associative algebra}.)  If an algebra
possesses a multiplicative identity (that is, a vector $\vc 1$ such that $\vc 1\,x = x\,\vc 1
= x$ for all $x \in V$), then it is a
 \index{unital!algebra}%
 \index{algebra!unital}%
\df{unital algebra}.  A subset of an algebra $A$ which is closed under the operations of
addition, multiplication, and scalar multiplication is a
 \index{subalgebra}%
\df{subalgebra} of~$A$.  If $A$ is a unital algebra and $B$ is a subalgebra of $A$ which
contains the multiplicative identity of $A$, then $B$ is a
 \index{unital!subalgebra}%
 \index{subalgebra!unital}%
\df{unital subalgebra} of~$A$.
\end{defn}

\begin{exam}\label{exam_cont_alg} If $M$ is a compact metric space, then the vector space
$\fml B(M,\R)$ of bounded functions on $M$ is a unital algebra under pointwise operations.
(The constant function $1$ is its multiplicative identity.)  We have already seen in
chapter~\ref{cpt} that the space $\fml C(M,\R)$ of continuous functions on $M$ is a vector
subspace of $\fml B(M,\R)$. Since the product of continuous functions is continuous (and
constant functions are continuous) $\fml C(M,\R)$ is a unital subalgebra of $\fml B(M,\R)$.
\end{exam}

\begin{prob}\label{prob_lt_alg} It has already been shown (in proposition~\ref{endo_vs}) that
if $V$ is a vector space, then so is~$\ofml L(V,V)$.  Show that with the additional operation
of composition serving as multiplication $\ofml L(V,V)$ is a unital algebra.
\end{prob}

\begin{prob} If $T \in \ofml L(V,W)$ is invertible, then so is $T^{-1}$ and
$\bigl(T^{-1})\bigr)^{-1} = T$.
\end{prob}

\begin{prob} If $S \in \ofml L(U,V)$ and $T \in \ofml L(V,W)$ are both invertible, then so is
$TS$ and $(TS)^{-1} = S^{-1}T^{-1}$.
\end{prob}

\begin{prob} If $T \in \ofml L(V,V)$ satisfies the equation
  \[ T^2 - T + I = \vc 0 \]
then it is invertible.  What is $T^{-1}$?
\end{prob}

\begin{prob}\label{lt_transfer} Let $V$ be a vector space; let $W$ be a set which is provided
with operations $(u,v) \mapsto u + v$ from $W \times W$ into $W$ and $(\alpha,u) \mapsto
\alpha u$ from $\R \times W$ into $W$; and let $T \colon V \sto W$. If $T$ is bijective and it
preserves operations (that is, $T(x + y) = Tx + Ty$ and $T(\alpha x) = \alpha Tx$ for all $x$,
$y \in V$ and $\alpha \in \R$), then $W$ is a vector space which is isomorphic to~$V$.
\emph{Hint.}  Verify the eight defining axioms for a vector space. The first axiom is
associativity of addition.  Let $u$, $v$, $w \in W$. Write $(u + v) + w$ as $\bigl(T(T^{-1}u)
+ T(T^{-1}v)\bigr) + T(T^{-1}w)$ and use the hypothesis that $T$ is operation preserving.
\end{prob}

\begin{prob}\label{prob_vsp_from_bij}  Let $V$ be a vector space, $W$ be a set, and $T \colon
V \sto W$ be a bijection. Explain carefully how $W$ can be made into a vector space
isomorphic to~$V$. \emph{Hint.} Use problem \ref{lt_transfer}.
\end{prob}











\section{MATRICES}\label{matr}
The purpose of this section and the next two is almost entirely computational.  Many (but by
no means all!) of the linear transformations we will consider in the sequel are maps between
various Euclidean spaces; that is, between $\R^n$ and $\R^m$ where $m$, $n \in \N$.  Such
transformations may be represented by matrices.  This of great convenience in dealing with
specific examples because matrix computations are so very simple. We begin by reviewing a few
elementary facts about matrices and matrix operations.

For each $n \in \N$ let $\N_n$ be $\{1, \dots, n\}$.  An $m \times n$ (read ``$m$ by $n$'')
 \index{matrix}%
\df{matrix} is a function whose domain is $\N_m \times \N_n$.  We deal here only with matrices
of real numbers; that is, with real valued functions on $\N_m \times \N_n$.  If $a \colon
\N_m \times \N_n \sto \R$ is an $m \times n$ matrix, its value at $(i,j) \in \N_m \times \N_n$
will be denoted by~$a_j^i$.  (Occasionally we use the notation $a_{ij}$ instead.)  The matrix
$a$ itself may be denoted by ${\bigl[a_j^i\bigr]_{i=1}^m}_{j=1}^n$,
 \index{<@$[a_j^i]$, $[a_{ij}]$ (matrix notation)}%
by $[a_j^i]$,  or by a rectangular array whose entry in row $i$ and
column $j$ is $a_j^i$.
 \[ \begin{bmatrix}
             a_1^1  &  a_2^1  &  \hdots &  a_n^1  \\
             a_1^2  &  a_2^2  &  \hdots &  a_n^2  \\
             \vdots &  \vdots &  \ddots &  \vdots \\
             a_1^m  &  a_2^m  &  \hdots &  a_n^m
    \end{bmatrix} \]
In light of this notation it is reasonable to refer to the index $i$ in the expression $a_j^i$
as the
 \index{row!index}%
\df{row!index} and to call $j$ the
 \index{column!index}%
\df{column index}.  (If you are accustomed to thinking of a matrix as \emph{being} a
rectangular array, no harm will result.  The reason for defining a matrix as a function is to
make good on the boast made in appendix~\ref{sets} that everything in the sequel can be
defined ultimately in terms of sets.) We denote the family of all $m \times n$ matrices of
real numbers
 \index{matrices@$M_{m \times n}$ ($m$ by $n$ matrices)}%
by~$M_{m \times n}$.  For families of square matrices we shorten $M_{n \times n}$
 \index{matrices@$M_n$ (square matrices)}%
to~$M_n$.

Two $m \times n$ matrices $a$ and $b$ may be added. Addition is done pointwise. The sum
 \index{<add@$a+b$ (sum of matrices)}%
$a+b$ is the $m \times n$ matrix whose
value at $(i,j)$ is $a_j^i + b_j^i$,  That is,
  \[ (a + b)_j^i =  a_j^i + b_j^i \]
for $1 \le i \le m$ and $1 \le j \le n$.  Scalar multiplication is also defined pointwise.  If
$a$ is an $m \times n$ matrix and $\alpha \in \R$, then
 \index{<@$\alpha a$ (scalar multiple of a matrix)}%
$\alpha a$ is the $m \times n$ matrix whose
value at $(i,j)$ is $\alpha a_j^i$.  That is,
  \[ (\alpha a)_j^i  =  \alpha a_j^i \]
for $1 \le i \le m$ and $1 \le j \le n$.  We may also subtract matrices. By $-b$ we mean
$(-1)b$, and by $a - b$ we mean~$a + (-b)$.

\begin{exer}\label{ex_mat_add} Let
 \[ \begin{bmatrix}
              4  &  2  &  0  & -1  \\
             -1  & -3  &  1  &  5
    \end{bmatrix} \qquad \text{ and } \qquad
    \begin{bmatrix}
              1  & -5  &  3  & -1  \\
              3  &  1  &  0  & -1
    \end{bmatrix}\,. \]
Find $a + b$, $3a$, and $a - 2b$.   (Solution~\ref{sol_ex_mat_add}.)
\end{exer}

If $a$ is an $m \times n$ matrix and $b$ is an $n \times p$ matrix, the product of $a$ and $b$
is the $m \times p$ matrix whose value at $(i,j)$ is $\sum_{k=1}^n a_k^i b_j^k$.  That is,
 \[ (ab)_j^i = \sum_{k=1}^n a_k^i b_j^k\]
for $1 \le i \le m$ and $1 \le j \le p$.  Notice that in order for the product
 \index{<@$ab$ (product of matrices)}%
 \index{product!of matrices}%
$ab$ to be defined the number of columns of $a$ must be the same as the number of rows of~$b$.
Here is a slightly different way of thinking of the product of $a$ and~$b$.  Define the
 \index{inner product}%
 \index{product!inner}%
\df{inner product} (or
 \index{dot product}%
 \index{product!dot}%
\df{dot product}) of two $n$-tuples $(x_1,x_2,\dots, x_n)$ and $(y_1,y_2,\dots, y_n)$ to be
$\sum_{k=1}^n x_k y_k$.  Regard the rows of the matrix $a$ as $n$-tuples (read from left to
right) and the columns of $b$ as $n$-tuples (read from top to bottom).  Then the entry in the
$i^{\text{th}}$ row and $j^{\text{th}}$ column of the product $ab$ is the dot product of the
$i^{\text{th}}$ row of $a$ and the $j^{\text{th}}$ column of~$b$.

\begin{exam} Matrix multiplication is not commutative.  If $a$ is a $2 \times 3$ matrix and
$b$ is a $3 \times 4$ matrix, then $ab$ is defined but $ba$ is not.  Even in situations where
both products $ab$ and $ba$ are defined, they need not be equal.  For example, if $a =
\begin{bmatrix} 1 & 2 \\ 1 & 0 \end{bmatrix}$ and $b = \begin{bmatrix} -1 & 1 \\ 2 & 3
\end{bmatrix}$, then $ab =
\begin{bmatrix} 3 & 7 \\ -1 & 1 \end{bmatrix}$ whereas $ba =
\begin{bmatrix} 0 & -2 \\ 5 & 4 \end{bmatrix}$.
\end{exam}

\begin{exer}\label{ex_mat_mult} Let $a = \begin{bmatrix} 2 & 3 & -1 \\ 0 & 1 & 4 \end{bmatrix}$
and $b = \begin{bmatrix} 1 & 0 \\ 2 & -1 \\ 1 & -\end{bmatrix}$.  Find $ab$.
(Solution~\ref{sol_ex_mat_mult}.)
\end{exer}

\begin{prob} Let $a = \begin{bmatrix}
                            4 &  3 &  1 &  2 \\
                            0 & -1 & -1 &  1 \\
                            2 &  0 &  1 &  3
                      \end{bmatrix}$
and $b = \begin{bmatrix}
                   2 & -1 \\
                   0 &  1 \\
                   1 &  0 \\
                  -3 &  2
         \end{bmatrix}$.
 \begin{enumerate}
  \item[(a)] Find the product $ab$ (if it exists).
  \item[(b)] Find the product $ba$ (if it exists).
 \end{enumerate}
\end{prob}

\begin{defn} Let $a$ be an $m \times n$ matrix.  The
 \index{transpose}%
\df{transpose} of $a$, denoted by $a^t$, is the $n \times m$ matrix obtained by interchanging
the rows and columns of~$a$.  That is, if $b = a^t$, then $b_j^i = a_i^j$ for $1 \le i \le n$
and $1 \le j \le  m$.
\end{defn}

\begin{exam} Let $a =  \begin{bmatrix}
                            1 &  2 &  0 & -4 \\
                            3 &  0 & -1 &  5
                       \end{bmatrix}$.  Then
$a^t = \begin{bmatrix}
               1 &  3 \\
               2 &  0 \\
               0 & -1 \\
              -4 &  5
       \end{bmatrix}$.
\end{exam}

For material in the sequel the most important role played by matrices will be as
(representations of) linear transformations on finite dimensional vector spaces. Here is how
it works.

\begin{defn}\label{def_act_matr} We define the \emph{action} of a matrix on a vector.  If
$a \in M_{m \times n}$ and $x \in \R^n$, then $ax$, the
 \index{action of a matrix}%
\df{result of} $a$ \df{acting on} $x$, is defined to be the vector in $\R^m$ whose
$j^{\text{th}}$ coordinate is $\sum_{k=1}^n a_k^j x_k$ (this is just the dot product of the
$j^{\text{th}}$ row of $a$ with~$x$).  That is,
  \[ (ax)_j := \sum_{k=1}^n a_k^jx_k \]
for $1 \le j \le m$.  Here is another way of saying the same thing: Regard $x$ as an $n \times
1$ matrix
  \[ \begin{bmatrix} x_1 \\ x_2 \\ \vdots \\ x_n \end{bmatrix} \]
(sometimes called a
 \index{column!vector}%
 \index{vector!column}%
\df{column vector}).  Now multiply the $m \times n$ matrix $a$ by the $n \times 1$ matrix~$x$.
The result will be an $m \times 1$ matrix (another column vector), say
  \[ \begin{bmatrix} y_1 \\ y_2 \\ \vdots \\ y_m \end{bmatrix}\,. \]
Then
 \index{<@$ax$ (action of a matrix on a vector)}%
$ax$ is the $m$-tuple $(y_1,\dots, y_m)$.  Thus $a$ may be thought of as a mapping from $\R^n$
into $\R^m$.
\end{defn}

\begin{exer}\label{ex_mat_act} Let $a =
        \begin{bmatrix}
             3 &  0 & -1 & -4 \\
             2 &  1 & -1 & -2 \\
             1 & -3 &  0 &  2
        \end{bmatrix}$ and $x = (2, 1, -1, 1)$.  Find $ax$.  (Solution~\ref{sol_ex_mat_act}.)
\end{exer}

\begin{prob}  Let $a = \begin{bmatrix}
                            2 &  0 \\
                            1 & -3 \\
                            5 &  1
                       \end{bmatrix}$
and $x = (1,-2)$.  Find $ax$.
\end{prob}

From the definition of the action of a matrix on a vector, we derive several formulas which
will be useful in the sequel. Each is a simple computation.

\begin{prop}\label{prop_matr_arith} Let $a$, $b \in M_{m \times n}$; $c \in M_{n \times p}$;
$x$, $y \in \R^n$; $z \in \R^p$; and $\alpha \in \R$.  Then
 \begin{enumerate}
  \item[(a)] $a(x + y) = ax + ay$;
  \item[(b)] $a(\alpha x) = \alpha(ax)$;
  \item[(c)] $(a + b)x = ax + bx$;
  \item[(d)] $(\alpha a)x = \alpha(ax)$;
  \item[(e)] $(ac)z = a(cz)$.
 \end{enumerate}
\end{prop}

\begin{proof} Part (a) is an exercise. (Solution~\ref{sol_prop_matr_arith}.)
Parts (b)--(e) are problems.  \ns \end{proof}

Next we show that a sufficient (and obviously necessary) condition for two $m \times n$
matrices to be equal is that they have the same action on the standard basis vectors
in~$\R^n$.

\begin{prop}\label{prop_mrlt_bas}  Let $a$ and $b$ be $m \times n$ matrices and, as usual,
let $e^1, \dots, e^n$ be the standard basis vectors in $\R^n$.  If $ae^k = be^k$ for $1 \le k
\le n$, then $a = b$.
\end{prop}

\begin{proof} Problem.  \emph{Hint.}  Compute $(ae^k)_j$ and $(be^k)_j$ for $1 \le j \le m$
and $1 \le k \le n$.  Remember that $(e^k)_l = 0$ if $k \ne l$ and that $(e^k)_k = 1$.  \ns
\end{proof}

\begin{rem} The definition~\ref{def_act_matr} of the action of a matrix on a vector technically
requires us to think of vectors as ``column vectors''.  It is probably more likely that most
of us think of vectors in $\R^n$ as
 \index{row!vector}%
 \index{vector!row}%
``row vectors'', that is, as $n$-tuples or as $1 \times n$ matrices. Then for the matrix
multiplication $ax$ to make sense and for the result to again be a ``row vector'' we really
should write
  \[ \bigl(a(x^t)\bigr)^t \]
for the action of the matrix $a \in M_{m \times n}$ on the vector $x \in \R^n$. We won't do
this.  We will regard vectors as ``row vectors'' or ``column vectors'' as convenience
dictates.
\end{rem}

\begin{defn}  In our later work we will have occasion to consider
the action of a
 \index{square matrix}%
\df{square matrix} (one with the same number of rows as columns) on a pair of vectors.  Let $a
\in  M_n$ and $x,y \in \R^n$.  We denote by $xay$ the number $\sum_{j,k=1}^n a_k^jx_jy_k$.

Since
 \[\sum_{j,k=1}^n a_k^jx_jy_k = \sum_{j=1}^n x_j \sum_{k=1}^na_k^jy_k\]
and since $\sum_{k=1}^na_k^jy_k$ is just $(ay)_j$, we may write
 \[xay = \sum_{j=1}^n x_j(ay)_j.\]
\end{defn}

In other words $xay$ is just the dot product of the vectors $x$ and
$ay$.  If we identify $n$-tuples (row vectors) with $1 \times n$
matrices, then $xay$ is the product of the three matrices $x$, $a$,
and $y^t$.  That is,
 \[ xay = \begin{bmatrix} x_1 \dots x_n \end{bmatrix}
   \begin{bmatrix}
        a_1^1 & \hdots & a_n^1  \\
       \vdots & \ddots & \vdots \\
        a_1^n & \hdots & a_n^n
   \end{bmatrix}
   \begin{bmatrix}
           y_1 \\
           \vdots \\
           y_n
   \end{bmatrix}\,.\]

\begin{exer}\label{ex_mat_act3} Let $a = \begin{bmatrix}
                                              1 & 3 & -1 \\
                                              0 & 2 & 4  \\
                                              1 & -1 & 1
                                         \end{bmatrix}$, $x = (1,-2,0)$, and $y = (3,0,1)$.
Find the action of $a$ on the pair of vectors $x$ and $y$; that is, find~$xay$.
(Solution~\ref{sol_ex_mat_act3}.)

\end{exer}

\begin{prob} Let $a = \begin{bmatrix}
                               1 &  2 &  0 & -1 \\
                               3 & -3 &  1 &  0 \\
                               2 &  0 &  1 & -4 \\
                              -1 &  1 & -1 &  1
                      \end{bmatrix}$, $x = (1,-1,0,2)$, and $y = (1,0,3,1)$.  Find~$xay$.
\end{prob}

\begin{defn} The
 \index{main diagonal}%
 \index{diagonal!main}%
\df{main} (or
 \index{principal diagonal}%
 \index{diagonal!principal}%
\df{principal}) \df{diagonal} of a square matrix is the diagonal running from the upper left
corner to the lower right corner.  That is, it consists of all the elements of the
form~$a_k^k$.  If each entry on the main diagonal of an $n \times n$ matrix is $1$ and all its
other entries are $0$, then the matrix is the $n \times n$
 \index{identity matrix}%
 \index{matrix!identity}%
\df{identity matrix}.  This matrix is denoted
 \index{identity@$I_n$, $I$ (identity matrix}%
by~$I_n$ (or just by $I$ if no confusion will result).  If $c$ is a real number, it is
conventional to denote the matrix $cI_n$ by~$c$.  It is clear that $a\,I_n = I_n\,a = a$ for
every $n \times n$ matrix~$a$.  The $m \times n$
 \index{zero!matrix}%
 \index{matrix!zero}%
\df{zero matrix} is the $m \times n$ matrix all of whose entries are~$0$.  It is denoted
 \index{zero@$\vc 0_{m \times n}$ (zero matrix)}%
by~$\vc 0_{m \times n}$ or just by $\vc 0$. Certainly $\vc 0 + a = a + \vc 0 = a$ for every $m
\times n$ matrix~$a$.
\end{defn}

\begin{defn} A square matrix $a$ in $\ofml M_{n \times n}$ is invertible if there exists an
$n \times n$ matrix $a^{-1}$ such that
  \[ aa^{-1} = a^{-1}a = I_n\,. \]
The matrix
 \index{<@$a^{-1}$ (inverse of a matrix)}%
$a^{-1}$ is the
 \index{inverse!of a matrix}%
 \index{matrix!inverse of a}%
\df{inverse} of~$a$.
\end{defn}


\begin{prop}\label{mmult_uniq_inv} An $n \times n$-matrix has at most one inverse.
\end{prop}

\begin{proof} Exercise.  (Solution~\ref{sol_mmult_uniq_inv}.)  \ns  \end{proof}

\begin{exer}\label{ex_mat_inv}  Show that the matrix $b =
    \begin{bmatrix}
        -1/2 &  1/2 &  3/2 \\
         1/4 &  1/4 & -1/4 \\
         3/4 & -1/4 & -3/4
    \end{bmatrix}$ is the inverse of the matrix $a =
             \begin{bmatrix}
                    1 &  0 &  2 \\
                    0 &  3 & -1 \\
                    1 & -1 &  1
             \end{bmatrix}.$ (Solution~\ref{sol_ex_mat_inv}.)
\end{exer}


\begin{prob}\label{prob_matr_inv}  Show that the matrix $a =
          \begin{bmatrix}
                  1 &  3 & -1 \\
                  0 &  2 &  1 \\
                  1 & -2 &  1
          \end{bmatrix}$ satisfies the equation
  \[ a^3 - 4a^2 + 8a - 9 = \vc 0. \]
Use this fact to find the inverse of~$a$.
\end{prob}











\section{DETERMINANTS}\label{dets} A careful development of the properties of the determinant
function on $n \times n$ matrices is not a central concern of this course.  In this section we
record without proof some of its elementary properties. (Proofs of these facts can be found in
almost any linear algebra text. Two elegant (if not entirely elementary) presentations can be
found in~\cite{Halmos:1958} and~\cite{HoffmanK:1971}.)

\begin{fact}\label{mat_fact1} Let $n \in \N$.  There is exactly one function
 \[ \det \colon  M_{n \times n} \sto \R \colon  a \mapsto \det a \]
which satisfies
 \begin{enumerate}
  \item[(a)] $\det I_n = 1$.
  \item[(b)] If $a \in \ofml M_{n \times n}$ and $a'$ is the matrix obtained by interchanging two
rows of $a$, then $\det a' = - \det a$.
  \item[(c)] If $a \in \ofml M_{n \times n}$, $c \in \R$, and $a'$ is the matrix obtained by
multiplying each element in one row of $a$ by $c$, then $\det a' = c \det a$.
  \item[(d)] If $a \in \ofml M_{n \times n}$, $c \in \R$, and $a'$ is the matrix obtained from $a$
by multiplying one row of $a$ by $c$ and adding it to another row of $a$ (that is, choose $i,j
\in \N_n$ with $i \ne j$ and replace $a_k^j$ by $a_k^j + ca_k^i$ for each $k$ in $N_n$), then
$\det a' = \det a$.
 \end{enumerate}
\end{fact}

\begin{defn} The unique function $\det \colon  \ofml M_{n \times n} \sto \R$ described above is
the $n \times n$
 \index{determinant@$\det a$ (determinant of a matrix)}%
\df{determinant function}.
\end{defn}

\begin{fact} If $a \in \R$ ( = $\ofml M_{1 \times 1}$), then $\det a = a$; if
$a \in \ofml M_{2 \times 2}$, then $\det a = a_1^1 a_2^2 - a_2^1 a_1^2$.
\end{fact}

\begin{fact} If $a, b \in \ofml M_{n \times n}$, then $\det(ab) = (\det a)(\det b)$.
\end{fact}

\begin{fact}\label{mat_fact4} If $a \in \ofml M_{n \times n}$, then $\det a^t = \det a$.
(An obvious corollary of this: in conditions (b), (c), and (d) of fact \ref{mat_fact1} the
word ``columns'' may be substituted for the word ``rows''.)
\end{fact}

\begin{defn} Let $a$ be an $n \times n$ matrix.  The
 \index{minor}%
\df{minor} of the element $a_k^j$, denoted by $M_k^j$, is the determinant of the $(n-1) \times
(n-1)$ matrix which results from the deletion of the j$^{\text{th}}$ row and k$^{\text{th}}$
column of~$a$.  The
 \index{cofactor}%
\df{cofactor} of the element~$a_k^j$, denoted by $C_k^j$ is defined
by
  \[ C_k^j := (-1)^{j+k}M_k^j\,. \]
\end{defn}

\begin{fact} If $a \in \ofml M_{n \times n}$ and $1 \le j \le n$, then
  \[ \det a = \sum_{k=1}^n a_k^j C_k^j\,. \]
This is the
 \index{Laplace expansion}%
 \index{expansion!of a determinant}%
\df{(Laplace) expansion} of the determinant along the j$^{\text{th}}$ row.
\end{fact}

In light of fact~\ref{mat_fact4}, it is clear that expansion along columns works as well as
expansion along rows.  That is,
  \[ \det a = \sum_{j=1}^n a_k^j C_k^j \]
for any $k$ between $1$ and~$n$.  This is the \df{(Laplace) expansion} of the determinant
along the k$^{\text{th}}$ column.

\begin{fact}\label{fact_matr_inv} An $n \times n$ matrix $a$ is invertible if and only if
$\det a \ne 0$.  If $a$ is invertible, then
 \[ a^{-1} =  (\det a)^{-1} C^t \]
where $C = \bigl[C_k^j\bigr]$ is the matrix of cofactors of elements of~$a$.
\end{fact}

\begin{exer}\label{ex_mat_inv2} Let $a =
         \begin{bmatrix}
               1  &  0  &  2 \\
               0  &  3  & -1 \\
               1  & -1  &  1
         \end{bmatrix}$. Use the preceding facts to show that $a$ is invertible and to compute
the inverse of~$a$.  (Solution~\ref{sol_ex_mat_inv2}.)

\end{exer}

\begin{prob}  Let $a$ be the matrix given in problem~\ref{prob_matr_inv}.  Use the facts stated in
section~\ref{dets} to show that $a$ is invertible and to compute~$a^{-1}$.
\end{prob}









\section{MATRIX REPRESENTATIONS OF LINEAR TRANSFORMATIONS}  We are now in a position to represent
members of $\ofml L(\R^n,\R^m)$ by means of matrices.  This will simplify computations
involving such linear transformations.

\begin{defn} If $T \in \ofml L(\R^n,\R^m)$, we define $[T]$ to be the $m \times n$ matrix whose
entry in the $j^{\text{th}}$ row and $k^{\text{th}}$ column is $(Te^k)_j$, the $j^{\text{th}}$
component of the vector $Te^k$ in $\R^m$.  That is, if $a = [T]$, then $a_k^j = (Te^k)_j$.
The matrix
 \index{<@$[T]$ (matrix representation of a linear map)}%
$[T]$ is the
 \index{matrix!representation}%
 \index{representation!matrix}%
\df{matrix representation} of $T$ (with respect to the standard bases in $\R^n$ and~$\R^m$).
\end{defn}

\begin{exam}  Let $T \colon \R^4 \sto \R^3 \colon  (w,x,y,z) \mapsto (w+2x+3y, 5w+6x+7y+8z,
-2x-3y-4z)$.  Then $T$ is linear and
 \begin{align*}
      Te^1 &= T(1,0,0,0) = (1,5,0)  \\
      Te^2 &= T(0,1,0,0) = (2,6,-2) \\
      Te^3 &= T(0,0,1,0) = (3,7,-3) \\
      Te^4 &= T(0,0,0,1) = (0,8,-4).
 \end{align*}
Having computed $Te^1,\dots, Te^4$, we use these as the successive columns of~$[T]$.  Thus
 \[ T = \begin{bmatrix}
             1 &  2 &  3 &  0 \\
             5 &  6 &  7 &  8 \\
             0 & -2 & -3 & -4
        \end{bmatrix}. \]
\end{exam}

\begin{exam} If $I \colon \R^n \sto \R^n$ is the identity map on $\R^n$, then its matrix
representation $[I]$ is just the $n \times n$ identity matrix~$I_n$.
\end{exam}

\begin{exer}\label{exer_mrlt_exam}  Let $T \colon \R^2 \sto \R^4 \colon  (x,y) \mapsto
(x - 3y, 7y, 2x + y, -4x + 5y)$.  Find $[T]$.  (Solution~\ref{sol_exer_mrlt_exam}.)

\end{exer}

The point of the representation just defined is that if we compute the action of the matrix
$[T]$ on a vector $x$ (as defined in~\ref{def_act_matr}), what we get is the value of $T$
at~$x$. Moreover, this representation is unique; that is, two distinct matrices cannot
represent the same linear map.

\begin{prop}\label{prop_mrT_is_T} If $T \in \ofml L(\R^n,\R^m)$, then for all $x$ in~$\R^n$
  \[ Tx = [T]x\,. \]
Furthermore, if $a$ is any $m \times n$ matrix which satisfies
  \[ Tx = ax  \qquad \text{for all $x \in \R^n$},  \]
then $a = [T]$.
\end{prop}

\begin{proof} Exercise. \emph{Hint.}  For simplicity of notation let $b = [T]$.  The map
$S \colon \R^n \sto \R^m \colon  x \mapsto bx$ is linear. Why?  To show that $Sx = Tx$ for all
$x$ in $\R^n$ it suffices to show that $(Se^k)_j = (Te^k)_j$ for $1 \le k \le n$ and $1 \le j
\le m$.  Why? (Solution~\ref{sol_prop_mrT_is_T}.)  \ns
\end{proof}

\begin{prop}\label{prop_matrep_bij}  Let $m$, $n \in \N$.  The map $T \mapsto [T]$ from
$\ofml L(\R^n,\R^m)$ into $\ofml M_{m \times n}$ is a bijection.
\end{prop}

\begin{proof} Exercise.  (Solution~\ref{sol_prop_matrep_bij}.)  \ns  \end{proof}

\begin{prop}\label{prop_arith_matrep}  Let $m$, $n \in \N$; let $S$, $T \in \ofml L(\R^n,\R^m)$;
and let $\alpha \in \R$.  Then
 \begin{enumerate}
  \item[(a)] $[S + T] = [S] + [T]$, and
  \item[(b)] $[\alpha T] = \alpha [T]$.
 \end{enumerate}
\end{prop}

\begin{proof} Exercise. \emph{Hint.}  For (a) use propositions \ref{prop_mrlt_bas},
\ref{prop_mrT_is_T}, and~\ref{prop_matr_arith}(c). (Solution~\ref{sol_prop_arith_matrep}.) \ns
\end{proof}

\begin{thm}  Under the operations of addition and scalar multiplication (defined in
section~\ref{matr}) $\ofml M_{m \times n}$ is a vector space and the map $T \mapsto [T]$,
which takes a linear transformation to its matrix representation, is an isomorphism between
$\ofml L(\R^n,\R^m)$ and~$\ofml M_{m \times n}$.
\end{thm}

\begin{proof} Problem. \emph{Hint.} Use problem~\ref{lt_transfer}. \ns  \end{proof}

\begin{prob} Define $T \colon \R^3 \sto \R^4$ by
  \[ T(x,y,z) = (x - 2y, x + y - 3z, y + 4z, 3x - 2y + z)\,. \]
Find $[T]$.
\end{prob}

\begin{prob} Define $T \colon \R^4 \sto \R^3$ by
  \[ T(w,x,y,z) = (w - 3x + z, 2w + x + y - 4z, w + y + z)\,. \]
 \begin{enumerate}
  \item[(a)] Find $[T]$.
  \item[(b)] Use proposition~\ref{prop_mrT_is_T} to calculate $T(4,0,-3,1)$.
 \end{enumerate}
\end{prob}

\begin{prob} Let $f \colon \R^2 \sto \R^4$ be defined by
  \[ f(x) = (x_1x_2, (x_1)^2 - 4(x_2)^2, (x_1)^3, x_1\sin(\pi x_2)) \]
for all $x = (x_1, x_2)$ in $\R^2$, and let $T \colon \R^2 \sto \R^4$ be the linear
transformation whose matrix representation is
 \[ \begin{bmatrix}
          4 &  0 \\
          2 & -1 \\
          5 & -8 \\
         -1 &  2
    \end{bmatrix}\,. \]
Find $f(a + h) - f(a) - Th$ when $a = (-2,1/2)$ and $h = (-1,1)$.
\end{prob}

\begin{prop}\label{prop_comp_mrlt} If $S \in \ofml L(\R^p,\R^n)$ and
$T \in \ofml L(\R^n,\R^m)$, then
  \[ [TS] = [T] [S]\,. \]
\end{prop}

\begin{proof} Problem.  \emph{Hint.}  Why does it suffice to show that $[TS]x = ([T][S])x$
for all $x$ in $\R^p$?  Use propositions~\ref{prop_mrT_is_T} and~\ref{prop_matr_arith}(e). \ns
\end{proof}

\begin{prob} Let
 \begin{align*}
     T \colon \R^3 \sto \R^4 & \colon  (x,y,z) \mapsto (2x+y, x-z, y+z, 3x) \\
  \intertext{and}
     S \colon \R^4 \sto \R^3 & \colon  (w,x,y,z) \mapsto (x-y, y+z, z-w).
 \end{align*}
 \begin{enumerate}
  \item[(a)] Use proposition~\ref{prop_comp_mrlt} to find~$[TS]$.
  \item[(b)] Use proposition~\ref{prop_comp_mrlt} to find~$[ST]$.
 \end{enumerate}
\end{prob}

\begin{prob}\label{prob_madd_assoc}  Show that matrix multiplication is associative; that is,
show that if $a \in \ofml M_{m \times n}$, $b \in \ofml M_{n \times p}$, and $c \in \ofml M_{p
\times r}$, then $(ab)c = a(bc)$. \emph{Hint.} Don't make a complicated and messy computation
of this by trying to prove it directly.  Use propositions~\ref{prop_assoc_comp},
\ref{prop_matrep_bij}, and~\ref{prop_comp_mrlt}.  \ns
\end{prob}

\begin{prob}  Show that $\ofml M_{n \times n}$ is a unital algebra.  \emph{Hint.}  Use the
definition~\ref{def_alg}. Notice that problem~\ref{prob_madd_assoc} establishes condition (c)
of~\ref{prob_madd_assoc}.  Verify the other conditions in a similar fashion.
\end{prob}

\begin{prop}\label{prop_matr_inv} A linear map $T \in \ofml L(\R^n,\R^n)$ is invertible if and
only if $\det[T] \ne 0$.  If $T$ is invertible, then $[T^{-1}] = [T]^{-1}$.
\end{prop}

\begin{proof} Problem. \emph{Hint.}  Show that $T$ is invertible if and only if its matrix
representation is.  Then use~\ref{fact_matr_inv}. \ns \end{proof}

\begin{prob} Let $T \colon \R^3 \sto \R^3 \colon  (x,y,z) \mapsto (x + 2z, y - z, x + y)$.
 \begin{enumerate}
  \item[(a)] Compute $[T]$ by calculating $T$ and then writing down its matrix representation.
  \item[(b)] Use proposition~\ref{prop_matr_inv} to find~$[T]$.
 \end{enumerate}
\end{prob}

\begin{prob}\label{prob_p4_vs} Let $\fml P_4$ be the family of all polynomial functions on $\R$
with degree (strictly) less than~$4$.
 \begin{enumerate}
  \item[(a)] Show that (under the usual pointwise operations) $\fml P_4$ is a vector space which
is isomorphic to~$\R^4$. \emph{Hint.} Problem~\ref{lt_transfer}.
  \item[(b)] Let $D \colon \fml P_4 \sto \fml P_4 \colon  f \mapsto f'$ (where $f'$ is the
derivative of $f$).  Using part (a) to identify the spaces $\fml P_4$ and $\R^4$, find a
matrix representation for the (obviously linear) differentiation operator~$D$.
  \item[(c)] Use your answer to part (b) to differentiate the polynomial $7x^3 - 4x^2 + 5x - 81$.
 \end{enumerate}
\end{prob}

\begin{prob} Let $\fml P_4$ be as in problem~\ref{prob_p4_vs}.  Consider the map
 \[ K \colon \fml P_4 \sto \R \colon  f \mapsto \int_0^1 f(x)\,dx\,. \]
 \begin{enumerate}
  \item[(a)] Show that $K$ is linear.
  \item[(b)] Find a way to represent $K$ as a matrix. \emph{Hint.} Use problem~\ref{prob_p4_vs}(a).
  \item[(c)] Use your answer to part (b) to integrate the polynomial
$8x^3 - 5x^2 - 4x + 6$ over the interval~$[0,1]$.
  \item[(d)] Let $D$ be as in problem~\ref{prob_p4_vs}.  Find $[KD]$ by two different techniques.
 \end{enumerate}
\end{prob}

\begin{prob} Let $T \in \ofml L(\R^4,\R^4)$ satisfy:
 \begin{align*}
       Te^1 &= (1, 2, 0, -1) \\
       Te^2 &= (1, 0, -3, 2) \\
       Te^3 &= (1, -1,-1, 1) \\
       Te^4 &= (0, 2, -1, 0).
 \end{align*}
Also let $x = (1,-2,3,-1)$ and $y = (0,1,2,1)$.  Find~$x[T]y$.
\end{prob}
