\chapter{PARTIAL DERIVATIVES AND ITERATED INTEGRALS}

In this chapter we consider questions which arise concerning a function $f$ whose domain is
the product $V_1 \times V_2$ of normed linear spaces.  What relationship (if any) exists
between the differentiability of $f$ and the differentiability of the functions $x \mapsto
f(x,b)$ and $y \mapsto f(a,y)$, where $a$ and $b$ are fixed points in $V_1$ and $V_2$,
respectively? What happens in the special case $V_1 = V_2 = \R$ if we integrate $f$ first with
respect to $y$ (that is, integrate, for arbitrary $x$, the function $y \mapsto f(x,y)$ over
the interval $[c,d]$) and then integrate with respect to $x$ (that is, integrate the function
$x \mapsto \int_c^d f(x,y)\,dy$)?  Does this produce the same result as integrating first with
respect to $x$ and then with respect to $y$?  Before we can answer these and similar questions
we must develop a fundamental tool of analysis: the \emph{mean value theorem}.




\section{THE MEAN VALUE THEOREM(S)}
Heretofore we have discussed differentiability only of functions defined on \emph{open}
subsets of normed linear spaces.  It is occasionally useful to consider differentiability of
functions defined on other types of subsets.  The business from beginning calculus of right
and left differentiability at endpoints of intervals does not extend in any very natural
fashion to functions with more complicated domains in higher dimensional spaces.  Recall that
according to definition~\ref{defn_nbhd_pt} a \emph{neighborhood} of a point in a metric space
is an open set containing that point. It will be convenient to expand slightly our use of this
word.

\begin{defn} A
 \index{neighborhood!of a set}%
\df{neighborhood} of a subset $A$ of a metric space is any open set which contains~$A$.
\end{defn}

\begin{defn} Let $V$ and $W$ be normed linear spaces and $A \subseteq V$.  A $W$ valued function
$f$ is said to be
 \index{differentiable!on a set}%
\df{differentiable on} $A$ if it is (defined and) differentiable on some neighborhood of~$A$.
The function $f$ is
 \index{continuously differentiable}%
 \index{differentiable!continuously}%
\df{continuously differentiable on} $A$ if it is differentiable on (a neighborhood of) $A$ and
if its differential $df\colon x \mapsto df_x$ is continuous at each point of~$A$.  The family
of all $W$ valued continuously differentiable functions on $A$ is denoted
 \index{c@$\fml C^1(A,W)$ (continuously differentiable functions)}%
by~$\fml C^1(A,W)$.  (Keep in mind that in order for a function $f$ to belong to $\fml
C^1(A,W)$ its domain must contain a \emph{neighborhood} of~$A$.)
\end{defn}

Given a function $f$ for which both ``differential'' and ``derivative'' make sense it is
natural to ask if there is any difference between requiring $df$ to be continuous and
requiring $Df$ to be.  It is the point of the next problem to show that there is not.

\begin{prob} Let $A$ be a subset of $\R$ with nonempty interior and $W$ be a normed linear space.
A function $f$ mapping a neighborhood of $A$ into $W$ belongs to $\fml C^1(A,W)$ if and only
if its derivative $Df$ (exists and) is continuous on some neighborhood of~$A$.
\end{prob}

Not every function that is differentiable is continuously differentiable.

\begin{exam} Let $f(x) = x^2 \sin 1/x$ for $x \ne 0$ and $f(0) = 0$.  Then $f$ is differentiable
on $\R$ but does not belong to $\fml C^1(\R,\R)$.
\end{exam}

\begin{proof} Problem.   \ns  \end{proof}

We have already encountered one version of the \emph{mean value theorem} (see
theorem~\ref{mvthm}): if $a < b$ and $f$ is a real valued function continuous on the interval
$[a,b]$ and differentiable on the interior of that interval, then
 \begin{equation}\label{eqn_mvt1}
     \frac{f(b) - f(a)}{b - a} = Df(c)
 \end{equation}
for at least one number $c$ between $a$ and~$b$.

We consider the problem of generalizing this formula to
 \index{scalar!field}%
 \index{field!vector}%
\emph{scalar fields} (that is, real valued functions of a vector
variable), to
 \index{curve}%
\emph{curves} (vector valued functions of a real variable), and to
 \index{vector!field}%
 \index{field!vector}%
\emph{vector fields} (vector valued functions of a vector variable).  There is no difficulty
in finding an entirely satisfactory variant of~\eqref{eqn_mvt1} which holds for scalar fields
whose domain lies in $\R^n$.  This is done in chapter~\ref{comp_Rn} once we have the notion of
\emph{gradient} (see proposition~\ref{prop_mvt_sf}).  On the other hand we show in
exercise~\ref{exer_no_mvt} that for curves (\emph{a fortiori}, vector fields)
formula~\eqref{eqn_mvt1} does \emph{not} hold. Nevertheless, the most useful aspect of the
\emph{mean value theorem}, that changes in $f$ over the interval $[a,b]$ cannot exceed the
maximum value of $\abs{Df}$ multiplied by the length of the interval does have a direct
generalization to vector fields (see proposition~\ref{thm_mvt_curv}).  A somewhat different
generalization can be produced by considering the version of the \emph{fundamental theorem of
calculus} most used in beginning calculus: if $f$ is a function whose derivative exists and is
continuous on $[a,b]$, then
 \begin{equation}\label{eqn_mvt2}
     f(b) - f(a) = \int_a^bDft\,dt.
 \end{equation}
(We have yet to prove this result.  In fact we need the \emph{mean value theorem} to do so.
See theorem~\ref{thm_ftc_II}.)  It is conventional to define $(b - a)^{-1}\int_a^bg$ to be the
 \index{mean!value}%
 \index{value!mean}%
\df{mean value} (or
 \index{average value}%
 \index{value!average}%
\df{average value}) of a function $g$ over the interval $[a,b]$. Thus~\eqref{eqn_mvt2} may be
regarded as a ``mean value theorem'' saying that the Newton quotient $(b-a)^{-1}(f(b)-f(a))$
is just the mean value of the \emph{derivative} of~$f$.  Since functions between Banach spaces
do not in general have ``derivatives'' it is better for purposes of generalization to rewrite
\eqref{eqn_mvt2} in terms of differentials.  For a curve $f$ with continuous derivative
  \[ \int_a^bDf(t)\,dt = \int_a^bdf_t(1)\,dt
                       \quad\text{(by~\ref{prop_diff_der})} \]
so~\eqref{eqn_mvt2} becomes
 \begin{equation}\label{eqn_mvt3}
          f(b) - f(a) = \int_a^bdf_t(1)\,dt.
 \end{equation}
If we divide both sides by $b - a$ this result says, briefly, that the Newton quotient of $f$
is the mean value of the \emph{differential} of $f$ over~$[a,b]$.

There is one more thing to consider in seeking to generalize~\eqref{eqn_mvt1}.  In
chapter~\ref{C_int} we defined the integral only for vector valued functions of a \emph{real}
variable.  So if $a$ and $b$ are points in a general Banach space, the expression $\int_a^b
df_t(1)\,dt$ in~\eqref{eqn_mvt3} is meaningless (and so, of course, is the Newton quotient
$(f(b)-f(a))/(b-a)$).  This is easily dealt with.  In order to integrate over a closed segment
$[a,b]$ in a Banach space, parametrize it: let $l(t) = (1 - t)a + tb$ for $0 \leq t \leq 1$
and let $g = f\circ l$. Then
 \begin{equation}\label{eqn_mvt4}
     \begin{aligned}
                dg_t(1) &= df_{l(t)}(dl_t(1)) \\
                        &= df_{l(t)}(b - a)\,.
     \end{aligned}
 \end{equation}
Apply~\eqref{eqn_mvt3} to the function $g$ over the interval $[0,1]$ to obtain
 \begin{equation}\label{eqn_mvt5}
                g(1) - g(0) = \int_0^1dg_t(1)\,dt\,.
 \end{equation}
Substituting \eqref{eqn_mvt4} in \eqref{eqn_mvt5} leads to
 \begin{equation}\label{eqn_mvt6}
     \begin{aligned}
           f(b) - f(a) &= g(1) - g(0)  \\
                       &= \int_0^1df_{l(t)}(b - a)\,dt.
     \end{aligned}
 \end{equation}
Notice that if we let $h = b - a$, then~\eqref{eqn_mvt6} may be
written
 \begin{equation}\label{eqn_mvt7}
     \begin{aligned}
          \Delta f_a(h) &= \int_0^1df_{l(t)}(h)\,dt\\
                        &= \left(\int_0^1df_{l(t)}\,dt\right)(h)
     \end{aligned}
 \end{equation}
by corollary~\ref{cor_1param_fam}.  It is in this form (either~\eqref{eqn_mvt6}
or~\eqref{eqn_mvt7}) that the \emph{mean value theorem} holds for a function $f$ between
Banach spaces which is continuously differentiable on a segment~$[a,b]$. It is worth noticing
that this generalization is weaker in three respects than the classical \emph{mean value
theorem}: we are not able to conclude that there exists a particular point where the
differential of $f$ is equal to $f(b)-f(a)$; we assume differentiability at $a$ and $b$; and
we assume continuity of the differential.  Nonetheless, this will be adequate for our
purposes.

 \index{mean value theorem}%
\begin{exer}\label{exer_no_mvt}  Show that the classical \emph{mean value theorem}
(theorem~\ref{mvthm}) fails for vector valued functions.  That is, find an interval $[a,b]$, a
Banach space $E$, and a continuous function $f\colon [a,b] \sto E$ differentiable on $(a,b)$
such that the equation $(b - a)Df(c) = f(b) - f(a)$ holds for \emph{no} point $c$ in~$(a,b)$.
\emph{Hint.} Consider a parametrization of the unit circle. (Solution~\ref{sol_exer_no_mvt}.)
\end{exer}

Here is our first generalization of the \emph{mean value theorem}.  Others will occur
in~\ref{prop_mvt_nls}, \ref{cor_mvt_nls}, and~\ref{prop_mvt_Bs}.

 \index{mean value theorem!for curves}%
\begin{thm}[Mean Value Theorem for Curves]\label{thm_mvt_curv}  Let $a < b$ and $W$ be a normed
linear space.  If a continuous function $f \colon [a,b] \sto W$ has a derivative at each point
of $(a,b)$ and if there is a constant $M$ such that $\norm{Df(t)} \le M$ for all $t \in
(a,b)$, then
  \[ \norm{f(b) - f(a)} \le M(b - a)\,. \]
\end{thm}

\begin{proof} Exercise.  \emph{Hint.}  Given $\epsilon > 0$ define $h(t) =
\norm{f(t) - f(a)} - (t - a)(M + \epsilon)$ for $a \le t \le b$.  Let $A =
h^{\gets}(-\infty,\epsilon]$.  Show that:
 \begin{enumerate}
  \item[(i)] $A$ has a least upper bound, say $l$;
  \item[(ii)] $l > a$;
  \item[(iii)] $l \in A$; and
  \item[(iv)] $l = b$.
 \end{enumerate}
To prove (iv) argue by contradiction. Assume $l < b$.  Show that $\norm{(t - l)^{-1}(f(t) -
f(l))} < M + \epsilon$ for $t$ sufficiently close to and greater than~$l$.  For such $t$ show
that $t \in A$ by considering the expression
  \[ \norm{f(t)-f(l)} + \norm{f(l)-f(a)} - (t - l)(M + \epsilon)\, . \]
Finally, show that the desired conclusion follows from (iii) and~(iv).
(Solution~\ref{sol_thm_mvt_curv}.)   \ns
\end{proof}

Next we extend the \emph{mean value theorem} from curves to vector fields.

 \index{mean value theorem!for vector fields}%
\begin{prop}\label{prop_mvt_nls} Let $V$ and $W$ be normed linear spaces and $a$ and $h$ be points
in $V$.  If a $W$ valued function $f$ is continuously differentiable on the segment $[a,a +
h]$ , then
  \[ \norm{\Delta f_a(h)} \le M\norm h\,. \]
whenever $M$ is a number such that $\norm{df_z} \le M$ for all $z$ in~$[a,a + h]$.
\end{prop}


\begin{proof} Problem.   \emph{Hint.}   Use $l\colon t \mapsto a + th$ (where $0 \le t \le 1$)
to parametrize the segment~$[a,a + h]$. Apply~\ref{thm_mvt_curv} to the function $g = f \circ
l$.  \ns
\end{proof}

 \index{mean value theorem}%
\begin{cor}\label{cor_mvt_nls}  Let $V$ and $W$ be normed linear
spaces, $a$ and $h$ be points of $V$, the operator $T$ belong to $\ofml B(V,W)$, and $g$ be a
$W$ valued function continuously differentiable on the segment~$[a,a+h]$.  If $M$ is any
number such that $\norm{dg_z - T} \le M$ for all $z$ in $[a,a + h]$, then
  \[ \norm{\Delta g_a(h) - Th} \le M\norm h\,. \]
\end{cor}

\begin{proof} Problem.  \emph{Hint.}  Apply~\ref{prop_mvt_nls} to the function $g - T$.    \ns
\end{proof}

The next proposition is an important application of the \emph{mean value theorem}.

\begin{prop}\label{prop_sc_const} Let $V$ and $W$ be normed linear spaces, $U$ be a nonempty
connected open subset of $V$, and $f\colon U \sto W$ be differentiable.  If $df_x = \vc 0$ for
every $x \in U$, then $f$ is constant.
\end{prop}

\begin{proof} Problem.   \emph{Hint.}  Choose $a \in U$ and set $G = f^\gets \{f(a)\}$.  Show
that $G$ is both an open and closed subset of~$U$.  Then use~\ref{prop_conn_oc} to conclude
that $U=G$.

To prove that $G$ is open in $U$, take an arbitrary point $y$ in $G$, find an open ball $B$
about $y$ which is contained in $U$, and use~\ref{prop_mvt_nls} to show that $\Delta f_y(w -
y) = 0$ for every $w$ in~$B$.   \ns
\end{proof}

\begin{cor}\label{cor_const_diff}  Let $V$, $W$, and $U$ be as in the preceding proposition.
If $f$, $g \colon U \sto W$ are differentiable on $U$ and have the same differentials at each
point of $U$, then $f$ and $g$ differ by a constant.
\end{cor}

\begin{proof} Problem.   \ns  \end{proof}

\begin{prob}  Show that the hypothesis that $U$ be connected cannot be deleted in
proposition~\ref{prop_sc_const}.
\end{prob}

Corollary~\ref{cor_const_diff} makes possible a proof of the version of the \emph{fundamental
theorem of calculus} which is the basis for the procedures of ``formal integration'' taught in
beginning calculus.

\begin{defn}  A differentiable curve $f$ in a Banach space whose domain is an open interval in
$\R$ is an
 \index{antiderivative}%
\df{antiderivative} of a function $g$ if $Df = g$.
\end{defn}

 \index{fundamental theorem of calculus}%
\begin{thm}[Fundamental Theorem of Calculus - Version II]
\label{thm_ftc_II}  Let $a$ and $b$ be points in an open interval $J  \subseteq \R$ with $a <
b$.  If $g\colon J \sto E$ is a continuous map into a Banach space and $f$ is an
antiderivative of $g$ on $J$, then
  \[ \int_a^b g = f(b) - f(a)\,. \]
\end{thm}

\begin{proof} Problem.   \emph{Hint.}  Let $h(x) = \int_a^xg$ for $x$ in~$J$.  Use
corollary~\ref{cor_const_diff} to show that $h$ and $f$ differ by a constant.  Find the value
of this constant by setting $x = a$.   \ns
\end{proof}

With this second version of the \emph{fundamental theorem of calculus} in hand, we are in a
position to prove the version of the \emph{mean value theorem} discussed in the introduction
to this section.

 \index{mean value theorem!for vector fields}%
\begin{prop}\label{prop_mvt_Bs}  Suppose that $E$ and $F$ are Banach spaces, and $a$, $h \in E$.
If an $F$ valued function $f$ is continuously differentiable on the segment $[a,a + h]$, then
 \begin{equation}\label{eqn_mvt_Bs}
    \Delta f_a(h) = \left( \int_0^1 df_{l(t)}\,dt\right)(h)
 \end{equation}
where $l(t) = a + th$  for $0 \leq t \leq 1$.
\end{prop}

\begin{proof} Problem.   \emph{Hint.}  Let $g = f\circ l$.  Show that $Dg(t)=df_{l(t)}(h)$.
Apply~\ref{cor_1param_fam} to the right side of equation~\eqref{eqn_mvt_Bs}.
Use~\ref{thm_ftc_II}.  \ns
\end{proof}


It is conventional when we invoke any of the results \ref{thm_mvt_curv}, \ref{prop_mvt_nls},
\ref{cor_mvt_nls}, or~\ref{prop_mvt_Bs} to say that we have used ``the''
 \index{mean value theorem}%
\emph{mean value theorem}.

\begin{prob} Verify proposition~\ref{prop_mvt_Bs} directly for the function $f(x,y) = x^2 + 6xy
- 2y^2$ by computing both sides of equation~\eqref{eqn_mvt_Bs}.
\end{prob}

\begin{prob}  Let $W$ be a normed linear space and $a < b$. For each $f$ in $\fml C^1([a,b],W)$
define
  \[ \trinorm f = \sup\{\norm{f(x)}
                   + \norm{Df(x)}\colon  a \leq x \leq b\}\,. \]
 \begin{enumerate}
  \item[(a)] Show that $\fml C^1([a,b],W)$ is a vector space and that the map $f \mapsto \trinorm f$
is a norm on this space.
  \item[(b)] Let $j(x) = \sqrt{1 + x^2}$ for all $x$ in~$\R$.  Use the \emph{mean value theorem}
to show that $\abs{\Delta j_x(y)} \leq \abs y$ for all $x$, $y \in \R$.
  \item[(c)]  Let $k$ be a continuous real valued function on the interval~$[a,b]$.  For each $f$
in $\fml C^1([a,b],\R)$ define
     \[ \Phi (f) = \int_a^bk(x)\sqrt {1 + (f'(x))^2}\,dx\,. \]
Show that the function $\Phi \colon  \fml C^1([a,b],\R) \sto \R$ is uniformly continuous.
 \end{enumerate}
\end{prob}

 \index{change of variables}%
\begin{prop}[Change of Variables]\label{prop_changvar}  If $f \in \fml C(J_2,E)$ and $g \in
\fml C^1(J_1,J_2)$ where $J_1$ and $J_2$ are open intervals in $\R$ and $E$ is a Banach space,
and if $a$, $b \in J_1$ with $a < b$, then
  \[ \int_a^b g'(f \circ g) = \int_{g(a)}^{g(b)}f\,. \]
\end{prop}

\begin{proof} Problem.  \emph{Hint.}  Use~\ref{thm_ftc_II}.  Let $F(x) = \int_{g(a)}^x f$ for all
$x$ in $J_2$.  Compute $(F \circ g)'(t)$.  \ns
\end{proof}

 \index{integration by parts}%
\begin{prop}[Integration by Parts] If $f \in \fml C^1(J,\R)$ and $g \in \fml C^1(J,E)$ where
$J$ is an open interval in $\R$ and $E$ is a Banach space, and if $a$ and $b$ are points in
$J$ with $a < b$, then
  \[ \int_a^b fg' = f(b)g(b) - f(a)g(a) - \int_a^b f'g\,. \]
\end{prop}

\begin{proof} Problem.   \emph{Hint.}  Differentiate the function $t \mapsto f(t)g(t)$.   \ns
\end{proof}

\begin{prop}  If $f \in \fml C(J,\R)$ where $J$ is an interval, and if $a$, $b \in J$ with
$a < b$, then there exists $c \in (a,b)$ such that
  \[ \int_a^b f = f(c)(b - a)\,. \]
\end{prop}

\begin{proof} Problem.  \ns  \end{proof}

\begin{prob} Show by example that the preceding proposition is no longer true if it is assumed
that $f$ is a continuous function from $J$ into~$\R^2$.
\end{prob}








\section{PARTIAL DERIVATIVES} Suppose that $V_1, V_2, \dots, V_n$, and $W$ are normed linear spaces.
Let $V = V_1\times \dots \times V_n$.  In the following discussion we will need the
 \index{canonical injection maps}%
 \index{injection maps!canonical}%
 \index{<@$j_{{}_{\sst k}}$ ($k^{\text{th}}$ canonical injection)}%
 \index{j@$j_{{}_{\sst k}}$ ($k^{\text{th}}$ canonical injection)}%
\df{canonical injection maps} $j_{{}_{\sst 1}}, \dots, j_{{}_{\sst n}}$ which map the
coordinate spaces $V_1, \dots, V_n $ into the product space~$V$.  For $1\le k \le n $ the map
$j_{{}_{\sst k}}\colon V_k \sto V$ is defined by $j_{{}_{\sst k}}(x) = (0, \dots, 0, x, 0,
\dots, 0)$ (where $x$ appears in the $k^{\text {th}}$ coordinate). It is an easy exercise to
verify that each $j_{{}_{\sst k}}$ is a bounded linear transformation and that if $V_k$ does
not consist solely of the zero vector then $\norm{j_{{}_{\sst k}}} = 1$.  Also worth noting is
the relationship given in the following exercise between the injections $j_{{}_{\sst k}}$ and
the projections~$\pi_k$.

\begin{exer}\label{exer_inj_proj}  Let $V_1, \dots, V_n$ be normed linear spaces and $V =
V_1 \times \dots \times V_n$.  Then
 \begin{enumerate}
  \item[(a)] For $k = 1, \dots, n$ the injection $j_{{}_{\sst k}}$ is a right inverse of the
projection~$\pi_k$.
  \item[(b)] $\sum\limits_{k=1}^n (j_{{}_{\sst k}} \circ \pi_k) = I_V$.
 \end{enumerate}
(Solution~\ref{sol_exer_inj_proj}.)
\end{exer}

\begin{prob}\label{prob_proj_ball} Let $V = V_1 \times \dots \times V_n$ where $V_1, \dots, V_n$
are normed linear spaces.  Also let $a \in V$, $r > 0$, and $1 \le k \le n$.  Then the image
under the projection map $\pi_k$ of the open ball in $V$ about $a$ of radius $r$ is the open
ball in $V_k$ about $a_k$ of radius $r$.
\end{prob}

\begin{defn} A mapping $f \colon M_1 \sto M_2$ between metric spaces is
 \index{open!mapping}%
\df{open}
 \index{closed!mapping}%
[resp., \df{closed}] if $f^\sto(U)$ is an open [resp., closed] subset of $M_2$ whenever
$U$ is an open [resp., closed] subset of~$M_1$.
\end{defn}

\begin{prob} Show that each projection mapping $\pi_k\colon  V_1 \times \dots \times V_n \sto V_k$
on the product of normed linear spaces is an open mapping.  Construct an example to show that
projection mappings need not be closed.
\end{prob}

Now suppose that $f$ belongs to $\fml F_a(V,W)$ and that $1 \le k \le n$.  Let $B$ be an open
ball about $a$ which is contained in the domain of $f$ and $B_k = \left(\pi_k^{\sto}(B)\right)
- a_k$. From problem~\ref{prob_proj_ball} and the fact that translation by $a_k$ is an
isometry (see problem~\ref{prob_transl}) we see that $B_k$ is an open ball in $V_k$ about the
origin (whose radius is the same as the radius of $B$).  Define
  \[ g \colon B_k \sto W \colon x \mapsto f(a + j_{{}_{\sst k}}(x))\,. \]
Notice that as $x$ changes, only the $k^{\text {th}}$ variable of the domain of $f$ is
changing; the other $k - 1$ variables are fixed. Also notice that we can write $g = f \circ
T_a \circ j_{{}_{\sst k}}$, where $T_a$ is translation by $a$ (that is, $T_a \colon x \mapsto
x + a$).

\begin{exer}\label{exer_Del_inj}  With notation as above show that
  \[ \Delta g_{\vc 0} = \Delta f_a \circ j_{{}_{\sst k}}\,. \]
in some neighborhood of the origin in~$V_k$.  (Solution~\ref{sol_exer_Del_inj}.)
\end{exer}


\begin{prop}   Let $f$, $g$, and $a$ be as above.  If $f$ is differentiable at $a$, then $g$ is
differentiable at $\vc 0$ and
  \[ dg_{\vc 0} = df_a \circ j_{{}_{\sst k}}\,. \]
\end{prop}

\begin{proof} Problem.   \emph{Hint.}  Use exercise~\ref{exer_Del_inj} and
proposition~\ref{prop_tan_comp_O}.
\ns  \end{proof}

\begin{notn}  Suppose that the function $g\colon  B_k \sto W\colon x \mapsto f(a +
j_{{}_{\sst k}}(x))$ is differentiable at $\vc 0$.  Since $g$ depends only on the function
$f$, the point $a$, and the index $k$, it is desirable to have a notation for $dg_{\vc 0}$
which does not require the use of the extraneous letter ``$g$''.  A fairly common convention
is to write $d_kf_a$ for $dg_{\vc 0}$; this bounded linear map is the $k^{\text {th}}$
 \index{partial!differential}%
 \index{differential!partial}%
 \index{<@$d_kf_a$ ($k^{\text{th}}$ partial differential)}%
 \index{d@$d_kf_a$ ($k^{\text{th}}$ partial differential)}%
\df{partial differential} of $f$ at~$a$.  Thus $d_kf_a$ is the unique bounded linear map which
is tangent to $\Delta f_a \circ j_{{}_{\sst k}}$.  We restate the preceding proposition using
this notation.
\end{notn}

\begin{cor}\label{cor_dffntl_inj}  Let $V_1, V_2, \dots ,V_n$, and $W$ be normed linear spaces.
If the function $f$ belongs to $ \fml D_a(V_1 \times \dots \times V_n, W)$, then for each $k =
1,\dots ,n$ the $k^{\text{th}}$ partial differential of $f$ at $a$ exists and
  \[ d_kf_a = df_a \circ j_{{}_{\sst k}}\;. \]
\end{cor}

\begin{cor}   Let $V_1, V_2, \dots ,V_n$, and $W$ be normed linear spaces.  If the function $f$
belongs to $ \fml D_a(V_1 \times \dots \times V_n, W)$, then
  \[ df_a(x) = \sum\limits_{k=1}^n d_kf_a(x_k) \]
for each $x$ in $V_1\times \dots\times V_n$.
\end{cor}


\begin{proof} Problem.    \emph{Hint.}  Write $df_a$ as $df_a \circ I$ (where $I$ is the identity
map on $V_1 \times \dots \times V_n$) and use exercise~\ref{exer_inj_proj}(b).    \ns
\end{proof}

The preceding corollaries assure us that if $f$ is differentiable at a point, then it has
partial differentials at that point.  The converse is not true.  (Consider the function
defined on $\R^2$ whose value is $1$ everywhere except on the coordinate axes, where its value
is $0$.  This function has partial differentials at the origin, but is certainly not
differentiable---or even continuous---there.)  The following proposition shows that if we
assume continuity (as well as the existence) of the partial differentials of a function in an
open set, then the function is differentiable (in fact, continuously differentiable) on that
set. To avoid complicated notation we prove this only for the product of two spaces.

\begin{prop}\label{prop_par_dffntl}  Let $V_1$, $V_2$, and $W$ be normed linear spaces, and let
$f\colon U \sto W$ where $\open U{V_1 \times V_2}$.  If the partial differentials $d_1f$ and
$d_2f$ exist and are continuous on $U$, then $f$ is continuously differentiable on $U$ and
 \begin{equation}\label{eqn_par_dffntl1}
   df_{(a,b)} = d_1f_{(a,b)} \circ \pi_1 + d_2f_{(a,b)} \circ \pi_2
 \end{equation}
at each point $(a,b)$ in~$U$.
\end{prop}

\begin{proof}  Exercise.  \emph{Hint.}   To show that $f$ is differentiable at a point $(a,b)$
in $U$ and that its differential there is the expression on the right side
of~\eqref{eqn_par_dffntl1}, we must establish that $\Delta f_{(a,b)}$ is tangent to $R = S
\circ \pi_1 + T \circ \pi_2$ where $S = d_1f_{(a,b)}$ and $T = d_2f_{(a,b)}$.  Let $g \colon t
\mapsto f(a,b + t)$ for all $t$ such that $(a,b + t) \in U$ and $h^t \colon s \mapsto f(a + s,
b + t)$ for all $s$ and $t$ such that $(a + s, b + t) \in U$. Show that $\Delta f_{(a,b)}
(s,t)$ is the sum of $\Delta \left(h^t \right)_{\vc 0}(s)$ and $\Delta g_{\vc 0}(t)$. Conclude
from this that it suffices to show that: given $\epsilon > 0$ there exists $\delta > 0$ such
that if $\norm{(u,v)}_1 < \delta$ , then
 \begin{equation}\label{eqn_par_dffntl2}
   \norm{\Delta\left(h^v\right)_{\vc 0}(u) - Su} \leq \epsilon\,\norm u
 \end{equation}
and
 \begin{equation}\label{eqn_par_dffntl3}
    \norm{\Delta g_{\vc 0}(v) - Tv} \leq \epsilon\,\norm v\,.
 \end{equation}
Find $\delta_1 > 0$ so that~\eqref{eqn_par_dffntl3} holds whenever $\norm v < \delta_1$.
(This is easy.) Then find $\delta_2 > 0$ so that~\eqref{eqn_par_dffntl2} holds whenever
$\norm{(u,v)}_1 < \delta_2$.  This requires a little thought.  Notice
that~\eqref{eqn_par_dffntl2} follows from the \emph{mean value theorem}
(corollary~\ref{cor_mvt_nls}) provided that we can verify that
  \[ \norm{d\left(h^v\right)_z - S} \leq \epsilon \]
for all $z$ in the segment~$[\vc 0,u]$.  To this end show that $d\left(h^v\right)_z = d_1f_{(a
+ z, b + v)}$ and use the fact that $d_1f$ is assumed to be continuous.
(Solution~\ref{sol_prop_par_dffntl}.)    \ns
\end{proof}

\begin{notn}  Up to this point we have had no occasion to consider the problem of the various ways
in which a Euclidean space $\R^n$ may be regarded as a product.  Take $\R^5$ for example.  If
nothing to the contrary is specified it is natural to think of $\R^5$ as being the product of
5 copies of $\R$; that is, points of $\R^5$ are 5-tuples $x = (x_1, \dots, x_5)$ of real
numbers.  If, however we wish to regard $\R^5$ as the product of $\R^2$ and $\R^3$, then a
point in $\R^5$ is an ordered pair $(x,y)$ with $x \in \R^2$ and $y \in \R^3$.  One good way
of informing a reader that you wish $\R^5$ considered as a product in this particular fashion
is to write $\R^2 \times \R^3$; another is to
 \index{<@$\R^{m+n}$ ($\R^m \times \R^n$)}%
 \index{realnumbersss@$\R^{m+n}$ ($\R^m \times \R^n$)}%
write~$\R^{2+3}$.  (Note the distinction between $\R^{2+3}$  and $\R^{3+2}$.)  In many
concrete problems the names of variables are given in the statement of the problem: for
example, suppose we encounter several equations involving the variables $u$, $v$, $w$, $x$,
and $y$ and wish to solve for the last three variables in terms of the first two.  We are then
thinking of $\R^5$ as the product of $\R^2$ (the space of independent variables) and $\R^3$
(the space of dependent variables).  This particular factorization may be emphasized by
writing a point $(u,v,w,x,y)$ of the product as $((u,v),(w,x,y))$.  And if you wish to avoid
such an abundance of parentheses you may choose to write $(u,v;w,x,y)$ instead. Ordinarily
when $\R^n$ appears, it is clear from context what factorization (if any) is intended.  The
preceding notational devices are merely reminders designed to ease the burden on the reader.
In the next exercise a function $f$ of 4 variables is given and you are asked to compute
$d_1f_a$ in  several different circumstances; it is important to realize that the value of
$d_1f_a$ \emph{depends on the factorization of}~$\R^4$ \emph{that is assumed}.
\end{notn}

\begin{exer}\label{exer_pd_R4}   Let
 \begin{equation}\label{eqn_pd_R4}
        f(t,u,v,w) = tu^2 + 3\,vw
 \end{equation}
for all $t$, $u$, $v$, $w \in \R$, and let $a = (1,1,2,-1)$.  Find $d_1f_a$ assuming that the
domain of $f$ is:
 \begin{enumerate}
  \item[(a)] $\R^4$.
  \item[(b)] $\R \times \R^3$.
  \item[(c)] $\R^2 \times \R^2$.
  \item[(d)] $\R^2 \times \R \times \R$.
  \item[(e)] $\R^3 \times \R$.
 \end{enumerate}
\emph{Hint.}  First compute $df_a$.  Then use~\ref{cor_dffntl_inj}.  \textbf{Note.}  The
default case is (a); that is, if we were given only equation~\eqref{eqn_pd_R4} we would assume
that the domain of $f$ is $\R \times \R \times \R \times \R$.  In this case it is possible to
compute $d_kf_a$ for $k = 1,2,3, 4$.  In cases (b), (c), and (e) only $d_1f_a$ and $d_2f_a$
make sense; and in (d) we can compute $d_kf_a$ for $k = 1$, $2$, $3$.
(Solution~\ref{sol_exer_pd_R4}.)
\end{exer}

\begin{prob}  Let
  \[ f(t,u,v,w) = tuv - 4\,u^2w \]
for all $t$, $u$, $v$, $w \in \R$ and let $a = (1,2,-1,3)$.  Compute $d_kf_a$ for all $k$ for
which this expression makes sense, assuming that the domain of $f$ is:
 \begin{enumerate}
  \item[(a)] $\R^4$.
  \item[(b)] $\R \times \R^3$.
  \item[(c)] $\R^2 \times \R^2$.
  \item[(d)] $\R \times \R^2 \times \R$.
  \item[(e)] $\R^3 \times \R$.
 \end{enumerate}
\end{prob}

\begin{defn}  We now consider the special case of a function $f$ which maps an open subset of
$\R^n$ into a normed linear space. For $1 \le k \le n$ the injection $j_{{}_{\sst k}}\colon \R
\sto \R^n$ takes the real number $t$ to the vector $t\,e^k$.  Thus the function $g \equiv f
\circ T_a \circ j_{{}_{\sst k}}$ is just $f \circ l$ where as usual $l$ is the parametrized
line through $a$ in the direction of $e^k$.  [Proof: $g(t) = f\left(a + j_{{}_{\sst
k}}(t)\right) = f(a + t\,e^k) = (f \circ l)(t)$.]  We define
 \index{<@$f_k$ ($k^{\text{th}}$ partial derivative of~$f$)}%
 \index{<@$\pd f{x_k}$ ($k^{\text{th}}$ partial derivative of~$f$)}%
 \index{<@$D_kf$ ($k^{\text{th}}$ partial derivative of~$f$)}%
$f_k(a)$ (or $\pd f{x_k}(a)$, or $D_kf(a))$, the $k^{\text {th}}$
 \index{partial!derivative}%
 \index{derivative!partial}%
\df{partial derivative} of $f$ at $a$, to be $d_kf_a(1)$.  Using
propositions~\ref{prop_diff_der} and~\ref{prop_diff_dd} we see that
  \begin{align*}
            f_k(a) &= d_kf_a(1)        \\
                   &= dg_0(1)          \\
                   &= Dg(0)            \\
                   &= D(f \circ l)(0)  \\
                   &= D_{e^k}f(a).
  \end{align*}
That is, the $k^{\text {th}}$ partial derivative of $f$ is the directional derivative of $f$
in the direction of the $k^{\text {th}}$ coordinate axis of~$\R^n$.  Thus the notation $D_kf$
for the $k^{\text {th}}$ partial derivative can be regarded as a slight abbreviation of the
usual notation $D_{e^k}f$ used for directional derivatives of functions on~$\R^n$.
\end{defn}

It is also useful to note that
 \begin{equation}\label{eqn_def_pd}
   \begin{aligned}
        f_k(a) &= Dg(0)        \\
               &= \lim_{t \sto 0} \frac {g(t) - g(0)}t  \\
               &= \lim_{t \sto 0} \frac {f(a + t\,e^k) - f(a)}t.
   \end{aligned}
 \end{equation}
This is the usual definition given for partial derivatives in beginning calculus.  The
mechanics of computing partial derivatives is familiar and is justified by~\eqref{eqn_def_pd}:
pretend that a function of $n$ variables is a function only of its $k^{\text {th}}$ variable,
then take the ordinary derivative.

One more observation: if $f$ is differentiable at $a$ in $\R^n$, then by
proposition~\ref{prop_dd_diff}
  \[ f_k(a) = D_{e^k}f(a) = df_a(e^k)\,. \]

Let $f \in \fml D_a(\R^n,W)$ where $W = W_1 \times \dots \times W_m$ is the product of $m$
normed linear spaces.  From proposition~\ref{prop_diff_coord} (and induction) we know that
  \[ df_a = \left(d(f^1)_a, \dots ,d(f^m)_a\right)\,. \]
From this it is an easy step to the next proposition.

\begin{prop}\label{prop_pd_vvf}  If $f \in \fml D_a(\R^n,W)$ where $W = W_1 \times \dots \times
W_m$ is the product of $m$ normed linear spaces, then
  \[ \left(f^j\right)_k = \left(f_k\right)^j \]
for $1 \le j \le m$ and $1 \le k \le n$.
\end{prop}

\begin{proof} Problem.   \ns  \end{proof}

The point of the preceding proposition is that no ambiguity arises if we write the
 \index{<@$f^j_k$ ($k^{\text{th}}$ partial derivative of
$j^{\text{th}}$ component)}%
expression~$f^j_k$.  It may correctly be interpreted either as the $k^{\text {th}}$ partial
derivative of the $j^{\text {th}}$ component function $f^j$ or as the $j^{\text {th}}$
component of the $k^{\text {th}}$ partial derivative $f_k$.

\begin{exam}  Let $f(x,y) = (x^5y^2,x^3 - y^3)$.  To find $f_1(x,y)$ hold $y$ fixed and
differentiate with respect to $x$ using corollary~\ref{cor_diff_prodcurv}.  Then
  \[ f_1(x,y) = (5\,x^4y^2,3\,x^2)\,. \]
Similarly,
  \[ f_2(x,y) = (2\,x^5y,-3\,y^2)\,. \]
\end{exam}

\begin{exer}\label{exer_pd_R3} Let $f(x,y,z) = (x^3y^2\sin z, x^2 + y\cos z)$ and $a =
(1,-2,\frac\pi2)$.  Find $f_1(a)$, $f_2(a)$, and $f_3(a)$. (Solution~\ref{sol_exer_pd_R3}.)
\end{exer}

\begin{prob} Let $f(w,x,y,z) = (wxy^2z^3, w^2 + x^2 + y^2, wx + xy + yz)$ and $a = (-3,1,-2,1)$.
Find $f_k(a)$ for all appropriate~$k$.
\end{prob}

\begin{prob}\label{prob_dk}  Let $V_1,\dots,V_n,W$ be normed linear spaces, $\open UV = V_1 \times
\dots \times V_n$, $\alpha \in \R$, and $f$, $g\colon U \sto W$. If the $k^{\text {th}}$
partial derivatives of $f$ and $g$ exist at a point $a$ in $U$, then so do the $k^{\text
{th}}$ partial differentials of $f+g$ and $\alpha f$, and
 \begin{enumerate}
  \item[(a)] $d_k(f + g)_a = d_kf_a + d_kg_a$;
  \item[(b)] $d_k(\alpha f)_a = \alpha d_kf_a$;
  \item[(c)] $(f + g)_k(a) = f_k(a) + g_k(a)$; and
  \item[(d)] $(\alpha f)_k(a) = \alpha f_k(a)$.
 \end{enumerate}
\emph{Hint.}  In (a), consider the function $(f + g) \circ T_a \circ j_{{}_{\sst k}}$.
\end{prob}

\begin{prob} Let $f$, $g \in \fml F_a(V,W)$, $\alpha \in \R$, and $\vc 0 \ne v \in V$.  Suppose
that $D_vf(a)$ and $D_vg(a)$ exist.
 \begin{enumerate}
  \item[(a)] Show that $D_v(f + g)(a)$ exists and is the sum of $D_vf(a)$ and $D_vg(a)$.
\emph{Hint.}  Use the definition of directional derivative and proposition~\ref{prop_del_sum}.
We can \emph{not} use either~\ref{prop_diff_dd} or~\ref{prop_dd_diff} here because we are
\emph{not} assuming that $f$ and $g$ are differentiable at~$a$.
  \item[(b)] Show that $D_v(\alpha f)(a)$ exists and is equal to $\alpha D_vf(a)$.
  \item[(c)] Use (a) and (b) to prove parts (c) and (d) of problem~\ref{prob_dk}.
 \end{enumerate}
\end{prob}








\section{ITERATED INTEGRALS}
In the preceding section we considered partial differentiation of functions defined on the
product of two or more spaces.  In this section we consider the integration of such functions.
To take the partial derivative of a function $f\colon (x,y) \mapsto f(x,y)$ with respect to
$x$ we hold $y$ fixed and differentiate the function $x \mapsto f(x,y)$.  Partial integration
works in very much the same way.  If $f$ is a continuous function mapping a rectangular subset
$[a,b]\times [c,d]$ of $\R^2$ into a Banach space $E$, we may, for each fixed $y$ in $[c,d]$,
integrate the function $x \mapsto f(x,y)$ over the interval $[a,b]$. (This function is
continuous since it is the composite of the continuous functions $x \mapsto (x,y)$ and~$f$.)
The integration will result in a vector which depends on $y$, call it~$h(y)$.  We will show
shortly that the function $y \mapsto h(y)$ is also continuous and so may be integrated
over~$[c,d]$.  The resulting vector in $E$ is denoted by $\int_c^d\left(\int_a^b
f(x,y)\,dx\right)\,dy$ or just $\int_c^d\int_a^b f(x,y)\,dx\,dy$.  The two integrals operating
successively are called
 \index{<@$\int_c^d\left(\int_a^b f(x,y)\,dx\right)\,dy$ (iterated integrals)}%
 \index{iterated integrals}%
 \index{integral!iterated}%
\df{iterated integrals}.


\begin{notn} Let $f \colon (x,y) \mapsto f(x,y)$ be a continuous function defined on a subset
of $\R^2$ containing the rectangle $[a,b] \times [c,d]$.  Throughout this section we denote by
 \index{<@$f\,^y$, ${}^xf$}%
$f\,^y$ the function of $x$ which results from holding $y$ fixed and by ${}^xf$ the function
of $y$ resulting from fixing~$x$.  That is, for each $y$
  \[ f\,^y \colon x \mapsto f(x,y) \]
and for each $x$
  \[ {}^xf\colon y\mapsto f(x,y)\,. \]
For each $y$ we interpret  $\int_a^b f(x,y)\,dx$ to mean $\int_a^b f\,^y$, and for each $x$ we
take $\int_c^d f(x,y)\,dy$ to be $\int_c^d\, {}^xf$.  Thus
  \[ \int_c^d\int_a^b f(x,y)\,dx\,dy = \int_c^d g \]
where $g(y) = \int_a^b f\,^y$ for all $y \in [c,d]$.  In order for $\int_c^d g$ to make sense
we must know that $g$ is a regulated function.  It will suffice for our needs to show that if
$f$ is continuous, then so is~$g$.
\end{notn}

\begin{lem}\label{lem_int_2cont}  Let $f \colon [a,b] \times [c,d] \sto E$ be a continuous
function into a Banach space.  For each $y \in [c,d]$ let
  \[ g(y) = \int_a^b f\,^y\,. \]
Then $g$ is uniformly continuous on~$[c,d]$.
\end{lem}

\begin{proof} Exercise.   \emph{Hint.}  Use proposition~\ref{prop_cont_ucont}.
(Solution~\ref{sol_lem_int_2cont}.) \ns
\end{proof}

Perhaps the most frequently used result concerning iterated integrals is that if $f$ is
continuous, then the order of integration does not matter.

\begin{prop}\label{prop_chord_int}  If $E$ is a Banach space, if $a < b$ and $c < d$, and if \linebreak
$f\colon [a,b]\times [c,d] \sto E$ is continuous, then
  \[ \int_a^b \int_c^d f(x,y)\,dy\,dx = \int_c^d \int_a^b f(x,y)\,dx\,dy\,. \]
\end{prop}

\begin{proof} Problem.  \emph{Hint.}    Define functions $j$ and $k$ for all $z$ in $[c,d]$ by
the formulas
  \begin{align*}
               j(z) &= \int_a^b \int_c^z f(x,y)\,dy\,dx \\
   \intertext{and}
               k(z) &= \int_c^z \int_a^b f(x,y)\,dx\,dy\,.
  \end{align*}
It suffices to show that $j = k$.  (Why?)  One may accomplish this by showing that $j\,'(z) =
k'(z)$ for all $z$ and that $j(c) = k(c)$ (see corollary~\ref{cor_const_diff}).  Finding
$k'(z)$ is easy.  To find $j\,'(z)$ derive the formulas
  \begin{align}
       \frac 1h \Delta j_{{}_{\sst z}}(h)
              &= \int_a^b \frac 1h \int_z^{z+h}f(x,y)\,dy\,dx \label{eqn_chord_int1} \\
   \intertext{and}
       \int_a^b f(x,z)\,dx
             &= \int_a^b \frac 1h \int_z^{z+h}f(x,z)\,dy\,dx\,. \label{eqn_chord_int2}
  \end{align}
Subtract \eqref{eqn_chord_int2}) from~\eqref{eqn_chord_int1} to obtain a new equation.  Show
that the right side of this new equation can be made arbitrarily small by choosing $h$
sufficiently small.  (For this use an argument similar to the one used in the proof of
lemma~\ref{lem_int_2cont}.  Conclude that $j\,'(z) = \int_a^b f(x,z)\,dx$.    \ns
\end{proof}


Now that we have a result (proposition~\ref{prop_chord_int}) allowing us to change the order
of two integrals, we can prove a result which justifies changing the order of integration and
differentiation. We show that if $f$ is continuous and $f_2$ (exists and) is continuous, then
  \[ \dfrac d{dy} \int_a^b f(x,y)\,dx = \int_a^b \pd fy (x,y)\,dx\,. \]

\begin{prop}\label{prop_diff_und_int}  Let $E$ be a Banach space, $a < b$, $c < d$, and $f \colon
[a,b]\times [c,d] \sto E$.  If $f$ and $f_2$ are continuous then the function $g$ defined for
all $y \in [c,d]$ by
  \[ g(y) = \int_a^b f\,^y \]
is continuously differentiable in $(c,d)$ and for $c < y < d$
  \[ g'(y) = \int_a^b f_2(x,y)\,dx\,. \]
\end{prop}

\begin{proof} Exercise.    \emph{Hint.}  Let $h(y) = \int_a^b f_2(x,y)\,dx$ for $c \le y \le d$.
Use proposition~\ref{prop_chord_int} to reverse the order of integration in $\int_c^z h$
(where $c < z < d$).  Use the version of the \emph{fundamental theorem of calculus} given in
Theorem~\ref{thm_ftc_II} to obtain $\int_c^z h = g(z) - g(c)$. Differentiate.
(Solution~\ref{sol_prop_diff_und_int}.) \ns
\end{proof}


\begin{prob}  Compute
  \[ \int_0^1\int_0^1 \frac {x^2-y^2}{(x^2+y^2)^2}\,dx\,dy  \qquad \text{and} \qquad
                                \int_0^1\int_0^1 \frac {x^2-y^2}{(x^2+y^2)^2}\,dy\,dx\,. \]
Why does the result not contradict the assertion made in proposition~\ref{prop_chord_int}?
\end{prob}

\begin{prob} (a)\quad Suppose that the functions $g\colon \R^n \sto \R$ and $h \colon \R \sto \R$
are differentiable and $f = h \circ g$.  Show that
  \[ f_k(x) = g_k(x)\,(Dh)(g(x)) \]
whenever $x \in \R^n$ and $1 \le k \le n$.

(b)\quad Let $g \colon \R^n \sto \R$ be differentiable and $j \colon \R \sto \R$ be
continuous. Prove that
  \[ \pd{}{x_k}\int_c^{g(x)}j(t)\,dt = g_k(x)\,j(g(x)) \]
whenever $c \in \R$, $x \in \R^n$, and $1 \le k \le n$. \emph{Hint.}  The expression on the
left denotes $f_k(x)$ where $f$ is the function $x \mapsto \int_c^{g(x)}j(t)\,dt$.

(c)\quad  Use part (b) to compute
  \[ \pd{}x\int_{x^3y}^{x^2 + y^2}\frac 1{1 + t^2 + cos^2t}\,dt\,. \]
\end{prob}

 \index{Leibniz's formula}%
\begin{prop}[Leibniz's formula]  Let $f\colon [a,b] \times [c,d] \sto \R$ and $h \colon [c,d]
\sto \R$.  If $f$ and $f_2$ are continuous, if $h$ is continuously differentiable on $(c,d)$,
and if $h(y) \in [a,b]$ for every $y \in (c,d)$, then
  \[ \frac d{dy}\int_a^{h(y)}f(x,y)\,dx = \int_a^{h(y)}f_2(x,y)\,dx + Dh(y)\,f(h(y),y)\,. \]
\end{prop}

\begin{proof} Problem.  \ns  \end{proof}
