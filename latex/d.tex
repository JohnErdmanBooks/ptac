\chapter{LOGICAL CONNECTIVES}\label{log_con}

\section{DISJUNCTION AND CONJUNCTION}\label{d_and_c} The word ``or'' in English has two
distinct uses.  Suppose you ask a friend how he intends to spend the evening, and he replies,
``I'll'' walk home or I'll take in a film.'' If you find that he then walked home and on the
way stopped to see a film, it would not be reasonable to accuse him of having lied. He was
using the inclusive ``or'', which is true when one or both alternatives are. On the other hand
suppose that while walking along a street you are accosted by an armed robber who says, ``Put
up your hands or I'll shoot.''  You obligingly raise your hands. If he then shoots you, you
have every reason to feel ill-used. Convention and context dictate that he used the  exclusive
``or'': either alternative, but not both.

Since it is undesirable for ambiguities to arise in mathematical discourse, the inclusive
``or'' has been adopted as standard for mathematical (and most scientific) purposes. A
convenient way of defining logical connectives such as ``or'' is by means of a truth table.
The formal definition of ``or'' looks like this.
 \begin{center}
  \begin{tabular}{|c|c||c|}\hline
        \,$P$\,    &    \,$Q$\,    &   \,$P \lor Q$\,   \\
    \hline\hline
          $T$      &      $T$      &     $T$            \\
    \hline
          $T$      &      $F$      &     $T$            \\
    \hline
          $F$      &      $T$      &     $T$            \\
    \hline
          $F$      &      $F$      &     $F$            \\
    \hline
  \end{tabular}
 \end{center}
Here $P$ and $Q$ are any sentences. In the columns labeled $P$ and $Q$ we list all possible
combinations of truth values for $P$ and $Q$ ($T$ for true, $F$ for false). In the third
column appears the corresponding truth value for ``$P$ or $Q$''. According to the table ``$P$
or $Q$'' is true in all cases except when both $P$ and $Q$ are false. The notation
 \index{#@$\lor$ (or, disjunction)}%
``$P \lor Q$'' is frequently used for ``$P$ or $Q$''.  The
operation $\lor$ is called
 \index{disjunction}%
\df{disjunction}.

\begin{exer}\label{and} Construct a truth table giving the formal definition of ``and'',
frequently denoted
 \index{#@$\land$ (and, conjunction)}%
by~$\land$.  The operation $\land$ is called
 \index{conjunction}%
\df{conjunction}. (Solution~\ref{sol_and}.)
\end{exer}

We say that two sentences depending on variables $P,Q,\dots$ are
 \index{logical equivalence}%
 \index{equivalent!logically}%
\df{logically equivalent} if they have the same truth value no matter how truth values $T$ and
$F$ are assigned to the variables $P,Q,\dots$. It turns out that truth tables are quite
helpful in deciding whether certain statements encountered in mathematical reasoning are
logically equivalent to one another.  (But do \emph{not} clutter up mathematical proofs with
truth tables. Everyone is supposed to argue logically. Truth tables are only scratch work for
the perplexed.)

\begin{exam} Suppose that $P$, $Q$, and $R$ are any sentences. To a person who habitually uses
language carefully, it will certainly be clear that the following two assertions are
equivalent:
 \begin{enumerate}
   \item[(a)] $P$ is true and so is either $Q$ or $R$.
   \item[(b)] Either both $P$ and $Q$ are true or both $P$ and $R$ are
true.
 \end{enumerate}
Suppose for a moment, however, that we are in doubt concerning the relation between (a) and
(b).  We may represent (a) symbolically by $P \land (Q \lor R)$ and (b) by $(P \land Q) \lor
(P \land R)$.  We conclude that they are indeed logically equivalent by examining the
following truth table.  (Keep in mind that since there are 3 variables, $P$, $Q$, and $R$,
there are $2^3 = 8$ ways of assigning truth values to them; so we need 8 lines in our truth
table.)



 \begin{center}
  \begin{tabular}{|c|c|c||c|c|c|c|c|}\hline
       (1)  &   (2)   &   (3)   &      (4)       &       (5)                &       (6)       &      (7)        &                (8)                 \\
    \,$P$\, & \,$Q$\, & \,$R$\, & \,$Q \lor R$\, & \,$P \land (Q \lor R)$\, & \,$P \land Q$\, & \,$P \land R$\, & \,$(P \land Q) \lor (P \land R)$\, \\
    \hline\hline
      $T$   &   $T$   &   $T$   &     $T$        &       $T$                &      $T$        &      $T$        &                $T$                 \\
    \hline
      $T$   &   $T$   &   $F$   &     $T$        &       $T$                &      $T$        &      $F$        &                $T$                 \\
    \hline
      $T$   &   $F$   &   $T$   &     $T$        &       $T$                &      $F$        &      $T$        &                $T$                 \\
    \hline
      $T$   &   $F$   &   $F$   &     $F$        &       $F$                &      $F$        &      $F$        &                $F$                 \\
    \hline
      $F$   &   $T$   &   $T$   &     $T$        &       $F$                &      $F$        &      $F$        &                $F$                 \\
    \hline
      $F$   &   $T$   &   $F$   &     $T$        &       $F$                &      $F$        &      $F$        &                $F$                 \\
    \hline
      $F$   &   $F$   &   $T$   &     $T$        &       $F$                &      $F$        &      $F$        &                $F$                 \\
    \hline
      $F$   &   $F$   &   $F$   &     $F$        &       $F$                &      $F$        &      $F$        &                $F$                 \\
    \hline
  \end{tabular}
 \end{center}
Column (4) is obtained from (2) and (3), column (5) from (1) and (4), column (6) from (l) and
(2), column (7) from (1) and (3), and column (8) from (6) and (7).  Comparing the truth values
in columns (5) and (8), we see that they are exactly the same.  Thus $P \land (Q \lor R)$ is
logically equivalent to $(P \land Q) \lor (P \land R)$. This result is a
 \index{distributive law!in logic}%
\df{distributive law}; it says that conjunction distributes over
disjunction.
\end{exam}

\begin{prob} Use truth tables to show that the operation of disjunction is associative; that is,
show that $(P \lor Q) \lor R$ and $P \lor (Q \lor R)$ are logically equivalent.
\end{prob}

\begin{prob}\label{prob_disj_conj} Use truth tables to show that disjunction distributes over
conjunction; that is, show that $P \lor (Q \land R)$ is logically equivalent to $(P \lor Q)
\land (P \lor R)$.
\end{prob}

One final remark: quantifiers may be ``moved past'' portions of disjunctions and conjunctions
which do not contain the variable being quantified.  For example,
  \[ (\exists y)(\exists x)\,[(y^2 \le 9) \land (2<x<y)] \]
says the same thing as
  \[ (\exists y)\,[(y^2 \le 9) \land (\exists x)\,2<x<y]. \]







\section{IMPLICATION}\label{implication} Consider the assertion, ``If $1=2$, then $2=3$.''
If you ask a large number of people (not mathematically trained) about the truth of this,
you will probably find some who think it is true, some who think it is false, and some
who think it is meaningless (therefore neither true nor false).  This is another example
of the ambiguity of ordinary language. In order to avoid ambiguity and to insure that
``$P$ implies $Q$'' has a truth value whenever $P$ and $Q$ do, we define the operation of
 \index{implication}%
\df{implication}, denoted
 \index{#@$\Rightarrow$ (implies)}%
by~$\Rightarrow$, by means of the
following truth table.
 \begin{center}
  \begin{tabular}{|c|c||c|}\hline
        \,$P$\,    &    \,$Q$\,    &   \,$P \Rightarrow Q$\,   \\
    \hline\hline
          $T$      &      $T$      &         $T$            \\
    \hline
          $T$      &      $F$      &         $F$            \\
    \hline
          $F$      &      $T$      &         $T$            \\
    \hline
          $F$      &      $F$      &         $T$            \\
    \hline
  \end{tabular}
 \end{center}

There are many ways of saying that $P$ implies $Q$. The following assertions are all identical.
\begin{align*}
  &P \Rightarrow Q. \\
  &\text{$P$ implies $Q$.} \\
  &\text{If $P$, then $Q$.} \\
  &\text{$P$ is sufficient (or a sufficient condition) for $Q$.}\\
  &\text{Whenever $P$, then $Q$.} \\
  &Q \Leftarrow P \\
  &\text{$Q$ is implied by $P$.} \\
  &\text{$Q$ is a consequence of $P$.} \\
  &\text{$Q$ follows from $P$.} \\
  &\text{$Q$ is necessary (or a necessary condition) for $P$.} \\
  &\text{$Q$ whenever $P$.}
\end{align*}
The statement $Q \Rightarrow P$ is the
 \index{converse}%
\df{converse} of $P \Rightarrow Q$. It is a common mistake to confuse a statement with its
converse. This is a grievous error. For example: it is correct to say that if a geometric
figure is a square, then it is a quadrilateral; but it is \emph{not} correct to say that if a
figure is a quadrilateral it must be a square.

\begin{defn} If $P$ and $Q$ are sentences, we define the logical connective ``iff'' (read
``if and only if'') by the following truth table.


 \begin{center}
  \begin{tabular}{|c|c||c|}\hline
        \,$P$\,    &    \,$Q$\,    &   \,$P \text{ iff } Q$\,   \\
    \hline\hline
          $T$      &      $T$      &     $T$            \\
    \hline
          $T$      &      $F$      &     $F$            \\
    \hline
          $F$      &      $T$      &     $F$            \\
    \hline
          $F$      &      $F$      &     $T$            \\
    \hline
  \end{tabular}
 \end{center}
Notice that the sentence ``$P$ iff $Q$'' is true exactly in those cases where $P$ and $Q$ have
the same truth values. That is, saying that ``$P$ iff $Q$'' is a
 \index{tautology}%
\df{tautology} (true for all truth values of $P$ and $Q$) is the same as saying that $P$ and
$Q$ are equivalent sentences.  Thus the connective ``iff'' is called
 \index{logical equivalence}%
 \index{equivalent!logically}%
\df{equivalence}. An alternative notation
 \index{iff (logical equivalence)}%
for~``iff''
 \index{#@$\Leftrightarrow$ (logical equivalence)}%
is~``$\Leftrightarrow$''.
\end{defn}

\begin{exam} By comparing columns (3) and (6) of the following truth table, we see that
``$P$ iff $Q$'' is logically equivalent to ``$(P \Rightarrow Q) \land (Q \Rightarrow P)$''.
 \begin{center}
  \begin{tabular}{|c|c||c|c|c|c|}\hline
           (1)  &   (2)   &       (3)              &      (4)              &      (5)              &                      (6)                        \\
        \,$P$\, & \,$Q$\, & \,$P \text{ iff } Q$\, & \,$P \Rightarrow Q$\, & \,$Q \lor (\sim P)$\, & \,$(P \Rightarrow Q) \land (Q \Rightarrow P)$\, \\
    \hline\hline
          $T$   &   $T$   &       $T$              &      $T$              &       $T$             &                       $T$                       \\
    \hline
          $T$   &   $F$   &       $F$              &      $F$              &       $T$             &                       $F$                       \\
    \hline
          $F$   &   $T$   &       $F$              &      $T$              &       $F$             &                       $F$                       \\
    \hline
          $F$   &   $F$   &       $T$              &      $T$              &       $T$             &                       $T$                       \\
    \hline
  \end{tabular}
 \end{center}
This is a very important fact. Many theorems of the form $P$ iff $Q$ are most conveniently
proved by verifying separately that $P \Rightarrow Q$ and that $Q \Rightarrow P$.
\end{exam}






\section{RESTRICTED QUANTIFIERS}\label{restr_quan}  Now that we have the logical connectives
$\Rightarrow$ and $\land$ at our disposal, it is possible to introduce restricted quantifiers
formally in terms of unrestricted ones. This enables one to obtain properties of the former
from corresponding facts about the latter.

(See exercise~\ref{exer_ueq} and problems~\ref{prob_ord_restr_quan}
and~\ref{prob_neg_restr_quan}.)

\begin{defn}[of restricted quantifiers]
 \index{restricted quantifiers}%
Let $S$ be a set and $P(x)$ be an open sentence. We define $(\forall x \in S)\,P(x)$ to be
true if and only if $(\forall x)\,\bigl((x \in S) \Rightarrow P(x)\bigr)$ is true; and we
define $(\exists x \in S)\,P(x)$ to be true if and only if $(\exists x)\, \bigl((x \in S)
\land P(x)\bigr)$ is true.
\end{defn}

\begin{exer}\label{exer_ueq} Use the preceding definition and the fact (mentioned in
chapter~\ref{quan}) that the order of unrestricted existential quantifiers does not matter to
show that the order of restricted existential quantifiers does not matter. That is, show that
if $S$ and $T$ are sets and $P(x,y)$ is an open sentence, then $(\exists x \in S)\,(\exists y
\in T)\, P(x,y)$ holds if and only if $(\exists y \in T)\,(\exists x \in S)\, P(x,y)$ does.
(Solution~\ref{sol_exer_ueq}.)
\end{exer}






\section{NEGATION}\label{neg} If $P$ is a sentence,
 \index{#@$\sim P$ (negation of $P$)}
then $\sim P$, read ``the
 \index{negation}%
\df{negation} of $P$'' or ``the
 \index{denial}%
\df{denial} of $P$'' or just ``not $P$'', is the sentence whose truth values are the opposite
of~$P$.
 \begin{center}
  \begin{tabular}{|c||c|}\hline
        \,$P$\,    &    \,$\sim P$\,  \\
    \hline\hline
          $T$      &        $F$       \\
    \hline
          $F$      &        $T$       \\
    \hline
  \end{tabular}
 \end{center}

\begin{exam}\label{exam_demorg1} It should be clear that the denial of the disjunction of two
sentences $P$ and $Q$ is logically equivalent to the conjunction of their denials. If we were
in doubt about the correctness of this, however, we could appeal to a truth table to prove
that $\sim(P \lor Q)$ is logically equivalent to $\sim P \land \sim Q$.

 \begin{center}
  \begin{tabular}{|c|c||c|c|c|c|c|}\hline
           (1)  &   (2)   &       (3)      &      (4)             &      (5)     &     (6)      &               (7)              \\
        \,$P$\, & \,$Q$\, & \,$P \lor Q$\, & \,$\sim(P \lor Q)$\, & \,$\sim P$\, & \,$\sim Q$\, & \,$(\sim P) \land (\sim Q)$\,  \\
    \hline\hline
          $T$   &   $T$   &       $T$      &      $F$             &       $F$    &     $F$      &               $F$              \\
    \hline
          $T$   &   $F$   &       $T$      &      $F$             &       $F$    &     $T$      &               $F$              \\
    \hline
          $F$   &   $T$   &       $T$      &      $F$             &       $T$    &     $F$      &               $F$              \\
    \hline
          $F$   &   $F$   &       $F$      &      $T$             &       $T$    &     $T$      &               $T$              \\
    \hline
  \end{tabular}
 \end{center}
Columns (4) and (7) have the same truth values: That is, the denial of the disjunction of $P$
and $Q$ is logically equivalent to the conjunction of their denials. This result is one of
 \index{De Morgan's laws!in logic}%
\emph{De Morgan's laws}. The other is given as
problem~\ref{prob_demorg2}.
\end{exam}


 \index{De Morgan's laws!in logic}%
\begin{prob}[De Morgan's law]\label{prob_demorg2} Use a truth table to show that $\sim~(P \land Q)$
is logically equivalent to $(\sim P) \lor (\sim Q)$.
\end{prob}

\begin{prob} Obtain the result in problem \ref{prob_demorg2} without using truth tables.
\emph{Hint.} Use \ref{exam_demorg1} together with the fact that a proposition $P$ is logically
equivalent to $\sim\sim P$. Start by writing $(\sim P) \lor (\sim Q)$ iff $\sim\sim((\sim P)
\lor (\sim Q))$.
\end{prob}

\begin{exer}\label{equiv_impl} Let $P$ and $Q$ be sentences. Then $P \Rightarrow Q$ is logically
equivalent to $Q \lor (\sim P)$.  (Solution~\ref{sol_equiv_impl}.)
\end{exer}

One very important matter is the process of taking the negation of a quantified statement. Let
$P(x)$ be an open sentence. If it is not the case that $P(x)$ holds for every $x$, then it
must fail for some $x$, and conversely. That is, $\sim(\forall x)P(x)$ is logically equivalent
to $(\exists x)\sim P(x)$.

Similarly, $\sim(\exists x)P(x)$ is logically equivalent to $(\forall x)\sim P(x)$. (If it is
not the case that $P(x)$ is true for some $x$, then it must fail for all $x$, and conversely.)

\begin{exam}\label{exam_not_cont} In chapter \ref{cont_on_R} we define a real valued function
$f$ on the real line to be continuous provided that
  \[(\forall a)(\forall \epsilon)(\exists \delta)(\forall x)\,\abs{x-a}
          < \delta \Rightarrow \abs{f(x) - f(a)} < \epsilon.\]
How does one prove that a particular function $f$ is \emph{not} continuous?  Answer: find
numbers $a$ and $\epsilon$ such that for every $\delta$ it is possible to find an $x$ such
that $\abs{f(x) - f(a)} \ge \epsilon$ and $\abs{x - a} < \delta$. To see that this is in fact
what we must do, notice that each pair of consecutive lines in the following argument are
logically equivalent.
\begin{alignat*}{2}
  &\sim[(\forall a)(\forall \epsilon)(\exists \delta)(\forall x)\,
          &&\abs{x-a} < \delta \Rightarrow
          \abs{f(x) - f(a)} < \epsilon] \\
  &(\exists a)\sim[(\forall \epsilon)(\exists \delta)(\forall x)\,
          &&\abs{x-a} < \delta \Rightarrow
          \abs{f(x) - f(a)} < \epsilon] \\
  &(\exists a)(\exists \epsilon)\sim[(\exists \delta)(\forall x)\,
          &&\abs{x-a} <  \delta \Rightarrow
          \abs{f(x) - f(a)} < \epsilon] \\
  &(\exists a)(\exists \epsilon)(\forall \delta)\sim[(\forall x)\,
          &&\abs{x-a} <  \delta \Rightarrow
          \abs{f(x) - f(a)} < \epsilon] \\
  &(\exists a)(\exists \epsilon)(\forall \delta)(\exists x)\,
          \sim [&&\abs{x - a} <  \delta \Rightarrow
          \abs{f(x) - f(a)} < \epsilon] \\
  &(\exists a)(\exists \epsilon)(\forall \delta)(\exists x)\,
          \sim [(&&\abs{f(x) - f(a)} < \epsilon)
          \lor \sim (\abs{x - a} < \delta)] \\
  &(\exists a)(\exists \epsilon)(\forall \delta)(\exists x)\,
          [\sim (&&\abs{f(x) - f(a)} < \epsilon)
          \land \sim\sim(\abs{x - a} < \delta)] \\
  &(\exists a)(\exists \epsilon)(\forall \delta)(\exists x)
          \quad[(&&\abs{f(x) - f(a)} \ge \epsilon)
          \land (\abs{x - a} < \delta)]
\end{alignat*}
To obtain the third line from the end use exercise~\ref{equiv_impl};  the penultimate line is
a consequence of example \ref{exam_demorg1};  and the last line makes use of the obvious fact
that a sentence $P$ is always logically equivalent to $\sim\sim P$.
\end{exam}

\begin{prob} Two students, Smith and Jones, are asked to prove a mathematical theorem of the form,
``If $P$ then if $Q$ then~$R$.''  Smith assumes that $Q$ is a consequence of $P$ and tries to
prove~$R$. Jones assumes both $P$ and $Q$ are true and tries to prove~$R$. Is either of these
students doing the right thing? Explain carefully.
\end{prob}

\begin{prob} The
 \index{contrapositive}%
\df{contrapositive} of the implication $P \Rightarrow Q$ is the implication $(\sim
Q)\Rightarrow(\sim P)$.  Without using truth tables or assigning truth values show that an
implication is logically equivalent to its contrapositive.  (This is a very important fact. In
many cases when you are asked to prove a theorem of the form $P \Rightarrow Q$, rather than
assuming $P$ and proving $Q$ you will find it easier to assume that $Q$ is false and conclude
that $P$ must also be false.) \emph{Hint.} Use \ref{equiv_impl}. You may also use the obvious
facts that disjunction is a commutative operation ($P \lor Q$ is logically equivalent to $Q
\lor P$) and that $P$ is logically equivalent to $\sim\sim P$.)
\end{prob}

\begin{prob}\label{prob_ord_restr_quan}  Use the formal definition of restricted quantifiers
given in section~\ref{restr_quan} together with the fact mentioned in appendix \ref{quan}
that the order of unrestricted universal quantifiers does not matter to show that the
order of restricted universal quantifiers does not matter. That is, show that if $S$ and
$T$ are sets and $P(x,y)$ is an open sentence, then $(\forall x \in S)(\forall y \in
T)P(x,y)$ holds if and only if $(\forall y \in T)(\forall x \in S)P(x,y)$ does.
\end{prob}


\begin{prob}\label{prob_neg_restr_quan}  Let $S$ be a set and $P(x)$ be an open sentence. Show that
 \begin{enumerate}
   \item[(a)] $\sim(\forall x \in S)P(x)$ if and only if $(\exists x \in S) \sim P(x)$.
   \item[(b)] $\sim(\exists x \in S)P(x)$ if and only if $(\forall x \in S) \sim
P(x)$.
 \end{enumerate}

\noindent \emph{Hint.} Use the corresponding facts (given in the two paragraphs following
exercise \ref{equiv_impl}) for unrestricted quantifiers.
\end{prob}


\endinput
