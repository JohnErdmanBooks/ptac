\chapter{WRITING MATHEMATICS}

\section{PROVING THEOREMS}
Mathematical results are called \emph{theorems}---or \emph{propositions}, or \emph{lemmas}, or
\emph{corollaries}, or \emph{examples}.  All these are intended to be mathematical facts. The
different words reflect only a difference of emphasis. Theorems are more important than
propositions.  A lemma is a result made use of in a (usually more important) subsequent
result.  The German word for lemma is particularly suggestive: ``Hilfsatz,'' meaning ``helping
statement.''  A corollary (to a theorem or proposition) is an additional result we get
(almost) for free.  All these results are typically packaged in the form, ``If $P$, then
$Q$.''  The assertion $P$ is the \emph{hypothesis} (or \emph{premise}, or \emph{assumption},
or \emph{supposition}).  The assertion $Q$ is the \emph{conclusion}. Notice that the result
``Every object of type $A$ is of type $B$'' is in this form.  It can be rephrased as, ``If $x$
is an object of type $A$, then $x$ is of type~$B$.''

The statements, $P$ and $Q$ themselves may be complicated conjunctions, or disjunctions,
or conditionals of simpler statements.  One common type of theorem, for example, is, ``If
$P_1$, $P_2$, \dots, and $P_m$, then $Q_1$, $Q_2$, \dots, and~$Q_n$.''  (Personally, I
think such a theorem is clumsily stated if $m$ and $n$ turn out to be large.)

A
 \index{proofs}%
\emph{proof} of a result is a sequence of statements, each with justification, which leads to
the conclusion(s) of the desired result.  The statements that constitute a proof may be
definitions, or hypotheses, or statements which result from applying to previous steps of the
proof a valid
 \index{rule of inference}%
rule of inference.
 \index{modus ponens}%
\emph{Modus ponens} is the basic rule of inference.  It says that if you know a proposition
$P$ and you also know that $P$ implies $Q$, then you can conclude that $Q$ is true.  Another
important rule of inference (sometimes called \emph{universal instantiation}) is that if you
know a proposition $P(x)$ to be true for every $x$ in a set $S$ and you know that $a$ is a
member of $S$, then you can conclude that $P(a)$ is true.

Other rules of inference can be derived from \emph{modus ponens}. Let's look at an
example. Certainly, if we know that the proposition $P \land Q$ is true we should be able
to conclude that $P$ is true.  The reason is simple we know (or can easily check) that
  \begin{equation}\label{eqn_rul_inf}
        (P \land Q) \Rightarrow P
  \end{equation}
is a tautology (true for all truth values of $P$ and $Q$).  Since $P \land Q$ is known to be
true, $P$ follows from~\eqref{eqn_rul_inf} by \emph{modus ponens}.  No attempt is made here to
list every rule of inference that it is appropriate to use.  Most of them should be entirely
obvious.  For those that are not, truth tables may help.  (As an example consider
problem~\ref{prob_zero_prod}: If the product of two numbers $x$ and $y$ is zero, then either
$x$ or $y$ must be zero.  Some students feel obligated to prove two things: that if $xy=0$ and
$x \neq 0$ then $y=0$ \textbf{AND} that if $xy=0$ and $y \neq 0$ then $x=0$. Examination of
truth tables shows that this is not necessary.)

A proof in which you start with the hypotheses and reason until you reach the conclusion is a
 \index{direct proof}%
 \index{proof!direct}%
\emph{direct proof}.  There are two other proof formats, which are known as
 \index{indirect proof}%
 \index{proof!indirect}%
\emph{indirect proofs}.  The first comes about by observing that the proposition $\sim Q
\Rightarrow \sim P$ is logically equivalent to $P \Rightarrow Q$.  (We say that $\sim Q
\Rightarrow \sim P$ is the
 \index{contrapositive}%
\emph{contrapositive} of $P \Rightarrow Q$.)  To prove that $P$ implies $Q$, it suffices to
assume that $Q$ is false and prove, using this assumption that $P$ is false.  Some find it odd
that to prove something is true one starts by assuming it to be false.  A slight variant of
this is the
 \index{proof!by contradiction}%
 \index{contradiction, proof by}%
\emph{proof by contradiction}.  Here, to prove that $P$ implies $Q$, assume two things: that
$P$ is true and that $Q$ is false. Then attempt to show that these lead to a contradiction.
We like to believe that the mathematical system we work in is consistent (although we know we
can't prove it), so when an assumption leads us to a contradiction we reject it.  Thus in a
proof by contradiction when we find that $P$ and $\sim Q$ can't both be true, we conclude that
if $P$ is true $Q$ must also be true.

\begin{prob} Prove that in an inconsistent system, everything is true.  That is, prove that if
$P$ and $Q$ are propositions and that if both $P$ and $\sim P$ are true, then $Q$ is true.
\emph{Hint.} Consider the proposition $(P \land \sim P) \Rightarrow Q$.
\end{prob}

\begin{prob} What is wrong with the following proof that $1$ is the largest natural number.
 \begin{quote}
Let $N$ be the largest natural number.  Since $N$ is a natural number so is~$N^2$.  We see
that $N^2 = N \cdot N \ge N \cdot 1 = N$.  Clearly the reverse inequality $N^2 \le N$ holds
because $N$ is the largest natural number.  Thus $N^2 = N$.  This equation has only two
solutions: $N=0$ and $N=1$.  Since $0$ is not a natural number, we have $N=1$.  That is, $1$
is the largest natural number.
 \end{quote}
\end{prob}







\section{CHECKLIST FOR WRITING MATHEMATICS}  After you have solved a problem or discovered a
counterexample or proved a theorem, there arises the problem of writing up your result.  You
want to do this in a way that will be as clear and as easy to digest as possible. Check your
work against the following list of suggestions.
 \begin{enumerate}
   \item \emph{Have you clearly stated the problem you are asked to solve or the result you
are trying to prove?}

\vspace{10pt}

     \begin{quote}
Have an audience in mind.  Write to someone.  And don't assume the person you are writing to
remembers the problem.  (S)he may have gone on vacation, or been fired; or maybe (s)he just
has a bad memory.  You need not include every detail of the problem, but there should be
enough explanation so that a person not familiar with the situation can understand what you
are talking (writing) about.
      \end{quote}

\vspace{15pt}

   \item \emph{Have you included a paragraph at the beginning explaining the method you are
going to use to address the problem?}

\vspace{10pt}

     \begin{quote}
No one is happy being thrown into a sea of mathematics with no clue as to what is going on or
why.  Be nice.  Tell the reader what you are doing, what steps you intend to take, and what
advantages you see to your particular approach to the problem.
     \end{quote}

\vspace{15pt}

   \item \emph{Have you defined all the variables you use in your writeup?}

\vspace{10pt}
      \begin{quote}
\emph{Never} be so rude as to permit a symbol to appear that has not been properly introduced.
You may have a mental picture of a triangle with vertices labelled $A$, $B$, and~$C$.  When
you use those letters no one will know what they stand for unless you tell them.  (Even if you
have included a graph appropriately labelled, still tell the reader \emph{in the text} what
the letters denote.) Similarly, you may be consistent in always using the letter $j$ to denote
a natural number.  But how would you expect the reader to know?

\vspace{5pt}

It is good practice to italicize variables so that they can be easily distinguished from
regular text.
      \end{quote}

\vspace{15pt}

   \item \emph{Is the logic of your report entirely clear and entirely correct?}

\vspace{10pt}

     \begin{quote}
It is an unfortunate fact of life that the slightest error in logic can make a ``solution'' to
a problem totally worthless.  It is also unfortunate that even a technically correct argument
can be so badly expressed that no one will believe it.
     \end{quote}

\vspace{15pt}

   \item \emph{In your writeup are your mathematical symbols and mathematical terms all correctly
used in a standard fashion? And are all abbreviations standard?}

\vspace{10pt}

     \begin{quote}
Few things can make mathematics more confusing than misused or eccentrically used symbols.
Symbols should \emph{clarify} arguments not create yet another level of difficulty.  By the
way, symbols such as ``='' and ``$<$'' are used \emph{only} in formulas. They are not
substitutes for the words ``equals'' and ``less than'' in ordinary text.  Logical symbols such
as $\Rightarrow$ are rarely appropriate in mathematical exposition: write ``If A then B,'' not
``$A \Rightarrow B$.''  Occasionally they may be used in displays.
     \end{quote}

\vspace{15pt}

   \item \emph{Are the spelling, punctuation, diction, and grammar of your report all correct?}

\vspace{15pt}

   \item \emph{Is every word, every symbol, and every equation part of a sentence?  And is every
sentence part of a paragraph?}

\vspace{10pt}


     \begin{quote}
For some reason this seems hard for many students.  Scratchwork, of course, tends to be full
of free floating symbols and formulas. When you write up a result get rid of all this clutter.
Keep only what is necessary for a logically complete report of your work. And make sure any
formula you keep becomes (an intelligible) part of a sentence.  Study how the author of any
good mathematics text deals with the problem of incorporating symbols and formulas into text.
     \end{quote}

\vspace{15pt}

   \item \emph{Does every sentence start correctly and end correctly?}

\vspace{10pt}

     \begin{quote}
Sentences start with capital letters.  \emph{Never} start a sentence with a number or with a
mathematical or logical symbol. Every declarative sentence ends with a period.  Other
sentences may end with a question mark or (rarely) an exclamation mark.
     \end{quote}

\vspace{15pt}

   \item \emph{Is the \emph{function} of every sentence of your report clear?}

\vspace{10pt}

     \begin{quote}
Every sentence has a function.  It may be a definition.  Or it may be an assertion you are
about to prove.  Or it may be a consequence of the preceding statement.  Or it may be a
standard result your argument depends on.  Or it may be a summary of what you have just
proved.  \textbf{Whatever function a sentence serves, that function should be entirely clear
to your reader.}
     \end{quote}

\vspace{15pt}

   \item \emph{Have you avoided all unnecessary clutter?}

\vspace{10pt}

     \begin{quote}
Mindless clutter is one of the worst enemies of clear exposition.  No one wants to see all the
details of your arithmetic or algebra or trigonometry or calculus.  Either your reader knows
this stuff and could do it more easily than read it, or doesn't know it and will find it
meaningless.  In either case, get rid of it.  If you solve an equation, for example, state
what the solutions are; don't show how you used the quadratic formula to find them.  Write
only things that inform.  Logical argument informs, reams of routine calculations do not.  Be
ruthless in rooting out useless clutter.
     \end{quote}

\vspace{15pt}

   \item \emph{Is the word ``any'' used unambiguously?  And is the order in which quantifiers
act entirely clear?}

\vspace{10pt}

     \begin{quote}
Be careful with the word ``any'', especially when taking negations.  It can be surprisingly
treacherous.  ``Any'' may be used to indicate universal quantification.  (In the assertion,
``It is true for any $x$ that $x^2 - 1 = (x + 1)(x - 1)$'', the word ``any'' means ``every''.)
It may also be used for existential quantification.  (In the question, ``Does $x^2 + 2 = 2x$
hold for any $x$?'' the word ``any'' means ``for some''.  Also notice that answering
\emph{yes} to this question does not mean that you believe it is true that ``$x^2 + 2 = 2x$
holds for any~$x$.'') Negations are worse.  (What does ``It is not true that $x^2 + 2 = 2x$
for any $x$'' mean?)   One good way to avoid trouble: don't use ``any''.

\vspace{5pt}

And be careful of word order.  The assertions ``Not every element of $A$ is an element of
$B$'' and ``Every element of $A$ is not an element of $B$'' say quite different things.  I
recommend avoiding the second construction entirely.  Putting (some or all) quantifiers at the
end of a sentence can be a dangerous business. (The statement, ``$P(x)$ or $Q(x)$ fails to
hold for all~$x$,'' has at least two possible interpretations. So does ``$x \neq 2$ for all $x
\in A$,'' depending on whether we read $\neq$ as ``is not equal to'' or as ``is different
from''.)
     \end{quote}
 \end{enumerate}



\section{FRAKTUR AND GREEK ALPHABETS}

Fraktur is a typeface widely used until the middle of the twentieth century in Germany
and a few other countries.  In this text it is used to denote families of sets and
families of linear maps. In general it is not a good idea to try to reproduce these
letters in handwritten material or when writing on the blackboard. In these cases English
script letters are both easier to read and easier to write.  Here is a list of upper case
Fraktur letters:
\begin{align*} &\mathfrak A \quad (\text A)  &&\mathfrak B \quad (\text B)
               &&\mathfrak C \quad (\text C) &&\mathfrak D \quad (\text D)
               &&\mathfrak E \quad (\text E) &&\mathfrak F \quad (\text F)
               &&\mathfrak G \quad (\text G)   \\
               &\mathfrak H \quad (\text H)  &&\mathfrak I \quad (\text I)
               &&\mathfrak J \quad (\text J) &&\mathfrak K \quad (\text K)
               &&\mathfrak L \quad (\text L) &&\mathfrak M \quad (\text M)
               &&\mathfrak N \quad (\text N)   \\
               &\mathfrak O \quad (\text O)  &&\mathfrak P \quad (\text P)
               &&\mathfrak Q \quad (\text Q) &&\mathfrak R \quad (\text R)
               &&\mathfrak S \quad (\text S) &&\mathfrak T \quad (\text T)
               &&\mathfrak U \quad (\text U)   \\
               &\mathfrak V \quad (\text V)  &&\mathfrak W \quad (\text W)
               &&\mathfrak X \quad (\text X) &&\mathfrak Y \quad (\text Y)
               &&\mathfrak Z \quad (\text Z) && {}
               && {}
\end{align*}

\vskip .2 in

The following is a list of standard Greek letters used in mathematics. Notice that in
some cases both upper case and lower case are commonly used:
  \begin{align*} &\alpha \quad (\text{alpha})  &&\beta \quad (\text{beta})
               &&\Gamma, \gamma \quad (\text{gamma}) &&\Delta, \delta \quad (\text{delta})
               &&\epsilon \quad (\text{epsilon}) &&\zeta \quad (\text{zeta})  \\
               &\eta \quad (\text{eta}) &&\Theta, \theta \quad (\text{theta})
               &&\iota \quad (\text{iota})  &&\kappa \quad (\text{kappa})
               &&\Lambda, \lambda \quad (\text{lambda}) &&\mu \quad (\text{mu})   \\
               &\nu \quad (\text{nu})  &&\Xi, \xi \quad (\text{xi})
               &&\Pi, \pi \quad (\text{pi})  &&\rho \quad (\text{rho})
               &&\Sigma, \sigma \quad (\text{sigma}) &&\tau \quad (\text{tau})   \\
               &\Phi, \phi \quad (\text{phi}) &&\Psi, \psi \quad (\text{psi})
               &&\chi \quad (\text{chi})  &&\Omega, \omega \quad (\text{omega})
               && {}  &&  {}
  \end{align*}

\endinput
