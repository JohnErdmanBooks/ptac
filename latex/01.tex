\chapter{INTERVALS}\label{intervals}

The three great realms of calculus are differential calculus, integral calculus, and the
theory of infinite series.  Central to each of these is the notion of \emph{limit}:
derivatives, integrals, and infinite series can be defined as limits of appropriate objects.
Before entering these realms, however, one must have some background.  There are the basic
facts about the algebra, the order properties, and the topology of the real line~$\R$.  One
must know about functions and the various ways in which they can be combined (addition,
composition, and so on). And, finally, one needs some basic facts about continuous functions,
in particular the \emph{intermediate value theorem} and the \emph{extreme value theorem}.

%As was mentioned in the section \emph{How to Use This Book}, much
Much of this material appears in the appendices.  The remainder (the topology of $\R$,
continuity, and the \emph{intermediate} and \emph{extreme value theorems}) occupies the first
six chapters. After these have been completed we proceed with limits and the differential
calculus of functions of a single variable.

As you will recall from beginning calculus we are accustomed to calling certain intervals
``open'' (for example, $(0,1)$ and $(-\infty, 3)$ are open intervals) and other intervals
``closed'' (for example, $[0,1]$ and $[1,\infty)$ are closed intervals).  In this chapter and
the next we investigate the meaning of the terms ``open'' and ``closed''.  These concepts turn
out to be rather more important than one might at first expect.  It will become clear after
our discussion in subsequent chapters of such matters as continuity, connectedness, and
compactness just how important they really are.

\section{DISTANCE AND NEIGHBORHOODS}
\begin{defn} If $x$ and $a$ are real numbers, the
 \index{distance!between real numbers}%
\df{distance between} $x$ and $a$, which we denote by $d(x,a)$, is defined to be $\abs{x-a}$.
\end{defn}

\begin{exam} There are exactly two real numbers whose distance from the number $3$ is~$7$.
\end{exam}

\begin{proof} We are looking for all numbers $x \in \R$ such that $d(x,3)=7$. In other words
we want solutions to the equation $\abs{x-3}=7$.  There are two such solutions.  If $x-3 \ge
0$, then $\abs{x-3} = x-3$; so $x=10$ satisfies the equation.  On the other hand, if $x-3 <
0$, then $\abs{x-3} = -(x-3) = 3-x$, in which case $x=-4$ is a solution.
\end{proof}


\begin{exer}\label{1exer1} Find the set of all points on the real line which are within $5$
units of the number~$-2$. (Solution~\ref{sol_1exer1}.)
\end{exer}

\begin{prob} Find the set of all real numbers whose distance from $4$ is greater than~$15$.
\end{prob}

\begin{defn} Let $a$ be a point in $\R$ and $\epsilon > 0$.  The open interval
$(a - \epsilon, a + \epsilon)$ centered at $a$ is called the
 \index{epsilon@$\epsilon$-neighborhood}%
 \index{neighborhood!of a point}%
\df{$\epsilon$-neighborhood} of $a$ and is denoted
 \index{<@$J_\epsilon(a)$ ($\epsilon$-neighborhood about~$a$)}%
 \index{j@$J_\epsilon(a)$ ($\epsilon$-neighborhood about~$a$)}%
by~$J_\epsilon(a)$.  Notice that this neighborhood consists of all
numbers $x$ whose distance from $a$ is less than $\epsilon$; that
is, such that $\abs{x-a}<\epsilon$.
\end{defn}

\begin{exam} The $\frac12$\,-\,neighborhood of $3$ is the open interval
$\bigl(\frac52,\frac72\bigr)$.
\end{exam}

\begin{proof} We have $d(x,3) < \frac12$ only if $\abs{x-3} < \frac12$.  Solve this
inequality to obtain $J_\frac12(3) = \bigl(\frac52,\frac72\bigr)$.
\end{proof}

\begin{exam} The open interval $(1,4)$ is an $\epsilon$-neighborhood of an appropriate point.
\end{exam}

\begin{proof} The midpoint of $(1,4)$ is the point $\frac52$.  The distance from this point to
either end of the interval is $\frac32$. Thus $(1,4) = J_\frac32\bigl(\frac52\bigr)$.
\end{proof}

\begin{prob} Find, if possible, a number $\epsilon$ such that the $\epsilon$-neigh\-bor\-hood
of $\frac13$ contains both $\frac14$ and $\frac12$ but does not contain $\frac{17}{30}$.
If such a neighborhood does not exist, explain why.
\end{prob}

\begin{prob} Find, if possible, a number $\epsilon$ such that the $\epsilon$-neigh\-bor\-hood
of $\frac13$ contains $\frac{11}{12}$ but does not contain either $\frac12$ or $\frac58$.  If
such a neighborhood does not exist, explain why.
\end{prob}

\begin{prob} Let $U = \bigl(\frac14, \frac23\bigr)$ and $V = \bigl(\frac12,\frac65\bigr)$.
Write $U$ and $V$ as $\epsilon$-neighborhoods of appropriate points. (That is, find
numbers $a$ and $\epsilon$ such that $U = J_\epsilon(a)$ and find numbers $b$ and
$\delta$ such that $V = J_\delta(b)$.\,)  Also write the sets $U \cup V$ and $U \cap V$
as $\epsilon$-neighborhoods of appropriate points.
\end{prob}

\begin{prob} Generalize your solution to the preceding problem to show that the union and the
intersection of any two $\epsilon$-neighborhoods which overlap is itself an
$\epsilon$-neighborhood of some point.  \emph{Hint.} Since $\epsilon$-neighborhoods are open
intervals of finite length, we can write the given neighborhoods as $(a,b)$ and $(c,d)$.
There are really just two distinct cases.  One neighborhood may contain the other; say, $a \le
c < d \le b$.  Or each may have points that are not in the other; say $a < c < b < d$.  Deal
with the two cases separately.
\end{prob}

\begin{prop}\label{d_less_e} If $a \in \R$ and $0 < \delta \le \epsilon$, then $J_\delta(a)
\subseteq J_\epsilon(a)$.
\end{prop}

\begin{proof} Exercise. (Solution~\ref{sol_d_less_e}.)
 \ns \end{proof}










\section{INTERIOR OF A SET}
\begin{defn} Let $A \subseteq \R$. A point $a$ is an
 \index{interior!point}%
\df{interior point} of $A$ if some $\epsilon$-neighborhood of $a$ lies entirely in~$A$.  That
is, $a$ is an interior point of $A$ if and only if there exists $\epsilon > 0$ such that
$J_\epsilon(a) \subseteq A$. The set of all interior points of $A$ is denoted by $A^\circ$ and
is called the
 \index{<@$A^\circ$ (interior of~$A$)}%
 \index{interior}%
\df{interior} of~$A$.
\end{defn}

\begin{exam} Every point of the interval $(0,1)$ is an interior point of that interval.  Thus
$(0,1)^\circ = (0,1)$.
\end{exam}

\begin{proof} Let $a$ be an arbitrary point in $(0,1)$.  Choose $\epsilon$ to be the smaller of
the two (positive) numbers $a$ and $1-a$.  Then $J_\epsilon(a) = (a - \epsilon, a + \epsilon)
\subseteq (0,1)$ (because $\epsilon \le a$ implies $a-\epsilon \ge 0$, and $\epsilon \le 1-a$
implies $a+\epsilon \le 1$).
\end{proof}

\begin{exam}\label{interior_open_int} If $a < b$, then every point of the interval $(a,b)$ is an
interior point of the interval.  Thus $(a,b)^\circ = (a,b)$.
\end{exam}

\begin{proof} Problem. \ns  \end{proof}

\begin{exam} The point $0$ is not an interior point of the interval~$[0,1)$.
\end{exam}

\begin{proof} Argue by contradiction.  Suppose $0$ belongs to the interior of~$[0,1)$.  Then for
some $\epsilon > 0$ the interval $(-\epsilon,\epsilon) = J_\epsilon(0)$ is contained in
$[0,1)$. But this is impossible since the number $-\frac12\epsilon$ belongs to
$(-\epsilon,\epsilon)$ but not to $[0,1)$.
\end{proof}

\begin{exam} Let $A = [a,b]$ where $a < b$.  Then $A^\circ = (a,b)$.
\end{exam}

\begin{proof} Problem. \ns \end{proof}

\begin{exam}\label{1exer3} Let $A = \{x \in \R\colon x^2-x-6 \ge 0\}$.  Then $A^\circ \ne A$.
\end{exam}

\begin{proof} Exercise. (Solution~\ref{sol_1exer3}.)
  \ns \end{proof}

\begin{prob} Let $A = \{x \in \R\colon x^3 - 2x^2 - 11x + 12 \le 0\}$.  Find $A^\circ$.
\end{prob}

\begin{exam} The interior of the set $\Q$ of rational numbers is empty.
\end{exam}

\begin{proof} No open interval contains only rational numbers.
\end{proof}

\begin{prop}\label{intr_subsets} If $A$ and $B$ are sets of real numbers with $A \subseteq B$,
then $\intr A \subseteq \intr B$.
\end{prop}

\begin{proof} Let $a \in \intr A$.  Then there is an $\epsilon > 0$ such that $J_\epsilon(a)
\subseteq A \subseteq B$.  This shows that $a \in \intr B$.
\end{proof}

\begin{prop} If $A$ is a set of real numbers, then $A^{\circ\circ} = \intr A$.
\end{prop}

\begin{proof} Problem.  \ns  \end{proof}

\begin{prop}\label{intr_intersection} If $A$ and $B$ are sets of real numbers, then
  \[\intr{(A \cap B)} = \intr A \cap \intr B.\]
\end{prop}

\begin{proof} Exercise.  \emph{Hint.}  Show separately that $\intr{(A \cap B)} \subseteq
\intr A \cap \intr B$ and that $\intr A \cap \intr B \subseteq \intr{(A \cap B)}$.
(Solution~\ref{sol_intr_intersection}.) \ns
\end{proof}

\begin{prop}\label{intr_union} If $A$ and $B$ are sets of real numbers, then
   \[ \intr{(A \cup B)} \supseteq \intr A \cup \intr B\,. \]
\end{prop}

\begin{proof} Exercise. (Solution~\ref{sol_intr_union}.) \ns \end{proof}

\begin{exam} Equality may fail in the preceding proposition.
\end{exam}

\begin{proof} Problem.  \emph{Hint.}  See if you can find sets $A$ and $B$ in $\R$ both of
which have empty interior but whose union is all of~$\R$.  \ns
\end{proof}


\endinput
