\chapter{UNIFORM CONVERGENCE}\label{unif_conv}



\section{THE UNIFORM METRIC ON THE SPACE OF BOUNDED FUNCTIONS}
\begin{defn} Let $S$ be a nonempty set. A function $f \colon S \sto \R$ is
 \index{bounded!function}%
 \index{function!bounded}%
\df{bounded} if there exists a number $M \ge 0$ such that
  \[ \abs{f(x)} \le M \qquad \text{for all $x \in S$}\,. \]
We denote by
 \index{bounded@$\fml B(S)$, $\fml B(S,\R)$ (bounded real valued functions on a set)}%
$\fml B(S,\R)$ (or just by $\fml B(S)$) the set of all bounded real valued functions on~$S$.
\end{defn}

\begin{prop}\label{uc_prop1} If $f$ and $g$ are bounded real valued functions on a nonempty set
$S$ and $\alpha$ is a real number, then the functions $f + g$, $\alpha f$, and $fg$ are all
bounded.
\end{prop}

\begin{proof} There exist numbers $M$, $N \ge 0$ such that $\abs{f(x)} \le M$ and $\abs{g(x)}
\le N$ for all $x \in S$. Then, for all $x \in S$
 \begin{align*}
         \abs{(f + g)(x)} &\le \abs{f(x)} + \abs{g(x)} \le M + N, \\
     \abs{(\alpha f)(x)}  &= \abs{\alpha}\,\abs{f(x)} \le \abs{\alpha}\,M, \text{ and}\\
            \abs{(fg)(x)} &= \abs{f(x)}\,\abs{g(x)} \le MN.  \qedhere
 \end{align*}
\end{proof}

\begin{defn}\label{unif_met} Let $S$ be a nonempty set. We define a metric $d_u$ on the set
$\fml B(S,\R)$ by
  \[ d_u(f,g) \equiv \sup\{\abs{f(x) - g(x)} \colon x \in S\} \]
whenever $f$, $g \in \fml B(S,\R)$. The metric $d_u$ is the
 \index{du@$d_u$ (uniform metric)}%
 \index{uniform!metric}%
 \index{metric!uniform}%
\df{uniform metric} on~$\fml B(S,\R)$.
\end{defn}

\begin{exam} Let $S = [-1,1]$ and for all $x \in S$ let $f(x) = \abs{x}$ and $g(x) = \frac12(x-1)$.
Then $d_u(f,g) = 2$.
\end{exam}

\begin{proof} It is clear from the graphs of $f$ and $g$ that the functions are farthest apart at
$x = -1$. Thus
  \begin{align*}
    d_u(f,g) &= \sup\{\abs{f(x) - g(x)} \colon -1 \le x \le 1\} \\
             &= \abs{f(-1) - g(-1)} = 2.   \qedhere
  \end{align*}
\end{proof}

\begin{exam} Let $f(x) = x^2$ and $g(x) = x^3$ for $0 \le x \le 1$.  Then $d_u(f,g) = 4/27$.
\end{exam}

\begin{proof} Let $h(x) = \abs{f(x) - g(x)} = f(x) - g(x)$ for $0 \le x \le 1$.  To maximize $h$
on $[0,1]$ use elementary calculus to find critical points.  Since $h'(x) = 2x - 3x^2 = 0$
only if $x=0$ or $x=\frac23$, it is clear that the maximum value of $h$ occurs at $x =
\frac23$. Thus
  \[ d_u(f,g) = \sup\{h(x) \colon 0 \le x \le 1\}
              = h\biggl(\frac23\biggr) = \frac4{27}\,.  \qedhere \]
\end{proof}

\begin{exer}\label{uc_exer1} Suppose $f$ is the constant function defined on $[0,1]$ whose
value is~$1$.  Asked to describe those functions in $\fml B([0,1])$ which lie in the open ball
about $f$ of radius $1$, a student replies (somewhat incautiously) that $B_1(f)$ is the set of
all real-valued functions $g$ on $[0,1]$ satisfying $0 < g(x) < 2$ for all $x \in [0,1]$. Why
is this response wrong?  (Solution~\ref{sol_uc_exer1}.)
\end{exer}

\begin{prob} Let $f(x) = \sin x$ and $g(x) = \cos x$ for $0 \le x \le \pi$. Find $d_u(f,g)$
in the set of functions $\fml B([0,\pi])$.
\end{prob}

\begin{prob} Let $f(x) = 3x - 3x^3$ and $g(x) = 3x - 3x^2$ for $0 \le x \le 2$. Find $d_u(f,g)$
in the set of functions $\fml B([0,2])$.
\end{prob}

\begin{prob} Explain why it is reasonable to use the same notation $d_u$ (and the same name)
for both the metric in example~\ref{unif_rn} and the one defined in~\ref{unif_met}.
\end{prob}

The terminology in \ref{unif_met} is somewhat optimistic. We have not yet verified that the
``uniform metric'' is indeed a metric on $\fml B(S,\R)$. We now remedy this.

\begin{prop}\label{uc_exer2} Let $S$ be a nonempty set. The function $d_u$ defined
in~\ref{unif_met} is a metric on the set of functions $\fml B(S,\R)$.
\end{prop}

\begin{proof} Exercise. (Solution~\ref{sol_uc_exer2}.)   \ns \end{proof}

\begin{prob} Let $f(x) = x$ and $g(x) = 0$ for all $x \in [0,1]$.  Find a function $h$ in
$\fml B([0,1])$ such that
  \[ d_u(f,h) = d_u(f,g) = d_u(g,h)\,. \]
\end{prob}

\begin{defn}\label{def_uc_R} Let $(f_n)$ be a sequence of real valued functions on a nonempty
set~$S$.  If there is a function $g$ in $\fml F(S,\R)$ such that
  \[ \sup\{\abs{f_n(x) - g(x)} \colon x \in S\} \sto 0 \qquad \text{ as $n \sto \infty$}\]
we say that the sequence $(f_n)$
 \index{convergence!uniform}%
 \index{uniform!convergence}%
\df{converges uniformly} to $g$ and we write
 \index{<@$f_n \sto g \text{\,(unif)}$ (uniform convergence)}%
   \[ f_n \sto g \text{\,(unif)}\,. \]
The function $g$ is the
 \index{uniform!limit}%
 \index{limit!uniform}%
\df{uniform limit} of the sequence $(f_n)$. Notice that if $g$ and all the $f_n$'s belong to
$\fml B(S,\R)$, then uniform convergence of $(f_n)$ to $g$ is the same thing as convergence of
$f_n$ to $g$ in the uniform metric.
\end{defn}

\begin{exam} For each $n \in \N$ and $x \in \R$ let
  \[ f_n(x) = \frac1n\,\sin(nx)\,. \]
Then $f_n \sto \vc 0 \text{\,(unif)}$. (Here $\vc 0$ is the constant function zero.)
\end{exam}

\begin{proof}
  \begin{align*}
     d_u(f_n,0) &= \sup\bigl\{\tfrac1n \abs{\sin nx} \colon x\in \R\bigr\} \\
                &= \frac1n \sto 0 \qquad\text{as $n \sto \infty$}\,.   \qedhere
  \end{align*}
\end{proof}

\begin{exam} Let $g(x) = x$ and $f_n(x) = x + \frac1n$ for all $x \in \R$ and $n \in \N$.
Then $f_n \sto g \text{\,(unif)}$ since
   \[ \sup\{\abs{f_n(x) - g(x)}\colon  x \in \R\} = \frac1n \sto 0
                      \qquad\text{as $n \sto \infty$}.\]
It is not correct (although perhaps tempting) to write $d_u(f_n,g) \sto 0$.  This expression
is meaningless since the functions $f_n$ and $g$ do not belong to the metric space $\fml
B(\R)$ on which $d_u$ is defined.
\end{exam}





\section{POINTWISE CONVERGENCE}
Sequences of functions may converge in many different and interesting ways.  Another mode of
convergence that is frequently encountered is ``pointwise convergence''.


\begin{defn}\label{def_pc_R}  Let $(f_n)$ be a sequence of real valued functions on a nonempty
set~$S$.  If there is a function $g$ such that
   \[ f_n(x) \sto g(x) \qquad\text{for all $x \in S$} \]
then $(f_n)$
 \index{pointwise!convergence}%
 \index{convergence!pointwise}%
\df{converges pointwise} to~$g$. In this case we write
 \index{<@$f_n \sto g \text{\,(ptws)}$ (pointwise convergence)}%
   \[ f_n \sto g \text{\,(ptws)} \,. \]
The function $g$ is the
 \index{pointwise!limit}%
 \index{limit!pointwise}%
\df{pointwise limit} of the sequence~$f_n$.
\end{defn}

In the following proposition we make the important, if elementary, observation that uniform
convergence is stronger than pointwise convergence.

\begin{prop}\label{uc_vs_pwc} Uniform convergence implies pointwise convergence, but not conversely.
\end{prop}

\begin{proof} Exercise. (Solution~\ref{sol_uc_vs_pwc}.)   \ns \end{proof}

\begin{prob} Find an example of a sequence $(f_n)$ of functions in $\fml B([0,1])$ which
converges pointwise to the zero function $\vc 0$ but satisfies
  \[ d_u(f_n,\vc 0) \sto \infty \qquad \text{as $n \sto \infty$}\,. \]
\end{prob}

Next we show that the uniform limit of a sequence of bounded functions is itself bounded.

\begin{prop}\label{unif_lim_bdd} Let $(f_n)$ be a sequence in $\fml B(S)$ and $g$ be a
real valued function on~$S$.
 \begin{enumerate}
  \item[(a)] If $f_n \sto g \text{\,(unif)}$, then $g \in \fml B(S)$.
  \item[(b)] The assertion in (a) does not hold if uniform convergence is replaced by
pointwise convergence.
 \end{enumerate}
\end{prop}

\begin{proof} Exercise.  (Solution~\ref{sol_unif_lim_bdd}.) \ns \end{proof}

\begin{exer}\label{uc_exer3} Let $f_n(x) = x^n - x^{2n}$ for $0 \le x \le 1$ and $n \in \N$.
Does the sequence $(f_n)$ converge pointwise on $[0,1]$? Is the convergence uniform?
(Solution~\ref{sol_uc_exer3}.)
\end{exer}

\begin{prob} Given in each of the following is the $n^{\text{th}}$ term of a sequence of
real valued functions defined on~$[0,1]$. Which of these converge pointwise on~$[0,1]$?  For
which is the convergence uniform?
 \begin{enumerate}
  \item[(a)] $x \mapsto x^n$.
  \item[(b)] $x \mapsto nx$.
  \item[(c)] $x \mapsto xe^{-nx}$.
 \end{enumerate}
\end{prob}

\begin{prob} Given in each of the following is the $n^{\text{th}}$ term of a sequence of real
valued functions defined on $[0,1]$.  Which of these converge pointwise on $[0,1]$? For which
is the convergence uniform?
 \begin{enumerate}
  \item[(a)] $x \mapsto \dfrac1{\mathstrut nx+1}$.
\vskip 2 pt
  \item[(b)] $x \mapsto \dfrac{\mathstrut x}{\mathstrut nx+1}$.
\vskip 2 pt
  \item[(c)] $x \mapsto \dfrac{\mathstrut x^2}n - \dfrac{\mathstrut x}{n^2}$.
 \end{enumerate}
\end{prob}

\begin{prob} Let $f_n(x) = \dfrac{(n-1)x + x^2}{\mathstrut n + x}$ for all $x \ge 1$ and
$n \in \N$. Does the sequence $(f_n)$ have a pointwise limit on $\mathstrut [1,\infty)$?  A
uniform limit?
\end{prob}
