\chapter{THE IMPLICIT FUNCTION THEOREM}\label{IFT}

This chapter deals with the problem of solving equations and systems of equations for some of
the variables which appear in terms of the others.  In a few very simple cases this can be
done explicitly.  As an example, consider the equation
  \begin{equation}\label{eqn_circ5}
      x^2 + y^2 = 25
  \end{equation}
for the circle of radius $5$ centered at the origin in~$\R^2$.  Although~\eqref{eqn_circ5} can
\emph{not} be solved \emph{globally} for $y$ in terms of $x$ (that is, there is no function
$f$ such that $y = f(x)$ for all points $(x,y)$ satisfying~\eqref{eqn_circ5}),  it
nevertheless \emph{is} possible at most points on the circle to solve \emph{locally} for $y$
in terms of~$x$.  For example, if $(a,b)$ lies on the circle and $b > 0$, then there exist
open intervals $J_1$ and $J_2$ containing $a$ and $b$, respectively, and a function $f \colon
J_1 \sto J_2$ such that every point $(x,y)$ in the rectangular region $J_1 \times J_2$ will
satisfy $y = f(x)$ if and only if it satisfies equation~\eqref{eqn_circ5}.  In particular, we
could choose $J_1 = (-5,5)$, $J_2 = (0,6)$, and $f \colon x \mapsto \sqrt{25 - x^2}$. In case
$b < 0$ the function $f$ would be replaced by  $f:x \mapsto - \sqrt{25 - x^2}$.  If $b = 0$
then there is \emph{no} local solution for $y$ in terms of~$x$: each rectangular region about
either of the points $(5,0)$ or $(-5,0)$ will contain pairs of points symmetrically located
with respect to the  $x$-axis which satisfy~\eqref{eqn_circ5}; and it is not possible for the
graph of a function to contain such a pair.

Our attention in this chapter is focused not on the relatively rare cases where it is possible
to compute explicit (local) solutions for some of the variables in terms of the remaining ones
but on the more typical situation where no such computation is possible.  In this latter
circumstance it is valuable to have information concerning the \emph{existence} of (local)
solutions and the differentiability of such solutions.

The simplest special case is a single equation of the form $y = f(x)$ where $f$ is a
continuously differentiable function.  The \emph{inverse function theorem}, derived in the
first section of this chapter, provides conditions under which this equation can be solved
locally for $x$ in terms of~$y$, say $x = g(y)$, and gives us a formula allowing us to compute
the differential of $g$. More complicated equations and systems of equations require the
\emph{implicit function theorem}, which is the subject of the second section of the chapter.






\section{THE INVERSE FUNCTION THEOREM} Recall that in chapter~\ref{diff_calc} formulas concerning
the function $f \mapsto \Delta f$ lead to corresponding formulas involving differentials. For
example, $d(f+g)_a = df_a + dg_a$ followed from $\Delta (f+g)_a = \Delta f_a + \Delta g_a$
(see~\ref{prop_diff_sum}).  It is natural to ask whether the formula
  \[ \Delta \bigl(f^{-1}\bigr)_{f(x)} = (\Delta f_x)^{-1} \]
derived for bijective functions $f$ in proposition~\ref{prop_del_inv} leads to a corresponding
formula
 \begin{equation}\label{eqn_diff_inv}
     d\bigl(f^{-1}\bigr)_{f(x)} = (df_x)^{-1}
 \end{equation}
for differentials.  Obviously, a necessary condition for~\eqref{eqn_diff_inv} to hold for all
$x$ in some neighborhood of a point $a$ is that the linear map $df_a$ be invertible.  The
 \index{inverse!function theorem}%
\emph{inverse function theorem} states that for continuously differentiable (but not
necessarily bijective) functions this is all that is required.  The proof of the \emph{inverse
function theorem} is a fascinating application of the \emph{contractive mapping theorem}
(theorem~\ref{cmthm}).  First some terminology.

\begin{defn}  Let $E$ and $F$ be Banach spaces and $\emptyset \ne \open UE$.  A function $f$
belonging to  $\fml C^1(U,F)$  is
 \index{c@$\fml C^1$-invertible}%
 \index{invertible!$\fml C^1$-}%
$\fml C^1$-\df{invertible} if $f$ is a bijection between $U$ and an open subset $V$ of $F$ and
if $f^{-1}$ belongs to $\fml C^1(V,E)$.  Such a function is also called a
 \index{c@$\fml C^1$-isomorphism}%
 \index{isomorphism!$\fml C^1$-}%
$\fml C^1$-\df{isomorphism} between $U$ and~$V$.
\end{defn}

\begin{exer}\label{exer_not_c1}  Find nonempty open subsets $U$ and $V$ of $\R$ and a
continuously differentiable bijection $f \colon U \sto V$ which is \emph{not} a $\fml
C^1$-isomorphism between $U$ and~$V$. (Solution~\ref{sol_exer_not_c1}.)
\end{exer}

\begin{defn}  Let $E$ and $F$ be Banach spaces.  A function $f$ in $\fml F_a(E,F)$ is
 \index{locally $\fml C^1$-invertible}%
 \index{c@$\fml C^1$-invertible!locally}%
 \index{invertible!locally $\fml C^1$-}%
\df{locally $\fml C^1$-invertible} (or a
 \index{local!$\fml C^1$-isomorphism}%
 \index{c@$\fml C^1$-isomorphism!local}%
 \index{isomorphism!local $\fml C^1$-}%
\df{local $\fml C^1$-isomorphism}) at a point $a$ in $E$ if there exists a neighborhood of $a$
on which the restriction of $f$ is $\fml C^1$-invertible.  The inverse of this restriction is
a
 \index{local!$\fml C^1$-inverse}%
 \index{c@$\fml C^1$-inverse!local}%
 \index{inverse!local $\fml C^1$-}%
\df{local $\fml C^1$-inverse} of $f$ at $a$ and is denoted
 \index{<@$f_{\text{loc}}^{-1}$ (local $\fml C^1$-inverse of~$f$)}%
by~$f_{\text{loc}}^{-1}$.
\end{defn}

\begin{exer}\label{exer_c1_inv}  Let $f(x) = x^2 - 6x + 5$  for all $x$ in~$\R$.  Find a local
$\fml C^1$-inverse of $f$ at $x = 1$. (Solution~\ref{sol_exer_c1_inv}.)
\end{exer}

\begin{prob}  Let $f(x) = x^6 - 2x^3 -7$ for all $x$ in~$\R$.  Find local $\fml C^1$-inverses
for $f$ at $0$ and at~$10$.
\end{prob}

\begin{prob}  Find a nonempty open subset $U$ of $\R$ and a function $f$ in $\fml C^1(U,\R)$
which is \emph{not} $\fml C^1$-invertible but is locally $\fml C^1$-invertible at every point
in~$U$.
\end{prob}

Before embarking on a proof of the \emph{inverse function theorem} it is worthwhile seeing why
a naive ``proof'' of this result using the chain rule fails---even in the simple case of a
real valued function of a real variable.

\begin{exer}\label{exer_der_finv}  If $f \in \fml F_a(\R,\R)$ and if $Df(a) \ne 0$, then $f$
is locally $\fml C^1$-invertible at $a$ and
  \begin{equation}\label{eqn_der_finv}
      Df^{-1}_{\text{loc}}(b) = \frac1{Df(a)}
  \end{equation}
where $\locinv f$ is a local $\fml C^1$-inverse of $f$ at $a$ and $b = f(a)$.

This assertion is correct.  Criticize the following ``proof'' of the result:  Since $\locinv
f$ is a local $\fml C^1$-inverse of $f$ at $a$
  \[ \locinv f(f(x)) = x \]
for all $x$ in some neighborhood $U$ of~$a$.  Applying the chain rule (proposition~\ref{c_ru})
we obtain
  \[ (D\locinv f)(f(x)) \cdot Df(x)  =  1 \]
for all $x$ in~$U$.  Letting $x = a$ we have
  \[ (D\locinv f)(b)\,Df(a)  =  1 \]
and since $Df(a) \ne 0$  equation~\eqref{eqn_der_finv} follows.
(Solution~\ref{sol_exer_der_finv}.)
\end{exer}

The \emph{inverse function theorem} (\ref{thm_inft3}) deals with a continuously differentiable
function $f$ which is defined on a neighborhood of a point $a$ in a Banach space $E$ and which
maps into a second Banach space $F$.  We assume that the differential of $f$ at $a$ is
invertible.  Under these hypotheses we prove that $f$ is locally $\fml C^1$-invertible at $a$
and in some neighborhood of $a$ equation~\eqref{eqn_diff_inv} holds.  To simplify the proof we
temporarily make some additional assumptions: we suppose that $H$ is a continuously
differentiable function which is defined on a neighborhood of $0$ in a Banach space $E$ and
which maps into this same space $E$, that $H(\vc 0) = \vc 0$, and that the differential of $H$
at $0$ is the identity map on $E$.  Once the conclusion of the \emph{inverse function theorem}
has been established in this restricted case the more general version follows easily. The
strategy we employ to attack the special case is straightforward, but there are numerous
details which must be checked along the way.  Recall that in chapter~\ref{fpt} we were able to
solve certain systems of simultaneous linear equations by putting them in the form $Ax = b$
where $A$ is a square matrix and $x$ and $b$ are vectors in the Euclidean space of appropriate
dimension. This equation was rewritten in the form $Tx = x$ where $Tx := x - b + Ax$, thereby
reducing the problem to one of finding a fixed point of the mapping $T$. When $T$ is
contractive a simple application of the \emph{contractive mapping theorem} (\ref{cmthm}) is
all that is required. We make use of exactly the same idea here.  We want a local inverse of
$H$.  That is, we wish to solve the equation $H(x) = y$ for $x$ in terms of $y$ in some
neighborhood of~$\vc 0$. Rewrite the equation $H(x) = y$ in the form $\phi_y(x) = x$ where for
each $y$ near $\vc 0$ the function $\phi_y$ is defined by $\phi_y(x) := x - H(x) + y$.  Thus,
as before, the problem is to find for each $y$ a unique fixed point of $\phi_y$. In order to
apply the \emph{contractive mapping theorem} to $\phi_y$, the domain of this function must be
a \emph{complete} metric space. For this reason we choose temporarily to take the domain of
$\phi_y$ to be a \emph{closed} ball about the origin in~$E$.

In lemma~\ref{lem_inft1} we find such a closed ball~$C$.  It must satisfy two conditions:
first, $C$ must lie in the domain of $H$; and second, if $u$ belongs to $C$, then $dH_u$ must
be close to the identity map on $E$, say, $\norm{dH_u - I} < \frac12$.  (This latter condition
turns out to be a crucial ingredient in proving that $\phi_y$ is contractive.)  In
lemma~\ref{lem_inft2} we show that (for $y$ sufficiently small) $\phi_y$ maps the closed ball
$C$ into itself; and in~\ref{lem_inft3} the basic task is to show that $\phi_y$ is contractive
and therefore has a unique fixed point. The result of all this is that there exists a number
$r > 0$ such that for every $y$ in $B = B_r(0)$ there exists a unique $x$ in the closed ball
$C = C_{2r}(\vc 0)$ such that $y = H(x)$. Now this is not quite the end of the story.  First
of all we do \emph{not} know that $H$ restricted to $C$ is injective: some points in $C$ may
be mapped to the region outside $B$, about which the preceding says nothing. Furthermore, the
definition of local $\fml C^1$-invertibility requires a homeomorphism between \emph{open}
sets, and $C$ is not open.  This suggests we restrict our attention to points lying in the
interior of $C$ which map into~$B$.  So let $V = \intr C \cap H^{\gets}(B)$ and consider the
restriction of $H$ to $V$, which we denote by $H_{\text{loc}}$.  In lemma~\ref{lem_inft4} we
show that $V$ is a neighborhood of $0$ and that $H_{\text{loc}}$ is injective.  Thus the
inverse function $\locinv H: H^{\sto}(V) \sto V$ exists.  The succeeding lemma is devoted to
showing that this inverse is continuous.

In order to conclude that $H$ is locally $C^1$-invertible we still need two things: we must
know that $H_{\text{loc}}$ is a homeomorphism between \emph{open} sets and that $\locinv H$ is
continuously differentiable.  Lemma~\ref{lem_inft6} shows that $H^{\sto}(V)$ is open in $E$.
And in lemma~\ref{lem_inft7} we complete the proof of this special case of the \emph{inverse
function theorem} by showing that $\locinv H$ is continuously differentiable and that in the
open set $V$ its differential is given by~\eqref{eqn_diff_inv}.

Corollary~\ref{cor_inft} shows that the conclusions of the preceding result remain true even
when one of the hypotheses is eliminated and another weakened.  Here we prove the
\emph{inverse function theorem} for a function $G$ whose domain $E$ and codomain $F$ are
\emph{not} required to be identical.  Of course, if $E \ne F$ we cannot assume that the
differential of $G$ at $\vc 0$ is the identity map; we assume only that it is invertible.

Finally, in theorem~\ref{thm_inft3} we prove our final version of
the \emph{inverse function theorem}.  Here we drop the requirement
that the domain of the function in question be a neighborhood of
the origin.

In \ref{lem_inft1}--\ref{lem_inft7} the following hypotheses are in
force:
 \begin{enumerate}
  \item[(1')] $\vc 0 \in \open{U_1}E$ (where $E$ is a Banach space);
  \item[(2')] $H \in \fml C^1(U_1,E)$;
  \item[(3')] $H(\vc 0) = \vc 0$; and
  \item[(4')] $dH_{\vc 0} = I$.
 \end{enumerate}

\begin{lem}\label{lem_inft1}  There exists $r > 0$ such that $B_{3r}(\vc 0) \subseteq U_1$ and
$\norm{dH_u - I} < \frac12$ whenever $\norm u \le 2r$.
\end{lem}

\begin{proof} Problem.   \ns  \end{proof}

\begin{lem}\label{lem_inft2}  For $\norm y < r$ define a function $\phi_y$ by
  \[ \phi_y(x) := x - H(x) + y \]
for all $x$ such that $\norm x \le 2r$.  Show that $\phi_y$ maps $C_{2r}(\vc 0)$ into itself.
\end{lem}

\begin{proof} Problem.   \ns  \end{proof}

\begin{lem}\label{lem_inft3} For every $y$ in $B_r(\vc 0)$ there exists a unique $x$ in
$C_{2r}(\vc 0)$ such that $y = H(x)$.
\end{lem}

\begin{proof} Problem.  \emph{Hint.}  Show that the function $\phi_y$ defined in
lemma~\ref{lem_inft2} is contractive on the metric space $C_{2r}(\vc 0)$ and has $\frac12$ as
a contraction constant.  To find an appropriate inequality involving $\norm{\phi_y(u) -
\phi_y(v)}$ apply corollary~\ref{cor_mvt_nls} to $\norm{H(v) - H(u) - dH_{\vc 0}(v - u)}$.
\ns
\end{proof}

\begin{lem}\label{lem_inft4} Show that if $V := \{x \in B_{2r}(\vc 0) \colon \norm{H(x)} < r\}$,
then $\vc 0 \in \open VE$.  Let $H_{\text{loc}}$ be the restriction of $H$ to~$V$.  Show that
$H_{\text{loc}}$ is a bijection between $V$ and $H^{\sto}(V)$.
\end{lem}

\begin{proof} Problem.  \ns  \end{proof}

\begin{lem}\label{lem_inft5} The function $\locinv H \colon H^{\sto}(V) \sto V$ is continuous.
\end{lem}

\begin{proof} Problem.  \emph{Hint.}  Prove first that if $u$, $v \in C_{2r}(\vc 0)$, then
$\norm{u - v} \le 2\norm{H(u) - H(v)}$.  In order to do this, look at
  \[ 2\norm{\phi_{\vc 0}(u) + H(u) - \phi_{\vc 0}(v) - H(v)} - \norm{u - v} \]
and recall (from the proof of \ref{lem_inft3}) that $\phi_{\vc 0}$ has contraction constant
$\frac12$.  Use this to conclude that if $w$, $z \in H^{\sto}(V)$,  then
  \[ \norm{\locinv H(w) - \locinv H(z)} \le 2\,\norm{w - z} \]
where $H_{\text{loc}}$ is the restriction of $H$ to $V$ (see lemma~\ref{lem_inft3}.  \ns
\end{proof}

\begin{lem}\label{lem_inft6} Show that $\open{H^{\sto}(V)}E$.
\end{lem}

\begin{proof} Problem.  \emph{Hint.}  Show that if a point $b$ belongs to $H^{\sto}(V)$, then
so does the open ball $B_{r - \norm b}(b)$.  Proceed as follows: Show that if a point $y$ lies
in this open ball, then $\norm y < r$ and therefore $y = H(x)$ for some (unique) $x$ in
$C_{2r}(\vc 0)$.  Prove that $y \in H^{\sto}(V)$ by verifying $\norm x < 2r$.  To do this look
at
  \[ \norm{x - \locinv H(b)} + \norm{\locinv H(b) - \locinv H(\vc 0)} \]
and use the first inequality given in the hint to the preceding problem.    \ns
\end{proof}

\begin{lem}\label{lem_inft7}  The function $H$ is locally $\fml C^1-invertible$ at~$0$.  Furthermore,
  \[ d\bigl(\locinv H\bigr)_{H(x)} = (dH_x)^{-1} \]
for every $x$ in~$V$.
\end{lem}

\begin{proof} Problem.   \emph{Hint.}  First prove the differentiability of $\locinv H$ on
$H^{\sto}(V)$.  If $y \in H^{\sto}(V)$, then there exists a unique $x$ in $V$ such that $y =
H(x)$.  By hypothesis $\Delta H_x \sim dH_x$. Show that multiplication on the right by $\Delta
\bigl(\locinv H\bigr)_y$ preserves tangency.  (For this it must be established that $\Delta
\bigl(\locinv H\bigr)_y$ belongs to $\ofml O(E,E)$.) Then show that multiplication on the left
by $(dH_x)^{-1}$ preserves tangency. (How do we know that this inverse exists for all $x$ in
$V$?) Finally show that the map $y \mapsto d\bigl(\locinv H\bigr)_y$ is continuous on
$H^{\sto}(V)$ by using \eqref{eqn_diff_inv} to write it as the composite of $\locinv H$, $dH$,
and the map $T \mapsto T^{-1}$ on $\inv\,\ofml B(E,E)$ (see proposition~\ref{prop_inv_cont}).
\ns \end{proof}

\begin{cor}[A second, more general, version of the inverse function theorem]\label{cor_inft}
Let $E$ and $F$ be Banach spaces. If
 \begin{enumerate}
  \item[(1'')] $\vc 0 \in \open{U_1}E$,
  \item[(2'')] $G \in \fml C^1(U_1,F)$,
  \item[(3'')] $G(\vc 0) = \vc 0$, and
  \item[(4'')] $dG_{\vc 0} \in \inv\,\ofml B(E,F)$,
 \end{enumerate}
then $G$ is locally $\fml C^1$-invertible at~$\vc 0$.  Furthermore,
  \[ d\bigl(\locinv G\bigr)_{G(x)} = (dG_x)^{-1} \]
for all $x$ in some neighborhood of $\vc 0$, where $\locinv G$ is a local $\fml C^1$-inverse
of $G$ at~$\vc 0$.
\end{cor}

\begin{proof} Problem. \emph{Hint.}  Let $H = (dG_{\vc 0})^{-1} \circ G$. Apply lemma~\ref{lem_inft7}.
\ns
\end{proof}

 \index{inverse function theorem}%
\begin{thm}[Inverse Function Theorem (third, and final, version)]\label{thm_inft3} Let $E$ and
$F$ be Banach spaces.  If
 \begin{enumerate}
  \item[(1)] $a \in \open UE$,
  \item[(2)] $f \in \fml C^1(U,F)$, and
  \item[(3)] $df_a \in \inv\,\ofml B(E,F)$,
 \end{enumerate}
then $f$ is locally $\fml C^1$-invertible at $a$.  Furthermore,
  \[ d\bigl(\locinv f\bigr)_{f(x)} = (df_x)^{-1} \]
for all $x$ in some neighborhood of~$a$.
\end{thm}

\begin{proof}  Problem.    \emph{Hint.}   Let $U_1 = U - a$ and $G = \Delta f_a$.  Write $G$ as a
composite of $f$ with translation maps.  Apply corollary~\ref{cor_inft}.   \ns
\end{proof}

\begin{prob} Let $P \colon \R^2 \sto \R^2 \colon (r,\theta) \mapsto (r\cos\theta, r\sin\theta)$
and $a$ be a point in $\R^2$ such that $P(a) = (1, \sqrt 3)$.
 \begin{enumerate}
  \item[(a)] Show that $P$ is locally $\fml C^1$-invertible at $a$ by finding a local $\fml
C^1$-inverse $\locinv P$ of $P$ at~$a$.  For the inverse you have found, compute
$d\bigl(\locinv P\bigr)_{(1,\sqrt 3)}$.
  \item[(b)] Use the \emph{inverse function theorem}~\ref{thm_inft3} to show that $P$ is locally
$\fml C^1$-invertible at $a$.  Then use the formula given in that theorem to compute
$d\bigl(\locinv P\bigr)_{(1,\sqrt 3)}$ (where $\locinv P$ is a local $\fml C^1$-inverse of $P$
at~$a$).  \emph{Hint.}  Use proposition~\ref{prop_matr_inv}.
 \end{enumerate}
\end{prob}

\begin{prob} Let $U = \{(x,y,z) \in \R^3 \colon  x,y,z > 0\}$ and let $g \colon U \sto \R^3$ be
defined by
  \[ g(x,y,z) =\biggl(\frac x{y^2z^2}\,,\,yz\,,\, \ln y\biggr)\,. \]
Calculate separately $\bigl[dg_{(x,y,z)}\bigr]^{-1}$ and $\bigl[d(g^{-1})_{g(x,y,z)}\bigr]$.
\end{prob}












\section{THE IMPLICIT FUNCTION THEOREM}
In the preceding section we derived the \emph{inverse function theorem}, which gives
conditions under which an equation of the form $y = f(x)$ can be solved locally for $x$ in
terms of $y$. The \emph{implicit function theorem} deals with the local solvability of
equations that are not necessarily in the form $y = f(x)$ and of systems of equations. The
\emph{inverse function theorem} is actually a special case of the \emph{implicit function
theorem}. Interestingly, the special case can be used to prove the more general one.

This section consists principally of exercises and problems which illustrate how the
\emph{inverse function theorem} can be adapted to guarantee the existence of local solutions
for various examples of equations and systems of equations. Once the computational procedure
is well understood for these examples it is a simple matter to explain how it works in
general; that is, to prove the \emph{implicit function theorem}.

Suppose, given an equation of the form $y = f(x)$ and a point $a$ in the domain of~$f$, we are
asked to show that \emph{the equation can be solved for $x$ in terms of $y$ near $a$}. It is
clear what we are being asked: to show that $f$ is locally invertible at $a$. (Since the
function $f$ will usually satisfy some differentiability condition---continuous
differentiability, for example---it is natural to ask for the local inverse to satisfy the
same condition.)

As we have seen, local invertibility can be established by explicitly computing a local
inverse, which can be done only rarely, or by invoking the \emph{inverse function theorem}.
Let us suppose we are given an equation of the form
  \begin{equation}\label{ift_eq1}
     f(x,y) = 0
  \end{equation}
and a point $(a,b)$ in $\R^2$ which satisfies~\eqref{ift_eq1}.

\noindent\emph{Question.} What does it mean to say that \eqref{ift_eq1} can be solved for $y$
near $b$ in terms of $x$ near~$a$?  (Alternative wording: What does it mean to say that
\eqref{ift_eq1} can be solved for $y$ in terms of $x$ near the point~$(a,b)$?)

\noindent\emph{Answer.} There exist a neighborhood $V$ of $a$ and a function  $h \colon V \sto
\ R$ which satisfy
  \begin{alignat*}{2}
    &\text{(i)}  &&\quad h(a) = b\\
    &\text{(ii)} &&\quad f(x,h(x)) = 0
  \end{alignat*}
for all $x$ in~$V$.

\begin{exam} In the introduction to this chapter we discussed the problem of solving the equation
  \[ x^2 + y^2 = 25 \]
for $y$ in terms of~$x$.  This equation can be put in the form \eqref{ift_eq1} by setting
$f(x,y) = x^2 + y^2 - 25$.  Suppose we are asked to show that \eqref{ift_eq1} can be solved
for $y$ near $4$ in terms of $x$ near~$3$.  As in the introduction, take $V = (- 5,5)$ and
$h(x) = \sqrt{25 - x^2}$ for all $x$ in~$V$.  Then $h(3) = 4$ and $f(x,h(x)) = x^2 +
\bigl(\sqrt{25 - x^2}\bigr)^2 -25 = 0$; so $h$ is the desired local solution
to~\eqref{ift_eq1}. If we are asked to show that \eqref{ift_eq1} can be solved for $y$ near
$-4$ in terms of $x$ near 3, what changes? We choose $h(x) = -\sqrt{25 - x^2}$.  Notice that
condition (i) above dictates the choice of $h$. (Either choice will satisfy~(ii).)
\end{exam}

As was pointed out in the introduction, the preceding example is atypical in that it is
possible to \emph{specify} the solution. We can actually \emph{solve} for $y$ in terms of~$x$.
\emph{Much} more common is the situation in which an explicit solution is not possible.  What
do we do then?

To see how this more complicated situation can be dealt with, let us pretend just for a moment
that our computational skills have so totally deserted us that in the preceding example we are
unable to specify the neighborhood $V$ and the function $h$ required to solve~\eqref{ift_eq1}.
The problem is still the same: show that the equation
  \begin{equation}\label{ift_eq2}
     x^2 + y^2 - 25 = 0
  \end{equation}
can be solved for $y$ near $4$ in terms of $x$ near~$3$.  A good start is to define a function
$G \colon \R^2 \sto \R^2$ by
   \begin{equation}\label{ift_eq3}
      G(x,y) = (x,f(x,y))
   \end{equation}
where as above $f(x,y) = x^2 + y^2 - 25$, and apply the \emph{inverse function theorem}
to~$G$.  It is helpful to make a sketch here.  Take the $xy$-plane to be the domain of $G$ and
in this plane sketch the circle $x^2 + y^2 = 25$.  For the codomain of $G$ take another plane,
letting the horizontal axis be called ``$x$'' and the vertical axis be called~``$z$''.  Notice
that in this second plane the image under $G$ of the circle drawn in the $xy$-plane is the
line segment $[- 5,5]$ along the $x$-axis and that the image of the $x$-axis is the parabola
$z = x^2 - 25$. Where do points in the interior of the circle go?  What about points outside
the circle? If you think of the action of $G$ as starting with a folding of the $xy$-plane
along the $x$-axis, you should be able to guess the identity of those points where $G$ is not
locally invertible.  In any case we will find these points using the \emph{inverse function
theorem}.

The function $G$ is continuously differentiable on $\R^2$ and (the matrix representation of)
its differential at a point $(x,y)$ in $\R^2$ is given by
  \[ \bigl[dG_{(x,y)}\bigr] = \bmatrix 1  & 0 \\ 2x & 2y \endbmatrix\,. \]
Thus according to the \emph{inverse function theorem} $G$ is locally $\fml C^1$-invertible at
every point $(x,y)$ where this matrix is invertible, that is, everywhere except on the
$x$-axis. In particular, at the point $(3,4)$ the function $G$ is locally $\fml
C^1$-invertible. Thus there exist a neighborhood $W$ of $G(3,4) = (3,0)$ in $\ R^2$ and a
local $\fml C^1$-inverse $H \colon W \sto \R^2$ of $G$.  Write $H$ in terms of its component
functions $H = \bigl(H^1,H^2\bigr)$ and set $h(x) = H^2(x,0)$ for all $x$ in $V := \{x:(x,0)
\in W\}$.  Then $V$ is a neighborhood of $3$ in $\R$ and the function $h$ is continuously
differentiable (because $H$ is). To show that $h$ is a solution of the equation
\eqref{ift_eq2} for $y$ in terms of $x$ we must show that
  \begin{alignat*}{2}
     &\text{(i)}  &&\quad h(3) = 4; \quad \text{ and} \\
     &\text{(ii)} &&\quad f(x,h(x)) = 0
  \end{alignat*}
for all $x$ in $V$.

To obtain (i) equate the second components of the first and last terms of the following
computation.
  \begin{align*}
     (3,4) &= H(G(3,4)) \\
           &= H(3, f(3,4)) \\
           &= H(3,0) \\
           &= (H^1(3,0)\,,\,H^2(3,0)) \\
           &= (H^1(3,0)\,,\,h(3)) .
  \end{align*}
To obtain (ii) notice that for all $x$ in $V$
  \begin{align*}
    (x,0) &= G(H(x,0)) \\
          &= G(H^1(x,0)\,,\,H^2(x,0)) \\
          &= G(H^1(x,0)\,,\,h(x)) \\
          &= (H^1(x,0)\,,\,f(H^1(x,0),h(x)))\,.
  \end{align*}
Equating first components we see that $H^1(x,0) = x$. So the
preceding can be written
  \[ (x,0) = (x,f(x,h(x))) \]
from which (ii) follows by equating second components.

Although the preceding computations demonstrate the existence of a local solution $y = h(x)$
without specifying it, it is nevertheless possible to calculate the value of its derivative
$\dfrac{dy}{dx}$ at the point $(3,4)$, that is, to find $h'(3)$. Since $h'(3) = \bigl(H^2
\circ j_{{}_{\sst 1}}\bigr)'(3) = d\bigl(H^2 \circ j_{{}_{\sst 1}}\bigr)_3(1) =
\bigl(dH^2{}_{(3,0)} \circ j_{{}_{\sst 1}}\bigr)(1) = dH^2{}_{(3,0)}(1,0) = \pd {H^2}x(3,0)$
(where $j_{{}_{\sst 1}}$ is the inclusion map $x \mapsto (x,0)$ ) and since $H$ is a local
inverse of $G$, the \emph{inverse function theorem} tells us that
  \begin{align*}
     \bigl[dH_{(3,0)}\bigr] &= \bigl[dH_{G(3,4)}\bigr] \\
                            &= \bigl[dG_{(3,4)}\bigr]^{-1} \\
                            &= {\begin{bmatrix}
                                   1         &      0   \\
                                   f_1(3,4)  &  f_2(3,4)
                                 \end{bmatrix}}^{-1} \\
                            &= {\begin{bmatrix}
                                   1  &  0 \\
                                   6  &  8
                                \end{bmatrix}}^{-1} \\
                            &= \begin{bmatrix}
                                   1         &     0 \\
                                   -\frac34  &  \frac18
                               \end{bmatrix} \,.
  \end{align*}
But the entry in the lower left corner is $\pd{H^2}x(3,0)$.  Therefore, $h'(3) = - \frac34$.

In the next exercise we consider an equation whose solution cannot be easily calculated.

\begin{exer}\label{xmpl_ift_12} Consider the equation
  \begin{equation}\label{ift_12_eq1}
        x^2y + \sin\left(\frac{\pi}2 xy^2\right) = 2
  \end{equation}
where $x$, $y \in \ R$
 \begin{enumerate}
  \item[(a)] What does it mean to say that equation~\eqref{ift_12_eq1} can be solved for $y$ in
terms of $x$ near the point~$(1,2)$?
  \item[(b)]  Show that it is possible to solve~\eqref{ift_12_eq1} as described in (a).
\emph{Hint.}  Proceed as in the second solution to the preceding example.
  \item[(c)] Use the \emph{inverse function theorem} to find the value of $\frac{dy}{dx}$
at the point~$(1,2)$.
 \end{enumerate}
(Solution~\ref{sol_xmpl_ift_12}.)
\end{exer}

\begin{prob}  Consider the equation
  \begin{equation}\label{ift_12_eq2}
       e^{xy^2} - x^2y + 3x = 4
  \end{equation}
where $x$, $y \in \R$.
 \begin{enumerate}
  \item[(a)] What does it mean to say that equation~\eqref{ift_12_eq2} can be solved for $y$
near 0 in terms of $x$ near~$1$?
  \item[(b)] Show that such a solution does exist.
  \item[(c)] Use the \emph{inverse function theorem} to compute the value of $\frac{dy}{dx}$ at
the point~$(1,0)$.
 \end{enumerate}
\end{prob}

The preceding examples have all been equations involving only two variables.  The technique
used in dealing with these examples works just as well in cases where we are given an equation
in an arbitrary number of variables and wish to demonstrate the existence of local solutions
for one of the variables in terms of the remaining ones.

\begin{exer}\label{xmpl_ift_13} Consider the equation
  \begin{equation}\label{ift_13_eq1}
        x^2z + yz^2 - 3z^3 = 8
  \end{equation}
for $x$, $y$, $z \in \R$.
 \begin{enumerate}
  \item[(a)] What does it mean to say that equation~\eqref{ift_13_eq1} can be solved for $z$
near $1$ in terms of $x$ and $y$ near~$(3,2)$?  (Alternatively: that~\eqref{ift_13_eq1} can be
solved for $z$ in terms of $x$ and $y$ near the point $(x,y,z) = (3,2,1)$?)
  \item[(b)] Show that such a solution does exist. \emph{Hint.} Follow the preceding technique,
but instead of using~\eqref{ift_eq3}, define $G(x,y,z) := (x,y,f(x,y,z))$ for an appropriate
function~$f$.
  \item[(c)] Use the \emph{inverse function theorem} to find the values of $\left(\pd zx \right)_y$
and $\left(\pd zy \right)_x$ at the point~$(3,2,1)$.
 \end{enumerate}
 (Solution~\ref{sol_xmpl_ift_13}.)
\end{exer}

\begin{prob}\label{ift_prob2} Let $f(x,y,z) = xz + xy + yz - 3$. By explicit computation find a
neighborhood $V$ of $(1,1)$ in $\R^2$ and a function $h \colon V \sto \R$ such that $h(1,1) =
1$ and $f(x,y,h(x,y)) = 0$ for all $x$ and $y$ in~$V$.  Find $h_1(1,1)$ and $h_2(1,1)$.
\end{prob}

\begin{prob}  Use the \emph{inverse function theorem}, not direct calculation, to show that the equation
  \[ xz + xy + yz = 3 \]
has a solution for $z$ near $1$ in terms of $x$ and $y$ near $(1,1)$ and to find $\left(\pd zx
\right)_y$ and $\left(\pd zy \right)_x$ at the point $(1,1,1)$.
\end{prob}

\begin{prob}(a) What does it mean to say that the equation
  \[ wx^2y + \sqrt wy^2z^4 = 3xz + 6x^3z^2 + 7 \]
can be solved for $z$ in terms of $w$, $x$, and $y$ near the point $(w,x,y,z) = (4,1,2,1)$?
 \begin{enumerate}
  \item[(b)] Show that such a solution exists.
  \item[(c)] Use the \emph{inverse function theorem} to find $\left(\pd zw \right)_{x,y}$,
$\left(\pd zx \right)_{w,y}$, and $\left(\pd zy \right)_{w,x}$ at the point $(4,1,2,1)$.
 \end{enumerate}
\end{prob}

\begin{prob}\label{ift_prob5} Let $\open U{\R^{n-1} \times \R}$ and $f \in \fml C^1(U,\R)$.
Suppose that the point $(a,b) = (a_1, \dots ,a_{n-1};b)$ belongs to $U$, that  $f(a,b) = 0$,
and that $f_n(a,b) \ne 0$.  Show that there exists a neighborhood $V$ of $a$ in $\R^{n-1}$ and
a function $h$ in  $\fml C^1(V,\R)$ such that $h(a) = b$ and $f(x,h(x)) = 0$ for all $x$
in~$V$. \emph{Hint.} Show that the function
 \[G \colon U \sto \ R^{n-1} \times \R \colon (x,y)
                    = (x_1, \dots ,x_{n-1};y) \mapsto (x,f(x,y)) \]
has a local $\fml C^1$-inverse, say $H$, at~$(a,b)$.
\end{prob}

\begin{prob} Show that the equation
  \[ uvy^2z + w\sqrt xz^{10} + v^2e^yz^4 = 5 + uw^2\cos\bigl(x^3y^5\bigr) \]
can be solved for $x$ near 4 in terms of  $u$, $v$, $w$, $y$, and $z$ near $-3$, $2$, $-1$,
$0$, and  $1$, respectively. \emph{Hint.} Use the preceding problem.
\end{prob}

\begin{prob}(a) Use problem~\ref{ift_prob5} to make sense of the following ``theorem'':
If $f(x,y,z) = 0$, then
  \[ \cpd zxy = -\frac{\cpd fx{y,z}}{\cpd fz{x,y}}\,. \]
\emph{Hint.} After determining that there exists a function $h$ which satisfies the equation
$f(x,y,h(x,y)) = 0$ on an appropriate neighborhood of a point, use the chain rule to
differentiate both sides of the equation with respect to $x$.)
 \begin{enumerate}
   \item[(b)] Verify part (a) directly for the function $f$ given in problem~\ref{ift_prob2}
by computing each side independently.
   \item[(c)] Restate and prove the following ``theorem'': If $f(x,y,z) = 0$, then
     \[ \cpd zxy \cpd xyz \cpd yzx = -1\,. \]
 \end{enumerate}
\end{prob}


%\begin{prob} Here is a problem taken from a thermodynamics text:
%
%\vskip 10 pt
% {\narrower\narrower Show that if $f(x,y,z)=0$, then
% \[\left(\frac{\partial x}{\partial y}\right)_z
%     \left(\frac{\partial y}{\partial z}\right)_x
%     \left(\frac{\partial z}{\partial x}\right)_y = -1\,.\]
%\par}
%
%\vskip 10 pt
%
%Your old pal, Fred Dimm, is taking a class in thermodynamics and
%is baffled by the statement of the problem.  Among other things he
%notices that there is a mysterious $f$ in the hypothesis that does
%not appear in the conclusion.  He wonders, not unreasonably, how
%assuming something about $f$ is going to help him prove something
%that makes no reference whatever to that quantity.  He is also
%convinced that the answer is wrong: he thinks that the product of
%the partial derivatives should be $+1$.  He claims it should be
%just like the single variable case: the chain rule says
%$\dfrac{dy}{dt} = \dfrac{dy}{dx}\dfrac{dx}{dt}$ because we can
%cancel the $dx$'s.
%
%Help Fred by explaining the exercise to him.  And comment on his
%``correction'' of the problem.
%
%Unfortunately, once you have explained it all him, he still can't
%do the problem.  So also show him how to solve it.
%\end{prob}

\begin{prob} A commonly used formula in scientific work is
  \[ \cpd xyz = \frac 1{\cpd yxz}\,. \]
Recast this as a carefully stated theorem.  Then prove the theorem.  \emph{Hint.} Use
problem~\ref{ift_prob5} twice to obtain appropriate functions $h$ and $j$ satisfying
$f(h(y,z),y,z) = 0$ and $f(x,j(x,z),z) = 0$ on appropriate neighborhoods. Differentiate these
equations using the chain rule. Evaluate at a particular point. Solve.
\end{prob}


\begin{prob}(a) Make sense of the formula
  \[ \cpd xzy = -\frac{\cpd yzx}{\cpd yxz} \]
and prove it.
 \begin{enumerate}
   \item[(b)] Illustrate the result in part (a) by computing separately $\cpd PVT$ and
$-\dfrac{\cpd TVP}{\cpd TPV}$ from the equation of state $PV = RT$ for an ideal gas. (Here $R$
is a constant.)
 \end{enumerate}
\end{prob}

We have dealt at some length with the problem of solving a single equation for one variable in
terms of the remaining ones.  It is pleasant to discover that the techniques used there can be
adapted with only the most minor modifications to give local solutions for systems of $n$
equations in $p$ variables (where $p > n$) for $n$ of the variables in terms of the remaining
$p-n$ variables. We begin with some examples.

\begin{exer}\label{xmpl_ift_24} Consider the following system of equations:
  \begin{equation}\label{ift_24_eq1}
    \begin{cases}
            2u^3vx^2 + v^2x^3y^2 - 3u^2y^4 = 0 \\
            2uv^2y^2 - uvx^2 + u^3xy = 2 \,.
    \end{cases}
  \end{equation}
 \begin{enumerate}
   \item[(a)]  What does it mean to say that the system~\eqref{ift_24_eq1} can be solved for $x$
and $y$ near $(c,d)$ in terms of $u$ and $v$ near~$(a,b)$?
   \item[(b)]  Show that the system~\eqref{ift_24_eq1} can be solved for $x$ and $y$ near $(1,1)$
in terms of $u$ and $v$ near~$(1,1)$. \emph{Hint.} Try to imitate the technique of
problem~\ref{ift_prob5}, except in this case define $G$ on an appropriate subset of $\R^2
\times \R^2$.
 \end{enumerate}
 (Solution~\ref{sol_xmpl_ift_24}.)
\end{exer}

\begin{prob}\label{ift_prob10}  Consider the following system of equations
   \begin{equation}\label{ift_23_eq1}
     \begin{cases}
            4x^2 + 4y^2 &= z \\
             x^2 + y^2 &= 5 - z \,.
     \end{cases}
   \end{equation}
 \begin{enumerate}
  \item[(a)] What does it mean to say that \eqref{ift_23_eq1} can be solved for  $y$ and $z$ near
$(1,4)$ in terms of $x$ near~0?
  \item[(b)] Show that such a solution exists by direct computation.
  \item[(c)] Compute $\cpd yxz$ and $\cpd zxy$ at the point $x = 0$, $y = 1$, $z = 4$.
 \end{enumerate}
\end{prob}

\begin{prob}\label{ift_prob11}(a) Repeat problem~\ref{ift_prob10}(b), this time using the
\emph{inverse function theorem} instead of direct computation.
  \begin{enumerate}
   \item[(b)] Use the \emph{inverse function theorem} to find $\cpd yxz$ and $\cpd zxy$ at the
point $x = 0$, $y = 1$, $z = 4$.
  \end{enumerate}
\end{prob}

\begin{prob}\label{ift_prob12} Discuss the problem of solving the system of equations
  \begin{equation}
     \begin{cases}
          ux^2 + vwy + u^2w &= 4 \\
         uvy^3 + 2wx - x^2y^2 &= 3
     \end{cases}
  \end{equation}
for $x$ and $y$ near $(1,1)$ in terms of $u$, $v$, and $w$ near~$(1,2,1)$.
\end{prob}

As the preceding examples indicate, the first step in solving a system of $n$ equations in
$n+k$ unknowns (where $k>0$) for $n$ of the variables in terms of the remaining $k$ variables
is to replace the system of equations by a function $f:U \sto \R^n$ where $\open U{\R^k \times
\R^n}$. If the finite dimensional spaces $\R^k$ and $\R^n$ are replaced by arbitrary Banach
spaces, the subsequent calculations can be carried out in exactly the same fashion as in the
examples we have just considered.  The result of this computation is the \emph{implicit
function theorem}.

\begin{thm}[The Implicit Function Theorem]\label{thm_impft}  Let $E_1$, $E_2$, and $F$ be Banach
spaces, $(a,b) \in \open U{E_1 \times E_2}$, and $f \in \fml C^1(U,F)$. If $f(a,b) = \vc 0$
and $d_2f_{(a,b)}$ is invertible in $\ofml B(E_2,F)$, then there exist a neighborhood $V$ of
$a$ in $E_1$ and a continuously differentiable function $h \colon V \sto E_2$ such that $h(a)
= b$ and $f(x,h(x)) = \vc 0$ for all $x \in V$.
\end{thm}

\begin{proof} Problem.  \emph{Hint.} Let $G(x,y) = (x,f(x,y))$ for all $(x,y) \in U$. Show that
$dG_{(a,b)} = (x, Sx + Ty)$ where $S = d_1f_{(a,b)}$ and $T = d_2f_{(a,b)}$.  Show that the
map $(x,z) \mapsto (x,T^{-1}(z-Sx))$ from $E_1 \times F$ to $E_1 \times E_2$ is the inverse of
$dG_{(a,b)}$.  (One can guess what the inverse of $dG_{(a,b)}$ should be by regarding
$dG_{(a,b)}$ as the matrix $\begin{bmatrix}
                                 I_{E_1} & 0 \\
                                    S    & T
                            \end{bmatrix}$ acting on $E_1 \times E_2$.) Apply the \emph{inverse
function theorem}~\ref{thm_inft3} and proceed as in exercise~\ref{xmpl_ift_24} and
problems~\ref{ift_prob11} and~\ref{ift_prob12}.    \ns
\end{proof}

\begin{prob} Suppose we are given a system of $n$ equations in $p$ variables where $p > n$. What
does the \emph{implicit function theorem}~\ref{thm_impft} say about the possibility of solving
this system locally for $n$ of the variables in terms of the remaining $p - n$ variables?
\end{prob}


\endinput
