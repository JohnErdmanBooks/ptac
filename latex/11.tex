\chapter{THE TOPOLOGY OF METRIC SPACES}

\section{OPEN AND CLOSED SETS}
\begin{defn} A subset $A$ of a metric space $M$ is
 \index{open!subset of a metric space}%
\df{open} in $M$ if $\intr A = A$. That is, a set is open if it contains an open ball about
each of its points. To indicate that $A$ is open in $M$ we
 \index{<@$\open{}{}$ (open subset of)}%
write~$\open{A}{M}$.
\end{defn}

\begin{exam}\label{subsp_vs_sp} Care must be taken to claim that a particular set is open
(or not open) \emph{only} when the metric space in which the set ``lives'' is clearly
understood. For example, the assertion ``the set $[0,1)$ is open'' is false if the metric
space in question is $\R$. It is true, however, if the metric space being considered is
$[0,\infty)$ (regarded as a subspace of $\R$). The reason: In the space $[0,\infty)$ the point
0 is an interior point of $[0,1)$; in $\R$ it is not.
\end{exam}

\begin{exam} In a metric space every open ball is an open set.  Notice that this is exactly what
lemma \ref{sm_in_lar} says:  each point of an open ball is an interior point of that ball.
\end{exam}

The fundamental properties of open sets may be deduced easily from information we already
possess concerning interiors of sets. Three facts about open sets are given in
\ref{op_un_bl}--\ref{in_lar}. The first of these is very simple.

\begin{prop}\label{op_un_bl} Every nonempty open set is a union of open balls.
\end{prop}

\begin{proof} Let $U$ be an open set. For each $a$ in $U$ there is an open ball $B(a)$ about
$a$ contained in~$U$. Then clearly
  \[ U = \textstyle\bigcup\{B_a\colon  a \in U\}\,. \]
\end{proof}

\begin{prop}\label{op_in_ms} Let $M$ be a metric space.
 \begin{enumerate}
  \item[(a)] The union of any family of open subsets of $M$ is open.
  \item[(b)] The intersection of any \emph{finite} family of open subsets of $M$ is open.
 \end{enumerate}
\end{prop}

\begin{proof}
(a)\quad Let $\sfml U$ be a family of open subsets of $M$.  Since the interior of a set
is always contained in the set, we need only show that $\bigcup \sfml U \subseteq
\intr{\bigl(\bigcup \sfml U\bigr)}$. By \ref{un_int}
 \begin{align*}
   \textstyle\bigcup \sfml U
              &= \textstyle\bigcup\{U\colon U \in \sfml U\} \\
              &= \textstyle\bigcup\{\intr U\colon U \in \sfml U\} \\
              &\subseteq \intr{\bigl(\textstyle\bigcup \sfml U\bigr)}.
 \end{align*}

(b)\quad It is enough to show that the intersection of \emph{two} open sets is open.  Let
$\open{U,V}M$. Then by \ref{in_in}
  \[ \intr{(U \cap V)} = \intr U \cap \intr V = U \cap V\,. \]
\end{proof}

\begin{prop}\label{in_lar} The interior of a set $A$ is the largest open set contained in~$A$.
\textup{( Precisely:} $\intr A$ is the union of all the open sets contained in $A$.\textup{)}
\end{prop}

\begin{proof} Exercise.  (Solution~\ref{sol_in_lar}.)
  \ns \end{proof}

\begin{defn} A subset $A$ of a metric space is
 \index{closed}%
\df{closed} if $\clo A = A$. That is, a set is closed if it contains all its accumulation
points.
\end{defn}

\begin{exam} As is the case with open sets, care must be taken when affirming or denying that
a particular set is closed. It must be clearly understood in which metric space the set
``lives''. For example the interval $(0,1]$ is not closed in the metric space $\R$, but it is
a closed subset of the metric space $(0,\infty)$ (regarded as a subspace of $\R$).
\end{exam}


\textbf{REMINDER.}  Recall the remarks made after example~\ref{cl_int_cl}: \emph{sets are not
like doors or windows; they are not necessarily either open or closed.}  One can not show that
a set is closed, for example, by showing that it fails to be open.

\begin{prop}\label{op_vs_cl} A subset of a metric space is open if and only if its complement
is closed.
\end{prop}

\begin{proof} Exercise. \emph{Hint.} Use problem \ref{int_vs_cl}. (Solution~\ref{sol_op_vs_cl}.) \ns
\end{proof}

Facts already proved concerning closures of sets give us one way of dealing with closed sets;
the preceding proposition gives us another. To illustrate this, we give two proofs of the next
proposition.

\begin{prop} The intersection of an arbitrary family of closed subsets of a metric space is closed.
\end{prop}

\begin{proof}[First proof] Let $\sfml A$ be a family of closed subsets of a metric space.  Then
$\bigcap \sfml A$ is the complement of $\bigcup \{A^c\colon A \in \sfml A\}$.  Since each
set $A^c$ is open (by \ref{op_vs_cl}), the union of $\{A^c \colon A \in \sfml A\}$ is
open (by \ref{op_in_ms}(a)); and its complement $\bigcap \sfml A$ is closed
(\ref{op_vs_cl} again).
\end{proof}

\begin{proof}[Second proof] Let $\sfml A$ be a family of closed subsets of a metric space. Since a
set is always contained in its closure, we need only show that $\clo{\bigcap \sfml A}
\subseteq \bigcap \sfml A$. Use problem~\ref{intr_cl}(a):
 \begin{align*}
    \clo{\textstyle\bigcap \sfml A}
                   &\subseteq \textstyle\bigcap\{\,\clo A \colon A \in \sfml A\,\} \\
                   &= \textstyle\bigcap\{\,A \colon A \in \sfml A\,\} \\
                   &= \textstyle\bigcap \sfml A\,.   \qedhere
 \end{align*}
\end{proof}

\begin{prob} The union of a finite family of closed subsets of a metric space is closed.
 \begin{enumerate}
  \item[(a)] Prove this assertion using propositions \ref{op_in_ms}(b) and~\ref{op_vs_cl}.
  \item[(b)] Prove this assertion using problem \ref{un_clo}.
 \end{enumerate}
\end{prob}

\begin{prob} Give an example to show that the intersection of an arbitrary family of open subsets
of a metric space need not be open.
\end{prob}

\begin{prob} Give an example to show that the union of an arbitrary family of closed subsets of a
metric space need not be closed.
\end{prob}

\begin{defn} Let $M$ be a metric space, $a \in M$, and $r > 0$.  The
 \index{cra@$C_r(a)$ (closed ball of radius $r$ about~$a$)}%
 \index{closed!ball}%
 \index{ball!closed}%
\df{closed ball} $C_r(a)$ about $a$ of radius $r$ is $\{x \in M \colon d(a,x) \le r\}$. The
 \index{sra@$S_r(a)$ (sphere or radius $r$ about~$a$)}%
 \index{sphere}%
\df{sphere} $S_r(a)$ about $a$ of radius $r$ is $\{x \in M \colon d(a,x) = r\}$.
\end{defn}

\begin{prob} Let $M$ be a metric space, $a \in M$, and $r > 0$.
 \begin{enumerate}
  \item[(a)] Give an example to show that the closed ball about $a$ of radius $r$ need not be the
same as the closure of the open ball about $a$ of radius $r$.  That is, the sets $C_r(a)$ and
$\clo{B_r(a)}$ may differ.
  \item[(b)] Show that every closed ball in $M$ is a closed subset of~$M$.
  \item[(c)] Show that every sphere in $M$ is a closed subset of~$M$.
 \end{enumerate}
\end{prob}

\begin{prop} In a metric space the closure of a set $A$ is the smallest closed set
containing $A$.  \textup{(Precisely:} $\clo A$ is the intersection of the family of all closed
sets which contain $A$.\textup{)}
\end{prop}

\begin{proof} Problem.  \ns  \end{proof}

\begin{prop} If $A$ is a subset of a metric space, then its boundary $\partial A$ is equal
to~$\clo A \setminus \intr A$. Thus $\partial A$ is closed.
\end{prop}

\begin{proof} Problem. \ns  \end{proof}

\begin{prop}  Let $A$ be a subset of a metric space $M$. If $A$ is closed in $M$ or if it is
open in $M$, then $\intr{(\partial A)} = \emptyset$.
\end{prop}

\begin{proof} Problem. \ns  \end{proof}

\begin{prob} Give an example of a subset $A$ of the metric space $\R$ the interior of whose
boundary is all of~$\R$.
\end{prob}

\begin{defn} Let $A \subseteq B \subseteq M$ where $M$ is a metric space. We say that $A$ is
 \index{dense}%
\df{dense} in $B$ if $\clo A \supseteq B$.  Thus, in particular, $A$ is dense in the space $M$
if~$\clo A = M$.
\end{defn}

\begin{exam} The rational numbers are dense in the reals; so are the irrationals.
\end{exam}

\begin{proof} That $\clo Q = \R$ was proved in \ref{q_dense}.  The proof that $\clo{Q^c} = \R$
is similar.
\end{proof}

The following proposition gives a useful and easily applied criterion for determining when a
set is dense in a metric space.

\begin{prop}\label{cnd_dns} A subset $D$ of a metric space $M$ is dense in $M$ if and only if
every open ball contains a point of~$D$.
\end{prop}

\begin{proof} Exercise.  (Solution~\ref{sol_cnd_dns}.)
  \ns \end{proof}

\begin{prob}  Let $M$ be a metric space.  Prove the following.
 \begin{enumerate}
  \item[(a)] If $A \subseteq M$ and $\open UM$, then $U \cap \clo A \subseteq \clo{U \cap A}$.
  \item[(b)] If $D$ is dense in $M$ and $\open UM$, then $U \subseteq \clo{U \cap D}$.
 \end{enumerate}
\end{prob}

\begin{prop} Let $A \subseteq B \subseteq C \subseteq M$ where $M$ is a metric space. If $A$
is dense in $B$ and $B$ is dense in $C$, then $A$ is dense in~$C$.
\end{prop}

\begin{proof} Problem.  \ns  \end{proof}






\section{THE RELATIVE TOPOLOGY}
In example \ref{subsp_vs_sp} we considered the set $A = [0,1)$ which is contained in both the
metric spaces  $M = [0,\infty)$ and $N = \R$.  We observed that the question ``Is $A$ open?''
is ambiguous; it depends on whether we mean ``open in $M$'' or ``open in $N$''. Similarly, the
notation $B_r(a)$ is equivocal.  In $M$ the open ball $B_\frac12(0)$ is the interval
$[0,\frac12)$ while in $N$ it is the interval $(-\frac12,\frac12)$.  When working with sets
which are contained in two different spaces, considerable confusion can be created by
ambiguous choices of notation or terminology.  In the next proposition, where we examine the
relationship between open subsets of a metric space $N$ and open subsets of a subspace~$M$, it
is necessary, in the proof, to consider open balls in both $M$ and~$N$.  To avoid confusion we
use the usual notation $B_r(a)$ for open balls in $M$ and a different one $D_r(a)$ for those
in~$N$.

The point of the following proposition is that even if we are dealing with a complicated or
badly scattered subspace of a metric space, its open sets are easily identified. When an open
set in the larger space is intersected with the subspace $M$ what results is an open set in
$M$; and, less obviously, \emph{every} open set in $M$ can be produced in this fashion.

\begin{prop}\label{open_in_subsp} Let $M$ be a subspace of a metric space $N$.  A set
$U \subseteq M$ is open in $M$ if and only if there exists a set $V$ open in $N$ such that $U
= V \cap M$.
\end{prop}

\begin{proof} Let us establish some notation. If $a \in M$ and $r > 0$ we write $B_r(a)$ for
the open ball about $a$ of radius $r$ in the space~$M$. If $a \in N$ and $r>0$ we write
$D_r(a)$ for the corresponding open ball in the space~$N$.  Notice that $B_r(a) = D_r(a) \cap
M$. Define a mapping $f$ from the set of all open balls in $M$ into the set of open balls in
$N$ by
  \[ f(B_r(a)) = D_r(a)\,. \]
Thus $f$ is just the function which associates with each open ball in the space $M$ the
corresponding open ball (same center, same radius) in $N$; so $f(B) \cap M = B$ for each open
ball $B$ in $M$.

Now suppose $U$ is open in $M$. By proposition \ref{op_un_bl} there exists a family
$\sfml B$ of open balls in $M$ such that $U = \bigcup\sfml B$.  Let $\sfml D = \{f(B)
\colon B \in \sfml B\}$ and $V = \bigcup\sfml D$. Then $V$, being a union of open balls
in $N$, is an open subset of $N$ and
 \begin{align*}
   V \cap M
        &= \bigl(\textstyle\bigcup \sfml D\bigr) \cap M \\
        &= \bigl(\textstyle\bigcup\{f(B):B \in \sfml B\}\bigr)\cap M\\
        &= \textstyle\bigcup\{f(B) \cap M: B \in \sfml B\} \\
        &= \textstyle\bigcup \sfml B \\
        &= U\,.
 \end{align*}
(For the third equality in the preceding string, see proposition~\ref{ioveru_inf}.)

The converse is even easier. Let $V$ be an open subset of $N$ and $a \in V \cap M$. In order
to show that $V \cap M$ is open in $M$, it suffices to show that, in the space $M$, the point
$a$ is an interior point of the set $V \cap M$. Since $V$ is open in $N$, there exists $r>0$
such that $D_r(a) \subseteq V$. But then
  \[ B_r(a) \subseteq D_r(a) \cap M \subseteq V \cap M\,.   \qedhere \]
\end{proof}

The family of all open subsets of a metric space is called the
 \index{topology}%
\df{topology} on the space. As was the case for the real numbers, the concepts of
continuity, compactness, and connectedness can be characterized entirely in terms of the
open subsets of the metric spaces involved and without any reference to the specific
metrics which lead to these open sets. Thus we say that continuity, compactness, and
connectedness are topological concepts.  The next proposition~\ref{mtr_vs_top} tells us
that strongly equivalent metrics on a set produce identical topologies. Clearly, no
topological property of a metric space is affected when we replace the given metric with
another metric which generates the same topology.

\begin{defn}\label{def_equiv_metrics} Two metrics $d_1$ and $d_2$ on a set $M$ are
 \index{equivalent!metrics}%
 \index{metrics!equivalent}%
\df{equivalent} if they induce the same topology on~$M$.
\end{defn}

We now prove that strongly equivalent metrics are equivalent.

\begin{prop}\label{mtr_vs_top} Let $d_1$ and $d_2$ be metrics on a set $M$ and $\sfml T_1$ and
$\sfml T_2$ be the topologies on $(M,d_1)$ and $(M,d_2)$, respectively. If $d_1$ and
$d_2$ are strongly equivalent, then $\sfml T_1 = \sfml T_2$.
\end{prop}

\begin{proof} Exercise. (Solution~\ref{sol_mtr_vs_top}.)
  \ns \end{proof}



\begin{prob} Give an example to show that equivalent metrics need not be strongly
equivalent.
\end{prob}

\begin{defn} If $M$ is a subspace of the metric space $(N,d)$, the family of open subsets of $M$
induced by the metric $d$ is called the
\index{relative!topology}%
\index{topology!relative}%
\df{relative topology} on~$M$.  According to proposition~\ref{open_in_subsp}, the relative
topology on $M$ is~$\{V \cap M \colon \open VN\}$.
\end{defn}
