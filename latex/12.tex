\chapter{SEQUENCES IN METRIC SPACES}\label{seqs_msps}



In chapter \ref{seqs_rn} we were able to characterize several topological properties of the
real line $\R$ by means of sequences. The same sort of thing works in general metric spaces.
Early in this chapter we give sequential characterizations of open sets, closed sets, dense
sets, closure, and interior.  Later we discuss products of metric spaces.


\section{CONVERGENCE OF SEQUENCES}
Recall that a
 \index{sequence}%
\df{sequence} is any function whose domain is the set $\N$ of natural numbers.  If $S$ is a
set and $x \colon \N \sto S$, then we say that $x$ is a sequence of members of $S$.  A map $x
\colon \N \sto \R$, for example, is a sequence of real numbers.  In dealing with sequences one
usually (but not always) writes $x_n$ for $x(n)$.  The element $x_n$ in the range of a
sequence is the $n^{\text{th}}$
 \index{term!of a sequence}%
\df{term} of the sequence. Frequently we use the notations $(x_n)_{n=1}^{\infty}$ or just
$(x_n)$ to denote the sequence~$x$.

\begin{defn}\label{defn_nbhd_pt} A
 \index{neighborhood!of a point}%
\df{neighborhood} of a point in a metric space is any open set containing the point.  Let $x$
be a sequence in a set $S$ and $B$ be a subset of~$S$.  The sequence $x$ is
 \index{eventually}%
\df{eventually} in the set $B$ if there is a natural number $n_0$ such that $x_n \in B$
whenever $n \ge n_0$.  A sequence $x$ in a metric space $M$
 \index{convergence!of a sequence}%
\df{converges} to a point $a$ in $M$ if $x$ is eventually in every neighborhood of $a$
(equivalently, if it is eventually in every open ball about~$a$). The point $a$ is the
 \index{limit!of a sequence}%
\df{limit} of the sequence~$x$. (In proposition \ref{lim_uniq} we find that limits of
sequences are unique, so references to ``the'' limit of a sequence are justified.)

If a sequence $x$ converges to a point $a$ in a metric space we write
 \index{<@$x_n \sto a \text{ as $n \sto \infty$}$ (limit of a sequence)}%
   \[ x_n \sto a  \quad \text{ as } \quad n \sto \infty \]
or
 \index{limit@$\lim_{n \sto \infty}x_n$ (limit of a sequence)}%
    \[ \lim_{n \sto \infty}x_n = a\,. \]
It should be clear that the preceding definition may be rephrased as follows: The sequence $x$
converges to the point $a$ if for every $\epsilon > 0$ there exists $n_0 \in \N$ such that
$d(x_n,a) < \epsilon$ whenever $n \ge n_0$.  It follows immediately that $x_n \sto a$ if and
only if $d(x_n,a) \sto 0$ as $n \sto \infty$.  Notice that in the metric space $\R$ the
current definition agrees with the one given in chapter~\ref{seqs_rn}.
\end{defn}








\section{SEQUENTIAL CHARACTERIZATIONS OF TOPOLOGICAL PROPERTIES}
Now we proceed to characterize some metric space concepts in terms of sequences.  The point of
this is that sequences are often easier to work with than arbitrary open sets.

\begin{prop} A subset $U$ of a metric space $M$ is open if and only if every sequence in $M$
which converges to an element of $U$ is eventually in~$U$.
\end{prop}

\begin{proof} Suppose that $U$ is an open subset of $M$. Let $(x_n)$ be a sequence in $M$ which
converges to a point $a$ in $U$. Then $U$ is a neighborhood of $a$. Since $x_n \sto a$, the
sequence $(x_n)$ is eventually in~$U$.

To obtain the converse suppose that $U$ is not open. Some point, say $a$, of $U$ is not an
interior point of $U$. Then for every $n$ in $\N$ we may choose an element $x_n$ in
$B_{1/n}(a)$ such that $x_n \notin U$. Then the sequence $(x_n)$ converges to $a \in U$, but
it is certainly not true that $(x_n)$ is eventually in~$U$.
\end{proof}

\begin{prop}\label{scm_closed} A subset $A$ of a metric space is closed if and only if $b$
belongs to $A$ whenever $(a_n)$ is a sequence in $A$ which converges to $b$.
\end{prop}

\begin{proof} Exercise. (Solution~\ref{sol_scm_closed}.)  \ns \end{proof}

\begin{prop}\label{scm_dense} A subset $D$ of a metric space $M$ is dense in $M$ if and only
if every point of $M$ is the limit of a sequence of elements of~$D$.
\end{prop}

\begin{proof} Problem. \ns \end{proof}

\begin{prop}\label{lim_uniq} In metric spaces limits of sequences are unique.  (That is,
if $a_n \sto b$ and $a_n \sto c$ in some metric space, then $b = c$.)
\end{prop}

\begin{proof} Problem. \ns \end{proof}

\begin{prob}\label{prob_clo_seq} Show that a point $p$ is in the closure of a subset $A$ of a
metric space if and only if there is a sequence of points in $A$ which converges to~$p$.
Also, give a characterization of the interior of a set by means of sequences.
\end{prob}

\begin{prob} Since the rationals are dense in $\R$, it must be possible, according to
proposition~\ref{scm_dense}, to find a sequence of rational numbers which converges to the
number~$\pi$. Identify one such sequence.
\end{prob}

\begin{prop}\label{conv_eqm} Let $d_1$ and $d_2$ be strongly equivalent metrics on a
set~$M$.  If a sequence of points in $M$ converges to a point $b$ in the metric space
$(M,d_1)$, then it also converges to $b$ in the space $(M,d_2)$.
\end{prop}

\begin{proof} Problem. \ns \end{proof}

\begin{prob} Use proposition \ref{conv_eqm} to show that the metric $\rho$ defined in example
\ref{Gam} is not strongly equivalent on $\R^2$ to the usual Euclidean
(example~\ref{Eucl_met}). You gave a (presumably different) proof of this in
problem~\ref{d_ne_rho}.
\end{prob}

\begin{prob} Let $M$ be a metric space with the discrete metric.  Give a simple characterization
of the convergent sequences in~$M$.
\end{prob}

\begin{defn}\label{dist_sets} Let $A$ and $B$ be nonempty subsets of a metric space~$M$.  The
 \index{distance!between sets}%
\df{distance} between $A$ and $B$, which we denote
 \index{dab@$d(A,B)$ (distance between two sets}%
by~$d(A,B)$, is defined to be $\inf\{d(a,b) \colon a \in A \text{ and } b \in B\}$. If $a \in
M$ we write $d(a,B)$ for $d(\{a\},B)$.
\end{defn}

\begin{prob}\label{dist_sets_prob} Let $B$ be a nonempty subset of a metric space~$M$.
 \begin{enumerate}
  \item[(a)] Show that if $x \in B$, then $d(x,B) = 0$.
  \item[(b)] Give an example to show that the converse of (a) may fail.
  \item[(c)] Show that if $B$ is closed, the converse of (a) holds.
 \end{enumerate}
\end{prob}







\section{PRODUCTS OF METRIC SPACES}
\begin{defn}\label{3prods} Let $(M_1, \rho_1)$ and $(M_2,\rho_2)$ be metric spaces. We define three
 \index{product!metrics}%
 \index{metrics!on products}%
metrics, $d$, $d_1$, and $d_u$, on the product $M_1 \times M_2$.  For $x = (x_1,x_2)$ and $y =
(y_1,y_2)$ in $M_1 \times M_2$ let
 \index{done@$d_1$ (taxicab metric, product metric)}%
 \index{du@$d_u$ (uniform metric)}%
   \begin{align*}
          d(x,y)   &= \bigl((\rho_1(x_1,y_1))^2 +
                   (\rho_2(x_2,y_2))^2\bigr)^{\frac12},
                               \displaybreak[0]\\[.1 true in]
          d_1(x,y) &= \rho_1(x_1,y_1) + \rho_2(x_2,y_2),
                \text{\quad and}
                               \displaybreak[0]\\[.1 true in]
          d_u(x,y) &= \max\{\rho_1(x_1,y_1), \rho_2(x_2,y_2)\}\,.
 \end{align*}
It is not difficult to show that these really are metrics. They are just generalizations, to
arbitrary products, of the metrics on $\R \times \R$ defined in \ref{Eucl_met}, \ref{taxicab},
and \ref{unif_rn}.
\end{defn}


\begin{prop}\label{3equiv_ms} The three metrics on $M_1 \times M_2$ defined in \ref{3prods}
are strongly equivalent.
\end{prop}

\begin{proof} Exercise.  \emph{Hint.}  Review proposition \ref{3equiv}.
(Solution~\ref{sol_3equiv_ms}.) \ns
\end{proof}

In light of the preceding result and proposition \ref{mtr_vs_top} the three metrics defined on
$M_1 \times M_2$ (in \ref{3prods}) all give rise to exactly the same topology on $M_1 \times
M_2$. Since we will be concerned primarily with \emph{topological properties} of product
spaces, it makes little difference which of these metrics we officially adopt as ``the''
product metric. We choose $d_1$ because it is arithmetically simple (no square roots of sums
of squares).

\begin{defn}\label{prod_met} If $M_1$ and $M_2$ are metric spaces, then we say that the metric
space $(M_1 \times M_2, d_1)$, where $d_1$ is defined in~\ref{3prods}, is the
 \index{product!of metric spaces}%
 \index{space!product}%
\df{product (metric) space} of $M_1$ and $M_2$; and the metric $d_1$ is the
 \index{done@$d_1$ (taxicab metric, product metric)}%
 \index{product!metric, ``the''}%
 \index{metric!product, ``the''}%
\df{product metric}.  When we encounter a reference to ``the metric space $M_1 \times M_2$''
we assume, unless the contrary is explicitly stated, that this space is equipped with
\emph{the} product metric $d_1$.
\end{defn}

A minor technical point, which is perhaps worth mentioning, is that the usual (Euclidean)
metric on $\R^2$ is not (according to the definition just given) the product metric. Since
these two metrics \emph{are} equivalent and since most of the properties we consider are
topological ones, this will cause little difficulty.

It is easy to work with sequences in product spaces. This is a consequence of the fact, which
we prove next, that a necessary and sufficient condition for the convergence of a sequence in
a product space is the convergence of its coordinates.

\begin{prop}\label{seq_prod} Let $M_1$ and $M_2$ be metric spaces.  A sequence
$\bigl(\,(x_n,y_n)\,\bigr)_{n=1}^{\infty}$ in the product space converges to a point $(a,b)$
in $M_1 \times M_2$ if and only if $x_n \sto a$ and $y_n \sto b$.
\end{prop}

\begin{proof} For $k=1,2$ let $\rho_k$ be the metric on the space $M_k$. The product metric
$d_1$ on $M_1 \times M_2$ is defined in~\ref{3prods}.

Suppose $(x_n,y_n) \sto (a,b)$. Then
 \begin{align*}
     \rho_1(x_n,a) &\le \rho_1(x_n,a) + \rho_2(y_n,b) \\
                   &= d_1\bigl((x_n,y_n)\,,\,(a,b)\bigr) \sto 0;
 \end{align*}
so $x_n \sto a$. Similarly, $y_n \sto b$.

Conversely, suppose $x_n \sto a$ in $M_1$ and $y_n \sto b$ in $M_2$. Given $\epsilon > 0$ we
may choose $n_1,n_2 \in \N$ such that $\rho_1(x_n,a) < \frac12\epsilon$ when $n \ge n_1$ and
$\rho_2(y_n,b) < \frac12\epsilon$ when $n \ge n_2$. Thus if $n \ge \max\{n_1,n_2\}$,
   \[ d_1\bigl((x_n,y_n)\,,\,(a,b)\bigr)
                       = \rho_1(x_n,a) + \rho_2(y_n,b)
                 < \tfrac12\epsilon + \tfrac12\epsilon = \epsilon\,; \]
so $(x_n,y_n) \sto (a,b)$ in $M_1 \times M_2$.
\end{proof}

\begin{rem}  By virtue of proposition \ref{conv_eqm}, the truth of the preceding proposition
would not have been affected had either $d$ or $d_u$ (as defined in \ref{3prods}) been chosen
as the product metric for $M_1 \times M_2$.
\end{rem}

\begin{prob} Generalize definitions \ref{3prods} and \ref{prod_met} to $\R^n$ where $n \in \N$.
That is, write appropriate formulas for $d(x,y)$, $d_1(x,y)$, and $d_u(x,y)$ for $x,y \in
\R^n$, and explain what we mean by the product metric on an arbitrary finite product $M_1
\times M_2 \times \cdots \times M_n$ of metric spaces.

Also state and prove generalizations of propositions \ref{3equiv_ms} and \ref{seq_prod} to
arbitrary finite products.
\end{prob}
