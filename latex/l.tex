\chapter{PROPERTIES OF FUNCTIONS}\label{images}

\section{IMAGES AND INVERSE IMAGES}
\begin{defn} If $f\colon S \sto T$ and $A \subseteq S$, then $f^{\sto} (A)$, the
 \index{<@$f^{\sto}(A)$ (image of a set $A$ under a function $f$)}%
\df{image} of $A$ under $f$, is $\{f(x)\colon x \in A\}$. It is common practice to write
$f(A)$ for $f^{\sto}(A)$.  The set $f^{\sto}(S)$ is the
 \index{range@$\ran f$ (range of a function $f$)}%
\df{range} (or
 \index{image!of a function}%
\df{image}) of $f$; usually we write $\ran f$ for $f^{\sto}(S)$.
\end{defn}

\begin{exer}\label{exer_fcn1} Let
 \[ f(x) =
       \begin{cases} -1, &\text{for $x<-2$}\\
                7 - x^2, &\text{for $-2 \le x < 1$}\\
             \dfrac1x\,, &\text{for $x \ge 1$}
       \end{cases}\]
and $A = (-4,4)$.  Find $f^{\sto}(A)$. (Solution~\ref{sol_exer_fcn1}.)
\end{exer}

\begin{exer}\label{exer_fcn2} Let $f(x) = 3x^4 + 4x^3 - 36x^2 + 1$.  Find $\ran f$. (
Solution~\ref{sol_exer_fcn2}.)
\end{exer}

\begin{defn}  Let $f\colon S \sto T$ and $B \subseteq T$. Then
$f^\gets (B)$, the
 \index{<@$f^{\gets}(B)$ (inverse image of a set $B$ under a function $f$)}%
\df{inverse image} of $B$ under~$f$, is $\{x \in S\colon f(x) \in B\}$. In many texts
$f^{\gets}(B)$ is denoted by $f^{-1}(B)$. This may cause confusion by suggesting that
functions always have inverses (see section~\ref{sec_inv_fcns} of chapter~\ref{inverses}).
\end{defn}

\begin{exer}\label{exer_fcn3}  Let $f(x) = \arctan x$ and $B = (\frac\pi4, 2)$. Find
$f^{\gets}(B)$. (Solution~\ref{sol_exer_fcn3}.)
\end{exer}

\begin{exer}\label{exer_fcn4} Let $f(x) = -\sqrt{9 - x^2}$ and $B = (1, 3)$. Find
$f^{\gets}(B)$.  (Solution~\ref{sol_exer_fcn4}.)
\end{exer}

\begin{prob}  Let
 \[f(x) =
      \begin{cases} -x-4, &\text{for $x \le 0$}\\
                  x^2+3,  &\text{for $0 < x \le 2$}\\
             (x-1)^{-1},  &\text{for $x>2$}
     \end{cases}\]
and $A = (-3,4)$.  Find $f^{\sto}(A)$.
\end{prob}

\begin{prob}  Let $f(x) = 4 - x^2$ and $B= (1,3]$.  Find $f^{\gets}(B)$.
\end{prob}

\begin{prob}  Let $f(x) = \dfrac x{1-x}$.
 \begin{enumerate}
  \item[(a)] Find $f^{\gets}([0,a])$  for $a > 0$.
  \item[(b)] Find $f^{\gets}([-\frac32, -\frac12])$.
 \end{enumerate}
\end{prob}

\begin{prob} Let $f(x) = -x^2 + 4\arctan x$. Find $\ran f$.
\end{prob}

\begin{prob} Let
 \[f(x) =
      \begin{cases}  x + 1,  &\text{for $x < 1$}\\
              8 + 2x - x^2,  &\text{for $x \ge 1$}.
      \end{cases}\]
Let $A = (-2, 3)$ and $B = [0,1]$.  Find $f^{\sto}(A)$ and $f^{\gets}(B)$.
\end{prob}







\section{COMPOSITION OF FUNCTIONS}  Let $f \colon S \sto T$ and $g \colon T \sto U$. The
 \index{composite!function}%
\df{composite} of $g$ and $f$, denoted by $g \circ f$, is the function taking $S$ to $U$
defined by
 \[(g \circ f)(x) = g(f(x))\]
for all $x$ in $S$.  The operation $\circ$ is
 \index{<@$\circ$ (composition)}%
\df{composition}.  We again make a special convention for real valued functions of a real
variable:  The domain of $g \circ f$ is the set of all $x$ in $\R$ for which the expression
$g(f(x))$ makes sense.

\begin{exam}  Let $f(x) = (x - 1)^{-1}$ and $g(x) = \sqrt x$. Then
the domain of $g \circ f$ is the interval $(1,\infty)$, and for all
$x$ in that interval
  \[(g \circ f)(x) = g(f(x)) = \frac1{\sqrt{x - 1}}.\]
\end{exam}

\begin{proof} The square root of $x-1$ exists only when $x \ge 1$; and since we take its
reciprocal, we exclude $x=1$.  Thus $\dom(g \circ f) = (1,\infty)$.
\end{proof}

\begin{exer}\label{exer_fcn5}  Let
 \[f(x) =
      \begin{cases}  0,  &\text{for $x<0$}\\
                     3x, &\text{for $0 \le x \le 2$}\\
                     2,  &\text{for $x>2$}
      \end{cases}\]
and
 \[g(x) =
      \begin{cases} x^2, &\text{for $1<x<3$}\\
                    -1,  &\text{otherwise.}
      \end{cases}\]
Sketch the graph of $g \circ f$.  (Solution~\ref{sol_exer_fcn5}.)
\end{exer}

\begin{prop}\label{prop_assoc_comp}  Composition of functions is associative but not necessarily
commutative.
\end{prop}

\begin{proof} Exercise. \emph{Hint.} Let $f\colon S \sto T$, $g\colon T \sto U$,
and $h\colon U \sto V$. Show that $h \circ (g \circ f) = (h \circ g) \circ f$. Give an example
to show that $f \circ g$ and $g \circ f$ may fail to be equal.
(Solution~\ref{sol_prop_assoc_comp}.) \ns
\end{proof}

\begin{prob} Let $f(x) = x^2 + 2x^{-1}$, $g(x) = 2(2x + 3)^{-1}$,
and $h(x) = \sqrt{2x}$. Find $(h \circ g \circ f)(4)$.
\end{prob}

\begin{prob}  If $f\colon S \sto T$ and $g:T \sto U$, then
 \begin{enumerate}
  \item[(a)] $(g \circ f)^{\sto}(A) = g^{\sto}(f^{\sto}(A))$ for every $A \subseteq S$.
  \item[(b)] $(g \circ f)^{\gets}(B) = f^{\gets}(g^{\gets}(B))$ for every $B \subseteq U$.
 \end{enumerate}
\end{prob}








\section{The IDENTITY FUNCTION} The family of all functions mapping a set $S$ into a set $T$
is denoted by $\fml F(S,T)$.  One member of $\fml F(S,S)$ is particularly noteworthy, the
 \index{identityfunction@$I_S$ (identity function on a set $S$)}%
\df{identity function} on $S$. It is defined by
 \[I_S\colon S \sto S\colon x \mapsto x.\]
When the set $S$ is understood from context, we write $I$ for~$I_S$.

The identity function is characterized algebraically by the conditions:
 \[\text{if $f\colon R \sto S$, then $I_S \circ f = f$}\]
and
 \[\text{if $g\colon S \sto T$, then $g \circ I_S = g$}.\]

\begin{defn}\label{def_incl_map} More general than the identity function are the inclusion maps.
If $A \subseteq S$, then the
 \index{inclusionmap@$\iota_{A,S}$ (inclusion map of $A$ into $S$)}%
\df{inclusion map} taking $A$ into $S$ is defined by
 \[\iota_{A,S}:A \sto S\colon x \mapsto x.\]
When no confusion is likely  to result, we abbreviate $\iota_{A,S}$ to $\iota$.  Notice that
$\iota_{S,S}$ is just the identity map~$I_S$.
\end{defn}





\section{DIAGRAMS}  It is frequently useful to think of functions as arrows in diagrams. For example,
the situation $f\colon R \sto S$, $h\colon R \sto T$, $j\colon T \sto U$, $g\colon S \sto U$
may be represented by the following diagram.
 \[\xy
   \square[R`S`T`U;f`h`g`j]
   \morphism<500,500>[T`S;]
   \endxy\]
The diagram is said to
 \index{commute}%
 \index{commuting!diagram}%
\df{commute} if $j \circ h = g \circ f$.

Diagrams need not be rectangular. For instance,
 \[\xy
     \btriangle[R`S`T;f`k`g]
   \endxy\]
is a commutative diagram if $k = g \circ f$.

\begin{exam}  Here is a diagrammatic way of stating the associative law for composition of functions.
If the triangles in the diagram
 \[\xy
     \square[R`U`S`T;j`f`h`k]
     \morphism<500,500>[S`U;g]
   \endxy\]
commute, then so does the rectangle.
\end{exam}







\section{RESTRICTIONS AND EXTENSIONS}\label{sec_res_ext}
If $f \colon S \sto T$ and $A\subseteq S$, then the
 %\index{<@$\bigl.f\bigr"|_A$ (restriction of $f$ to $A$)}%
 \index{<@$f\mid_{{}_A}$ (restriction of $f$ to $A$)}%
 \index{restriction}%
\df{restriction} of $f$ to $A$, denoted by $\bigl.f\bigr|_A$, is the function
$f\circ\iota_{{}_{A,S}}$. That is, it is the mapping from $A$ into $T$ whose value at each $x$
in $A$ is $f(x)$.
  \[ \xy
      \btriangle/<-`>`>/[S`A`T;\iota_{{}_{AS}}`f`\bigl.f\bigr|_A]
     \endxy \]

Suppose that $g \colon A \sto T$ and $A \subseteq S$. A function $f\colon S \sto T$ is an
 \index{extension}%
\df{extension} of $g$ to~$S$ if $f\big|_A = g$, that is, if the diagram
  \[ \xy
     \btriangle/<-`>`>/[S`A`T;\iota_{{}_{AS}}`f`g]
    \endxy \]
commutes.


\endinput
