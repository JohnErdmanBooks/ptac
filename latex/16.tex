\chapter{SEQUENTIAL CHARACTERIZATION OF COMPACTNESS}

We have previously characterized open sets, closed sets, closure, and continuity by means of
sequences.  Our next goal is to produce a characterization of compactness in terms of
sequences.  This is achieved in theorem~\ref{cpt_scpt} where it is shown that compactness in
metric spaces is equivalent to something called \emph{sequential compactness}.  For this
concept we need to speak of \emph{subsequences} of sequences of points in a metric space. In
\ref{subseq_r} we defined ``subsequence'' for sequences of real numbers.  There is certainly
no reason why we cannot speak of subsequences of arbitrary sequences.



\section{SEQUENTIAL COMPACTNESS}
\begin{defn} If $a$ is a sequence of elements of a set $S$ and $n \colon \N \sto \N$ is
strictly increasing, then the composite function $a \circ n$ is a
 \index{subsequence}%
\df{subsequence} of the sequence~$a$. (Notice that $a \circ n$ is itself a sequence of
elements of $S$ since it maps $\N$ into~$S$.) The $k^{\text{th}}$ term of $a \circ n$ is
frequently written as $a_{n_k}$; other acceptable notations are $a_{n(k)}$ and $a(n(k))$.
\end{defn}

Notice that it is possible for a sequence which fails to converge to have subsequences which
do converge.  For example, if $a_n = (-1)^n + (1/n)$ for each $n$, then the subsequence
$(a_{2n})$ converges while the sequence $(a_n)$ itself does not.

\begin{defn} A metric space $M$ is
 \index{sequentially compact}%
 \index{compact!sequentially}%
\df{sequentially compact} if every sequence in $M$ has a convergent subsequence.
\end{defn}

\begin{exam} It is important to understand that for a space to be sequentially compact the
preceding definition requires that every sequence in the space have a subsequence which
converges to a point \emph{in that space}.  It is not enough to find a subsequence which
converges in some larger space. For example, we know that the metric space $(0,1]$ regarded as
a subspace of $\R$ is not sequentially compact because the sequence $\bigl(\frac1n \bigr)$ has
no subsequence which converges to something in~$(0,1]$.  That $\bigl(\frac1n \bigr)$ happens
to converge to $0$ in $\R$ is completely irrelevant.
\end{exam}

A major goal of this chapter is to demonstrate that in metric spaces compactness and
sequential compactness are the same thing. This is done in theorem~\ref{cpt_scpt}.  Be aware
however that in general topological spaces this is \emph{not} true.  An essential ingredient
of the proof is the following chain of implications
 \[ \text{sequentially compact} \Longrightarrow \text{totally bounded}
                                            \Longrightarrow \text{separable}. \]
So the next order of business is to define the last two concepts and prove that the preceding
implications do hold.

\begin{defn} A metric space $M$ is
 \index{totally bounded}%
 \index{bounded!totally}%
\df{totally bounded} if for every $\epsilon > 0$ there exists a finite subset $F$ of $M$ such
that for every $a \in M$ there is a point $x \in F$ such that $d(x,a) < \epsilon$.  This
definition has a more or less standard paraphrase: A space is totally bounded if it can be
kept under surveillance by a finite number of arbitrarily near-sighted policemen.
\end{defn}

\begin{prop}\label{tbdd_bdd} Every totally bounded metric space is bounded.
\end{prop}

\begin{proof} Problem.   \ns  \end{proof}

\begin{exam} The converse of the preceding proposition is false. Any infinite set with the
discrete metric is an example of a bounded metric space which is not totally bounded.  (Why?)
\end{exam}

\begin{prop}\label{scpt_tbdd} Every sequentially compact metric space is totally bounded.
\end{prop}

\begin{proof} Exercise. \emph{Hint.} Assume that a metric space $M$ is not totally bounded.
Inductively construct a sequence in $M$ no two terms of which are closer together than some
fixed distance $\epsilon > 0$. (Solution~\ref{sol_scpt_tbdd}.)    \ns
\end{proof}

\begin{defn} A metric space is
 \index{separable}%
\df{separable} if it possesses a countable dense subset.
\end{defn}

\begin{exam} The space $\R^n$ is separable.  The set of points $(q_1, \dots, q_n)$ such that
each coordinate $q_k$ is rational is a countable dense subset of~$\R^n$.
\end{exam}

\begin{exam} It follows easily from proposition~\ref{cnd_dns} that the real line (or indeed any
uncountable set) with the discrete metric is \emph{not} separable. (Consider the open balls of
radius $1$ about each point.)
\end{exam}

\begin{prop}\label{tbdd_sep} Every totally bounded metric space is separable.
\end{prop}

\begin{proof} Problem.  \emph{Hint.} Let $M$ be a totally bounded metric space.  For each $n$
in $\N$ choose a finite set $F_n$ such that for each $a \in M$ the set $F_n \cap
B_{\frac1n}(a)$ is nonempty.   \ns
\end{proof}

\begin{prob} The metric space $\R_d$ comprising the real numbers under the discrete metric is
not separable.
\end{prob}

\begin{cor}\label{scpt_sep} Every sequentially compact metric space is separable.
\end{cor}

\begin{proof} Propositions~\ref{scpt_tbdd} and~\ref{tbdd_sep}.
\end{proof}







\section{CONDITIONS EQUIVALENT TO COMPACTNESS}
We are now ready for a major result of this chapter---a sequential characterization of
compactness.  We show that a space $M$ is compact if and only if every sequence in $M$ has a
convergent subsequence. In other words, a space is compact if and only if it is sequentially
compact. We will see later in the section how useful this result is when we use it to prove
that a finite product of compact spaces is compact.

The following theorem also provides a second characterization of compactness: a space $M$ is
compact if and only if every infinite subset of $M$ has a point of accumulation in~$M$. The
proof of the theorem is fairly straightforward except for one complicated bit. It is not easy
to prove that every sequentially compact space is compact. This part of the proof comes
equipped with an lengthy hint.

\begin{thm}\label{cpt_scpt} If $M$ is a metric space then the following are equivalent:
 \begin{enumerate}
  \item[(1)] $M$ is compact;
  \item[(2)] every infinite subset of $M$ has an accumulation point in $M$;
  \item[(3)] $M$ is sequentially compact.
 \end{enumerate}
\end{thm}

\begin{proof} Exercise. \emph{Hint.} Showing that (3) implies (1) is not so easy.  To
show that a sequentially compact metric space $M$ is compact start with an arbitrary open
cover $\sfml U$ for $M$ and show first that
  \[ \text{(A)\quad $\sfml U$ has a {\sl countable\/} subfamily $\sfml V$ which covers $M$.} \]
Then show that
  \[ \text{(B)\quad there is a finite subfamily of $\sfml V$ which covers $M$.} \]
The hard part is (A).  According to corollary~\ref{scpt_sep} we may choose a countable
dense subset $A$ of $M$. Let $\sfml B$ be the family of all open balls $B_r(a)$ such that
 \begin{align*}
       \text{(i)\quad} &a \in A;\\
      \text{(ii)\quad} &r \in \Q \text{ and;}\\
     \text{(iii)\quad} &B_r(a) \subseteq U \text{\quad for some }U \in \sfml U.
 \end{align*}
For each $B$ in $\sfml B$ choose a set $U_B$ in $\sfml U$ such that $B \subseteq U_B$ and
let
   \[ \sfml V = \{U_B \colon B \in \sfml B\}. \]
Then verify that $\sfml V$ is a countable subfamily of $\sfml U$ which covers~$M$.

To show that $\sfml V$ covers $M$, start with an arbitrary point $x \in M$ and a set $U
\in \sfml U$ containing $x$.  All that is needed is to find an open ball $B_s(a)$ in
$\sfml B$ such that $x \in B_s(a) \subseteq U$. In order to do this the point $a \in A$
must be taken sufficiently close to $x$ so that it is possible to choose a rational
number $s$ which is both
 \begin{align*}
      \text{(i) } &\text{ small enough for $B_s(a)$ to be a subset of~$U$, and} \\
      \text{(ii)} &\text{ large enough for $B_s(a)$ to contain $x$.}
 \end{align*}
To establish (B) let $(V_1, V_2, V_3, \dots)$ be an enumeration of $\sfml V$ and $W_n =
\cup_{k=1}^n V_k$ for each $n \in \N$. If no one set $W_n$ covers $M$, then for every $k
\in \N$ there is a point $x_k \in {W_k}^c$.    (Solution~\ref{sol_cpt_scpt}.)  \ns
\end{proof}

\begin{prob} Use theorem~\ref{cpt_scpt} to give three different proofs that the metric space
$[0,1)$  (with the usual metric inherited from~$\R$) is not compact.
\end{prob}


\begin{cau} It is a common (and usually helpful) mnemonic device to reduce statements of
complicated theorems in analysis to brief paraphrases.  In doing this considerable care should
be exercised so that crucial information is not lost. Here is an example of the kind of thing
that can go wrong.

Consider the two statements:
 \begin{enumerate}
  \item[(1)] In the metric space $\R$ every infinite subset of the open unit interval has a point
of accumulation; and
  \item[(2)] A metric space is compact if every infinite subset has a point of accumulation.
 \end{enumerate}
Assertion (1) is a special case of proposition~\ref{Bolzano_thm}; and (2) is just part of
theorem~\ref{cpt_scpt}. The unwary tourist might be tempted to conclude from (1) and (2) that
the open unit interval $(0,1)$ is compact, which, of course, it is not. The problem here is
that (1) is a correct assertion about $(0,1)$ regarded as a subset of the space $\R$; every
infinite subset of $(0,1)$ does have a point of accumulation \emph{lying in}~$\R$.

If, however, the metric space under consideration is $(0,1)$ itself, then (1) is no longer
true. For example, the set of all numbers of the form $1/n$ for $n \ge 2$ has no accumulation
point in $(0,1)$. When we use (2) to establish the compactness of a metric space~$M$, what we
must verify is that every infinite subset of $M$ has a point of accumulation \emph{which lies
in}~$M$. Showing that these points of accumulation exist in some space which contains $M$ just
does not do the job. The complete statements of~\ref{Bolzano_thm} and~\ref{cpt_scpt} make this
distinction clear; the paraphrases (1) and (2) above do not.
\end{cau}





\section{PRODUCTS OF COMPACT SPACES}
It is possible using just the definition of compactness to prove that the product of two
compact metric spaces is compact. It is a pleasant reward for the effort put into proving
theorem~\ref{cpt_scpt} that it can be used to give a genuinely simple proof of this important
result.

\begin{thm} If $M_1$ and $M_2$ are compact metric spaces, then so is $M_1 \times M_2$.
\end{thm}

\begin{proof}  Problem.  \ns  \end{proof}

\begin{cor}\label{prod_cpt} If $M_1, \dots, M_n$ are compact metric spaces, then the product
space $M_1 \times \dots \times M_n$ is compact.
\end{cor}

\begin{proof} Induction. \end{proof}

\begin{prob} Let $A$ and $B$ be subsets of a metric space. Recall from definition~\ref{dist_sets}
that the distance between $A$ and $B$ is defined by
  \[ d(A,B) := \inf\{d(a,b) \colon a \in A \text{ and } b \in B\}\,. \]
Prove or disprove:
 \begin{enumerate}
  \item[(a)] If $A \cap B = \emptyset$, then $d(A,B) > 0$.
  \item[(b)] If $A$ and $B$ are closed and $A \cap B = \emptyset$, then $d(A,B) > 0$.
  \item[(c)] If $A$ is closed, $B$ is compact, and $A \cap B = \emptyset$, then $d(A,B) > 0$.
\emph{Hint.} If $d(A,B) = 0$ then there exist sequences $a$ in $A$ and $b$ in $B$ such that
$d(a_n,b_n) \sto 0$.
  \item[(d)] If $A$ and $B$ are compact and $A \cap B = \emptyset$, then $d(A,B) > 0$.
 \end{enumerate}
\end{prob}

\begin{prob}\label{elps_cpt} Let $a$, $b > 0$.  The elliptic disk
  \[ D := \biggl\{(x,y) \colon \frac{x^2}{a^2} + \frac{y^2}{b^2} \le 1 \biggr\} \]
is a compact subset of~$\R^2$. \emph{Hint.} Write the disk as a continuous image of the unit
square $[0,1] \times [0,1]$.
\end{prob}

\begin{prob} The unit sphere
  \[ S^2 \equiv \{(x,y,z) \colon x^2 + y^2 + z^2 = 1\} \]
is a compact subset of $\R^3$. \emph{Hint.} Spherical coordinates.
\end{prob}

\begin{prob} Show that the interval $[0,\infty)$ is not a compact subset of $\R$ using each
of the following:
 \begin{enumerate}
  \item[(a)] The definition of compactness.
  \item[(b)] Proposition~\ref{mtc}.
  \item[(c)] Proposition~\ref{cpt_clbdd}.
  \item[(d)] The \emph{extreme value theorem}~(\ref{EVThm}).
  \item[(e)] Theorem~\ref{cpt_scpt}, condition (2).
  \item[(f)] Theorem~\ref{cpt_scpt}, condition (3).
  \item[(g)] The \emph{finite intersection property}. (See problem~\ref{fip}.)
  \item[(h)] \emph{Dini's theorem.} (See problem~\ref{Dini_thm}.)
 \end{enumerate}
\end{prob}







\section{THE HEINE-BOREL THEOREM} We have seen in proposition~\ref{cpt_clbdd} that in an
arbitrary metric space compact sets are always closed and bounded.  Recall also that the
converse of this is not true in general (see problem~\ref{cpt_prb2}).  In $\R^n$, however, the
converse does indeed hold. This assertion is the \emph{Heine-Borel theorem}. Notice that in
example~\ref{HBthm} we have already established its correctness for the case $n=1$. The proof
of the general case is now just as easy---we have done all the hard work in proving that the
product of finitely many compact spaces is compact (corollary~\ref{prod_cpt}).

\index{Heine-Borel theorem}%
\begin{thm}[Heine-Borel Theorem]\label{HBThm} A subset of $\R^n$ is compact if and only if it
is closed and bounded.
\end{thm}

\begin{proof} Exercise.  (Solution~\ref{sol_HBThm}.) \ns  \end{proof}

\begin{exam} The triangular region $T$ whose vertices are $(0,0)$, $(1,0)$, and $(0,1)$ is a
compact subset of~$\R^2$.
\end{exam}

\begin{proof} Define functions $f,g,h: \R^2 \sto \R$ by $f(x,y) = x$, $g(x,y) = y$, and $h(x,y)
= x + y$. Each of these functions is continuous. Thus the sets
 \begin{align*}
     A &:= f^{\gets}[0,\infty) = \{(x,y) : x \ge 0\}, \\
     B &:= g^{\gets}[0,\infty) = \{(x,y) : y \ge 0\}\text{,\quad
                       and} \\
     C &:= h^{\gets}(-\infty,1] = \{(x,y) : x+y \le 1\}
 \end{align*}
are all closed sets. Thus $T = A \cap B \cap C$ is closed. It is bounded since it is contained
in the open ball about the origin with radius~2.  Thus by the \emph{Heine-Borel
theorem}~(\ref{HBThm}), $T$~is compact.
\end{proof}

\begin{prob} Do problem~\ref{elps_cpt} again, this time using the \emph{Heine-Borel theorem}.
\end{prob}

\index{Bolzano-Weierstrass theorem}%
\begin{prob}[Bolzano-Weierstrass Theorem]\label{BWThm} Every bounded infinite subset of $\R^n$
has at least one point of accumulation in $\R^n$ (compare proposition~\ref{Bolzano_thm}).
\end{prob}

\begin{prob}[Cantor Intersection Theorem]\label{CIThm} If $(A_n)$ is a nested sequence of
 \index{Cantor intersection theorem}%
nonempty closed bounded subsets of $\R^n$, then $\cap_{n=1}^{\infty}A_n$ is nonempty.
Furthermore, if $\diam A_n \sto 0$, then $\cap_{n=1}^{\infty}A_n$ is a single point.
\end{prob}

\begin{prob} Use the \emph{Cantor intersection theorem} (problem~\ref{CIThm}) to show that the
medians of a triangle are concurrent.
\end{prob}

\begin{prob}\label{cpt_prb3} Let $m$ be the set of all bounded sequences of real numbers under
the uniform metric:
   \[ d_u(a,b) = \sup \{\abs{a_n - b_n} \colon n \in \N\} \]
whenever $a$, $b \in m$.
 \begin{enumerate}
  \item[(a)] Show by example that the \emph{Heine-Borel theorem} (\ref{HBThm}) does not hold for
subsets of~$m$.
  \item[(b)] Show by example that the \emph{Bolzano-Weierstrass theorem} does not hold for subsets
of~$m$. (See problem~\ref{BWThm}.)
  \item[(c)] Show by example that the \emph{Cantor intersection theorem} does not hold for subsets
of~$m$. (See problem~\ref{CIThm}.)
 \end{enumerate}
\end{prob}

\begin{prob}\label{cpt_prb4} Find a metric space $M$ with the property that every infinite subset
of $M$ is closed and bounded but not compact.
\end{prob}

\begin{prob} Prove or disprove: If both the interior and the boundary of a set $A \subseteq \R$ are
compact, then so is~$A$.
\end{prob}



\endinput
