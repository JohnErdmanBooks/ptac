\chapter{INFINITE SERIES}\label{inf_ser}

It is perhaps tempting to think of an infinite series
  \[ \sum_{k=1}^\infty a_k = a_1 + a_2 + a_3 + \dots \]
as being nothing but ``addition performed infinitely often''.  This view of series is
misleading, can lead to curious errors, and should not be taken seriously.  Consider the
infinite series
  \begin{equation}\label{nonsense_ser}
      1 - 1 + 1 - 1 + 1 - 1 + ...   .
  \end{equation}
If we think of the ``+'' and ``-'' signs as functioning in essentially the same manner as the
symbols we encounter in ordinary arithmetic, we might be led to the following ``discovery''.
  \begin{align}\label{nonsense_ser2}
           0 &=  (1 - 1) + (1 - 1) + (1 - 1) + \dots \\
             &= 1 - 1 + 1 - 1 + 1 - 1 + \dots \\
             &= 1 - (1 - 1) - (1 - 1) - \dots \\
             &= 1 - 0 - 0 - \dots \\
             &=1\,.
  \end{align}
One can be more inventive: If $S$ is the sum of the series~\eqref{nonsense_ser}, then
  \[ 1 - S = 1 - (1 - 1 + 1 - 1 + \dots) = 1 - 1 + 1 - 1 + 1 - 1 + \dots = S \]
from which it follows that $S = \frac12$.  This last result, incidentally, was believed (for
quite different reasons) by both Leibniz and Euler.  See~\cite{Moore:1932}. The point here is
that if an intuitive notion of infinite sums and plausible arguments lead us to conclude that
$1 = 0 = \frac12$, then it is surely crucial for us to exercise great care in defining and
working with convergence of infinite series.

In the first section of this chapter we discuss convergence of series in arbitrary normed
linear spaces.  One reason for giving the definitions in this generality is that doing so is
no more complicated than discussing convergence of series in $\R$.  A second reason is that it
displays with much greater clarity the role of completeness of the underlying space in
questions of convergence.  (See, in particular, propositions \ref{prop_Cauchy_crit} and
~\ref{prop_compl_abssum}.)  A final reason is that this generality is actually needed.  We use
it, for example, in the proofs of the \emph{inverse and implicit function theorems} in the
next chapter.






\section{CONVERGENCE OF SERIES}
\begin{defn}  Let $(a_k)$ be a sequence in a normed linear space. For each $n$ in $\N$ let
$s_n = \sum_{k=1}^n a_k$.  The vector $s_n$ is the
 \index{partial!sum}%
 \index{sum!partial}%
\df{$n^{\text{th}}$ partial sum} of the sequence~$(a_k)$.  As is true of sequences, we permit
variations of this definition.  For example, in section~\ref{pow_ser} on power series we
consider sequences $\bigl(a_k\bigr)_{k=0}^n$ whose first term has index zero.  In this case,
of course, the proper definition of $s_n$ is $\sum_{k=0}^n a_k$.
\end{defn}

\begin{exer}\label{exer_par_sum1}  Let $a_k = (-1)^{k+1}$ for each $k$ in $\N$.  For $n \in \N$
compute the $n^{\text{th}}$ partial sum of the sequence~$(a_k)$.
(Solution~\ref{sol_exer_par_sum1}.)
\end{exer}

\begin{exer}\label{exer_par_sum2}  Let $a_k = 2^{-k}$ for each $k$ in $\N$.  For $n \in \N$ show
that the $n^{\text{th}}$ partial sum of the sequence $(a_k)$ is $1 - 2^{-n}$.
(Solution~\ref{sol_exer_par_sum2}.)
\end{exer}

\begin{defn}  Let $(a_k)$ be a sequence in a normed linear space.  The
 \index{infinite!series}%
 \index{series!infinite}%
 \index{<@$\sum_{k=1}^\infty a_k$ (infinite series)}%
\df{infinite series} $\sum_{k=1}^\infty a_k$ is defined to be the sequence $(s_n)$ of partial
sums of the sequence $(a_k)$.  We may also write $a_1 + a_2 + a_3 + \dots$ for
$\sum_{k=1}^\infty a_k$. Again we permit variants of this definition.  For example, the
infinite series associated with the sequence $\bigl(a_k\bigr)_{k=0}^\infty$ is denoted by
$\sum_{k=0}^\infty a_k$.   Whenever the range of the summation index $k$ is understood from
context or is unimportant we may denote a series simply by $\sum a_k$.
\end{defn}

\begin{exer}\label{exer_inf_ser1}  What are the infinite series associated with the sequences
given in exercises \ref{exer_par_sum1} and~\ref{exer_par_sum2}?
(Solution~\ref{sol_exer_inf_ser1}.)
\end{exer}

\begin{defn}  Let $(a_k)$ be a sequence in a normed linear space~$V$.  If the infinite series
$\sum_{k=1}^\infty a_k$ (that is, the sequence of partial sums of $(a_k)$) converges to a
vector $b$ in $V$, then we say that the sequence $(a_k)$ is
 \index{summable!sequence}%
 \index{sequence!summable}%
\df{summable} or, equivalently, that the series $\sum_{k=1}^\infty a_k$ is a
 \index{convergence!of a series}%
 \index{series!convergent}%
\df{convergent series}.  The vector $b$ is called the
 \index{sum!of an infinite series}%
 \index{series!sum of a}%
\df{sum} of the series $\sum_{k=1}^\infty a_k$ and we write
  \[ \sum_{k=1}^\infty a_k = b\,. \]
It is clear that a necessary and sufficient condition for a series $\sum_{k=1}^\infty a_k$ to
be convergent or, equivalently, for the sequence $(a_k)$ to be summable, is that there exist a
vector $b$ in $V$ such that
  \begin{equation}\label{cond_ser_conv}
     \bignorm{b - \sum_{k=1}^n a_k} \sto 0 \qquad \text{as} \qquad n \sto \infty.
  \end{equation}
If a series does not converge we say that it is a
 \index{divergence!of a series}%
 \index{series!divergent}%
\df{divergent series} (or that it \df{diverges}).
\end{defn}

\begin{cau}  It is an old, if illogical, practice to use the same notation $\sum_{k=1}^\infty a_k$
for both the sum of a convergent series and the series itself.  As a result of this
convention, the statements ``$\sum_{k=1}^\infty a_k$ converges to $b$'' and
``$\sum_{k=1}^\infty a_k = b$'' are interchangeable.  It is possible for this to cause
confusion, although in practice it is usually clear from the context which use of the symbol
$\sum_{k=1}^\infty a_k$ is intended.  Notice however that since a divergent series has no sum,
the symbol $\sum_{k=1}^\infty a_k$ for such a series is unambiguous; it can refer only to the
series itself.
\end{cau}

\begin{exer}\label{exer_inf_ser2}  Are the sequences $(a_k)$ given in exercises
\ref{exer_par_sum1} and~\ref{exer_par_sum2} summable? If $(a_k)$ is summable, what is the sum
of the corresponding series $\sum_{k=1}^\infty a_k$?  (Solution~\ref{sol_exer_inf_ser2}.)
\end{exer}

\begin{prob}[Geometric Series]\label{prob_geomser2}
 \index{geometric!series}%
 \index{series!geometric}%
Let $a$ and $r$ be real numbers.
 \begin{enumerate}
  \item[(a)] Show that if $\abs r < 1$, then $\sum_{k=0}^\infty ar^k$ converges and
    \[ \sum_{k=0}^\infty ar^k = \frac a{1-r}\,. \]
\emph{Hint.}  To compute the $n^{\text{th}}$ partial sum $s_n$, use a technique similar
to the one used in exercise~\ref{exer_par_sum2}. See also problem~\ref{prob_geomser1}.
  \item[(b)] Show that if $\abs r \ge 1$ and $a \ne 0$, then $\sum_{k=0}^\infty ar^k$
diverges. \emph{Hint.} Look at the cases $r \ge 1$ and $r \le -1$ separately.
 \end{enumerate}
\end{prob}

\begin{prob} Let $\sum a_k$ and $\sum b_k$ be convergent series in a normed linear space.
 \begin{enumerate}
  \item[(a)] Show that the series $\sum(a_k + b_k)$ also converges and that
    \[ \sum(a_k + b_k) = \sum a_k + \sum b_k\,. \]
\emph{Hint.} Problem~\ref{prob_clo_subsp}(a).
  \item[(b)] Show that for every $\alpha \in \R$ the series $\sum (\alpha a_k)$ converges
and that
    \[ \sum (\alpha a_k) = \alpha \sum a_k\,. \]
 \end{enumerate}
\end{prob}

One very easy way of seeing that certain series do not converge is to observe that its terms
do not approach~$0$.  (The proof is given in the next proposition.)  It is important not to
confuse this assertion with its converse.  \emph{The condition $a_k \sto 0$ does not guarantee
that $\sum_{k=1}^\infty a_k$ converges.}  (An example is given in example~\ref{exam_harm_ser}.

\begin{prop}\label{prop_terms_zero}  If $\sum_{k=1}^\infty a_k$ is a convergent series in a
normed linear space, then $a_k \sto 0$ as $k \sto \infty$.
\end{prop}

\begin{proof}  Exercise. \emph{Hint.}  Write $a_n$ in terms of the partial sums $s_n$ and
$s_{n-1}$.   (Solution~\ref{sol_prop_terms_zero}.) \ns
\end{proof}

\begin{exam}\label{exam_harm_ser}  It is possible for a series to diverge even though the
terms $a_k$ approach~$0$.  A standard example of this situation in $\R$ is the harmonic series
$\sum_{k=1}^\infty 1/k$.  The harmonic series diverges.
\end{exam}

\begin{proof}  Exercise. \emph{Hint.}  Show that the difference of the partial sums $s_{2p}$
and $s_p$ is at least~$1/2$.  Assume that $(s_n)$ converges. Use proposition~\ref{conv_cau}.
 (Solution~\ref{sol_exam_harm_ser}.)  \ns
\end{proof}

\begin{prob} Show that if $0 < p \le 1$, then the series $\sum_{k=1}^\infty k^{-p}$ diverges.
\emph{Hint.}  Modify the argument used in~\ref{exam_harm_ser}.
\end{prob}

\begin{prob}  Show that the series $\sum_{k=1}^\infty \frac1{k^2+k}$ converges and find its sum.
\emph{Hint.} $\frac1{k^2 + k} = \frac1k - \frac1{k + 1}$.
\end{prob}

\begin{prob} Use the preceding problem to show that the series $\sum_{k=1}^\infty \frac1{k^2}$
converges and that its sum is no greater than $2$.
\end{prob}

\begin{prob} Show that the series $\sum_{k=4}^\infty\frac1{k^2-1}$ converges and find its sum.
\end{prob}

\begin{prob} Let $p \in \N$.  Find the sum of the series $\sum_{k=1}^\infty\frac{(k-1)!}{(k+p)!}$.
\emph{Hint.}  If $a_k = \frac{k!}{(k+p)!}$, what can you say about $\sum_{k=1}^n(a_{k-1} -
a_k)$?
\end{prob}

In \emph{complete} normed linear spaces the elementary fact that a sequence is Cauchy if and
only if it converges may be rephrased to give a simple necessary and sufficient condition for
the convergence of series in the space.

\begin{prop}[The Cauchy Criterion]\label{prop_Cauchy_crit}
 \index{Cauchy!criterion}%
 \index{criterion!Cauchy}%
Let $V$ be a normed linear space.  If the series $\sum a_k$ converges in $V$, then for every
$\epsilon > 0$ there exists $n_0 \in \N$ such that $\bignorm{\sum_{k=m+1}^n a_k} < \epsilon$
whenever $n > m \ge n_0$.  If $V$ is a Banach space, then the converse of this implication
also holds.
\end{prop}

\begin{proof} Exercise   (Solution~\ref{sol_prop_Cauchy_crit}.)  \ns  \end{proof}

The principal use of the preceding proposition is to shorten proofs.  By invoking the
\emph{Cauchy criterion} one frequently can avoid explicit reference to the partial sums of the
series involved. See, for example, proposition~\ref{prop_abssum_x_bdd}.

One obvious consequence of the \emph{Cauchy criterion} is that the convergence of an infinite
series is unaffected by changing any finite number of its terms.  If $a_n = b_n$ for all $n$
greater than some fixed integer $n_0$, then the series $\sum a_n$ converges if and only if the
series $\sum b_n$ does.

The examples of infinite series we have looked at thus far are all series of real numbers.  We
now turn to series in the Banach space $\fml B(S,E)$ of bounded $E$ valued functions on a set
$S$ (where $E$ is a Banach space).  Most of the examples we consider will be real valued
functions on subsets of the real line.

First a word of caution: the notations $\sum_{k=1}^\infty f_k$ and $\sum_{k=1}^\infty f_k(x)$
can, depending on context, mean many different things.  There are many ways in which sequences
(and therefore series) of functions can converge.  There are, among a host of others, uniform
convergence, pointwise convergence, convergence in mean, and convergence in measure.  Only the
first two of these appear in this text.  Since we regard $\fml B(S,E)$ as a Banach space under
the uniform norm $\norm{\hphantom{A}}_u$, it is not, strictly speaking, necessary for us to
write ``$f_n \sto g \text{\,(unif)}$'' when we wish to indicate that the sequence $(f_n)$
converges to $g$ in the space $\fml B(S,E)$; writing ``$f_n \sto g$'' is enough, because
unless the contrary is explicitly stated uniform convergence is understood.  Nevertheless, in
the sequel we will frequently add the redundant ``(unif)'' just as a reminder that in the
space $\fml B(S,E)$ we are dealing with uniform, and not some other type of, convergence of
sequences and series.


It is important to keep in mind that in the space $\fml B(S,E)$ the following assertions are
equivalent:
 \begin{enumerate}
   \item[(a)] $\sum_{k=1}^\infty f_k$ converges uniformly to $g$;
   \item[(b)] $g = \sum_{k=1}^\infty f_k$; and
   \item[(c)] $\norm{g - \sum_{k=1}^n f_k}_u \sto 0$ as $n \sto
\infty$.
 \end{enumerate}
Since uniform convergence implies pointwise convergence, but not conversely, each of the
preceding three conditions implies---\emph{but is not implied by}---the following three (which
are also equivalent):
 \begin{enumerate}
   \item[(a${}'$)] $\sum_{k=1}^\infty f_k(x)$ converges to $g(x)$ for every $x$ in~$S$;
   \item[(b${}'$)] $g(x) = \sum_{k=1}^\infty f_k(x)$ for every $x$ in~$S$; and
   \item[(c${}'$)] For every $x$ in $S$
       \[ \bignorm{g(x) - \sum_{k=1}^n f_k(x)} \sto 0 \quad \text{ as } \quad  n \sto \infty\,. \]
 \end{enumerate}


One easy consequence of the \emph{Cauchy criterion} (proposition~\ref{prop_Cauchy_crit}) is
called the \emph{Weierstrass M-test}.  The rather silly name which is attached to this result
derives from the fact that in the statement of the proposition, the constants which appear are
usually named~$M_n$.


\begin{prop}[Weierstrass M-test]\label{M_test}
 \index{Weierstrass M-test}%
 \index{M-test}%
Let $(f_n)$ be a sequence of functions in $\fml B(S,E)$ where $S$ is a nonempty set and $E$ is
a Banach space.  If there is a summable sequence of positive constants $M_n$ such that
$\bignorm{f_n}_u \le M_n$ for every $n$ in $\N$, then the series $\sum f_k$ converges
uniformly on~$S$.  Furthermore, if the underlying set $S$ is a metric space and each $f_n$ is
continuous, then $\sum f_k$ is continuous.
\end{prop}

\begin{proof} Problem.  \ns  \end{proof}

\begin{exer}\label{exer_M_test}  Let $0 < \delta < 1$.  Show that the series
$\sum_{k=1}^\infty \frac{x^k}{1+x^k}$ converges uniformly on the interval $[-\delta,\delta]$.
\emph{Hint.}  Use problem~\ref{prob_geomser2}.  (Solution~\ref{sol_exer_M_test}.)
\end{exer}

\begin{prob} Show that $\sum_{n=1}^\infty\frac1{1 + n^2x}$ converges uniformly on
$[\delta,\infty)$ for any $\delta$ such that $0 < \delta < 1$.
\end{prob}

\begin{prob} Let $M > 0$.  Show that $\sum_{n=1}^\infty\frac{n^2x^3}{n^4+x^4}$ converges
uniformly on~$[-M,M]$.
\end{prob}

\begin{prob} Show that $\sum_{n=1}^\infty\frac{nx}{n^4 + x^4}$ converges uniformly on~$\R$.
\end{prob}

We conclude this section with a generalization of the alternating series test, familiar from
beginning calculus.  Recall that an
 \index{alternating series}%
 \index{series!alternating}%
\df{alternating series} in $\R$ is a series of the form $\sum_{k=1}^\infty (-1)^{k+1}
\alpha_k$ where each $\alpha_k > 0$. The generalization here will not require that the
multipliers of the $\alpha_k$'s be $+1$ and $-1$ in strict alternation.  Indeed they need not
even be real numbers; they may be the terms of any sequence of vectors in a Banach space for
which the corresponding sequence of partial sums is bounded.  You are asked in
problem~\ref{prob_ast} to show that the \emph{alternating series test} actually follows from
the next proposition.


 \index{alternating series test!generalized}%
\begin{prop}\label{prop_genl_ast} Let $(\alpha_k)$ be a decreasing sequence of real numbers,
each greater than or equal to zero, which converges to zero.  Let $(x_k)$ be a sequence of
vectors in a Banach space for which there exists $M>O$ such that $\bignorm{\sum_{k=1}^n x_k}
\le M$ for all $n$ in $\N$.  Then $\sum_{k=1}^\infty \alpha_k x_k$ converges and
$\bignorm{\sum_{k=1}^\infty \alpha_k x_k} \le M\alpha_1$.
\end{prop}

\begin{proof} Problem.   \emph{Hint.}  This is a bit complicated.  Start by proving the
following very simple geometrical fact about a normed linear space $V$: if $[x,y]$ is a closed
segment in $V$, then one of its endpoints is at least as far from the origin as every other
point in the segment.  Use this to derive the fact that if $x$ and $y$ are vectors in $V$ and
$0 \le t \le 1$, then
  \[ \norm{x + ty} \le \max\{\norm x,\norm{x + y}\,. \]
Next prove the following result.

\begin{lem}  Let $(\alpha_k)$ be a decreasing sequence of real numbers with $\alpha_k \ge 0$
for every $k$, let $M > 0$, and let $V$ be a normed linear space.  If $x_1,\dots,x_n \in
V$ satisfy
  \begin{equation}\label{eqn_ast}
                  \bignorm{\sum_{k=1}^m x_k} \le M
  \end{equation}
for all $m \le n$, then
  \[ \bignorm{\sum_{k=1}^n \alpha_k x_k} \le M\alpha_1\,. \]
\end{lem}

To prove this result use mathematical induction.  Supposing the lemma to be true for $n = p$,
let $y_1,\dots,y_{p+1}$ be vectors in $V$ such that $\bignorm{\sum_{k=1}^m y_k} \le M$ for all
$m \le p+1$.  Let $x_k = y_k$ for $k = 1,\dots,p-1$ and let $x_p = y_p +
(\alpha_{p+1}/\alpha_p)y_{p+1}$.  Show that the vectors $x_1,\dots,x_p$
satisfy~\eqref{eqn_ast} for all $m \le p$ and invoke the inductive hypothesis.

Once the lemma is in hand, apply it to the sequence $\bigl(x_k\bigr)_{k=1}^\infty$ to obtain
$\bignorm{\sum_{k=1}^n \alpha_k x_k} \le M\alpha_1$ for all $n \in \N$, and apply it to the
sequence $\bigl(x_k\bigr)_{k=m+1}^\infty$ to obtain $\bignorm{\sum_{k = m+1}^n\alpha_kx_k} \le
2M\alpha_{m+1}$ for $0 < m < n$.  Use this last result to prove that the sequence of partial
sums of the series $\sum\alpha_k x_k$ is Cauchy. \ns
\end{proof}

 \index{alternating series test}%
\begin{prob}\label{prob_ast} Use proposition~\ref{prop_genl_ast} to derive the \emph{alternating
series test}: If $(\alpha_k)$ is a decreasing sequence of real numbers with $\alpha_k \ge 0$
for all $k$ and if $\alpha_k \sto 0$ as $k \sto \infty$, then the alternating series
$\sum_{k=1}^\infty (-1)^{k+1}\alpha_k$ converges. Furthermore, the absolute value of the
difference between the sum of the series and its $n^{\text{th}}$ partial sum is no greater
than $\alpha_{n+1}$.
\end{prob}

\begin{prob} Show that the series $\sum_{k=1}^\infty k^{-1}\sin(k\pi/4)$ converges.
\end{prob}

An important and interesting result in analysis is the \emph{Tietze extension theorem}.  For
compact metric spaces it says that any continuous real valued function defined on a closed
subset of the space can be extended to a continuous function on the whole space and that this
process can be carried out in such a way that the (uniform) norm of the extension does not
exceed the norm of the original function.  One proof of this uses both the M-\emph{test} and
the \emph{approximation theorem} of Weierstrass.

\begin{thm}[Tietze Extension Theorem]\label{Tet}
 \index{Tietze extension theorem}%
 \index{extension!from closed sets}%
Let $M$ be a compact metric space, $A$ be a closed subset of $M$, and $g\colon A \sto \R$ be
continuous.  Then there exists a continuous function $w\colon M \sto \R$ such that $w\big|_A =
g$ and $\norm w_u = \norm g_u$.
\end{thm}

\begin{proof} Problem.   \emph{Hint.}  First of all demonstrate that a continuous function can
be truncated without disturbing its continuity.  (Precisely: if $f \colon M \sto \R$ is a
continuous function on a metric space, if $A \subseteq M$, and if $f^{\sto}(A) \subseteq
[a,b]$, then there exists a continuous function $g \colon M \sto \R$ which agrees with $f$ on
$A$ and whose range is contained in $[a,b]$.) Let $\fml F = \{u\big|_A \colon  u \in \fml
C(M,\R)\}$. Notice that the preceding comment reduces the proof of~\ref{Tet} to showing that
$\fml F = \fml C(A,\R)$.  Use the \emph{Stone-Weierstrass theorem}~\ref{SW_thm} to prove that
$\fml F$ is dense in $\fml C(A,\R)$.  Next find a sequence $(f_n)$ of functions in $\fml F$
such that
  \[ \bignorm{g - \sum_{k=1}^n f_k}_u < \frac1{2^n} \]
for every $n$.  Then for each $k$ find a function $u_k$ in $\fml C(M,\R)$ whose restriction to
$A$ is $f_k$.  Truncate each $u_k$ (as above) to form a new function $v_k$ which agrees with
$u_k$ (and therefore $f_k$) on $A$ and which satisfies $\norm{v_k}_u = \norm{f_k}_u$.  Use the
\emph{Weierstrass M-test}~\ref{M_test} to show that $\sum_1^{\infty}v_k$ converges uniformly
on $M$. Show that $w = \sum_1^{\infty}v_k$ is the desired extension.   \ns
\end{proof}

Recall that in problem~\ref{prob_c_functor} we showed that if $\phi\colon M \sto N$ is a
continuous map between compact metric spaces, then the induced map $T_{\phi}\colon \fml
C(N,\R) \sto \fml C(M,\R)$ is injective if and only if $\phi$ is surjective, and $\phi$ is
injective if $T_{\phi}$ is surjective.  The ``missing part'' of this result ($T_{\phi}$ is
surjective if $\phi$ is injective) happens also to be true but requires the \emph{Tietze
extension theorem} for its proof.

\begin{prob} Let $\phi$ and $T_{\phi}$ be as in problem~\ref{prob_c_functor}.  Show that if
$\phi$ is injective, then $T_{\phi}$ is surjective.
\end{prob}

%This problem requires quotients.
%\begin{prob}  Let $M$ be a compact metric space
%and $A$ be a closed subset of~$M$.  Let $K =
%\{f \in \fml C(M,\R)\colon  f^{\sto}(A) = \{0\}\,\}$.
%Show that $\fml C(M,\R)/K$ is isometrically
%isomorphic to $\Cal C(A,\Bbb R)$.
%\end{prob}










\section{SERIES OF POSITIVE SCALARS}
In this brief section we derive some of the standard tests of beginning calculus for
convergence of series of positive numbers.

\begin{defn}  Let $S(n)$ be a statement in which the natural number $n$ is a variable.  We say
that $S(n)$ holds for $n$
 \index{sufficiently large}%
\df{sufficiently large} if there exists $N$ in $\N$ such that $S(n)$ is true whenever $n \ge
N$.
\end{defn}

\begin{prop}[Comparison Test]\label{comp_test}
 \index{comparison test}%
Let $(a_k)$ and $(b_k)$ be sequences in $[0, \infty)$ and suppose that there exists $M > 0$
such that $a_k \le Mb_k$ for sufficiently large $k \in \N$.  If $\sum b_k$ converges, then
$\sum a_k$ converges.  If $\sum a_k$ diverges, so does $\sum b_k$.
\end{prop}

\begin{proof} Problem.    \ns  \end{proof}

\begin{prop}[Ratio Test]\label{ratio_test}
 \index{ratio test}%
Let $(a_k)$ be a sequence in $(0,\infty)$.  If there exists $\delta \in (0,1)$ such that
$a_{k+1} \le \delta a_k$ for $k$ sufficiently large, then $\sum a_k$ converges.  If there
exists $M>1$ such that $a_{k+1} \ge Ma_k$ for $k$ sufficiently large, then $\sum a_k$
diverges.
\end{prop}

\begin{proof}  Exercise. (Solution~\ref{sol_ratio_test}.)
   \ns  \end{proof}

\begin{prop}[Integral Test]
 \index{integral test}%
Let  $f\colon [1,\infty) \sto [0,\infty)$ be decreasing; that is, $f(x) \ge f(y)$ whenever $x<
y$.  If $\lim_{M \sto \infty}\int_1^M f$ exists, then $\sum_1^\infty f(k)$ converges. If
$\lim_{M\sto\infty}\int_1^M f$ does not exist, then $\sum_1^\infty f(k)$ diverges.
\end{prop}

\begin{proof} Problem. \emph{Hint.}  Show that $\int_k^{k+1}f \le f(k) \le \int_{k-1}^kf$ for
$k \ge 2$.    \ns
\end{proof}

\begin{prop}[The Root Test]
 \index{root test}%
Let $\sum a_k$ be a series of numbers in $[0,\infty)$.  Suppose that the limit $L =
\lim_{k\sto\infty}(a_k)^{1/k}$ exists.
 \begin{enumerate}
  \item[(a)] If $L < 1$, then $\sum a_k$ converges.
  \item[(b)] If $L > 1$, then $\sum a_k$ diverges.
 \end{enumerate}
\end{prop}

\begin{proof} Problem.  \ns   \end{proof}

\begin{prob} Let $(a_k)$ and $(b_k)$ be decreasing sequences in $(0,\infty)$ and let $c_k =
\min\{a_k,b_k\}$ for each $k$.  If $\sum a_k$ and $\sum b_k$ both diverge, must $\sum c_k$
also diverge?
\end{prob}







\section{ABSOLUTE CONVERGENCE}  It is a familiar fact from beginning calculus that absolute
convergence of a series of real numbers implies convergence of the series.  The proof of this
depends in a crucial way on the completeness of~$\R$.  We show in the first proposition of
this section that for series in a normed linear space $V$ absolute convergence implies
convergence if \emph{and only if} $V$ is complete.

\begin{defn}  Let $(a_k)$ be a sequence in a normed linear space~$V$.  We say that $(a_k)$ is
 \index{absolute!summability}%
 \index{summable!absolutely}%
\df{absolutely summable} or, equivalently, that the series $\sum a_k$
 \index{convergence!absolute}%
 \index{absolute!convergence}%
\df{converges absolutely} if the series $\sum\norm{a_k}$ converges in~$\R$.
\end{defn}

\begin{prop}\label{prop_compl_abssum}  A normed linear space $V$ is complete if and only if
every absolutely summable sequence in $V$ is summable.
\end{prop}


\begin{proof}  Exercise. \emph{Hint.}  If $V$ is complete, the \emph{Cauchy
criterion}~\ref{prop_Cauchy_crit} may be used.  For the converse, suppose that every
absolutely summable sequence is summable.  Let $(a_k)$ be a Cauchy sequence in $V$.  Find a
subsequence $\bigl(a_{n_k}\bigr)$ such that $\bignorm{a_{n_{k+1}} - a_{n_k}} < 2^{-k}$ for
each $k$.  Consider the sequence $(y_k)$ where $y_k := a_{n_{k+1}} - a_{n_k}$ for all $k$.
(Solution~\ref{sol_prop_compl_abssum}.)  \ns
\end{proof}


% QUO
%The first use to which we put the preceding result
%is to fulfill the promise made in section 12.6 to
%prove that the quotient of a Banach space by a closed
%linear subspace is itself a Banach space.
%
%\proclaim{17.3.2 Proposition} Let $W$ be a closed
%linear subspace of a normed linear space $V$.
%If $V$ is complete, so is $V/W$.  \endproclaim
%
%\demo{Proof} Exercise. (Hint. Use 17.3.1.
%Let $([a_k])$ be an absolutely summable
%sequence in $V/W$.  Show that it is possible to choose,
%for each $k$ in $\N$, a vector $y_k$ in $[a_k]$ whose
%length differs from the length of $[a_k]$ by less
%than $2^{-k}$.) \enddemo



One of the most useful consequences of absolute convergence of a series is that the terms of
the series may be rearranged without affecting the sum of the series.  This is not true of
 \index{conditionally convergent}%
 \index{convergence!conditional}%
\df{conditionally convergent} series (that is, series which converge but do not converge
absolutely).  One can show, in fact, that a conditionally convergent series of real numbers
can, by rearrangement, be made to converge to any real number whatever, or, for that matter,
to diverge. We will not demonstrate this here, but a nice proof can be found
in~\cite{Apostol:1974}.

\begin{defn}  A series $\sum_{k=1}^\infty b_k$ is said to be a
 \index{rearrangement of a series}%
 \index{series!rearrangement of a}%
 \index{<@$\sum_{k=1}^\infty a_{\phi(k)}$ (rearrangement of a series)}%
\df{rearrangement} of the series $\sum_{k=1}^\infty a_k$ if there exists a bijection
$\phi\colon \N \sto \N$ such that $b_k = a_{\phi(k)}$ for all $k$ in~$\N$.
\end{defn}

\begin{prop}  If $\sum b_k$ is a rearrangement of an absolutely convergent series $\sum a_k$
in a Banach space, then $\sum b_k$ is itself absolutely convergent and it converges to the
same sum as $\sum a_k$.
\end{prop}

\begin{proof}  Problem    \emph{Hint.}   Let $\beta_n := \sum_{k=1}^n \|b_k\|$.  Show that the
sequence $(\beta_n)$ is increasing and bounded.  Conclude that $\sum b_k$ is absolutely
convergent.  The hard part of the proof is showing that if $\sum_1^\infty a_k$ converges to a
vector $A$, then so does $\sum_1^\infty b_k$.  Define partial sums as usual: $s_n := \sum_1^n
a_k$ and $t_n := \sum_1^n b_k$.  Given $\epsilon > 0$, you want to show that $\norm{t_n - A} <
\epsilon$ for sufficiently large~$n$.  Prove that there exists a positive $N$ such that
$\norm{s_n - A} < \frac12\epsilon$  and $\sum_n^\infty \norm{a_k} \le \frac12\epsilon$
whenever $n \ge N$.  Write
  \[ \norm{t_n - A} \le \norm{t_n - s_N} + \norm{s_N - A}\,. \]
Showing that $\norm{t_n - s_N} \le \frac12 \epsilon$ for $n$ sufficiently large takes a little
thought.  For an appropriate function $\phi$ write $b_k = a_{\phi(k)}$.  Notice that
  \[ \norm{t_n - s_N} = \bignorm{\sum_1^n a_{\phi(k)} - \sum_1^N a_k}\,. \]
The idea of the proof is to choose $n$ so large that there are enough terms $a_{\phi(j)}$ to
cancel \emph{all} the terms $a_k$ ($1 \le k \le N$).

If you have difficulty in dealing with sums like $\sum_{k=1}^n a_{\phi(k)}$ whose terms are
not consecutive ($a_1,a_2,\dots$ are consecutive terms of the sequence $(a_k)$;
$a_{\phi(1)},a_{\phi(2)},\dots$ in general are not), a notational trick may prove useful.  For
$P$ a \emph{finite} subset of $\N$, write $\underset P{\sum} a_k$ for the sum of all the terms
$a_k$ such that $k$ belongs to $P$.  This notation is easy to work with. It should be easy to
convince yourself that, for example, if $P$ and $Q$ are finite subsets of $\N$ and if they are
disjoint, then $\underset{P \cup Q}{\sum} a_k = \underset P{\sum}a_k + \underset Q{\sum} a_k$.
(What happens if $P \cap Q \ne \emptyset$?)  In the present problem, let $C := \{1,\dots,N\}$
(where $N$ is the integer chosen above).  Give a careful proof that there exists an integer
$p$ such that the set $\{\phi(1),\dots,\phi(p)\}$ contains $C$. Now suppose $n$ is any integer
greater than~$p$. Let $F := \{\phi(1),\dots,\phi(n)\}$ and show that
 \[\norm{t_n - s_N} \le \underset G{\sum} \norm{a_k}\]
where $G := F\setminus C$.  \ns
\end{proof}

\begin{prop}\label{prop_abssum_x_bdd}  If $(\alpha_n)$ is an absolutely summable sequence in $\R$
and $(x_n)$ is a bounded sequence in a Banach space $E$, then the sequence $(\alpha_n x_n)$ is
summable in $E$.
\end{prop}

\begin{proof} Problem.  \emph{Hint.}  Use the Cauchy criterion~\ref{prop_Cauchy_crit}.     \ns
\end{proof}

\begin{prob}  What happens in the previous proposition if the sequence $(\alpha_n)$ is assumed
only to be bounded and the sequence $(x_n)$ is absolutely summable?
\end{prob}

\begin{prob} Show that if the sequence $(a_n)$ of real numbers is
 \index{square summable}%
 \index{summable!square}%
\df{square summable} (that is, if the sequence $({a_n}^2)$ is summable), then the series $\sum
n^{-1} a_n$ converges absolutely. \emph{Hint.}  Use the \emph{Schwarz
inequality}~\ref{Schwarz_ineq}.
\end{prob}











\section{POWER SERIES}\label{pow_ser}
According to problem~\ref{prob_geomser2} we may express the reciprocal of the real number $1 -
r$ as the sum of a power series $\sum_0^\infty r^k$ provided that $\abs r < 1$.  One may
reasonably ask if anything like this is true in Banach spaces other than $\R$, in the space of
bounded linear maps from some Banach space into itself, for example.  If $T$ is such a map and
if $\norm T < 1$, is it necessarily true that $I- T$ is invertible?  And if it is, can the
inverse of $I- T$ be expressed as the sum of the power series $\sum_0^\infty T^k$?  It turns
out that the answer to both questions is \emph{yes}.  Our interest in pursuing this matter is
not limited to the fact that it provides an interesting generalization of facts concerning
$\R$ to spaces with richer structure.  In the next chapter we will need exactly this result
for the proof we give of the \emph{inverse function theorem}.

Of course it is not possible to study power series in arbitrary Banach spaces; there are, in
general, no powers of vectors because there is no multiplication.  Thus we restrict our
attention to those Banach spaces which (like the space of bounded linear maps) are equipped
with an additional operation $(x,y) \mapsto xy$ (we call it \emph{multiplication}) under which
they become linear associative algebras (see~\ref{def_alg} for the definition) and on which
the norm is
 \index{submultiplicative}%
\df{submultiplicative} (that is, $\norm{xy} \le \norm x\,\norm y$ for all $x$ and~$y$).  We
will, for simplicity, insist further that these algebras be unital and that the multiplicative
identity $\boldsymbol 1$ have norm one.  Any Banach space thus endowed is a \df{(unital)
Banach algebra}.  It is clear that Banach algebras have many properties in common with~$\R$.
It is important, however, to keep firmly in mind those properties \emph{not} shared with~$\R$.
Certainly there is, in general, no linear ordering $<$ of the elements of a Banach algebra (or
for that matter of a Banach space).  Another crucial difference is that in $\R$ every nonzero
element has a multiplicative inverse (its reciprocal); this is not true in general Banach
algebras (see, for example, proposition\ref{prop_matr_inv}).  Furthermore, Banach algebras may
have nonzero
 \index{nilpotent}%
\df{nilpotent} elements (that is, elements $x \ne 0$ such that $x^n = 0$ for some natural
number~$n$).  (Example: the $2 \times 2$ matrix
         $a = \begin{bmatrix} 0  &  1 \\
                               0  &  0
              \end{bmatrix}$ is not zero, but $a^2 = 0$.)  This,of course, prevents us from requiring
that the norm be multiplicative: while $\abs{xy} = \abs x\,\abs y$ holds in~$\R$, all that is
true in general Banach algebras is $\norm{xy} \le \norm x\,\norm y$.  Finally, multiplication
in Banach algebras need not be commutative.

\begin{defn} Let $A$ be a normed linear space.  Suppose there is an operation $(x,y) \mapsto xy$
from $A \times A$ into $A$ satisfying the following: for all $x$, $y$, $z \in A$ and $\alpha
\in \R$
 \begin{enumerate}
  \item[(a)] $(xy)z = x(yz)$,
  \item[(b)] $(x+y)z = xz+yz$,
  \item[(c)] $x(y+z) = xy+xz$,
  \item[(d)] $\alpha(xy) = (\alpha x)y = x(\alpha y)$, and
  \item[(e)] $\norm{xy} \le \norm x\,\norm y$.
 \end{enumerate}
Suppose additionally that there exists a vector $\vc 1$ in $A$ such that
 \begin{enumerate}
  \item[(f)] $x\,\vc 1 = \vc 1\,x = x$ for all $x \in A$, and
  \item[(g)] $\norm{\vc 1} = 1$.
 \end{enumerate}
Then $A$ is a
 \index{normed!algebra}%
 \index{algebra!normed}%
 \index{identity@$\vc 1$ (multiplicative identity in an algebra)}%
 \index{unital!normed algebra}%
\df{(unital) normed algebra}.  If $A$ is complete it is a
 \index{Banach!algebra}%
 \index{algebra!Banach}%
 \index{unital!Banach algebra}%
\df{(unital) Banach algebra}.  If all elements $x$ and $y$ in a normed (or Banach) algebra
satisfy $xy = yx$, then the algebra $A$ is
 \index{commutative!normed algebra}%
 \index{commutative!Banach algebra}%
\df{commutative}.
\end{defn}

\begin{exam}  The set $\R$ of real numbers is a commutative Banach algebra.  The number $1$
is the multiplicative identity.
\end{exam}

\begin{exam}  If $S$ is a nonempty set, then with pointwise multiplication
  \[ (fg)(x) :=  f(x)g(x) \qquad \text{for all $x \in S$} \]
the Banach space $\fml B(S,\R)$ becomes a commutative Banach algebra.  The constant function
$\vc 1 \colon x \mapsto 1$ is the multiplicative identity.
\end{exam}

\begin{exer}\label{exer_bddfn_ba}  Show that if $f$ and $g$ belong to $\fml B(S,\R)$, then
$\norm{fg}_u \le \norm f_u\norm g_u$.  Show by example that equality need not hold.
(Solution~\ref{sol_exer_bddfn_ba}.)
\end{exer}

\begin{exam} If $M$ is a compact metric space then (again with pointwise multiplication)
$\fml C(M,\R)$ is a commutative Banach algebra.  It is a
 \index{subalgebra}%
\df{subalgebra} of $\fml B(M,\R)$ (that is, a subset of $\fml B(M,\R)$ containing the
multiplicative identity which is a Banach algebra under the induced operations).
\end{exam}

\begin{exam} If $E$ is a Banach space, then $\ofml B(E,E)$ is a Banach algebra (with
composition as ``multiplication'').  We have proved in problem~\ref{prob_lt_alg} that the
space of linear maps from $E$ into $E$ is a unital algebra.  The same is easily seen to be
true of $\ofml B(E,E)$ the corresponding space of bounded linear maps.  The identity
transformation $I_E$ is the multiplicative identity.  Its norm is $1$
by~\ref{exer_exam_norm}(a).  In proposition~\ref{prop_opnorm} it was shown that $\ofml B(E,E)$
is a normed linear space; the submultiplicative property of the norm on this space was proved
in proposition~\ref{prop_comp_op}. Completeness was proved in
proposition~\ref{prop_BVW_compl}.
\end{exam}

\begin{prop}  If $A$ is a normed algebra, then the operation of multiplication
  \[ M \colon A \times A \sto A\colon  (x,y) \mapsto xy \]
is continuous.
\end{prop}

\begin{proof} Problem.   \emph{Hint.}  Try to adapt the proof of example~\ref{cont_mult}.  \ns
\end{proof}

\begin{cor}  If $x_n \sto a$ and $y_n \sto b$ in a normed algebra, then $x_ny_n \sto ab$.
\end{cor}

\begin{proof} Problem.  \ns  \end{proof}

\begin{defn}  An element $x$ of a unital algebra $A$ (with or without norm) is
 \index{invertible!element of an algebra}%
\df{invertible} if there exists an element
 \index{inverse!of an element of an algebra}%
 \index{<@$x^{-1}$ (inverse of an element of an algebra}%
$x^{-1}$ (called the
 \index{inverse!multiplicative}%
 \index{multiplicative inverse}%
\df{multiplicative inverse} of $x$) such that $xx^{-1} = x^{-1}x = \vc 1$.  The set of all
invertible elements of A is denoted
 \index{invertible@$\inv A$ (invertible elements of an algebra)}%
by~$\inv A$. We list several almost obvious properties of inverses.
\end{defn}

\begin{prop}\label{prop_prop_inv}  If $A$ is a unital algebra, then
 \begin{enumerate}
   \item[(a)]  Each element of $A$ has at most one multiplicative inverse.
   \item[(b)]  The multiplicative identity $\boldsymbol 1$ of $A$ is invertible and $\vc 1^{-1}
= \vc 1$.
   \item[(c)]  If $x$ is invertible, then so is $x^{-1}$ and $\bigl(x^{-1}\bigr)^{-1} = x$.
   \item[(d)]  If $x$ and $y$ are invertible, then so is $xy$ and $(xy)^{-1} = y^{-1}x^{-1}$.
   \item[(e)]  If $x$ and $y$ are invertible, then $x^{-1} - y^{-1} = x^{-1}(y - x)y^{-1}$.
 \end{enumerate}
\end{prop}

\begin{proof}  Let $\vc 1$ be the multiplicative identity of~$A$.
 \begin{enumerate}
   \item[(a)] If $y$ and $z$ are multiplicative inverses of $x$, then $y = y\,\vc 1 = y(xz)
= (yx)z = \vc 1\,z = z$.
   \item[(b)] $\vc 1 \cdot \vc 1 = \vc 1$ implies $\vc 1^{-1} = \vc 1$ by~(a).
   \item[(c)] $x^{-1}x = xx^{-1} = 1$ implies $x$ is the inverse of~$x^{-1}$ by~(a).
   \item[(d)] $(xy)(y^{-1}x^{-1}) = xx^{-1} = \vc 1$ and $(y^{-1}x^{-1})(xy) = y^{-1}y =
\vc 1$ imply $y^{-1}x^{-1}$ is the inverse of $xy$ (again by~(a)).
   \item[(e)] $x^{-1}(y - x)y^{-1} = (x^{-1}y - 1)y^{-1} = x^{-1} - y^{-1}$.
 \end{enumerate}
\end{proof}

\begin{prop}\label{prop_inv_ps} If $x$ is an element of a unital Banach algebra and $\norm x < 1$,
then $\vc 1 - x$ is invertible and $(\vc 1 - x)^{-1} = \sum_{k=0}^\infty x^k$.
\end{prop}

\begin{proof} Exercise. \emph{Hint.}  First show that the geometric series $\sum_{k=0}^\infty x^k$
converges absolutely.  Next evaluate $(\vc 1 - x)s_n$ and $s_n(\vc 1 - x)$ where $s_n =
\sum_{k=0}^n x^k$.  Then take limits as $n \sto \infty$.  (Solution~\ref{sol_prop_inv_ps}.)
\end{proof}

\begin{cor}\label{cor_inv_ps}  If $x$ is an element of a unital Banach algebra and
$\norm x < 1$, then
  \[ \norm{(\boldsymbol 1-x)^{-1} - \boldsymbol 1} \le \frac{\norm x}{1 - \norm x}\,. \]
\end{cor}

\begin{proof} Problem.   \ns  \end{proof}

Proposition~\ref{prop_inv_ps} says that anything close to $\vc 1$ in a Banach algebra $A$ is
invertible.  In other words $\vc 1$ is an interior point of~$\inv A$.
Corollary~\ref{cor_inv_ps} says that if $\norm x$ is small, that is, if $\vc 1 - x$ is close
to $\vc 1$, then $(\vc 1 - x)^{-1}$ is close to $\vc 1^{-1}$ ($= \vc 1$). In other words, the
operation of inversion $x \mapsto x^{-1}$ is continuous at~$\vc 1$. These results are actually
special cases of much more satisfying results: every point of $\inv A$ is an interior point of
that set (that is, $\inv A$ is open); and the operation of inversion is continuous at every
point of $\inv A$.  We use the special cases above to prove the more general results.

\begin{prop}  If $A$ is a Banach algebra, then $\open{\inv A}A$.
\end{prop}

\begin{proof} Problem.  \emph{Hint.}  Let $a \in \inv A$, $r = \norm{a^{-1}}^{-1}$, $y \in
B_r(a)$, and $x = a^{-1}(a - y)$.  What can you say about~$a(\boldsymbol 1 - x)$?  \ns
\end{proof}

\begin{prop}\label{prop_inv_cont} If $A$ is a Banach algebra, then the operation of inversion
  \[ r \colon \inv A \sto \inv A \colon  x \mapsto x^{-1} \]
is continuous.
\end{prop}

\begin{proof} Exercise.  \emph{Hint.}  Show that $r$ is continuous at an arbitrary point $a$ in
$\inv A$.  Given $\epsilon > 0$ find $\delta > 0$ sufficiently small that if $y$ belongs to
$B_{\delta}(a)$ and $x = \boldsymbol 1 - a^{-1}y$, then
  \[ \norm x < \norm{a^{-1}}\delta < \tfrac12 \]
and
  \begin{equation}\label{eqn_inv_cont1}
     \norm{r(y) - r(a)} \le \frac{\norm x \norm{a^{-1}}}{1 - \norm x} < \epsilon
  \end{equation}
For \eqref{eqn_inv_cont1} use \ref{prop_prop_inv}(e), the fact that $y^{-1}a =
\bigl(a^{-1}y\bigr)^{-1}$, and~\ref{cor_inv_ps}.  (Solution~\ref{sol_prop_inv_cont}.) \ns
\end{proof}

There are a number of ways of multiplying infinite series.  The most common is the
\emph{Cauchy product}.  And it is the only one we will consider.

\begin{defn}  If $\sum_{n=0}^\infty a_n$ and $\sum_{n=0}^\infty b_n$ are infinite series in a
Banach algebra, then their
 \index{Cauchy!product}%
 \index{product!Cauchy}%
\df{Cauchy product} is the series $\sum_{n=0}^\infty c_n$ where $c_n = \sum_{k=0}^n a_k
b_{n-k}$.  To see why this definition is rather natural, imagine trying to multiply two power
series $\sum_0^\infty a_kx^k$ and $\sum_0^\infty b_kx^k$ just as if they were infinitely long
polynomials.  The result would be another power series.  The coefficient of $x^n$ in the
resulting series would be the sum of all the products $a_ib_j$ where $i+j = n$. There are
several ways of writing this sum
  \[ \sum_{i+j = n}a_ib_j = \sum_{k=0}^n a_kb_{n-k} = \sum_{k=0}^n a_{n-k}b_k\,. \]
If we just forget about the variable $x$, we have the preceding definition of the Cauchy
product.
\end{defn}

The first thing we observe about these products is that convergence of both the series $\sum
a_k$ and $\sum b_k$ does \emph{not} imply convergence of their Cauchy product.

\begin{exam}  Let $a_k = b_k = (-1)^k(k+1)^{-1/2}$ for all $k \ge 0$.  Then $\sum_0^\infty a_k$
and $\sum_0^\infty b_k$ converge by the alternating series test (problem~\ref{prob_ast}).  The
$n^{\text{th}}$ term of their Cauchy product is
  \[ c_n = \sum_{k=0}^na_kb_{n-k} = (-1)^n\sum_{k=0}^n\frac1{\sqrt{k+1}\sqrt{n-k+1}}\,. \]
Since for $0 \le k \le n$
  \begin{align*}
      (k + 1)(n - k + 1) &= k(n - k) + n + 1 \\
                         &\le n^2 + 2n + 1 \\
                         &= (n + 1)^2
  \end{align*}
we see that
  \[ \abs{c_n} = \sum_{k=0}^n\frac1{\sqrt{k + 1}\sqrt{n - k + 1}}
                                        \ge \sum_{k=0}^n\frac1{n + 1} = 1\,. \]
Since $c_n \nrightarrow 0$, the series $\sum_0^\infty c_k$ does not converge (see
proposition~\ref{prop_terms_zero}).
\end{exam}

Fortunately quite modest additional hypotheses do guarantee convergence of the Cauchy product.
One very useful sufficient condition is that at least one of the series $\sum a_k$ or $\sum
b_k$ converge absolutely.

\index{Mertens' theorem}%
\begin{thm}[Mertens' Theorem]\label{thm_Mertens} If in a Banach algebra $\sum_0^\infty a_k$ is
absolutely convergent and has sum $a$ and the series $\sum_0^\infty b_k$ is convergent with
sum~$b$, then the Cauchy product of these series converges and has sum~$ab$.
\end{thm}

\begin{proof}  Exercise \emph{Hint.}  Although this exercise is slightly tricky, the difficulty
has nothing whatever to do with Banach algebras.  Anyone who can prove \emph{Mertens' theorem}
for series of real numbers can prove it for arbitrary Banach algebras.

For each $k \in \N$ let $c_k = \sum_{j=0}^k a_j b_{k-j}$.  Let $s_n$, $t_n$, and $u_n$ be the
$n^{\text{th}}$ partial sums of the sequences $(a_k)$, $(b_k)$, and $(c_k)$, respectively.
First verify that for every $n$ in $\N$
  \begin{equation}\label{eqn_Mertens1}
        u_n =  \sum_{k=0}^n a_{n-k}t_k
  \end{equation}
To this end define, for $0 \le j,k \le n$, the vector $d_{jk}$ by
  \[ d_{jk} =
        \begin{cases}  a_jb_{k-j},   &\text{if $j \le k$} \\
                            0,       &\text{if $j > k$.}
        \end{cases}\]
Notice that both the expression which defines $u_n$ and the expression on the right side of
equation~\eqref{eqn_Mertens1} involve only finding the sum of the elements of the matrix
$[d_{jk}]$ but in different orders.

From \eqref{eqn_Mertens1} it is easy to see that for every $n$
  \[ u_n = s_nb + \sum_{k=0}^n a_{n-k}(t_k - b). \]
Since $s_n b \sto ab$ as $n \sto \infty$, the proof of the theorem is reduced to showing that
  \[ \norm{\sum_{k=0}^n a_{n-k}(t_k - b)} \sto 0 \qquad \text{as $n \sto \infty$.} \]
Let $\alpha_k = \norm{a_k}$ and $\beta_k = \norm{t_k - b}$ for all~$k$.  Why does it suffice
to prove that $\sum_{k=0}^n \alpha_{n-k}\beta_k \sto 0$ as $n \sto \infty$? In the finite sum
 \begin{equation}\label{eqn_Mertens2}
     \alpha_n\beta_0 + \alpha_{n-1}\beta_1 + \dots + \alpha_0\beta_n
 \end{equation}
the $\alpha_k$'s towards the left are small (for large~$n$) and the $\beta_k$'s towards the
right are small (for large~$n$).  This suggests breaking the sum~\eqref{eqn_Mertens2} into two
pieces
  \[ p = \sum_{k=0}^{n_1} \alpha_{n-k}\beta_k \qquad \text{and}
                                 \qquad q = \sum_{k=n_1+1}^n \alpha_{n - k}\beta_k \]
and trying to make each piece small (smaller, say, than $\frac12\epsilon$ for a
preassigned~$\epsilon$).

For any positive number $\epsilon_1$ it is possible (since $\beta_k \sto 0$) to choose $n_1$
in $\N$ so that $\beta_k < \epsilon_1$ whenever $k \ge n_1$.  What choice of $\epsilon_1$ will
ensure that $q < \frac12\epsilon$?

For any $\epsilon_2 > 0$ it is possible (since $\alpha_k \sto 0$) to choose $n_2$ in $\N$ so
that $\alpha_k < \epsilon_2$ whenever $k \ge n_2$.  Notice that $n-k \ge n_2$ for all $k \le
n_1$ provided that $n \ge n_1 + n_2$.  What choice of $\epsilon_2$ will guarantee that $p <
\frac12\epsilon$?  (Solution~\ref{sol_thm_Mertens}.) \ns
\end{proof}

\begin{prop}\label{prop_bamult_diff}  On a Banach algebra $A$ the operation of multiplication
  \[ M\colon  A \times A \sto A\colon  (x,y) \mapsto xy \]
is differentiable.
\end{prop}

\begin{proof} Problem.   \emph{Hint.}   Fix $(a,b)$ in $A \times A$.  Compute the value of the
function $\Delta M_{(a,b)}$ at the point $(h,j)$ in $A \times A$.  Show that
$\norm{(h,j)}^{-1}hj \sto 0$ in A as $(h,j) \sto 0$ in $A \times A$.  How should $dM_{(a,b)}$
be chosen so that the Newton quotient
  \[ \frac{\Delta M_{(a,b)}(h,j) - dM_{(a,b)}(h,j)}{\norm{(h,j)}} \]
approaches zero (in $A$) as $(h,j) \sto 0$ (in $A \times A$)?  Don't forget to show that your
choice for $dM_{(a,b)}$ is a bounded linear map.     \ns
\end{proof}

\begin{prop}  Let $c \in V$ where $V$ is a normed linear space and let $A$ be a Banach algebra.
If $f$, $g \in \fml D_c(V,A)$, then their product $fg$ defined by
  \[ (fg)(x) := f(x)\, g(x) \]
(for all $x$ in some neighborhood of~$c$) is differentiable at $c$ and
  \[ d(fg)_c = f(c)dg_c + df_c \cdot g(c)\,. \]
\end{prop}

\begin{proof} Problem.  \emph{Hint.}  Use proposition~\ref{prop_bamult_diff}.  \ns
\end{proof}

\begin{prop}\label{prop_bapower_diff}  If $A$ is a commutative Banach algebra and $n \in \N$,
then the function $f \colon x \mapsto x^n$ is differentiable and $df_a(h) = na^{n-1}h$ for all
$a$, $h \in A$.
\end{prop}

\begin{proof} Problem.  \emph{Hint.}  A simple induction proof works.  Alternatively, you may
choose to convince yourself that the usual form of the \emph{binomial theorem}
(see~\ref{binom_thm}) holds in every commutative Banach algebra.   \ns
\end{proof}

\begin{prob}  What happens in \ref{prop_bapower_diff} if we do not assume that the Banach algebra
is commutative?  Is $f$ differentiable? \emph{Hint.}  Try the cases $n = 2$, $3$, and~$4$.
Then generalize.
\end{prob}

\begin{prob}  Generalize proposition~\ref{prop_abssum_x_bdd} to produce a theorem concerning the
convergence of a series $\sum a_kb_k$ in a Banach algebra.
\end{prob}


\begin{defn} If $(a_k)$ is a sequence in a Banach algebra $A$ and $x \in A$, then a series of the
form $\sum_{k=0}^\infty a_kx^k$ is a
 \index{power series}%
 \index{series!power}%
\df{power series} in~$x$.
\end{defn}

\begin{notn} It is a bad (but very common) habit to use the same notation for a polynomial function
and for its value at a point~$x$.  One often encounters an expression such as ``the function
$x^2 - x + 5$'' when clearly what is meant is ``the function $x \mapsto x^2 - x + 5$.''  This
abuse of language is carried over to power series. If $(a_k)$ is a sequence in a Banach
algebra $A$ and $D = \{x \in A\colon \sum_{k=0}^\infty a_kx^k \text{ converges}\}$, then the
function $x \mapsto \sum_{k=0}^\infty a_kx^k$ from $D$ into $A$ is usually denoted by
$\sum_{k=0}^\infty a_kx^k$.  Thus for example one may find the expression,
``$\sum_{k=0}^\infty a_kx^k$ converges uniformly on the set~$U$.''  What does this mean?
Answer: if $s_n(x) = \sum_{k=0}^n a_kx^k$ for all $n \in \N$ and $f(x) = \sum_{k=0}^\infty
a_kx^k$, then $s_n \sto f\text{\,(unif)}$.
\end{notn}

\begin{prop}\label{prop_pser_unif_conv}  Let $(a_k)$ be a sequence in a Banach algebra and
$r > 0$.  If the sequence $(\norm{a_k}r^k)$ is bounded, then the power series
$\sum_{k=0}^\infty a_kx^k$ converges uniformly on every open ball $B_s(0)$ such that $0 < s <
r$.  (And therefore it converges on the ball~$B_r(0)$. )
\end{prop}

\begin{proof} Exercise. \emph{Hint.} Let $p = s/r$ where $0 < s < r$.  Let $f_k(x) = a_kx^k$.
Use the \emph{Weierstrass M-test}~\ref{M_test}.  (Solution~\ref{sol_prop_pser_unif_conv}.) \ns
\end{proof}

Since the uniform limit of a sequence of continuous functions is continuous
(see~\ref{unif_lim_cont}), it follows easily from preceding proposition that (under the
hypotheses given) the function $f\colon x \mapsto \sum_0^\infty a_kx^k$ is continuous on
$B_r(0)$.  We will prove considerably more than this: the function is actually differentiable
on $B_r(0)$, and furthermore, the correct formula for its differential is found by
differentiating the power series term-by-term.  That is, $f$ behaves on $B_r(0)$ just like a
``polynomial'' with infinitely many terms.  This is proved in~\ref{term_by_term_diff}.  We
need however a preliminary result.

\begin{prop}\label{unif_lim_dffrntls}  Let $V$ and $W$ be normed linear spaces, $U$ be an open
convex subset of~$V$, and $(f_n)$ be a sequence of functions in $\fml C^1(U,W)$.  If the
sequence $(f_n)$ converges pointwise to a function $F\colon U \sto W$ and if the sequence
$\bigl(d(f_n)\bigr)$ converges uniformly on $U$, then $F$ is differentiable at each point $a$
of $U$ and
  \[ dF_a = \lim_{n \sto \infty} d\bigl(f_n\bigr)_a\,. \]
\end{prop}

\begin{proof}  Exercise. \emph{Hint.} Fix $a \in U$.  Let
$\phi\colon U \sto \ofml B(V,W)$ be the function to which the sequence $\bigl(d(f_n)\bigr)$
converges uniformly. Given $\epsilon
> 0$ show that there exists $N \in \N$ such that
$\bignorm{d\bigl(f_n - f_N\bigr)_x} < \epsilon/4$ for all $x \in
U$ and $n \in \N$.  Let $g_n = f_n - f_N$ for each~$n$.  Use one
version of the \emph{mean value theorem} to show that
\begin{equation}\label{eqn_uld}
   \norm{\Delta\bigl(g_n\bigr)_a(h) - d\bigl(g_n\bigr)_a(h)}
                \le \textstyle{\frac12}\epsilon\norm h
\end{equation}
whenever $n \ge N$ and $h$ is a vector such that $a + h \in U$. In~\eqref{eqn_uld} take the
limit as $n \sto \infty$.  Use the result of this together with the fact that
$\Delta\bigl(f_N\bigr)_a \simeq d\bigl(f_N\bigr)_a$ to show that $\Delta F_a \simeq T$ when $T
= \phi(a)$.  (Solution~\ref{sol_unif_lim_dffrntls}.)   \ns
\end{proof}

\begin{thm}[Term-by-Term Differentiation of Power Series]\label{term_by_term_diff} Suppose that
$(a_n)$ is a sequence in a commutative Banach algebra $A$ and that $r > 0$. If the sequence
$(\norm{a_k}r^k)$ is bounded, then the function $F\colon B_r(0) \sto A$ defined by
  \[ F(x) = \sum_{k=0}^\infty a_kx^k \]
is differentiable and
  \[ dF_x(h) = \sum_{k=1}^\infty k a_k x^{k-1}h \]
for every $x \in B_r(0)$ and $h \in A$.
\end{thm}

\begin{proof} Problem.  \emph{Hint.}  Let $f_n(x) = \sum_0^n a_kx^k$. Fix $x$ in $B_r(0)$ and
choose a number $s$ satisfying $\norm x < s < r$. Use propositions \ref{prop_bapower_diff}
and~\ref{unif_lim_dffrntls} to show that $F$ is differentiable at $x$ and to compute its
differential there. In the process you will need to show that the sequence
$\bigl(d(f_n)\bigr)$ converges uniformly on $B_s(0)$. Use propositions
\ref{prop_pser_unif_conv} and~\ref{4exam2}.  If $s < t < r$, then there exists $N \in \N$ such
that $k^{1/k} < r/t$ for all $k \ge N$. (Why ?) \ns
\end{proof}

\begin{prob} Give a definition of and develop the properties of the \emph{exponential function}
on a commutative Banach algebra~$A$.  Include at least the following:
\begin{enumerate}
  \item[(a)] The series $\sum_{k=0}^\infty \frac1{k!}x^k$ converges absolutely for all $x$
in~$A$ and uniformly on $B_r(0)$ for every $r > 0$.
  \item[(b)] If $\exp(x) := \sum_{k=0}^\infty \frac1{k!}x^k$, then $\exp\colon A \sto A$ is
differentiable and
    \[ d\exp_x(h) = \exp(x) \cdot h\,. \]
  This is the
 \index{exponential function}%
 \index{function!exponential}%
\df{exponential function} on~$A$.
  \item[(c)] If $x$, $y \in A$, then
    \[ \exp(x)\cdot\exp(y) = \exp(x + y)\,. \]
  \item[(d)] If $x \in A$, then $x$ is invertible and
    \[ \bigl(\exp(x)\bigr)^{-1} = \exp(-x)\,. \]
\end{enumerate}
\end{prob}



\begin{prob} Develop some trigonometry on a commutative Banach algebra~$A$.  (It will be convenient
to be able to take the
 \index{derivative!in Banach algebras}%
\df{derivative} of a Banach algebra valued function.  If $G \colon A \sto A$ is a
differentiable function, define $DG(a) := dG_a(1)$ for every $a \in A$.)  Include at least the
following:
\begin{enumerate}
  \item[(a)] The series $\sum_{k=0}^\infty \frac{(-1)^k}{(2k)!}x^k$ converges absolutely for all
$x$ in~$A$ and uniformly on every open ball centered at the origin.
  \item[(b)] The function $F\colon x \mapsto \sum_{k=0}^\infty \frac{(-1)^k}{(2k)!}x^k$ is
differentiable at every $x \in A$.   Find $dF_x(h)$.
  \item[(c)] For every $x\in A$, let $\cos x := F(x^2)$.  Let $\sin x := D\cos x$ for every~$x$.
Show that
    \[ \sin x = \sum_{k=0}^\infty \frac{(-1)^k}{(2k+1)!}x^{2k+1} \]
for every $x \in A$.
  \item[(d)] Show that $D\sin x = \cos x$ for every $x \in A$.
  \item[(e)] Show that $\sin^2 x + \cos^2 x = 1$ for every $x \in
  A$.
\end{enumerate}
\end{prob}




\endinput
