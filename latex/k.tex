\chapter{PRODUCTS, RELATIONS, AND FUNCTIONS}\label{fcns}

\section{CARTESIAN PRODUCTS}  Ordered pairs are familiar objects. They are used among other
things for coordinates of points in the plane.  In the first sentence of chapter~\ref{sets} it
was promised that all subsequent mathematical objects would be defined in terms of sets.  So
here just for the record is a formal definition of ``ordered pair''.

\begin{defn} Let $x$ and $y$ be elements of arbitrary sets.  Then the
 \index{<@$(x,y)$ (ordered pair)}%
 \index{ordered!pair}%
\df{ordered pair} $(x,y)$ is defined to be $\{\{x,y\},\{x\}\}$.  This definition reflects our
intuitive attitude: an ordered pair is a set $\{x,y\}$ with one of the elements, here $x$,
designated as being ``first''.  Thus we specify two things: $\{x,y\}$ and~$\{x\}$.
\end{defn}

Ordered pairs have only one interesting property: two of them are equal if and only if both
their first coordinates and their second coordinates are equal. As you will discover by
proving the next proposition, this fact follows easily from the definition.

\begin{prop}\label{lex1} Let $x$, $y$, $u$, and $v$ be elements of arbitrary sets.  Then
$(x,y) = (u,v)$ if and only if $x = u$ and $y = v$.
\end{prop}

\begin{proof} Exercise. \emph{Hint.}  Do not assume that the set $\{x,y\}$ has two elements.
If $x = y$, then $\{x,y\}$ has only one element. (Solution~\ref{sol_lex1}.)  \ns
\end{proof}

\begin{prob} Asked to define an ``ordered triple'', one might be tempted to try a definition
analogous to the definition of ordered pairs: let $(a,b,c)$ be $\{\{a,b,c\},\{a,b\},\{a\}\}$.
This appears to specify the entries, to pick out a ``first'' element, and to identify the
``first two'' elements.  Explain why this won't work.  (See \ref{triple} for a definition that
does work.)
\end{prob}

\begin{defn} Let $S$ and $T$ be sets.  The
 \index{<@$\times$ (Cartesian product)}%
 \index{product!Cartesian}%
\df{Cartesian product} of $S$ and $T$, denoted by $S \times T$, is defined to be
$\{(x,y)\colon x \in S \text{ and } y \in T\}$.  The set $S \times S$ is often denoted
by~$S^2$.
\end{defn}

\begin{exam} Let $S = \{1,x\}$ and $T = \{x,y,z\}$.  Then
  \[ S \times T = \{(1,x), (1,y), (1,z), (x,x), (x,y), (x,z)\}\,. \]
\end{exam}

\begin{prob} Let $S = \{0,1,2\}$ and $T = \{1,2,3\}$.  List all members of
$(T \times S) \setminus (S \times T)$.
\end{prob}

\begin{prob} Let $S$, $T$, $U$, and $V$ be sets.  Then
 \begin{enumerate}
   \item[(a)] $(S \times T) \cap (U \times V) = (S \cap U) \times (T \cap V)$;
   \item[(b)] $(S \times T) \cup (U \times V) \subseteq (S \cup U)
        \times (T \cup V)$; and
   \item[(c)] equality need not hold in (b).
 \end{enumerate}
\end{prob}

The proofs of (a) and (b) in the preceding problem are not particularly difficult.
Nonetheless, before one can write down a proof one must have a conjecture as to what is true.
How could we have \emph{guessed} initially that equality holds in (a) but not in~(b)?  The
answer is, as it frequently is in mathematics, \emph{by looking at pictures.}  Try the
following:  Make a sketch where $S$ and $U$ are overlapping intervals on the $x$-axis and $T$
and $V$ are overlapping intervals on the $y$-axis.  Then $S \times T$ and $U \times V$ are
overlapping rectangles in the plane.  Are not (a) and (b) almost obvious from your sketch?

We will also have occasion to use ordered $n$-tuples and $n$-fold Cartesian products for $n$
greater than~$2$.

\begin{defn}\label{triple} Let $n \ge 3$.  We define
 \index{ordered!$n$-tuple}%
\df{ordered} $n$-\df{tuples} inductively.  Suppose ordered $(n-1)$-tuples
$\bigl(x_1,\dots,x_{n-1}\bigr)$ have been defined. Let $\bigl(x_1,\dots,x_n\bigr) :=
\left(\bigl(x_1,\dots,x_{n-1}\bigr),x_n\right)$. An easy inductive proof shows that
$\bigl(x_1,\dots,x_n\bigr) = \bigl(y_1,\dots,y_n\bigr)$ if and only if $x_k = y_k$ for $k = 1,
\dots, n$.
\end{defn}

\begin{defn} If $S_1,\dots,S_n$ are sets, we define the
 \index{Cartesian product}%
 \index{product!Cartesian}%
 \index{<@$S_1 \times \dots \times S_n$ ($n$-fold Cartesian product)}%
\df{Cartesian product} $S_1 \times \dots \times S_n$ to be the set of all ordered
 \index{<@$(x_1,x_2,\dots,x_n)$ ($n$-tuples)}%
 \index{ntuple@$n$-tuple}%
$n$-tuples $\bigl(x_1,\dots,x_n)$ where $x_k \in S_k$  for $1 \le k \le n$.  We write
 \index{<@$S^n$ ($n$-fold Cartesian product)}%
$S^n$ for $S \times \dots \times S$ ($n$ factors).
\end{defn}

\begin{exam}  The $n$-fold Cartesian product of the set $\R$ of real numbers is the set
 \index{<@$\R^n$ ($n$-space)}%
 \index{realnumberss@$\R^n$ ($n$-space)}%
 \index{nspace@$n$-space}%
$\R^n$ of all $n$-tuples of real numbers and is often called (Euclidean) $n$-space.
\end{exam}







\section{RELATIONS}  Calculus is primarily about functions.  We differentiate functions,
we integrate them, we represent them as infinite series.  A function is a special kind of
relation.  So it is convenient before introducing functions to make a few observations
concerning the more general concept---relations.

\begin{defn} A
 \index{relation}%
\df{relation} from a set $S$ to a set $T$ is a subset of the Cartesian product $S \times T$. A
relation from the set $S$ to itself is often called a relation \emph{on} $S$ or a relation
\emph{among members of}~$S$.
\end{defn}

There is a notational oddity concerning relations.  To indicate that an ordered pair $(a,b)$
belongs to a relation $R \subseteq S \times T$, we almost always write something like $aRb$
rather than $(a,b) \in R$, which we would expect from the definition.  For example, the
relation ``less than'' is a relation on the real numbers.  (We discussed this relation in
appendix \ref{order_R}.) Technically then, since $<$ is a subset of $\R \times \R$, we could
(correctly) write expressions such as $(3,7) \in <$.  Of course we don't.  We write $3<7$
instead.  And we say, ``$3$ is less than~$7$'', not ``the pair $(3,7)$ belongs to the relation
\emph{less than}''.   This is simply a matter of convention; it has no mathematical or logical
content.









\section{FUNCTIONS} Functions are familiar from beginning calculus. Informally, a function
consists of a pair of sets and a ``rule'' which associates with each member of the first set
(the \emph{domain}) one and only one member of the second (the \emph{codomain}).  While this
informal ``definition'' is certainly adequate for most purposes and seldom leads to any
misunderstanding, it is nevertheless sometimes useful to have a more precise formulation.
This is accomplished by defining a function to be a special type of relation between two sets.

\begin{defn} A
 \index{function}%
\df{function} $f$ is an ordered triple $(S,T,G)$ where $S$ and $T$ are sets and $G$ is a
subset of $S \times T$ satisfying:
 \begin{enumerate}
   \item for each $s \in S$ there is a $t \in T$ such that $(s,t) \in G$, and
   \item if $(s,t_1)$ and $(s,t_2)$ belong to G, then $t_1 = t_2$.
 \end{enumerate}
In this situation we say that $f$ is a \emph{function from} $S$ \emph{into} $T$ (or that $f$
\emph{maps} $S$ \emph{into} $T$) and write $f: S \sto T$.  The set $S$ is the
 \index{domain}%
\df{domain} (or the
 \index{input space}%
 \index{space!input}%
\df{input space}) of~$f$.  The set $T$ is the
 \index{codomain}%
\df{codomain} (or
 \index{target space}%
 \index{space!target}%
\df{target space}, or the
 \index{output space}%
 \index{space!output}%
\df{output space}) of~$f$.  And the relation $G$ is the
 \index{graph}%
\df{graph} of~$f$.  In order to avoid explicit reference to the graph $G$ it is usual to
replace the expression ``$(x,y) \in G\,$'' by ``$y = f(x)$''; the element $f(x)$ is the
 \index{image!of a point}%
\df{image} of $x$ under~$f$. In this text (but not everywhere!) the words
 \index{transformation}%
``transformation'',
 \index{map}%
``map'', and
 \index{mapping}%
``mapping'' are synonymous with ``function''.  The domain of $f$ is denoted
 \index{domain@$\dom f$ (domain of a function $f$)}%
by~$\dom f$.
\end{defn}

\begin{exam} There are many ways of specifying a function.  Statements (1)--(4) below define
exactly the same function.  We will use these (and other similar) notations interchangeably.
 \begin{enumerate}
   \item[(1)] For each real number $x$ we let $f(x) =x^2$.
   \item[(2)] Let $f = (S,T,G)$ where $S = T = \R$ and $G = \{(x,x^2) \colon x \in \R\}$.
   \item[(3)] Let $f\colon \R \sto \R$ be defined by $f(x) = x^2$.
   \item[(4)] Consider the function
 \index{<@$f\colon A \sto B\colon x \mapsto f(x)$ (function notation)}%
$f\colon \R \sto \R\colon  x \mapsto x^2$.
 \end{enumerate}
\end{exam}

\begin{notn} If $S$ and $T$ are sets we denote
 \index{functions@$\fml F(S,T)$ (functions from $S$ into $T$)}%
by $\fml F(S,T)$ the family of all functions from $S$ into~$T$.
\end{notn}

\begin{conv} A
 \index{real valued function}%
\df{real valued} function is a function whose codomain lies in
$\R$. A function
 \index{function!of a real variable}%
\df{of a real variable} is a function whose domain is contained in~$\R$.  Some real valued
functions of a real variable may be specified simply by writing down a formula.  When the
domain and codomain are not specified, the understanding is that the domain of the function is
the largest set of real numbers for which the formula makes sense and the codomain is taken to
be~$\R$.

In the case of real valued functions on a set $S$, we frequently
 \index{functions@$\fml F(S)$, $\fml F(S,\R)$ (real valued functions on a set)}%
write $\fml F(S)$ instead of~$\fml F(S,\R)$.
\end{conv}

\begin{exam} Let $f(x) = (x^2 + x)^{-1}$. Since this formula is meaningful for all real numbers
except $-1$ and $0$, we conclude that the domain of $f$ is $\R \setminus \{-1,0\}$.
\end{exam}

\begin{exam} Let $f(x) = (x^2 + x)^{-1}$ for $x > 0$. Here the domain of $f$ is specified: it is
the interval $(0,\infty)$.
\end{exam}

\begin{exer}\label{lex2} Let $f(x)=(1- 2(1+(1-x)^{-1})^{-1})^{-1}$.
 \begin{enumerate}
  \item[(a)] Find $f(\frac12)$.
  \item[(b)] Find the domain of $f$.
 \end{enumerate}
(Solution~\ref{sol_lex2}.)
\end{exer}

\begin{exer}\label{lex3} Let $f(x) = (-x^2 - 4x -1)^{-1/2}$. Find the domain of~$f$.
(Solution~\ref{sol_lex3}.) \end{exer}

\begin{prob} Let $f(x)=(1-(2+(3-(1+x)^{-1})^{-1})^{-1})^{-1}$
 \begin{enumerate}
  \item[(a)] Find $f(\frac12)$.
  \item[(b)] Find the domain of $f$.
 \end{enumerate}
\end{prob}

\begin{prob} Let $f(x)=(-x^2-7x-10)^{-1/2}$.
 \begin{enumerate}
  \item[(a)] Find $f(-3)$.
  \item[(b)] Find the domain of $f$.
 \end{enumerate}
\end{prob}

\begin{prob} Let $f(x)=\dfrac{\sqrt{x^2-4}}{5-\sqrt{36-x^2}}$. Find the domain of~$f$.
Express your answer as a union of intervals.
\end{prob}

\begin{prob} Explain carefully why two functions $f$ and $g$ are equal if and only if
their domains and codomains are equal and $f(x) = g(x)$ for every $x$ in their common domain.
\end{prob}

\endinput
