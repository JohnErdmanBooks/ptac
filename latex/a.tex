\chapter{QUANTIFIERS}\label{quan}

 \setcounter{section}{1}
 \setcounter{thm}{0}

Certainly  ``$2 + 2 = 4$'' and  ``$2 + 2 = 5$'' are statements---one true, the other false. On
the other hand the appearance of the variable  $x$ prevents the expression  ``$x + 2 = 5$''
from being a statement.  Such an expression we will call an
 \index{open!sentence}%
\df{open sentence}; its truth is open to question since $x$ is unidentified.  There are three
standard ways of converting open sentences into statements.

The first, and simplest, of these is to give the variable a particular value. If we
``evaluate'' the expression  ``$x + 2 = 5$'' at $x = 4$, we obtain the (false) statement ``$4
+ 2 = 5$''.

A second way of obtaining a statement from an expression involving a variable is \df{universal
quantification}: we assert that the expression is true for all values of the variable.  In the
preceding example we get, ``For all  $x$, $x + 2 = 5$.'' This is now a statement (and again
false). The expression ``for all $x$'' (or equivalently, ``for every x'') is often denoted
symbolically
 \index{all@$\forall$ (for all, for every)}%
 \index{<@$\forall$ (for all, for every)}%
by~$(\forall x)$.  Thus the preceding sentence may be written, $(\forall x) x + 2 = 5$.  (The
parentheses are optional; they may be used in the interest of clarity.) We call $\forall$ a
 \index{universal quantifier}%
 \index{quantifier!universal}%
\df{universal quantifier}.

Frequently there are several variables in an expression.  They may all be universally
quantified.  For example
  \begin{equation}\label{eq:first}
             (\forall x)(\forall y)\, x^2 - y^2 = (x - y)(x + y)
  \end{equation}
is a (true) statement, which says that for every $x$ and for every $y$ the expression $x^2 -
y^2$ factors in the familiar way.  The order of consecutive universal quantifiers is
unimportant: the statement
  \[ (\forall y)(\forall x)\, x^2 - y^2 = (x - y)(x + y) \]
says exactly the same thing as~\eqref{eq:first}.  For this reason the notation may be
contracted slightly to read
  \[ (\forall x,y)\, x^2 - y^2 = (x - y)(x + y)\,. \]

A third way of obtaining a statement from an open sentence $P(x)$ is \df{existential
quantification}.  Here we assert that  $P(x)$ is true for \emph{at least one} value of~$x$.
This is often written
 \index{exists@$\exists$ (there exists)}%
 \index{<@$\exists$ (there exists)}%
``$(\exists x)$ such that $P(x)$'' or more briefly ``$(\exists x) P(x)$'', and is read ``there
exists an $x$ such that $P(x)$'' or ``$P(x)$ is true for some~$x$.''  For example, if we
existentially quantify the expression ``$x + 2 = 5$'' we obtain  ``$(\exists x)$ such that $x
+ 2 = 5$'' (which happens to be true).  We call $\exists$ an
 \index{existential quantifier}%
 \index{quantifier!existential}%
\df{existential quantifier}.

As is true for universal quantifiers, the order of consecutive existential quantifiers is
immaterial.

\begin{cau}  It is absolutely essential to realize that the order of an existential and a
universal quantifier may \emph{not} in general be reversed. For example,
  \[ (\exists x)(\forall y)\, x < y \]
says that there is a number $x$ with the property that no matter how $y$ is chosen, $x$ is
less than $y$; that is, there is a smallest real number. (This is, of course, false.) On the
other hand
  \[ (\forall y)(\exists x)\, x < y \]
says that for every $y$ we can find a number $x$ smaller than $y$.  (This is true: take $x$ to
be $y - 1$ for example.) \emph{The importance of getting quantifiers in the right order cannot
be overestimated.}
\end{cau}

There is one frequently used convention concerning quantifiers that should be mentioned.  In
the statement of definitions, propositions, theorems, \emph{etc.}, missing quantifiers are
assumed to be universal; furthermore, they are assumed to be the innermost quantifiers.

\begin{exam}  Let $f$ be a real valued function defined on the real line $\R$.  Many texts give
the following definition. The function $f$ is \emph{continuous} at a point $a$ in $\R$ if: for
every $\epsilon > 0$ there exists $\delta > 0$ such that
  \[\abs{f(x) - f(a)} < \epsilon \quad\text{ whenever } \abs{x - a} < \delta\,. \]
Here $\epsilon$ and $\delta$ are quantified; the function $f$ and the point $a$ are fixed for
the discussion, so they do not require quantifiers. What about $x$?  According to the
convention just mentioned, $x$ is universally quantified and that quantifier is the innermost
one. Thus the definition reads: for every $\epsilon > 0$ there exists $\delta > 0$ such that
for every $x$
  \[\abs{f(x) - f(a)} < \epsilon \quad\text{ whenever } \abs{x - a} < \delta\,. \]
\end{exam}


\begin{exam}Sometimes all quantifiers are missing.  In this case the preceding convention dictates
that all variables are universally quantified. Thus
  \[ \text{Theorem.}\quad  x^2 - y^2 = (x - y)(x + y) \]
is interpreted to mean
  \[\text{Theorem.}\quad  (\forall x)(\forall y)\, x^2 - y^2  = (x - y)(x+y)\,. \]
\end{exam}

