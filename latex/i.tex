\chapter{NATURAL NUMBERS AND MATHEMATICAL INDUCTION}\label{nat_nos}


 \setcounter{section}{1}
 \setcounter{thm}{0}


The \emph{principle of mathematical induction} is predicated on a knowledge of the natural
numbers, which we introduce in this section.  In what follows it is helpful to keep in mind
that the axiomatic development of the real numbers sketched in the preceding chapter says
nothing about the natural numbers; in fact, it provides explicit names for only three real
numbers: $0$, $1$, and $-1$.

\begin{defn}  A collection $J$ of real numbers is
 \index{inductive set}%
\df{inductive} if
 \begin{enumerate}
  \item $1 \in J$,\; and
  \item $x + 1 \in J$ whenever $x \in J$.
 \end{enumerate}
\end{defn}

\begin{exam}  The set $\R$ is itself inductive; so are the intervals $(0,\infty)$, $[-1, \infty)$,
and $[1,\infty)$.
\end{exam}

\begin{prop}\label{prop_intrs_ind}  Let $\sfml A$ be a family of inductive subsets of $\R$. Then
$\cap \sfml A$ is inductive.
\end{prop}

\begin{proof} Exercise. (Solution~\ref{sol_prop_intrs_ind}.)   \ns
\end{proof}

\begin{defn}  Let $J$ be the family of all inductive subsets of~$\R$.  Define
   \[ \N := \cap J\,. \]
We call $\N$ the set of
 \index{natural numbers}%
 \index{numbers!natural}%
\df{natural numbers}.
\end{defn}

Notice that according to proposition~\ref{prop_intrs_ind} the set $\N$ is inductive.  It is
the \emph{smallest} inductive set, in the sense that it is contained in \emph{every} inductive
set.  The elements of $\N$ have standard names. Define $2 := 1 + 1$.  Since $1$ belongs to
$\N$ and $\N$ is inductive, $2$~belongs to~$\N$. Define $3 := 2 + 1$.  Since $2$ belongs
to~$\N$, so does~$3$. Define $4 := 3 + 1$; \emph{etc.}

\begin{defn} The set of
 \index{integers}%
\df{integers}, denoted by $\Z$, is defined to be
  \[ -\N \cup \{0\} \cup \N \]
where $-\N := \{-n: n \in \N\}$.
\end{defn}

The next proposition is practically obvious; but as it is an essential ingredient of several
subsequent arguments (\emph{e.g.} problem~\ref{prob_less_pos}), we state it formally.

\begin{prop}\label{prop_natno_pos} If $n \in \N$, then $n \ge 1$.
\end{prop}

\begin{proof}  Since the set $[1,\infty)$ is inductive, it must contain~$\N$.
\end{proof}

The observation made previously that $\N$ is the smallest inductive set clearly implies that
no proper subset of $\N$ can be inductive. This elementary fact has a rather fancy name: it is
the \emph{principle of mathematical induction}.

\begin{thm}[Principle of Mathematical Induction]\label{thm_mth_ind}
 \index{mathematical induction}%
 \index{induction!mathematical}%
 \index{principle!of mathematical induction}
Every inductive subset of $\N$ equals $\N$.
\end{thm}

By spelling out the definition of ``inductive set'' in the preceding theorem we obtain a
longer, perhaps more familiar, statement of the \emph{principle of mathematical induction.}

\begin{cor}\label{cor_mi2}  If $S$ is a subset of $\N$ which satisfies
 \begin{enumerate}
  \item $1 \in S$,\quad and
  \item $n + 1 \in S$ whenever $n \in S$,
 \end{enumerate}
then $S = \N$.
\end{cor}

Perhaps even more familiar is the version of the preceding which refers to ``a proposition (or
assertion, or statement) concerning the natural number $n$''.

\begin{cor}\label{cor_mi3}  Let $P(n)$ be a proposition concerning the natural number~$n$.  If
$P(1)$ is true, and if $P(n + 1)$ is true whenever $P(n)$ is true, then $P(n)$ is true for all
$n \in \N$.
\end{cor}

\begin{proof}  In corollary~\ref{cor_mi2} let $S = \{n \in \N \colon P(n) \text{ is true}\}$.
Then $1 \in S$ and $n + 1$ belongs to $S$ whenever $n$ does.  Thus $S = \N$.  That is, $P(n)$
is true for all $n \in \N$.
\end{proof}

\begin{exer}\label{exer_mi1}  Use \emph{mathematical induction} to prove the following assertion:
The sum of the first $n$ natural numbers is $\frac 12n(n+1)$.  \emph{Hint.}  Recall that if
$p$ and $q$ are integers with $p \le q$ and if $c_p, c_{p+1}, \dots, c_q$ are real numbers,
then the sum $c_p + c_{p+1} + \dots + c_q$ may be denoted by $\sum_{k=p}^q c_k$. Using this
summation notation, we may write the desired conclusion as $\sum_{k=1}^n k = \frac 12
n(n+1)$.)  (Solution~\ref{sol_exer_mi1}.)
\end{exer}

It is essentially obvious that there is nothing crucial about starting inductions with $n =
1$. Let $m$ be any integer and $P(n)$ be a proposition concerning integers $n \ge m$.  If we
prove that $P(m)$ is true and that $P(n + 1)$ is true whenever $P(n)$ is true and $n \ge m$,
then we may conclude that $P(n)$ is true for all $n \ge m$. [Proof: Apply
corollary~\ref{cor_mi3} to the proposition $Q$  where $Q(n) = P(n + m - 1)$.]

\begin{prob}\label{prob_factor_ambm}  Let $a,b \in \R$ and $m \in \N$.  Then
  \[ a^m - b^m = (a - b) \sum_{k=0}^{m-1} a^k b^{m-k-1}\,. \]
\emph{Hint.}  Multiply out the right hand side. This is not an induction problem.
\end{prob}

\begin{prob}\label{prob_geomser1} If $r \in \R$, $r \ne 1$, and $n \in \N$, then
  \[ \sum_{k=0}^n r^k = \frac{1 - r^{n+1}}{1 - r}\,. \]
\emph{Hint.}  Use \ref{prob_factor_ambm}
\end{prob}

\begin{defn}  Let  $m$, $n \in \N$. We say that $m$ is a
 \index{factor}%
\df{factor} of $n$ if $n/m \in \N$.  Notice that  $1$ and $n$ are always factors of $n$; these
are the
 \index{factor!trivial}%
 \index{trivial!factor}%
\df{trivial factors} of~$n$.  The number $n$ is
 \index{composite!number}%
 \index{number!composite}%
\df{composite} if $n > 1$ and if it has at least one nontrivial factor. (For example, $20$ has
several nontrivial factors: $2$, $4$, $5$, and $10$.  Therefore it is composite.) If $n > 1$
and it is not composite, it is
 \index{prime number}%
 \index{number!prime}%
\df{prime}.  (For example, $7$ is prime; its only factors are $1$ and~$7$.)
\end{defn}

\begin{prob}  Prove that if $n \in \N$ and $2^n - 1$ is prime, then so is~$n$. (\emph{Hint.}
Prove the contrapositive.  Use problem~\ref{prob_factor_ambm}.)  Illustrate your technique by
finding a nontrivial factor of $2^{403} - 1$.
\end{prob}

\begin{prob}\label{prob_sum_sqrs}  Show that $\sum_{k=1}^nk^2 = \frac16 n(n+1)(2n+1)$ for every
$n \in \N$.
\end{prob}

\begin{prob}\label{prob_less_pos}  Use only the definition of $\N$ and the results given in this
section to prove (a) and (b).
 \begin{enumerate}
  \item[(a)] If $m,n \in \N$ and $m < n$, then $n - m \in \N$. \emph{Hint.}  One proof of this
involves an induction within an induction.  Restate the assertion to be proved as follows.
For every $m \in \N$ it is true that:
   \begin{equation}\label{prob_eqn1_mi}
         \text{if $n \in \N$ and $n > m$, then $n - m \in \N$.}
   \end{equation}
Prove this assertion by induction on~$m$.  That is, show that \eqref{prob_eqn1_mi} holds for
$m = 1$, and then show that it holds for $m = k + 1$ provided that it holds for $m = k$.  To
show that \eqref{prob_eqn1_mi} holds for $m = 1$, prove that the set
   \[ J := \{1\} \cup \{n \in \N \colon n - 1 \in \N\} \]
is an inductive set.
  \item[(b)] Let $n \in \N$.  There does not exist a natural number $k$ such that $n < k < n+1$.
\emph{Hint.}  Argue by contradiction. Use part~(a).
 \end{enumerate}
\end{prob}

 \index{binomial theorem}%
\begin{prob}[The binomial theorem.]\label{binom_thm} If $x$, $y \in \R$ and $n \in \N$, then
  \[ (x + y)^n = \sum_{k=0}^n\binom nkx^{n-k}y^k\,. \]
\emph{Hint.}  Use induction.  Recall that $0! = 1$, that $n! = n(n - 1)(n - 2) \dots 1$ for $n
\in \N$, and that
  \[ \binom nk = \frac{n!}{k!(n-k)!} \]
for  $0 \le k \le n$.
\end{prob}

The final result of this section is the
 \index{well-ordering}%
 \index{principle!of well-ordering}%
\emph{principle of well-ordering}.  It asserts that every nonempty subset of the natural
numbers has a smallest element.

\begin{prop}\label{prop_well_ord}  If $\emptyset \ne K \subseteq \N$, then there exists
$a \in K$ such that $a \le k$ for every $k \in K$.
\end{prop}

\begin{proof}  Exercise. \emph{Hint.}  Assume that $K$ has no smallest member.  Show that
this implies $K = \emptyset$ by proving that the set
  \[ J \equiv \{n \in \N \colon n < k  \text{ for every } k \in K\} \]
is inductive.  In the inductive step problem~\ref{prob_less_pos}(b) may prove useful.
(Solution~\ref{sol_prop_well_ord}.)  \ns
\end{proof}

\begin{prob}  (This slight modification of the \emph{principle of mathematical induction}
is occasionally useful.)  Let $P(n)$ be a proposition concerning the natural number $n$. If
$P(n)$ is true whenever $P(k)$ is true for all $k \in \N$ such that $k < n$, then $P(n)$ is
true for all $n$.  \emph{Hint.}  Use the \emph{well-ordering principle.}
\end{prob}



\endinput
