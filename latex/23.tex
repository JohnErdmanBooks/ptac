\chapter{CONTINUITY AND LINEARITY}



\section{BOUNDED LINEAR TRANSFORMATIONS}\label{blt}
Normed linear spaces have both algebraic and topological structure.  It is therefore natural
to be interested in those functions between normed linear spaces which preserve both types of
structure, that is, which are both linear and continuous.  In this section we study such
functions.
\begin{defn}  A linear transformation $T \colon V \sto W$ between normed linear spaces is
 \index{bounded!linear map}%
\df{bounded} if there exists a number $M > 0$ such that
  \[ \norm{Tx} \le M \norm x \]
for all $x \in V$.  The family of all bounded linear transformations from $V$ into $W$ is
denoted
 \index{bounded@$\ofml B(V,W)$ (bounded linear maps between normed linear spaces)}%
by~$\ofml B(V,W)$.
\end{defn}

\begin{cau}  There is a possibility of confusion.  Here we have defined bounded linear
transformations; in section~\ref{space_bdd} we gave a quite different definition for
``bounded'' as it applies to arbitrary vector valued functions; and certainly linear
transformations are such functions.  The likelihood of becoming confused by these two
different notions of boundedness is very small once one has made the following observation:
Except for the zero function, it is impossible for a linear transformation to be bounded in
the sense of section~\ref{space_bdd}.  (Proof. Let $T\colon V \sto W$ be a nonzero linear
transformation between normed linear spaces.  Choose $a$ in $V$ so that $Ta \ne 0$.  Then
$\norm{Ta} > 0$, so that by the Archimedean principle (proposition~\ref{arch_prop}) the number
$\norm{T(na)} = n\norm{Ta}$ can be made as large as desired by choosing $n$ sufficiently
large.  Thus there is certainly no $M > 0$ such that $\norm{Tx} \le M$ for all~$x$.)  Since
nonzero linear transformations cannot be bounded in the sense of section~\ref{space_bdd}, an
assertion that a linear map is bounded should always be interpreted in the sense of
boundedness introduced in this section.
\end{cau}

A function $f\colon S \sto W$, which maps a set $S$ into a normed linear space $W$, is bounded
if and only if it maps \emph{every} subset of $S$ into a bounded subset of~$W$.  However, a
\emph{linear} map $T\colon V \sto W$ from one normed linear space into another is bounded if
and only if it maps every bounded subset of $V$ into a bounded subset of~$W$.

\begin{prop}  A linear transformation $T\colon V \sto W$ between normed linear spaces is bounded
if and only if $T^\sto(A)$ is a bounded subset of $W$ whenever $A$ is a bounded subset of~$V$.
\end{prop}

\begin{proof} Problem.  \ns   \end{proof}

\begin{exam}\label{exam_bdd_lt}  Let $T \colon \R^2 \sto \R^3 \colon (x,y)
\mapsto (3x + y,x - 3y,4y)$.  It is easily seen that $T$ is linear. For all $(x,y)$ in~$\R^2$
 \begin{align*}
   \norm{T(x,y)}
           &= \norm{(3x + y,x - 3y,4y)} \\
           &= \bigl((3x + y)^2 + (x - 3y)^2 + (4y)^2\bigr)^{\frac12} \\
           &= (10x^2 + 26y^2)^{\frac12} \\
           &\le \sqrt{26}(x^2 + y^2)^{\frac12} \\
           &= \sqrt{26}\norm{(x,y)}\,.
 \end{align*}
So the linear transformation $T$ is bounded.
\end{exam}

Why is boundedness of linear transformations an important concept? Because it turns out to be
equivalent to continuity and is usually easier to establish.

\begin{prop}\label{prop_equiv_cont}  Let $T\colon V \sto W$ be a linear transformation between
normed linear spaces.  The following are equivalent:
 \begin{enumerate}
  \item[(a)] $T$ is continuous.
  \item[(b)] There is at least one point at which $T$ is continuous.
  \item[(c)] $T$ is continuous at $\vc 0$.
  \item[(d)] $T$ is bounded.
 \end{enumerate}
\end{prop}

\begin{proof}  Exercise. \emph{Hint.}  To prove that (c) implies (d) argue by contradiction.
Show that for each $n$ there exists $x_n$ in $V$ such that $\norm{Tx_n} > n\norm{x_n}$.  Let
$y_n = \bigl(n\norm{x_n}\bigr)^{-1}x_n$.  Give one argument to show that the sequence $(Ty_n)$
converges to zero as $n \sto\infty$.  Give another to show that it does not.
(Solution~\ref{sol_prop_equiv_cont}.)  \ns
\end{proof}

\begin{defn} Let $T \in \ofml B(V,W)$ where $V$ and $W$ are normed linear spaces.  Define
  \[ \norm T := \inf\{M > 0\colon \norm{Tx} \le M \norm x \text{for all $x \in V$}\}. \]
This number is called the
 \index{<@$\norm T$ (norm of a linear transformation)}%
 \index{linear!transformation, norm of a}%
 \index{norm!of a linear transformation}%
\df{norm} of~$T$.  We show in proposition~\ref{prop_opnorm} that the map $T \mapsto \norm T$
really is a norm on $\ofml B(V,W)$.
\end{defn}

There are at least four ways to compute the norm of a linear transformation~$T$.  Use the
definition or any of the three formulas given in the next lemma.

\begin{lem}\label{lem_equiv_norm} Let $T$ be a bounded linear map between nonzero normed linear
spaces. Then
  \begin{align*}
     \norm T &= \sup\{\norm x^{-1}\,\norm{Tx} \colon x \ne \vc 0\} \\
             &= \sup\{\norm{Tu} \colon \norm u = 1\} \\
             &= \sup\{\norm{Tu} \colon \norm u \le 1\}.
  \end{align*}
\end{lem}

\begin{proof} Exercise. \emph{Hint.}  To obtain the first equality, use the fact that if a subset
$A$ of $\R$ is bounded above then $\sup A = \inf\{M \colon \text{$M$ is an upper bound
for~$A$}\}$.  (Solution~\ref{sol_lem_equiv_norm}.)  \ns
\end{proof}

\begin{cor}\label{cor_norm_op} If $T \in \ofml B(V,W)$ where $V$ and $W$ are normed linear spaces, then
  \[ \norm{Tx} \le \norm T \, \norm x \]
for all $x$ in~$V$.
\end{cor}

\begin{proof} By the preceding lemma $\norm T \ge \norm x^{-1}\norm{Tx}$ for all $x \ne 0$.  Thus
$\norm{Tx} \le \norm T \,\norm x$ for all~$x$.
\end{proof}


The following example shows how to use lemma~\ref{lem_equiv_norm} (in conjunction with the
definition) to compute the norm of a linear transformation.

\begin{exam} Let $T$ be the linear map defined in example~\ref{exam_bdd_lt}.  We have already seen that
$\norm{T(x,y)} \le \sqrt{26}\norm{(x,y)}$ for all $(x,y)$ in $\R^2$.  Since $\norm T$ is
defined to be the infimum of the set of all numbers $M$ such that $\norm{T(x,y)} \le M
\norm{(x,y)}$ for all $(x,y) \in \R^2$, and since $\sqrt{26}$ is such a number, we know that
  \begin{equation}\label{eqn_norm_est1} \norm  T \le \sqrt{26}
  \end{equation}
On the other hand lemma~\ref{lem_equiv_norm} tells us that $\norm T$ is the supremum of the
set of all numbers $\norm{Tu}$ where $u$ is a unit vector. Since $(0,1)$ is a unit vector and
$\norm{T(0,1)} = \norm{(1,-3,4)} = \sqrt{26}$, we conclude that
  \begin{equation}\label{eqn_norm_est2} \norm  T \ge \sqrt{26}
  \end{equation}
Conditions \eqref{eqn_norm_est1} and \eqref{eqn_norm_est2} imply that $\norm T = \sqrt{26}$.
\end{exam}

\begin{prob} Let $T\colon \R^2 \sto \R^3$ be the linear transformation defined by $T(x,y) =
(3x, x + 2y, x - 2y)$.  Find $\norm T$.
\end{prob}

\begin{prob} Let $T\colon \R^3 \sto \R^4\colon x \mapsto (x_1 - 4x_2, 2x_1 + 3x_3, x_1 + 4x_2,
x_1- 6x_3)$.  Find $\norm T$.
\end{prob}

\begin{exer}\label{exer_exam_norm}  Find the norm of each of the
following.
 \begin{enumerate}
  \item[(a)] The identity map on a normed linear space.
  \item[(b)] The zero map in $\ofml B(V,W)$.
  \item[(c)] A coordinate projection $\pi_k: V_1 \times V_2 \sto V_k$
($k = 1,2$) where $V_1$ and $V_2$ are nontrivial normed linear spaces (that is, normed linear
spaces which contain vectors other than the zero vector.)
 \end{enumerate}
(Solution~\ref{sol_exer_exam_norm}.)
\end{exer}

\begin{exer}\label{exer_norm_int} Let $\fml C = \fml C([a,b],\R)$.  Define $J\colon
\fml C \sto \R$ by
  \[ Jf = \int_a^b f(x)\,dx\,. \]
Show that $J \in \ofml B(\fml C,\R)$ and find~$\norm J$. (Solution~\ref{sol_exer_norm_int}.)
\end{exer}

\begin{prob} Let $\fml C^1$ and $\fml C$ be as in problem~\ref{prob_intop_c1}.  Let $D$ be the
differentiation operator
  \[ D \colon \fml C^1 \sto \fml C \colon f \mapsto f' \]
(where $f'$ is the derivative of $f$).  Let both $\fml C^1$ and $\fml C$ have the uniform
norm.  Is the linear transformation $D$ bounded?  \emph{Hint.}  Let $[a,b] = [0,1]$ and
consider the functions $f_n(x) = x^n$ for $n \in \N$ and $0 \le x \le 1$.
\end{prob}

Next we show that the set $\ofml B(V,W)$ of all bounded linear transformations between two
normed linear spaces is itself a normed linear space.

\begin{prop}\label{prop_opnorm} If $V$ and $W$ are normed linear spaces then under pointwise
operations $\ofml B(V,W)$ is a vector space and the map $T \mapsto \norm T$ from $\ofml
B(V,W)$ into $\R$ defined above is a norm on $\ofml B(V,W)$.
\end{prop}

\begin{proof} Exercise.  (Solution~\ref{sol_prop_opnorm}.) \ns  \end{proof}

One obvious fact that we state for future reference is that the composite of bounded linear
transformations is bounded and linear.

\begin{prop}\label{prop_comp_op} If $S \in \ofml B(U,V)$ and $T \in \ofml B(V,W)$, then
$TS \in \ofml B(U,W)$ and $\norm{TS} \le \norm T \, \norm S$
\end{prop}

\begin{proof} Exercise.  (Solution~\ref{sol_prop_comp_op}.)  \ns  \end{proof}

In propositions~\ref{prop_ker_vsubsp} and ~\ref{lt_prop2} we saw that the kernel and range of
a linear map $T$ are vector subspaces, respectively, of the domain and codomain of~$T$.  It is
interesting to note that the kernel is always a closed subspace while the range need not be.

\begin{prop} If $V$ and $W$ are normed linear spaces and $T \in \ofml B(V,W)$, then $\ker T$
is a closed linear subspace of~$V$.
\end{prop}

\begin{proof} Problem.   \ns  \end{proof}

\begin{exam} Let $c_0$ be the vector space of all sequences $x$ of real numbers (with pointwise
operations) which converge to zero. Give $c_0$ the uniform norm (see example~\ref{exam_un_set}
so that for every $x \in c_0$
  \[ \norm x_u = \sup\{x_k: k \in \N\}\,. \]
The family $l$ of sequences of real numbers which have only finitely many nonzero coordinates
is a vector subspace of~$c_0$, but it is not a closed subspace.  Thus the range of the
inclusion map of $l$ into $c_0$ does not have closed range.
\end{exam}

\begin{proof} Problem.  \ns  \end{proof}

In general the calculus of infinite dimensional spaces is no more complicated than calculus on
$\R^n$.  One respect in which the Euclidean spaces $\R^n$ turn out to be simpler, however, is
the fact that every linear map from $\R^n$ into $\R^m$ is automatically continuous.  Between
finite dimensional spaces there are no discontinuous linear maps.  And this is true regardless
of the particular norms placed on these spaces.

\begin{prop}\label{prop_fd_cont}  Let $\R^n$ and $\R^m$ have any norms whatever.  If
$T \colon \R^n \sto \R^m$ is linear, then it is continuous.
\end{prop}

\begin{proof} Problem.  \emph{Hint.}   Let $\bigl[t_k^j\bigr] = [T]$ be the matrix representation
of $T$ and
  \[ M = \max\{\bigl| t_k^j \bigr| \colon 1 \le j \le m \text{ and } 1 \le k \le n\}\,. \]
Let $\R_u^m$ be $\R^m$ provided with the uniform norm
  \[ \norm x_u := \max\{\abs{x_k} \colon 1 \le k \le m\} \]
and $\R_1^n$ be $\R^n$ equipped with the norm
  \[ \norm x_1 := \sum_{k=1}^n \abs{x_k}\,. \]
Show that $T$ regarded as a map from $\R_1^n$ to $\R_u^m$ is bounded (with $\norm T \le M$).
Then use problem~\ref{prob_norms_Rn}.  \ns   \end{proof}

\begin{prob}\label{prob_eval_op} Let $V$ and $W$ be normed linear spaces and $x \in V$.
Define a map $E_x$ (called
 \index{evaluation map}%
\df{evaluation at} $x$) by
  \[ E_x\colon \ofml B(V,W) \sto W\colon T \mapsto Tx\,. \]
Show that $E_x \in \ofml B\bigl(\ofml B(V,W),W\bigr)$ and that $\norm{E_x} \le \norm x$.
\end{prob}

\begin{prob}\label{prob_Ex_cms}  What changes in the preceding problem if we let $M$ be a
compact metric space and $W$ be a nonzero normed linear space and consider the evaluation map
$E_x \colon \fml C(M,W) \sto W \colon f \mapsto f(x)$?
\end{prob}

\begin{prob}\label{prob_comp_op} Let $S$ be a nonempty set and $T \colon V \sto W$ be a bounded
linear transformation between normed linear spaces.  Define a function $C_T$ on the normed
linear space $\fml B(S,V)$ by
  \[ C_T(f) :=  T \circ f \]
for all $f$ in $\fml B(S,V)$.
 \begin{enumerate}
  \item[(a)] Show that $C_T(f)$ belongs to $\fml B(S,W)$ whenever $f$ is a member of
$\fml B(S,V)$.
  \item[(b)] Show that the map $C_T\colon \fml B(S,V) \sto \fml B(S,W)$ is linear and
continuous.
  \item[(c)] Find $\norm{C_T}$.
  \item[(d)] Show that if $f_n \sto g \text{\,(unif)}$ in $\fml B(S,V)$, then
$T \circ f_n \sto T \circ g \text{\,(unif)}$ in $\fml B(S,W)$.
  \item[(e)] Show that $C_T$ is injective if and only if $T$ is.
 \end{enumerate}
\end{prob}

\begin{prob}\label{prob_c_functor}  Let $M$ and $N$ be compact metric spaces and
$\phi\colon M \sto N$ be continuous.  Define $T_\phi$ on $\fml C(N,\R)$ by
  \[ T_\phi(g) := g \circ \phi \]
for all $g$ in $\fml C(N,\R)$.
 \begin{enumerate}
  \item[(a)] $T_\phi$ maps $\fml C(N,\R)$ into $\fml C(M,\R)$.
  \item[(b)] $T_\phi$ is a bounded linear transformation.
  \item[(c)] $\norm{T_\phi} = 1$.
  \item[(d)] If $\phi$ is surjective, then $T_\phi$ is injective.
  \item[(e)] If $T_\phi$ is injective, then $\phi$ is surjective.
\emph{Hint.}  Suppose $\phi$ is not surjective.  Choose $y$ in $N \setminus \ran\phi$.  Show
that problem~\ref{prob_urys_lem} can be applied to the sets $\{y\}$ and $\ran\phi$.
  \item[(f)] If $T_\phi$ is surjective, then $\phi$ is injective.  \emph{Hint.}  Here again
problem~\ref{prob_urys_lem} is useful. [It is also true that if $\phi$ is injective, then
$T_\phi$ is surjective.  But more machinery is needed before we can prove this.]
 \end{enumerate}
\end{prob}

\begin{prob}  Let $(S_k)$ be a sequence in $\ofml B(V,W)$ and $U$ be a member of $\ofml B(W,X)$
where $V$, $W$, and $X$ are normed linear spaces.  If $S_k \sto T$ in $\ofml B(V,W)$, then
$US_k \sto UT$.  Also, state and prove a similar result whose conclusion is, ``then $S_kU \sto
TU$.''
\end{prob}

\begin{defn} A family $\ofml T$ of linear maps from a vector space into itself is a
 \index{commuting!family}%
 \index{family!commuting}%
\df{commuting family} if $ST = TS$ for all $S$, $T \in \ofml T$.
\end{defn}

\begin{prob}[Markov-Kakutani Fixed Point Theorem]
Prove: If $\ofml T$ is a commuting family of bounded linear maps from a normed linear
space $V$ into itself and $K$ is a nonempty convex compact subset of $V$ which is mapped
into itself by every member of $\ofml T$, then there is at least one point in $K$ which
is fixed under every member of~$\ofml T$. \emph{Hint.}  For every $T$ in $\ofml T$ and
$n$ in $\N$ define
  \[ T_n = n^{-1}\sum_{j=0}^{n-1} T^j \]
(where $T^0 := I$).  Let $\ofml U= \{T_n\colon T \in \ofml T \text{ and } n \in \N\}$.
Show that $\ofml U$ is a commuting family of bounded linear maps on $V$ each of which
maps $K$ into itself and that if $U_1,\dots, U_n \in \ofml U$, then
  \[ (U_1 \dots U_n)^\sto (K) \subseteq \bigcap_{j=1}^n U_j^\sto (K)\,. \]
Let $\sfml C = \{U^\sto (K)\colon U \in \ofml U\}$ and use problem~\ref{fip} to show that
$\bigcap\sfml C \ne \emptyset$.

Finally, show that every element of $\bigcap\sfml C$ is fixed under each $T$ in $\ofml
T$; that is, if $a \in \bigcap\sfml C$ and $T \in \ofml T$, then $Ta = a$.  To this end
argue that for every $n \in \N$ there exists $c_n \in K$ such that $a = T_nc_n$ and
therefore $Ta - a$ belongs to $n^{-1}(K-K)$ for each~$n$.  Use
problems~\ref{prob_sum_cpt} and \ref{cpt_clbdd} to show that every neighborhood of $0$
contains, for sufficiently large $n$, sets of the form $n^{-1}(K-K)$.  What do these last
two observations say about $Ta - a$?
\end{prob}

















\section{THE STONE-WEIERSTRASS THEOREM}
In example~\ref{exam_approx_sqrt} we found that the square root function can be uniformly
approximated on $[0,1]$ by polynomials. In this section we prove the remarkable
\emph{Weierstrass approximation theorem} which says that \emph{every} continuous real valued
function can be uniformly approximated on compact intervals by polynomials.  We will in fact
prove an even stronger result due to M. H. Stone which generalizes the Weierstrass theorem to
arbitrary compact metric spaces.

\begin{prop}\label{prop_SW_prep}  Let $A$ be a subalgebra of $\fml C(M,\R)$ where $M$ is a compact
metric space.  If $f \in A$, then $\abs f \in \clo A$.
\end{prop}

\begin{proof}  Exercise.  \emph{Hint.}  Let $(p_n)$ be a sequence of polynomials converging uniformly
on $[0,1]$ to the square root function. (See~\ref{exam_approx_sqrt}.)  What can you say about
the sequence $(p_n \circ g^2)$ where $g = f / \|f\|$\,? (Solution~\ref{sol_prop_SW_prep}.) \ns
\end{proof}

\begin{cor}\label{cor_sup_clo}  If $A$ is a subalgebra of $\fml C(M,\R)$ where $M$ is a compact
metric space, and if $f$, $g \in A$, then $f \lor g$ and $f \land g$ belong to~$\clo A$.
\end{cor}

\begin{proof}   As in the solution to problem~\ref{prob_cont_sup}, write $f \lor g =
\frac12(f + g + \abs{f-g})$ and $f \land g = \frac12(f + g - \abs{f-g})$; then apply the
preceding proposition.
\end{proof}

\begin{defn}  A family $\fml F$ of real valued functions defined on a set $S$ is a
 \index{separating!family of functions}%
 \index{family!separating}%
\df{separating} family if corresponding to every pair of distinct points $x$ and $y$ and $S$
there is a function $f$ in $\fml F$ such that $f(x) \ne f(y)$. In this circumstance we may
also say that the family $\fml F$ \df{separates points} of~$S$.
\end{defn}


\begin{prop}\label{prop_sep_subalg}  Let $A$ be a separating unital subalgebra of
$\fml C(M,\R)$ where $M$ is a compact metric space. If $a$ and $b$ are distinct points in
$M$ and $\alpha,\beta \in \R$, then there exists a function $f \in A$ such that $f(a) =
\alpha$ and $f(b) = \beta$.
\end{prop}

\begin{proof} Problem.  \emph{Hint.}  Let $g$ be any member of $A$ such that $g(a) \ne g(b)$.
Notice that if $k$ is a constant, then the function $f \colon x \mapsto \alpha + k(g(x)-g(a))$
satisfies $f(a) = \alpha$. Choose $k$ so that $f(b) = \beta$.   \ns
\end{proof}

Suppose that $M$ is a compact metric space. The \emph{Stone-Weierstrass theorem} says that any
separating unital subalgebra of the algebra of continuous real valued functions on $M$ is
dense.  That is, if $A$ is such a subalgebra, then we can approximate each function $f \in
\fml C(M,\R)$ arbitrarily closely by members of~$A$.  The proof falls rather naturally into
two steps.  First (in lemma~\ref{lem_SW_prep}) we find a function in $\clo A$ which does not
exceed $f$ by much; precisely, given $a \in M$ and $\epsilon > 0$ we find a function $g$ in
$\clo A$ which agrees with $f$ at $a$ and satisfies $g(x) < f(x) + \epsilon$ elsewhere.  Then
(in theorem~\ref{SW_thm}) given $\epsilon > 0$ we find $h \in \clo A$ such that $f(x) -
\epsilon < h(x) < f(x) + \epsilon$.

\begin{lem}\label{lem_SW_prep}  Let $A$ be a unital separating subalgebra of $\fml C(M,\R)$
where $M$ is a compact metric space.  For every $f \in \fml C(M,\R)$, every $a \in M$, and
every $\epsilon > 0$ there exists a function $g \in \clo A$ such that $g(a) = f(a)$ and $g(x)
< f(x) + \epsilon$ for all $x \in M$.
\end{lem}

\begin{proof} Exercise.   \emph{Hint.} For each $y \in M$ find a
function $\phi_y$ which agrees with $f$ at $a$ and at $y$.  Then $\phi_y(x) < f(x) + \epsilon$
for all $x$ in some neighborhood $U_y$ of~$y$.  Find finitely many of these neighborhoods
$U_{y_1}, \dots, U_{y_n}$ which cover~$M$.  Let $g = \phi_{y_1} \land \dots \land \phi_{y_n}$.
(Solution~\ref{sol_lem_SW_prep}.)    \ns
\end{proof}

 \index{Stone-Weierstrass theorem}%
\begin{thm}[Stone-Weierstrass Theorem]\label{SW_thm}
 \index{Stone-Weierstrass theorem}%
Let $A$ be a unital separating subalgebra of $\fml C(M,\R)$ where $M$ is a compact metric
space.  Then $A$ is dense in~$\fml C(M,\R)$.
\end{thm}

\begin{proof} All we need to show is that $\fml C(M,\R) \subseteq \clo A$.  So we choose
$f \in \fml C(M,\R)$ and try to show that $f \in \clo A$.  It will be enough to show that for
every $\epsilon > 0$ we can find a function $h \in \clo A$ such that $\norm{f - h}_u <
\epsilon$. [Reason: then $f$ belongs to $\clo{\clo A}$ (by proposition\ref{cnd_dns}) and
therefore to $\clo A$ (by~\ref{prop_clo}(b)).]

Let $\epsilon > 0$.  For each $x \in M$ we may, according to lemma~\ref{lem_SW_prep}, choose a
function $g_x \in \clo A$ such that $g_x(x) = f(x)$ and
 \begin{equation}\label{SW_ineq1}
       g_x(y) < f(y) + \epsilon
 \end{equation}
for every $y \in M$.  Since both $f$ and $g_x$ are continuous and they agree at $x$, there
exists an open set $U_x$ containing $x$ such that
 \begin{equation}\label{SW_ineq2}
       f(y) < g_x(y) + \epsilon
 \end{equation}
for every $y \in U_x$.  (Why?)

Since the family $\{U_x\colon x \in M\}$ covers $M$ and $M$ is compact, there exist points
$x_1, \dots, x_n$ in $M$ such that $\{U_{x_1}, \dots U_{x_n}\}$ covers~$M$.  Let $h = g_{x_1}
\lor \dots \lor g_{x_n}$.  We know from \ref{prob_clo_alg} that $\clo A$ is a subalgebra of
$\fml C(M,\R)$.  So according to corollary~\ref{cor_sup_clo}, $h \in \clo{\clo A} = \clo A$.

By inequality~\eqref{SW_ineq1}
  \[ g_{x_k}(y) < f(y) + \epsilon \]
holds for all $y \in M$ and $1 \le k \le n$.  Thus
 \begin{equation}\label{SW_ineq3}
       h(y) < f(y) + \epsilon
 \end{equation}
for all $y \in M$.  Each $y \in M$ belongs to at least one of the open sets~$U_{x_k}$.  Thus
by~\eqref{SW_ineq2}
 \begin{equation}\label{SW_ineq4}
       f(y) < g_{x_k}(y) + \epsilon < h(y) + \epsilon
 \end{equation}
for every $y \in M$.  Together \eqref{SW_ineq3} and~\eqref{SW_ineq4} show that
  \[ -\epsilon < f(y) - h(y) < \epsilon \]
for all~$y \in M$.  That is, $\norm{f - h}_u < \epsilon$.
\end{proof}

\begin{prob} Give the missing reason for inequality~\eqref{SW_ineq2} in the preceding proof.
\end{prob}


\begin{thm}[Weierstrass Approximation Theorem]\label{Wat}
 \index{Weierstrass approximation theorem}%
 \index{approximation theorem!Weierstrass}%
Every continuous real valued function on $[a,b]$ can be uniformly approximated by polynomials.
\end{thm}

\begin{proof} Problem.  \ns  \end{proof}

\begin{prob} Let $\fml G$ be the set of all functions $f$ in $\fml C\bigl(\,[0,1]\,\bigr)$
such that $f$ is differentiable on $(0,1)$ and $f\,'(\frac12) = 0$. Show that $\fml G$ is
dense in~$\fml C\bigl(\,[0,1]\,\bigr)$.
\end{prob}

\begin{prop} If $M$ is a closed and bounded subset of $\R$, then the normed linear space
$\fml C(M,\R)$ is separable.
\end{prop}

\begin{proof} Problem.  \ns  \end{proof}















\section{BANACH SPACES}

\begin{defn} A
 \index{Banach!space}%
 \index{space!Banach}%
\df{Banach space} is a normed linear space which is complete with respect to the metric
induced by its norm.
\end{defn}

\begin{exam} In example~\ref{r_compl} we saw that $\R$ is complete; so it is a Banach space.
\end{exam}

\begin{exam} In example~\ref{Rn_compl} it was shown that the space $\R^n$ is complete with
respect to the Euclidean norm.  Since all norms on $\R^n$ are equivalent
(problem~\ref{prob_norms_Rn}) and since completeness of a space is not affected by changing to
an equivalent metric (proposition~\ref{eq_compl}), we conclude that $\R^n$ is a Banach space
under all possible norms.
\end{exam}

\begin{exam} If $S$ is a nonempty set, then (under the uniform norm) $\fml B(S,\R)$ is a
Banach space (see example~\ref{bdd_compl}).
\end{exam}

\begin{exam} If $M$ is a compact metric space, then (under the uniform norm) $\fml C(M,\R)$
is a Banach space (see example~\ref{CMR_cmpl}).
\end{exam}

At the beginning of section~\ref{sec_prod_nls} we listed 4 ways of making new normed linear
spaces from old ones. Under what circumstances do these new spaces turn out to be Banach
spaces?
 \begin{enumerate}
  \item[(i)] A closed vector subspace of a Banach space is a Banach space.
(See proposition~\ref{prop_clss_ms}.)
  \item[(ii)] The product of two Banach space is a Banach space.
(See proposition~\ref{prod_compl}.)
  \item[(iii)] If $S$ is a nonempty set and $E$ is a Banach space, then $\fml B(S,E)$ is a Banach
space. (See problem~\ref{prob_BSV_compl}.)
%\item[(v)] We defer until chapter 17 the proof that if $E$ is
%a Banach space and $F$ is a closed vector subspace of $E$, then
%$E/F$ is a Banach space.
  \item[(iv)] If $V$ is a normed linear space and $F$ is a Banach space, then $\ofml B(V,F)$ is
a Banach space. (See the following proposition.)
 \end{enumerate}

\begin{prop}\label{prop_BVW_compl}  Let $V$ and $W$ be normed linear spaces. Then $\ofml B(V,W)$
is complete if $W$ is.
\end{prop}

\begin{proof} Exercise. \emph{Hint.} Show that if $(T_n)$ is a Cauchy sequence in $\ofml B(V,W)$,
then $\lim\limits_{n \sto \infty} T_nx$ exists for every $x$ in $V$.  Let $Sx = \lim\limits_{n
\sto \infty} T_nx$.  Show that the map $S\colon x \mapsto Sx$ is linear. If $\epsilon > 0$
then for $m$ and $n$ sufficiently large $\norm{T_m - T_n} < \frac12 \epsilon$.  For such $m$
and $n$ show that
  \[ \norm{Sx - T_nx} \le \norm{Sx - T_mx} + \tfrac12 \epsilon \norm x \]
and conclude from this that $S$ is bounded and that $T_n \sto S$ in $\ofml B(V,W)$.
(Solution~\ref{sol_prop_BVW_compl}.) \ns
\end{proof}

\begin{prob}\label{prob_BSV_compl}  Let $S$ be a nonempty set and $V$ be a normed linear space.
If $V$ is complete, so is $\fml B(S,V)$.   \emph{Hint.}  Make suitable modifications in the
proof of example~\ref{bdd_compl}.
\end{prob}

\begin{prob}  Let $M$ be a compact metric space and $V$ be a normed linear space.  If $V$ is
complete so is $\fml C(M,V)$. \emph{Hint.} Use proposition~\ref{prop_unif_lim}.
\end{prob}

\begin{prob} Let $m$ be the set of all bounded sequences of real numbers. (That is, a sequence
$(x_1,x_2,\dots)$ of real numbers belongs to $m$ provided that there exists a constant $M>0$
such that $\abs{x_k} \le M$ for all $k \in \N$.) If $x = (x_k)$ and $y = (y_k)$ belong to $m$
and $\alpha$ is a scalar define
  \[ x + y = (x_1 + y_1,x_2 + y_2,\dots) \]
and
  \[ \alpha x = (\alpha x_1, \alpha x_2, \dots)\,. \]
Also, for each sequence $x$ in $m$ define
  \[ \norm x_u = \sup\{\abs{x_k}\colon k \in \N \}\,. \]
 \begin{enumerate}
  \item[(a)] Show that $m$ is a Banach space. \emph{Hint.} The proof should be very short.  This
is a special case of a previous example.
  \item[(b)] The space $m$ is not separable.
%The following hint should be rewritten
%and references located.
%\emph{Hint.} Consider the family $\sfml B$
%of all open balls $B_{1/2}(a)$ of radius
%$\frac12$ about those points $a$ in $m$
%which are discussed in problem 16 of chapter 4;
%that is, those sequences each of whose
%entries is either $0$ or~$1$.  Argue by
%contradiction: assume there does exist a
%countable dense set $D$ in~$m$.  Let $D_0$
%comprise those members of $D$ which lie in
%some member of $\sfml B$.  Find a surjection
%$f\colon D_0 \sto \sfml B$.  Use propositions 4.2.2,
%5.4.7, and 4.2.5.
  \item[(c)] The closed unit ball of the space $m$ is closed and bounded but not compact.
%The following hint should be rewritten
%and references located.
%\emph{Hint.}  Trivial modifications of the
%proof suggested in the hint for part (b) show
%that the closed unit ball in $m$ (that is,
%$\{x \in m\colon \norm x_u \le 1\}$) is not
%separable. Use problem 13(b) of chapter 9
%and proposition 7.4.3.
 \end{enumerate}
\end{prob}













\section{DUAL SPACES AND ADJOINTS}

\vskip .1 in

\begin{center}
\fbox{In this section $U$, $V$, and $W$ are normed linear spaces.}
\end{center}

\vskip .2 in

Particularly important among the spaces of bounded linear transformations introduced in
section~\ref{blt} are those consisting of maps from a space $V$ into its scalar field~$\R$.
The space $\ofml B(V,\R)$ is the
 \index{dual space}%
 \index{space!dual}%
 \index{<@$V^*$ (dual of a Banach space)}%
\df{dual space} of $V$ and is usually denoted by~$V^*$.  Members of $V^*$ are called
 \index{bounded!linear functionals}%
 \index{functionals!bounded linear}%
\df{bounded linear functionals}.  (When the word ``function'' is given the suffix ``al'' we
understand that the function is scalar valued.)

\begin{exam} Let $f\colon \R^3 \sto \R\colon (x,y,z) \mapsto x + y + z$. Then $f$ is a member
of $(\R^3)^*$. (It is obviously linear and is bounded by proposition~\ref{prop_fd_cont}.)
\end{exam}

\begin{exam} The familiar Riemann integral of beginning calculus is a bounded linear functional
on the space $\fml C = \fml C([a,b],\R)$.  That is, the functional $J$ defined in
exercise~\ref{exer_norm_int} belongs to~$\fml C^*$.
\end{exam}

\begin{exam} Let $M$ be a compact metric space and $\fml C = \fml C(M,\R)$.  For each $x$ in $M$
the evaluation functional $E_x\colon f \mapsto f(x)$ belongs to $\fml C^*$.  (Take $W = \R$ in
problem~\ref{prob_Ex_cms}.)
\end{exam}

\begin{defn}  Let $T \in \ofml B(V,W)$.  Define $T^*$ by $T^*(g) = g \circ T$ for every $g$ in~$W^*$.
The map $T^*$ is the
 \index{adjoint}%
\df{adjoint} of~$T$.  In the next two propositions and problem~\ref{prob_adj_bl} we state only
the most elementary properties of the adjoint map $T \mapsto T^*$. We will see more of it
later.
\end{defn}

\begin{prop}\label{prop_adj1} If $T \in \ofml B(V,W)$, then $T^*$ maps $W^*$ into~$V^*$.
Furthermore, if $g \in W^*$, then $\norm{T^*g} \le \norm T \, \norm g$.
\end{prop}

\begin{proof} Exercise. (Solution~\ref{sol_prop_adj1}.)  \ns   \end{proof}

\begin{prop} Let $S \in \ofml B(U,V)$, $T \in \ofml B(V,W)$, and $I_V$ be the identity map on $V$.
Then
 \begin{enumerate}
  \item[(a)] $T^*$ belongs to $\ofml B(W^*,V^*)$ and $\norm{T^*} \le \norm T$;
  \item[(b)] $(I_V)^*$ is $I_{V^*}$, the identity map on $V^*$; and
  \item[(c)] $(TS)^* = S^*T^*$.
 \end{enumerate}
\end{prop}

\begin{proof} Problem.   \ns  \end{proof}

\begin{prob}\label{prob_adj_bl} The adjoint map $T \mapsto T^*$ from $\ofml B(V,W)$ into
$\ofml B(W^*,V^*)$ is itself a bounded linear transformation, and it has norm not
exceeding~$1$.
\end{prob}







\endinput
