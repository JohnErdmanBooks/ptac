\chapter{COMPACTNESS AND THE EXTREME VALUE THEOREM}\label{evt}

One of the most important results in beginning calculus is the \emph{extreme value theorem}: a
continuous function on a closed and bounded subset of the real line achieves both a maximum
and a minimum value. In the present chapter we prove this result. Central to understanding the
\emph{extreme value theorem} is a curious observation: while neither boundedness nor the
property of being closed is preserved by continuity (see problems \ref{evt_pr1} and
\ref{evt_pr2}), the property of being closed \emph{and} bounded is preserved.  Once we have
proved this result it is easy to see that a continuous function defined on a closed and
bounded set attains a maximum and minimum on the set.

Nevertheless, there are some complications along the way.  To begin with, the proof that
continuity preserves the property of being closed and bounded turns out to be awkward and
unnatural.
%%(You can find the argument on pages 99--102 of [AET].)%
Furthermore, although this result can be generalized to $\R^n$, it does not hold in more
general metric spaces.  This suggests---even if it is not conclusive evidence---that we are
looking at the wrong concept.  One of the mathematical triumphs of the early twentieth century
was the recognition that indeed the very concept of closed-and-bounded is a manifestation of
the veil of \emph{m\={a}y\={a}}, a seductively simple vision which obscures the ``real''
topological workings behind the scenes.  Enlightenment, at this level, consists in piercing
this veil of illusion and seeing behind it the ``correct'' concept---compactness.  There is
now overwhelming evidence that compactness \emph{is} the appropriate concept.  First and most
rewarding is the observation that the proofs of the preservation of compactness by continuity
and of the \emph{extreme value theorem} now become extremely natural.  Also, the same proofs
work not only for $\R^n$ but for general metric spaces and even arbitrary topological spaces.
Furthermore, the property of compactness is an intrinsic one---if $A \subseteq B \subseteq C$,
then $A$ is compact in $B$ if and only if it is compact in $C$.  (The property of being closed
is not intrinsic: the interval $(0,1]$ is closed in $(0,\infty)$ but not in $\R$.) Finally,
there is the triumph of products.  In the early 1900's there were other contenders for the
honored place ultimately held by compactness---sequential compactness and countable
compactness. Around 1930 the great Russian mathematician Tychonov was able to show that
arbitrary products of compact spaces are compact, a powerful and useful property not shared by
the competitors.

There is, however, a price to be paid for the wonders of compactness---a frustratingly
unintuitive definition.  It is doubtless this lack of intuitive appeal which explains why it
took workers in the field so long to come up with the optimal notion. Do not be discouraged if
you feel you don't understand the definition.  I'm not sure anyone really ``understands'' it.
What is important is to be able to \emph{use} it.  And that is quite possible---with a little
practice. In this chapter we first define compactness (\ref{df_cpt}) and give two important
examples of compact sets: finite sets (see \ref{cpt_exm1}) and the interval $[0,1]$ (see
\ref{cpt_exm2}). Then we give a number of ways of creating new compact sets from old ones (see
\ref{cpt_cnd1}, \ref{cpt_cnd2}, \ref{cpt_cnd3}(b), and~\ref{cpt_cnd4}).  In \ref{cpt_cnd4} we
show that the continuous image of a compact set is compact and in \ref{evthm} we prove the
\emph{extreme value theorem}.  Finally (in \ref{HBthm}) we prove the \emph{Heine-Borel
theorem} for $\R$: the compact subsets of $\R$ are those which are both closed and bounded.





\section{COMPACTNESS}
\begin{prob}\label{evt_pr1} Give an example to show that if $f$ is a continuous real valued
function of a real variable and $A$ is a closed subset of $\R$ which is contained in the
domain of $f$, then it is not necessarily the case that $f^\sto(A)$ is a closed subset
of~$\R$.
\end{prob}

\begin{prob}\label{evt_pr2} Give an example to show that if $f$ is a continuous real valued
function of a real variable and $A$ is a bounded subset of the domain of $f$, then it is not
necessarily the case that $f^\sto(A)$ is bounded.
\end{prob}

\begin{defn}\label{df_cpt} A family $\sfml U$ of sets is said to
 \index{cover}%
\df{cover} a set $A$ if $\bigcup\,\sfml U \supseteq A$.  The phrases ``$\sfml U$ covers
$A$'', ``$\sfml U$ is a cover for $A$'',  ``$\sfml U$ is a covering of $A$'', and ``$A$
is covered by $\sfml U$'' are used interchangeably.  If $A$ is a subset of $\R$ and
$\sfml U$ is a cover for $A$ which consists entirely of open subsets of $\R$, then $\sfml
U$ is an
 \index{open!cover}%
 \index{cover!open}%
\df{open cover} for~$A$.  If $\sfml U$ is a family of sets which covers $A$ and $\sfml V$
is a subfamily of $\sfml U$ which also covers $A$, then $\sfml V$ is a
 \index{subcover}%
\df{subcover} of $\sfml U$ for~$A$.  A subset $A$ of $\R$ is
 \index{compact}%
\df{compact} if \emph{every} open cover of $A$ has a finite subcover.
\end{defn}

\begin{exam} Let $A = [0,1]$, $U_1 = (-3,\frac23)$, $U_2 = (-1,\frac12)$, $U_3 = (0,\frac12)$,
$U_4 = (\frac13,\frac23)$, $U_5 = (\frac12,1)$, $U_6 = (\frac9{10},2)$, and $U_7 =
(\frac23,\frac32)$.  Then the family $\sfml U = \{U_k \colon 1 \le k \le 7\}$ is an open
cover for $A$ (because each $U_k$ is open and $\bigcup_{k=1}^7U_k = (-3,2) \supseteq A$).
The subfamily $\sfml V = \{U_1, U_5, U_6\}$ of $\sfml U$ is a subcover for $A$ (because
$U_1 \cup U_5 \cup U_6 = (-3,2) \supseteq A$). The subfamily $\sfml W = \{U_1, U_2, U_3,
U_7\}$ of $\sfml U$ is \emph{not} a subcover for $A$ (because $U_1 \cup U_2 \cup U_3 \cup
U_7 = (-3,\frac23) \cup (\frac23,\frac32)$ does not contain $A$).
\end{exam}

\begin{prob} Let $J$ be the open unit interval $(0,1)$. For each $a$ let $U_a =
\left(a,a + \frac14 \right)$, and let $\sfml U = \{U_a \colon 0 \le a \le \frac34\}$.
Then certainly $\sfml U$ covers $J$.
 \begin{enumerate}
  \item[(a)] Find a finite subfamily of $\sfml U$ which covers $J$.
  \item[(b)] Explain why a solution to (a) does not suffice to show that $J$ is compact.
  \item[(c)] Show that $J$ is not compact.
 \end{enumerate}
\end{prob}








\section{EXAMPLES OF COMPACT SUBSETS OF $\R$}
It is easy to give examples of sets that are \emph{not} compact. For example, $\R$ itself
is not compact.  To see this consider the family $\sfml U$ of all open intervals of the
form $(-n,n)$ where $n$ is a natural number.  Then $\sfml U$ covers $\R$ (what property
of $\R$ guarantees this?); but certainly no finite subfamily of $\sfml U$ does.

What is usually a lot trickier, because the definition is hard to apply directly, is proving
that some particular compact set really is compact. The simplest examples of compact spaces
are the finite ones.  (See example \ref{cpt_exm1}.)  Finding nontrivial examples is another
matter.

In this section we guarantee ourselves a generous supply of compact subsets of $\R$ by
specifying some rather powerful methods for creating new compact sets from old ones. In
particular, we will show that a set is compact if it is
 \begin{enumerate}
  \item a closed subset of a compact set (\ref{cpt_cnd1}),
  \item a finite union of compact sets (\ref{cpt_cnd3}(b)), or
  \item the continuous image of a compact set (\ref{cpt_cnd4}).
 \end{enumerate}
Nevertheless it is clear that we need something to start with.  In example \ref{cpt_exm2} we
prove directly from the definition that the closed unit interval $[0,1]$ is compact.  It is
fascinating to see how this single example together with conditions (1)--(3) above can be used
to generate a great variety of compact sets in general metric spaces.  This will be done in
chapter~\ref{cpt}.

\begin{exam}\label{cpt_exm1} Every finite subset of $\R$ is compact.
\end{exam}

\begin{proof} Let $A = \{x_1,x_2, \dots, x_n\}$ be a finite subset of $\R$.  Let $\sfml U$ be a
family of open sets which covers $A$. For each $k = 1,\dots,n$ there is at least one set
$U_k$ in $\sfml U$ which contains $x_k$.  Then the family $\{U_1, \dots, U_n\}$ is a
finite subcover of $\sfml U$ for $A$.
\end{proof}

\begin{prop}\label{cpt_cnd1} Every closed subset of a compact set is compact.
\end{prop}

\begin{proof} Problem.  \emph{Hint.}  Let $A$ be a closed subset of a compact set $K$ and
$\sfml U$ be a cover of $A$ by open sets. Consider $\sfml U \cup \{A^c\}$.   \ns
\end{proof}

\begin{prop}\label{cpt_cb} Every compact subset of $\R$ is closed and bounded.
\end{prop}

\begin{proof} Exercise. \emph{Hint.}  To show that a compact set $A$ is closed, show that
its complement is open.  To this end let $y$ be an arbitrary point in~$A^c$.  For each $x$ in
$A$ take disjoint open intervals about $x$ and~$y$. (Solution~\ref{sol_cpt_cb}.)  \ns
\end{proof}

As we will see later, the preceding result holds in arbitrary metric spaces.  In fact, it is
true in very general topological spaces.  Its converse, the celebrated \emph{Heine-Borel
theorem} (see \ref{HBthm}), although also true in $\R^n$, (see \ref{HBThm}) does \emph{not}
hold in all metric spaces (see problems \ref{cpt_prb2}, \ref{cpt_prb3} and \ref{cpt_prb4}).
Thus it is important at a conceptual level not to regard the property closed-and-bounded as
being identical with compactness.

Now, finally, we give a nontrivial example of a compact space.  The proof requires verifying
some details, which at first glance may make it seem complicated.  The basic idea behind the
proof, however, is quite straightforward.  It repays close study since it involves an
important technique of proof that we will encounter again.  The first time you read the proof,
try to see its structure, to understand its basic logic.  Postpone for a second reading the
details which show that the conditions labelled (1) and (2) hold.  Try to understand instead
why verification of these two conditions is really all we need in order to prove that $[0,1]$
is compact.

\begin{exam}\label{cpt_exm2} The closed unit interval $[0,1]$ is compact.
\end{exam}

\begin{proof} Let $\sfml U$ be a family of open subsets of $\R$ which covers $[0,1]$ and let
$A$ be the set of all $x$ in $[0,1]$ such that the closed interval $[0,x]$ can be covered
by finitely many members of $\sfml U$. It is clear that $A$ is nonempty (since it
contains 0), and that if a number $y$ belongs to $A$ then so does any number in $[0,1]$
less than $y$. We prove two more facts about $A$:
 \begin{enumerate}
  \item[(1)] If $x\in A$ and $x<1$, then there exists a number $y > x$ such that $y \in A$.
  \item[(2)] If $y$ is a number in $[0,1]$ such that $x \in A$ for all $x < y$, then $y\in A$.
 \end{enumerate}

To prove (1) suppose that $x \in A$ and $x<1$. Since $x \in A$ there exists sets $U_1,
\dots, U_n$ in $\sfml U$ which cover the interval $[0,x]$. The number $x$ belongs to at
least one of these sets, say $U_1$. Since $U_1$ is an open subset of $\R$, there is an
interval $(a,b)$ such that $x \in (a,b) \subseteq U_1$.  Since $x<1$ and $x<b$, there
exists a number $y \in (0,1)$ such that $x < y < b$.  From $[0,x] \subseteq
\bigcup_{k=1}^n U_k$ and $[x,y] \subseteq (a,b) \subseteq U_1$ it follows that $U_1,
\dots, U_n$ cover the interval $[0,y]$. Thus $y > x$ and $y$ belongs to $A$.

The proof of (2) is similar. Suppose that $y \in [0,1]$ and that $[0,y) \subseteq A$. The
case $y=0$ is trivial so we suppose that $y>0$. Then $y$ belongs to at least one member
of $\sfml U$, say $V$. Choose an open interval $(a,b)$ in $[0,1]$ such that $y \in (a,b)
\subseteq V$. Since $a \in A$ there is a finite collection of sets $U_1, \dots, U_n$ in
$\sfml U$ which covers $[0,a]$. Then clearly $\{U_1, \dots, U_n, V\}$ is a cover for
$[0,y]$. This shows that $y$ belongs to $A$.

Finally, let $u = \sup A$. We are done if we can show that $u = 1$. Suppose to the contrary
that $u<1$. Then $[0,u) \subseteq A$. We conclude from (2) that $u \in A$ and then from (1)
that there is a point greater than $u$ which belongs to $A$.  This contradicts the choice of
$u$ as the supremum of $A$.
\end{proof}

\begin{prob} Let $A = \{0\} \cup \{1/n \colon n \in \N\}$ and $\sfml U$ be the family
$\{U_n \colon n \ge 0\}$ where $U_0 = (-1,0.1)$ and $U_n = \left(\dfrac 5{6n},\dfrac
7{6n}\right)$ for $n \ge 1$.
 \begin{enumerate}
  \item[(a)] Find a finite subfamily of $\sfml U$ which covers $A$.
  \item[(b)] Explain why a solution to (a) does not suffice to show
that $A$ is compact.
  \item[(c)] Use the \empty{definition} of compactness to show that
$A$ is compact.
  \item[(d)] Use proposition \ref{cpt_cnd1} to show that $A$ is
compact.
 \end{enumerate}
\end{prob}

\begin{exam} Show that the set $A = \{1/n \colon n \in \N\}$ is not a compact subset of~$\R$.
\end{exam}

\begin{proof} Problem.  \ns \end{proof}

\begin{prob}  Give two proofs that the interval $[0,1)$ is not compact---one making use of
proposition \ref{cpt_cb} and one not.
\end{prob}

\begin{prop}\label{cpt_cnd2} The intersection of a nonempty collection of compact subsets
of $\R$ is itself compact.
\end{prop}

\begin{proof} Problem.  \ns \end{proof}

\begin{prob}\label{cpt_cnd3} Let $\sfml K$ be the family of compact subsets of~$\R$.
 \begin{enumerate}
  \item[(a)] Show that $\bigcup\sfml K$ need not be compact.
  \item[(b)] Show that if $\sfml K$ contains only finitely many sets, then
$\bigcup\sfml K$ is compact.
 \end{enumerate}
\end{prob}







\section{THE EXTREME VALUE THEOREM}
\begin{defn} A real-valued function $f$ on a set $A$ is said to have a
 \index{maximum}%
\df{maximum} at a point $a$ in $A$ if $f(a) \ge f(x)$ for every $x$ in~$A$; the number $f(a)$
is the
 \index{maximum!value}%
\df{maximum value} of $f$. The function has a
 \index{minimum}%
\df{minimum} at $a$ if $f(a) \le f(x)$ for every $x$ in $A$; and in this case $f(a)$ is the
 \index{minimum!value}%
\df{minimum value} of $f$. A number is an
 \index{extreme!value}%
\df{extreme value} of $f$ if it is either a maximum or a minimum value. It is clear that a
function may fail to have maximum or minimum values.  For example, on the open interval
$(0,1)$ the function $f:x\mapsto x$ assumes neither a maximum nor a minimum.
\end{defn}

The concepts we have just defined are frequently called
 \index{global!maximum}%
 \index{absolute!maximum}%
 \index{maximum!global}%
 \index{maximum!absolute}%
\df{global} (or \df{absolute}) \df{maximum} and
 \index{global!minimum}%
 \index{absolute!minimum}%
 \index{minimum!global}%
 \index{minimum!absolute}%
\df{global} (or \df{absolute}) \df{minimum}.  This is to
distinguish them from two different ideas
 \index{local!maximum}%
 \index{relative!maximum}%
 \index{maximum!local}%
 \index{maximum!relative}%
\emph{local} (or \emph{relative}) \emph{maximum} and
 \index{local!minimum}%
 \index{relative!minimum}%
 \index{minimum!local}%
 \index{minimum!relative}%
\emph{local} (or \emph{relative}) \emph{minimum}, which we will encounter later.   In this
text, ``maximum'' and ``minimum'' without qualifiers will be the global concepts defined
above.

Our goal now is to show that every continuous function on a compact set attains both a maximum
and a minimum.  This turns out to be an easy consequence of the fact, which we prove next,
that the continuous image of a compact set is compact.

\begin{thm}\label{cpt_cnd4} Let $A$ be a subset of $\R$.  If $A$ is compact and $f\colon A \sto \R$
is continuous, then $f^{\sto}(A)$ is compact.
\end{thm}

\begin{proof} Exercise. (Solution~\ref{sol_cpt_cnd4}.)
   \ns \end{proof}

\begin{thm}[Extreme Value Theorem]\label{evthm} If $A$ is a compact subset of $\R$ and
$f \colon A \sto \R$ is continuous, then $f$ assumes both a maximum and a minimum value
on~$A$.
\end{thm}

\begin{proof} Exercise.  (Solution~\ref{sol_evthm}.)
  \ns \end{proof}

\begin{exam}  The closed interval $[-3,7]$ is a compact subset of~$\R$.
\end{exam}

\begin{proof} Let $A = [0,1]$ and $f(x) = 10x-3$.  Since $A$ is compact and $f$ is continuous,
theorem \ref{cpt_cnd4} tells us that the set $[-3,7] = f^\sto(A)$ is compact.
\end{proof}

\begin{exam}\label{cpt_exm3} If $a < b$, then the closed interval $[a,b]$ is a compact subset of~$\R$.
\end{exam}


\begin{proof} Problem.  \ns \end{proof}

\begin{exam}[Heine-Borel Theorem for $\R$]\label{HBthm} Every closed and bounded subset of $\R$ is compact.
\end{exam}

\begin{proof} Problem. \emph{Hint.}  Use \ref{cpt_exm3} and \ref{cpt_cnd1}. \ns
\end{proof}

\begin{exam} Define $f \colon [-1,1] \sto \R$ by
   \[f(x) = \begin{cases}
             x\sin\left(\dfrac1x\right), &\text{if $x \ne 0$}\\
                                      0, &\text{if $x=0$}.
            \end{cases}\]
The set of all $x$ such that $f(x) = 0$ is a compact subset of~$[-1,1]$.
\end{exam}

\begin{proof} Problem.  \ns \end{proof}

\begin{prob} Let $f \colon A \sto B$ be a continuous bijection between subsets of~$\R$.
 \begin{enumerate}
  \item[(a)] Show by example that $f$ need not be a homeomorphism.
  \item[(b)] Show that if $A$ is compact, then $f$ must be a homeomorphism.
 \end{enumerate}
\end{prob}

\begin{prob}\label{cpt_prb2} Find in $\Q$ a set which is both relatively closed and bounded
but which is not compact.
\end{prob}

\begin{prob} Show that the interval $[0,\infty)$ is not compact using each of the following:
 \begin{enumerate}
  \item[(a)] the definition of compactness;
  \item[(b)] proposition \ref{cpt_cb};
  \item[(c)] the \emph{extreme value theorem}.
 \end{enumerate}
\end{prob}

\begin{prob}Let $f$ and $g$ be two continuous functions mapping the interval $[0,1]$ into itself.
Show that if $f \circ g = g \circ f$, then $f$ and $g$ agree at some point of $[0,1]$.
\emph{Hint.} Argue by contradiction. Show that we may suppose, without loss of generality,
that $f(x) - g(x) > 0$ for all $x$ in~$[0,1]$.  Now try to show that there is a number $a > 0$
such that $f^n(x) \ge g^n(x) + na$ for every natural number $n$ and every $x$ in~$[0,1]$.
(Here $f^n = f \circ f \circ \dots \circ f$ ($n$ copies of $f$); and $g^n$ is defined
similarly.)
\end{prob}



\endinput
